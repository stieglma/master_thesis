\pdfbookmark[1]{Abstract}{Abstract}
\chapter*{Abstract}
Predicate analysis is a common approach to software model checking. Abstractions of programs are computed out of predicates found with craig interpolation. The found interpolants are, however,
in some cases not ideal, and lead to long-running verification runs. To reduce the reliance on interpolation this thesis evaluates the effects of using separately computed, auxiliary, invariants instead.

Our work is based on the CPA concept, \CPAchecker{} and the \PredicateCPA{}. It is split into two major parts, on the one hand we introduce a new algorithm for concurrent execution of several analysis in 
\CPAchecker{}, as well as communication between such analysis,
and on the other hand we show how the \PredicateCPA{} can be augmented with additional formulas in several ways. We chose to evaluate: appending invariants to the precision of the analysis and 
conjoining invariants either to the path formula or to the abstraction formula. The invariants we want to use are generated by some new approaches directly in \CPAchecker{}. They can
be separated in two classes, on the one hand, the on-the-fly and lightweight invariant generation heuristics which try to find invariants for a certain given program location only, and
on the other hand complete analyses, which results are then used for generating invariants for the whole program. 

While the lightweight on-the-fly approaches did not yield the expected results, analyses using concurrently computed invariants perform strictly better than comparable analyses without invariants.



