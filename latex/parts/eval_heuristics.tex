\subsection{Lightweight Heuristics}
With lightweight heuristics we mean all approaches that can be computed on the fly and which should --- compared to running more analyses in parallel or sequential combinations --- take only short amounts of time. The configurations we describe here are path invariants, inductive weakening of path formulas, checking the invariance of interpolants and checking the invariance of conjuncts of the path formula. All configurations have a limit of \SI{300}{\second} overall CPU time, there is no extra time for the invariant generation.

%%% predicate_base.2016-09-03_1927.results.pred-bitvectors %%%
 %% correct %%
\providecommand{\predicateBaseResultsPredBitvectorsCorrectPlain}{}
  \renewcommand{\predicateBaseResultsPredBitvectorsCorrectPlain}{1944\xspace}

  % cpu-time-sum
\providecommand{\predicateBaseResultsPredBitvectorsCorrectCpuTimeSumPlain}{}
  \renewcommand{\predicateBaseResultsPredBitvectorsCorrectCpuTimeSumPlain}{93675.79971751993\xspace}
\providecommand{\predicateBaseResultsPredBitvectorsCorrectCpuTimeSumPlainHours}{}
  \renewcommand{\predicateBaseResultsPredBitvectorsCorrectCpuTimeSumPlainHours}{26.02105547708887\xspace}

  % cpu-time-avg
\providecommand{\predicateBaseResultsPredBitvectorsCorrectCpuTimeAvgPlain}{}
  \renewcommand{\predicateBaseResultsPredBitvectorsCorrectCpuTimeAvgPlain}{48.18713977238679\xspace}
\providecommand{\predicateBaseResultsPredBitvectorsCorrectCpuTimeAvgPlainHours}{}
  \renewcommand{\predicateBaseResultsPredBitvectorsCorrectCpuTimeAvgPlainHours}{0.013385316603440776\xspace}

  % inv-succ
\providecommand{\predicateBaseResultsPredBitvectorsCorrectInvSuccPlain}{}
  \renewcommand{\predicateBaseResultsPredBitvectorsCorrectInvSuccPlain}{0\xspace}

  % inv-tries
\providecommand{\predicateBaseResultsPredBitvectorsCorrectInvTriesPlain}{}
  \renewcommand{\predicateBaseResultsPredBitvectorsCorrectInvTriesPlain}{0\xspace}

  % inv-time-sum
\providecommand{\predicateBaseResultsPredBitvectorsCorrectInvTimeSumPlain}{}
  \renewcommand{\predicateBaseResultsPredBitvectorsCorrectInvTimeSumPlain}{0.0\xspace}
\providecommand{\predicateBaseResultsPredBitvectorsCorrectInvTimeSumPlainHours}{}
  \renewcommand{\predicateBaseResultsPredBitvectorsCorrectInvTimeSumPlainHours}{0.0\xspace}

 %% incorrect %%
\providecommand{\predicateBaseResultsPredBitvectorsIncorrectPlain}{}
  \renewcommand{\predicateBaseResultsPredBitvectorsIncorrectPlain}{27\xspace}

  % cpu-time-sum
\providecommand{\predicateBaseResultsPredBitvectorsIncorrectCpuTimeSumPlain}{}
  \renewcommand{\predicateBaseResultsPredBitvectorsIncorrectCpuTimeSumPlain}{712.5640386970001\xspace}
\providecommand{\predicateBaseResultsPredBitvectorsIncorrectCpuTimeSumPlainHours}{}
  \renewcommand{\predicateBaseResultsPredBitvectorsIncorrectCpuTimeSumPlainHours}{0.19793445519361114\xspace}

  % cpu-time-avg
\providecommand{\predicateBaseResultsPredBitvectorsIncorrectCpuTimeAvgPlain}{}
  \renewcommand{\predicateBaseResultsPredBitvectorsIncorrectCpuTimeAvgPlain}{26.391260692481485\xspace}
\providecommand{\predicateBaseResultsPredBitvectorsIncorrectCpuTimeAvgPlainHours}{}
  \renewcommand{\predicateBaseResultsPredBitvectorsIncorrectCpuTimeAvgPlainHours}{0.007330905747911523\xspace}

  % inv-succ
\providecommand{\predicateBaseResultsPredBitvectorsIncorrectInvSuccPlain}{}
  \renewcommand{\predicateBaseResultsPredBitvectorsIncorrectInvSuccPlain}{0\xspace}

  % inv-tries
\providecommand{\predicateBaseResultsPredBitvectorsIncorrectInvTriesPlain}{}
  \renewcommand{\predicateBaseResultsPredBitvectorsIncorrectInvTriesPlain}{0\xspace}

  % inv-time-sum
\providecommand{\predicateBaseResultsPredBitvectorsIncorrectInvTimeSumPlain}{}
  \renewcommand{\predicateBaseResultsPredBitvectorsIncorrectInvTimeSumPlain}{0.0\xspace}
\providecommand{\predicateBaseResultsPredBitvectorsIncorrectInvTimeSumPlainHours}{}
  \renewcommand{\predicateBaseResultsPredBitvectorsIncorrectInvTimeSumPlainHours}{0.0\xspace}

 %% timeout %%
\providecommand{\predicateBaseResultsPredBitvectorsTimeoutPlain}{}
  \renewcommand{\predicateBaseResultsPredBitvectorsTimeoutPlain}{1414\xspace}

  % cpu-time-sum
\providecommand{\predicateBaseResultsPredBitvectorsTimeoutCpuTimeSumPlain}{}
  \renewcommand{\predicateBaseResultsPredBitvectorsTimeoutCpuTimeSumPlain}{432542.40992774273\xspace}
\providecommand{\predicateBaseResultsPredBitvectorsTimeoutCpuTimeSumPlainHours}{}
  \renewcommand{\predicateBaseResultsPredBitvectorsTimeoutCpuTimeSumPlainHours}{120.15066942437298\xspace}

  % cpu-time-avg
\providecommand{\predicateBaseResultsPredBitvectorsTimeoutCpuTimeAvgPlain}{}
  \renewcommand{\predicateBaseResultsPredBitvectorsTimeoutCpuTimeAvgPlain}{305.8998655783188\xspace}
\providecommand{\predicateBaseResultsPredBitvectorsTimeoutCpuTimeAvgPlainHours}{}
  \renewcommand{\predicateBaseResultsPredBitvectorsTimeoutCpuTimeAvgPlainHours}{0.08497218488286633\xspace}

  % inv-succ
\providecommand{\predicateBaseResultsPredBitvectorsTimeoutInvSuccPlain}{}
  \renewcommand{\predicateBaseResultsPredBitvectorsTimeoutInvSuccPlain}{0\xspace}

  % inv-tries
\providecommand{\predicateBaseResultsPredBitvectorsTimeoutInvTriesPlain}{}
  \renewcommand{\predicateBaseResultsPredBitvectorsTimeoutInvTriesPlain}{0\xspace}

  % inv-time-sum
\providecommand{\predicateBaseResultsPredBitvectorsTimeoutInvTimeSumPlain}{}
  \renewcommand{\predicateBaseResultsPredBitvectorsTimeoutInvTimeSumPlain}{0.0\xspace}
\providecommand{\predicateBaseResultsPredBitvectorsTimeoutInvTimeSumPlainHours}{}
  \renewcommand{\predicateBaseResultsPredBitvectorsTimeoutInvTimeSumPlainHours}{0.0\xspace}

 %% unknown-or-category-error %%
\providecommand{\predicateBaseResultsPredBitvectorsUnknownOrCategoryErrorPlain}{}
  \renewcommand{\predicateBaseResultsPredBitvectorsUnknownOrCategoryErrorPlain}{1517\xspace}

  % cpu-time-sum
\providecommand{\predicateBaseResultsPredBitvectorsUnknownOrCategoryErrorCpuTimeSumPlain}{}
  \renewcommand{\predicateBaseResultsPredBitvectorsUnknownOrCategoryErrorCpuTimeSumPlain}{440619.1379288969\xspace}
\providecommand{\predicateBaseResultsPredBitvectorsUnknownOrCategoryErrorCpuTimeSumPlainHours}{}
  \renewcommand{\predicateBaseResultsPredBitvectorsUnknownOrCategoryErrorCpuTimeSumPlainHours}{122.39420498024913\xspace}

  % cpu-time-avg
\providecommand{\predicateBaseResultsPredBitvectorsUnknownOrCategoryErrorCpuTimeAvgPlain}{}
  \renewcommand{\predicateBaseResultsPredBitvectorsUnknownOrCategoryErrorCpuTimeAvgPlain}{290.4542768153572\xspace}
\providecommand{\predicateBaseResultsPredBitvectorsUnknownOrCategoryErrorCpuTimeAvgPlainHours}{}
  \renewcommand{\predicateBaseResultsPredBitvectorsUnknownOrCategoryErrorCpuTimeAvgPlainHours}{0.08068174355982145\xspace}

  % inv-succ
\providecommand{\predicateBaseResultsPredBitvectorsUnknownOrCategoryErrorInvSuccPlain}{}
  \renewcommand{\predicateBaseResultsPredBitvectorsUnknownOrCategoryErrorInvSuccPlain}{0\xspace}

  % inv-tries
\providecommand{\predicateBaseResultsPredBitvectorsUnknownOrCategoryErrorInvTriesPlain}{}
  \renewcommand{\predicateBaseResultsPredBitvectorsUnknownOrCategoryErrorInvTriesPlain}{0\xspace}

  % inv-time-sum
\providecommand{\predicateBaseResultsPredBitvectorsUnknownOrCategoryErrorInvTimeSumPlain}{}
  \renewcommand{\predicateBaseResultsPredBitvectorsUnknownOrCategoryErrorInvTimeSumPlain}{0.0\xspace}
\providecommand{\predicateBaseResultsPredBitvectorsUnknownOrCategoryErrorInvTimeSumPlainHours}{}
  \renewcommand{\predicateBaseResultsPredBitvectorsUnknownOrCategoryErrorInvTimeSumPlainHours}{0.0\xspace}

 %% correct-false %%
\providecommand{\predicateBaseResultsPredBitvectorsCorrectFalsePlain}{}
  \renewcommand{\predicateBaseResultsPredBitvectorsCorrectFalsePlain}{553\xspace}

  % cpu-time-sum
\providecommand{\predicateBaseResultsPredBitvectorsCorrectFalseCpuTimeSumPlain}{}
  \renewcommand{\predicateBaseResultsPredBitvectorsCorrectFalseCpuTimeSumPlain}{38293.01043458401\xspace}
\providecommand{\predicateBaseResultsPredBitvectorsCorrectFalseCpuTimeSumPlainHours}{}
  \renewcommand{\predicateBaseResultsPredBitvectorsCorrectFalseCpuTimeSumPlainHours}{10.636947342940003\xspace}

  % cpu-time-avg
\providecommand{\predicateBaseResultsPredBitvectorsCorrectFalseCpuTimeAvgPlain}{}
  \renewcommand{\predicateBaseResultsPredBitvectorsCorrectFalseCpuTimeAvgPlain}{69.2459501529548\xspace}
\providecommand{\predicateBaseResultsPredBitvectorsCorrectFalseCpuTimeAvgPlainHours}{}
  \renewcommand{\predicateBaseResultsPredBitvectorsCorrectFalseCpuTimeAvgPlainHours}{0.019234986153598557\xspace}

  % inv-succ
\providecommand{\predicateBaseResultsPredBitvectorsCorrectFalseInvSuccPlain}{}
  \renewcommand{\predicateBaseResultsPredBitvectorsCorrectFalseInvSuccPlain}{0\xspace}

  % inv-tries
\providecommand{\predicateBaseResultsPredBitvectorsCorrectFalseInvTriesPlain}{}
  \renewcommand{\predicateBaseResultsPredBitvectorsCorrectFalseInvTriesPlain}{0\xspace}

  % inv-time-sum
\providecommand{\predicateBaseResultsPredBitvectorsCorrectFalseInvTimeSumPlain}{}
  \renewcommand{\predicateBaseResultsPredBitvectorsCorrectFalseInvTimeSumPlain}{0.0\xspace}
\providecommand{\predicateBaseResultsPredBitvectorsCorrectFalseInvTimeSumPlainHours}{}
  \renewcommand{\predicateBaseResultsPredBitvectorsCorrectFalseInvTimeSumPlainHours}{0.0\xspace}

 %% correct-true %%
\providecommand{\predicateBaseResultsPredBitvectorsCorrectTruePlain}{}
  \renewcommand{\predicateBaseResultsPredBitvectorsCorrectTruePlain}{1391\xspace}

  % cpu-time-sum
\providecommand{\predicateBaseResultsPredBitvectorsCorrectTrueCpuTimeSumPlain}{}
  \renewcommand{\predicateBaseResultsPredBitvectorsCorrectTrueCpuTimeSumPlain}{55382.789282936064\xspace}
\providecommand{\predicateBaseResultsPredBitvectorsCorrectTrueCpuTimeSumPlainHours}{}
  \renewcommand{\predicateBaseResultsPredBitvectorsCorrectTrueCpuTimeSumPlainHours}{15.384108134148907\xspace}

  % cpu-time-avg
\providecommand{\predicateBaseResultsPredBitvectorsCorrectTrueCpuTimeAvgPlain}{}
  \renewcommand{\predicateBaseResultsPredBitvectorsCorrectTrueCpuTimeAvgPlain}{39.81508934790515\xspace}
\providecommand{\predicateBaseResultsPredBitvectorsCorrectTrueCpuTimeAvgPlainHours}{}
  \renewcommand{\predicateBaseResultsPredBitvectorsCorrectTrueCpuTimeAvgPlainHours}{0.011059747041084764\xspace}

  % inv-succ
\providecommand{\predicateBaseResultsPredBitvectorsCorrectTrueInvSuccPlain}{}
  \renewcommand{\predicateBaseResultsPredBitvectorsCorrectTrueInvSuccPlain}{0\xspace}

  % inv-tries
\providecommand{\predicateBaseResultsPredBitvectorsCorrectTrueInvTriesPlain}{}
  \renewcommand{\predicateBaseResultsPredBitvectorsCorrectTrueInvTriesPlain}{0\xspace}

  % inv-time-sum
\providecommand{\predicateBaseResultsPredBitvectorsCorrectTrueInvTimeSumPlain}{}
  \renewcommand{\predicateBaseResultsPredBitvectorsCorrectTrueInvTimeSumPlain}{0.0\xspace}
\providecommand{\predicateBaseResultsPredBitvectorsCorrectTrueInvTimeSumPlainHours}{}
  \renewcommand{\predicateBaseResultsPredBitvectorsCorrectTrueInvTimeSumPlainHours}{0.0\xspace}

 %% incorrect-false %%
\providecommand{\predicateBaseResultsPredBitvectorsIncorrectFalsePlain}{}
  \renewcommand{\predicateBaseResultsPredBitvectorsIncorrectFalsePlain}{27\xspace}

  % cpu-time-sum
\providecommand{\predicateBaseResultsPredBitvectorsIncorrectFalseCpuTimeSumPlain}{}
  \renewcommand{\predicateBaseResultsPredBitvectorsIncorrectFalseCpuTimeSumPlain}{712.5640386970001\xspace}
\providecommand{\predicateBaseResultsPredBitvectorsIncorrectFalseCpuTimeSumPlainHours}{}
  \renewcommand{\predicateBaseResultsPredBitvectorsIncorrectFalseCpuTimeSumPlainHours}{0.19793445519361114\xspace}

  % cpu-time-avg
\providecommand{\predicateBaseResultsPredBitvectorsIncorrectFalseCpuTimeAvgPlain}{}
  \renewcommand{\predicateBaseResultsPredBitvectorsIncorrectFalseCpuTimeAvgPlain}{26.391260692481485\xspace}
\providecommand{\predicateBaseResultsPredBitvectorsIncorrectFalseCpuTimeAvgPlainHours}{}
  \renewcommand{\predicateBaseResultsPredBitvectorsIncorrectFalseCpuTimeAvgPlainHours}{0.007330905747911523\xspace}

  % inv-succ
\providecommand{\predicateBaseResultsPredBitvectorsIncorrectFalseInvSuccPlain}{}
  \renewcommand{\predicateBaseResultsPredBitvectorsIncorrectFalseInvSuccPlain}{0\xspace}

  % inv-tries
\providecommand{\predicateBaseResultsPredBitvectorsIncorrectFalseInvTriesPlain}{}
  \renewcommand{\predicateBaseResultsPredBitvectorsIncorrectFalseInvTriesPlain}{0\xspace}

  % inv-time-sum
\providecommand{\predicateBaseResultsPredBitvectorsIncorrectFalseInvTimeSumPlain}{}
  \renewcommand{\predicateBaseResultsPredBitvectorsIncorrectFalseInvTimeSumPlain}{0.0\xspace}
\providecommand{\predicateBaseResultsPredBitvectorsIncorrectFalseInvTimeSumPlainHours}{}
  \renewcommand{\predicateBaseResultsPredBitvectorsIncorrectFalseInvTimeSumPlainHours}{0.0\xspace}

 %% incorrect-true %%
\providecommand{\predicateBaseResultsPredBitvectorsIncorrectTruePlain}{}
  \renewcommand{\predicateBaseResultsPredBitvectorsIncorrectTruePlain}{0\xspace}

  % cpu-time-sum
\providecommand{\predicateBaseResultsPredBitvectorsIncorrectTrueCpuTimeSumPlain}{}
  \renewcommand{\predicateBaseResultsPredBitvectorsIncorrectTrueCpuTimeSumPlain}{0.0\xspace}
\providecommand{\predicateBaseResultsPredBitvectorsIncorrectTrueCpuTimeSumPlainHours}{}
  \renewcommand{\predicateBaseResultsPredBitvectorsIncorrectTrueCpuTimeSumPlainHours}{0.0\xspace}

  % cpu-time-avg
\providecommand{\predicateBaseResultsPredBitvectorsIncorrectTrueCpuTimeAvgPlain}{}
  \renewcommand{\predicateBaseResultsPredBitvectorsIncorrectTrueCpuTimeAvgPlain}{NaN\xspace}
\providecommand{\predicateBaseResultsPredBitvectorsIncorrectTrueCpuTimeAvgPlainHours}{}
  \renewcommand{\predicateBaseResultsPredBitvectorsIncorrectTrueCpuTimeAvgPlainHours}{NaN\xspace}

  % inv-succ
\providecommand{\predicateBaseResultsPredBitvectorsIncorrectTrueInvSuccPlain}{}
  \renewcommand{\predicateBaseResultsPredBitvectorsIncorrectTrueInvSuccPlain}{0\xspace}

  % inv-tries
\providecommand{\predicateBaseResultsPredBitvectorsIncorrectTrueInvTriesPlain}{}
  \renewcommand{\predicateBaseResultsPredBitvectorsIncorrectTrueInvTriesPlain}{0\xspace}

  % inv-time-sum
\providecommand{\predicateBaseResultsPredBitvectorsIncorrectTrueInvTimeSumPlain}{}
  \renewcommand{\predicateBaseResultsPredBitvectorsIncorrectTrueInvTimeSumPlain}{0.0\xspace}
\providecommand{\predicateBaseResultsPredBitvectorsIncorrectTrueInvTimeSumPlainHours}{}
  \renewcommand{\predicateBaseResultsPredBitvectorsIncorrectTrueInvTimeSumPlainHours}{0.0\xspace}

 %% all %%
\providecommand{\predicateBaseResultsPredBitvectorsAllPlain}{}
  \renewcommand{\predicateBaseResultsPredBitvectorsAllPlain}{3488\xspace}

  % cpu-time-sum
\providecommand{\predicateBaseResultsPredBitvectorsAllCpuTimeSumPlain}{}
  \renewcommand{\predicateBaseResultsPredBitvectorsAllCpuTimeSumPlain}{535007.5016851136\xspace}
\providecommand{\predicateBaseResultsPredBitvectorsAllCpuTimeSumPlainHours}{}
  \renewcommand{\predicateBaseResultsPredBitvectorsAllCpuTimeSumPlainHours}{148.61319491253155\xspace}

  % cpu-time-avg
\providecommand{\predicateBaseResultsPredBitvectorsAllCpuTimeAvgPlain}{}
  \renewcommand{\predicateBaseResultsPredBitvectorsAllCpuTimeAvgPlain}{153.38517823541102\xspace}
\providecommand{\predicateBaseResultsPredBitvectorsAllCpuTimeAvgPlainHours}{}
  \renewcommand{\predicateBaseResultsPredBitvectorsAllCpuTimeAvgPlainHours}{0.04260699395428084\xspace}

  % inv-succ
\providecommand{\predicateBaseResultsPredBitvectorsAllInvSuccPlain}{}
  \renewcommand{\predicateBaseResultsPredBitvectorsAllInvSuccPlain}{0\xspace}

  % inv-tries
\providecommand{\predicateBaseResultsPredBitvectorsAllInvTriesPlain}{}
  \renewcommand{\predicateBaseResultsPredBitvectorsAllInvTriesPlain}{0\xspace}

  % inv-time-sum
\providecommand{\predicateBaseResultsPredBitvectorsAllInvTimeSumPlain}{}
  \renewcommand{\predicateBaseResultsPredBitvectorsAllInvTimeSumPlain}{0.0\xspace}
\providecommand{\predicateBaseResultsPredBitvectorsAllInvTimeSumPlainHours}{}
  \renewcommand{\predicateBaseResultsPredBitvectorsAllInvTimeSumPlainHours}{0.0\xspace}

 %% equal-only %%
\providecommand{\predicateBaseResultsPredBitvectorsEqualOnlyPlain}{}
  \renewcommand{\predicateBaseResultsPredBitvectorsEqualOnlyPlain}{1754\xspace}

  % cpu-time-sum
\providecommand{\predicateBaseResultsPredBitvectorsEqualOnlyCpuTimeSumPlain}{}
  \renewcommand{\predicateBaseResultsPredBitvectorsEqualOnlyCpuTimeSumPlain}{64068.23991908711\xspace}
\providecommand{\predicateBaseResultsPredBitvectorsEqualOnlyCpuTimeSumPlainHours}{}
  \renewcommand{\predicateBaseResultsPredBitvectorsEqualOnlyCpuTimeSumPlainHours}{17.79673331085753\xspace}

  % cpu-time-avg
\providecommand{\predicateBaseResultsPredBitvectorsEqualOnlyCpuTimeAvgPlain}{}
  \renewcommand{\predicateBaseResultsPredBitvectorsEqualOnlyCpuTimeAvgPlain}{36.52693267906905\xspace}
\providecommand{\predicateBaseResultsPredBitvectorsEqualOnlyCpuTimeAvgPlainHours}{}
  \renewcommand{\predicateBaseResultsPredBitvectorsEqualOnlyCpuTimeAvgPlainHours}{0.010146370188630292\xspace}

  % inv-succ
\providecommand{\predicateBaseResultsPredBitvectorsEqualOnlyInvSuccPlain}{}
  \renewcommand{\predicateBaseResultsPredBitvectorsEqualOnlyInvSuccPlain}{0\xspace}

  % inv-tries
\providecommand{\predicateBaseResultsPredBitvectorsEqualOnlyInvTriesPlain}{}
  \renewcommand{\predicateBaseResultsPredBitvectorsEqualOnlyInvTriesPlain}{0\xspace}

  % inv-time-sum
\providecommand{\predicateBaseResultsPredBitvectorsEqualOnlyInvTimeSumPlain}{}
  \renewcommand{\predicateBaseResultsPredBitvectorsEqualOnlyInvTimeSumPlain}{0.0\xspace}
\providecommand{\predicateBaseResultsPredBitvectorsEqualOnlyInvTimeSumPlainHours}{}
  \renewcommand{\predicateBaseResultsPredBitvectorsEqualOnlyInvTimeSumPlainHours}{0.0\xspace}

%%% predicate_bitprecise_interpol_kind.2016-09-04_2044.results.RF_interpol-prec %%%
 %% correct %%
\providecommand{\predicateBitpreciseInterpolKindResultsRFInterpolPrecCorrectPlain}{}
  \renewcommand{\predicateBitpreciseInterpolKindResultsRFInterpolPrecCorrectPlain}{1817\xspace}

  % cpu-time-sum
\providecommand{\predicateBitpreciseInterpolKindResultsRFInterpolPrecCorrectCpuTimeSumPlain}{}
  \renewcommand{\predicateBitpreciseInterpolKindResultsRFInterpolPrecCorrectCpuTimeSumPlain}{105972.85310964301\xspace}
\providecommand{\predicateBitpreciseInterpolKindResultsRFInterpolPrecCorrectCpuTimeSumPlainHours}{}
  \renewcommand{\predicateBitpreciseInterpolKindResultsRFInterpolPrecCorrectCpuTimeSumPlainHours}{29.436903641567504\xspace}

  % cpu-time-avg
\providecommand{\predicateBitpreciseInterpolKindResultsRFInterpolPrecCorrectCpuTimeAvgPlain}{}
  \renewcommand{\predicateBitpreciseInterpolKindResultsRFInterpolPrecCorrectCpuTimeAvgPlain}{58.32297914674904\xspace}
\providecommand{\predicateBitpreciseInterpolKindResultsRFInterpolPrecCorrectCpuTimeAvgPlainHours}{}
  \renewcommand{\predicateBitpreciseInterpolKindResultsRFInterpolPrecCorrectCpuTimeAvgPlainHours}{0.016200827540763622\xspace}

  % inv-succ
\providecommand{\predicateBitpreciseInterpolKindResultsRFInterpolPrecCorrectInvSuccPlain}{}
  \renewcommand{\predicateBitpreciseInterpolKindResultsRFInterpolPrecCorrectInvSuccPlain}{1272\xspace}

  % inv-tries
\providecommand{\predicateBitpreciseInterpolKindResultsRFInterpolPrecCorrectInvTriesPlain}{}
  \renewcommand{\predicateBitpreciseInterpolKindResultsRFInterpolPrecCorrectInvTriesPlain}{6455\xspace}

  % inv-time-sum
\providecommand{\predicateBitpreciseInterpolKindResultsRFInterpolPrecCorrectInvTimeSumPlain}{}
  \renewcommand{\predicateBitpreciseInterpolKindResultsRFInterpolPrecCorrectInvTimeSumPlain}{25538.078999999994\xspace}
\providecommand{\predicateBitpreciseInterpolKindResultsRFInterpolPrecCorrectInvTimeSumPlainHours}{}
  \renewcommand{\predicateBitpreciseInterpolKindResultsRFInterpolPrecCorrectInvTimeSumPlainHours}{7.093910833333331\xspace}

 %% incorrect %%
\providecommand{\predicateBitpreciseInterpolKindResultsRFInterpolPrecIncorrectPlain}{}
  \renewcommand{\predicateBitpreciseInterpolKindResultsRFInterpolPrecIncorrectPlain}{23\xspace}

  % cpu-time-sum
\providecommand{\predicateBitpreciseInterpolKindResultsRFInterpolPrecIncorrectCpuTimeSumPlain}{}
  \renewcommand{\predicateBitpreciseInterpolKindResultsRFInterpolPrecIncorrectCpuTimeSumPlain}{1057.0374651350003\xspace}
\providecommand{\predicateBitpreciseInterpolKindResultsRFInterpolPrecIncorrectCpuTimeSumPlainHours}{}
  \renewcommand{\predicateBitpreciseInterpolKindResultsRFInterpolPrecIncorrectCpuTimeSumPlainHours}{0.29362151809305564\xspace}

  % cpu-time-avg
\providecommand{\predicateBitpreciseInterpolKindResultsRFInterpolPrecIncorrectCpuTimeAvgPlain}{}
  \renewcommand{\predicateBitpreciseInterpolKindResultsRFInterpolPrecIncorrectCpuTimeAvgPlain}{45.95815065804349\xspace}
\providecommand{\predicateBitpreciseInterpolKindResultsRFInterpolPrecIncorrectCpuTimeAvgPlainHours}{}
  \renewcommand{\predicateBitpreciseInterpolKindResultsRFInterpolPrecIncorrectCpuTimeAvgPlainHours}{0.012766152960567637\xspace}

  % inv-succ
\providecommand{\predicateBitpreciseInterpolKindResultsRFInterpolPrecIncorrectInvSuccPlain}{}
  \renewcommand{\predicateBitpreciseInterpolKindResultsRFInterpolPrecIncorrectInvSuccPlain}{0\xspace}

  % inv-tries
\providecommand{\predicateBitpreciseInterpolKindResultsRFInterpolPrecIncorrectInvTriesPlain}{}
  \renewcommand{\predicateBitpreciseInterpolKindResultsRFInterpolPrecIncorrectInvTriesPlain}{70\xspace}

  % inv-time-sum
\providecommand{\predicateBitpreciseInterpolKindResultsRFInterpolPrecIncorrectInvTimeSumPlain}{}
  \renewcommand{\predicateBitpreciseInterpolKindResultsRFInterpolPrecIncorrectInvTimeSumPlain}{496.91499999999996\xspace}
\providecommand{\predicateBitpreciseInterpolKindResultsRFInterpolPrecIncorrectInvTimeSumPlainHours}{}
  \renewcommand{\predicateBitpreciseInterpolKindResultsRFInterpolPrecIncorrectInvTimeSumPlainHours}{0.13803194444444444\xspace}

 %% timeout %%
\providecommand{\predicateBitpreciseInterpolKindResultsRFInterpolPrecTimeoutPlain}{}
  \renewcommand{\predicateBitpreciseInterpolKindResultsRFInterpolPrecTimeoutPlain}{1565\xspace}

  % cpu-time-sum
\providecommand{\predicateBitpreciseInterpolKindResultsRFInterpolPrecTimeoutCpuTimeSumPlain}{}
  \renewcommand{\predicateBitpreciseInterpolKindResultsRFInterpolPrecTimeoutCpuTimeSumPlain}{479053.91880363267\xspace}
\providecommand{\predicateBitpreciseInterpolKindResultsRFInterpolPrecTimeoutCpuTimeSumPlainHours}{}
  \renewcommand{\predicateBitpreciseInterpolKindResultsRFInterpolPrecTimeoutCpuTimeSumPlainHours}{133.07053300100907\xspace}

  % cpu-time-avg
\providecommand{\predicateBitpreciseInterpolKindResultsRFInterpolPrecTimeoutCpuTimeAvgPlain}{}
  \renewcommand{\predicateBitpreciseInterpolKindResultsRFInterpolPrecTimeoutCpuTimeAvgPlain}{306.1047404496055\xspace}
\providecommand{\predicateBitpreciseInterpolKindResultsRFInterpolPrecTimeoutCpuTimeAvgPlainHours}{}
  \renewcommand{\predicateBitpreciseInterpolKindResultsRFInterpolPrecTimeoutCpuTimeAvgPlainHours}{0.08502909456933487\xspace}

  % inv-succ
\providecommand{\predicateBitpreciseInterpolKindResultsRFInterpolPrecTimeoutInvSuccPlain}{}
  \renewcommand{\predicateBitpreciseInterpolKindResultsRFInterpolPrecTimeoutInvSuccPlain}{304\xspace}

  % inv-tries
\providecommand{\predicateBitpreciseInterpolKindResultsRFInterpolPrecTimeoutInvTriesPlain}{}
  \renewcommand{\predicateBitpreciseInterpolKindResultsRFInterpolPrecTimeoutInvTriesPlain}{7580\xspace}

  % inv-time-sum
\providecommand{\predicateBitpreciseInterpolKindResultsRFInterpolPrecTimeoutInvTimeSumPlain}{}
  \renewcommand{\predicateBitpreciseInterpolKindResultsRFInterpolPrecTimeoutInvTimeSumPlain}{159550.94300000003\xspace}
\providecommand{\predicateBitpreciseInterpolKindResultsRFInterpolPrecTimeoutInvTimeSumPlainHours}{}
  \renewcommand{\predicateBitpreciseInterpolKindResultsRFInterpolPrecTimeoutInvTimeSumPlainHours}{44.319706388888896\xspace}

 %% unknown-or-category-error %%
\providecommand{\predicateBitpreciseInterpolKindResultsRFInterpolPrecUnknownOrCategoryErrorPlain}{}
  \renewcommand{\predicateBitpreciseInterpolKindResultsRFInterpolPrecUnknownOrCategoryErrorPlain}{1648\xspace}

  % cpu-time-sum
\providecommand{\predicateBitpreciseInterpolKindResultsRFInterpolPrecUnknownOrCategoryErrorCpuTimeSumPlain}{}
  \renewcommand{\predicateBitpreciseInterpolKindResultsRFInterpolPrecUnknownOrCategoryErrorCpuTimeSumPlain}{487813.94804686395\xspace}
\providecommand{\predicateBitpreciseInterpolKindResultsRFInterpolPrecUnknownOrCategoryErrorCpuTimeSumPlainHours}{}
  \renewcommand{\predicateBitpreciseInterpolKindResultsRFInterpolPrecUnknownOrCategoryErrorCpuTimeSumPlainHours}{135.50387445746222\xspace}

  % cpu-time-avg
\providecommand{\predicateBitpreciseInterpolKindResultsRFInterpolPrecUnknownOrCategoryErrorCpuTimeAvgPlain}{}
  \renewcommand{\predicateBitpreciseInterpolKindResultsRFInterpolPrecUnknownOrCategoryErrorCpuTimeAvgPlain}{296.0036092517378\xspace}
\providecommand{\predicateBitpreciseInterpolKindResultsRFInterpolPrecUnknownOrCategoryErrorCpuTimeAvgPlainHours}{}
  \renewcommand{\predicateBitpreciseInterpolKindResultsRFInterpolPrecUnknownOrCategoryErrorCpuTimeAvgPlainHours}{0.0822232247921494\xspace}

  % inv-succ
\providecommand{\predicateBitpreciseInterpolKindResultsRFInterpolPrecUnknownOrCategoryErrorInvSuccPlain}{}
  \renewcommand{\predicateBitpreciseInterpolKindResultsRFInterpolPrecUnknownOrCategoryErrorInvSuccPlain}{308\xspace}

  % inv-tries
\providecommand{\predicateBitpreciseInterpolKindResultsRFInterpolPrecUnknownOrCategoryErrorInvTriesPlain}{}
  \renewcommand{\predicateBitpreciseInterpolKindResultsRFInterpolPrecUnknownOrCategoryErrorInvTriesPlain}{7811\xspace}

  % inv-time-sum
\providecommand{\predicateBitpreciseInterpolKindResultsRFInterpolPrecUnknownOrCategoryErrorInvTimeSumPlain}{}
  \renewcommand{\predicateBitpreciseInterpolKindResultsRFInterpolPrecUnknownOrCategoryErrorInvTimeSumPlain}{162211.13099999996\xspace}
\providecommand{\predicateBitpreciseInterpolKindResultsRFInterpolPrecUnknownOrCategoryErrorInvTimeSumPlainHours}{}
  \renewcommand{\predicateBitpreciseInterpolKindResultsRFInterpolPrecUnknownOrCategoryErrorInvTimeSumPlainHours}{45.05864749999999\xspace}

 %% correct-false %%
\providecommand{\predicateBitpreciseInterpolKindResultsRFInterpolPrecCorrectFalsePlain}{}
  \renewcommand{\predicateBitpreciseInterpolKindResultsRFInterpolPrecCorrectFalsePlain}{483\xspace}

  % cpu-time-sum
\providecommand{\predicateBitpreciseInterpolKindResultsRFInterpolPrecCorrectFalseCpuTimeSumPlain}{}
  \renewcommand{\predicateBitpreciseInterpolKindResultsRFInterpolPrecCorrectFalseCpuTimeSumPlain}{32002.18380171998\xspace}
\providecommand{\predicateBitpreciseInterpolKindResultsRFInterpolPrecCorrectFalseCpuTimeSumPlainHours}{}
  \renewcommand{\predicateBitpreciseInterpolKindResultsRFInterpolPrecCorrectFalseCpuTimeSumPlainHours}{8.889495500477773\xspace}

  % cpu-time-avg
\providecommand{\predicateBitpreciseInterpolKindResultsRFInterpolPrecCorrectFalseCpuTimeAvgPlain}{}
  \renewcommand{\predicateBitpreciseInterpolKindResultsRFInterpolPrecCorrectFalseCpuTimeAvgPlain}{66.25710932033122\xspace}
\providecommand{\predicateBitpreciseInterpolKindResultsRFInterpolPrecCorrectFalseCpuTimeAvgPlainHours}{}
  \renewcommand{\predicateBitpreciseInterpolKindResultsRFInterpolPrecCorrectFalseCpuTimeAvgPlainHours}{0.018404752588980895\xspace}

  % inv-succ
\providecommand{\predicateBitpreciseInterpolKindResultsRFInterpolPrecCorrectFalseInvSuccPlain}{}
  \renewcommand{\predicateBitpreciseInterpolKindResultsRFInterpolPrecCorrectFalseInvSuccPlain}{128\xspace}

  % inv-tries
\providecommand{\predicateBitpreciseInterpolKindResultsRFInterpolPrecCorrectFalseInvTriesPlain}{}
  \renewcommand{\predicateBitpreciseInterpolKindResultsRFInterpolPrecCorrectFalseInvTriesPlain}{2307\xspace}

  % inv-time-sum
\providecommand{\predicateBitpreciseInterpolKindResultsRFInterpolPrecCorrectFalseInvTimeSumPlain}{}
  \renewcommand{\predicateBitpreciseInterpolKindResultsRFInterpolPrecCorrectFalseInvTimeSumPlain}{7289.994999999995\xspace}
\providecommand{\predicateBitpreciseInterpolKindResultsRFInterpolPrecCorrectFalseInvTimeSumPlainHours}{}
  \renewcommand{\predicateBitpreciseInterpolKindResultsRFInterpolPrecCorrectFalseInvTimeSumPlainHours}{2.02499861111111\xspace}

 %% correct-true %%
\providecommand{\predicateBitpreciseInterpolKindResultsRFInterpolPrecCorrectTruePlain}{}
  \renewcommand{\predicateBitpreciseInterpolKindResultsRFInterpolPrecCorrectTruePlain}{1334\xspace}

  % cpu-time-sum
\providecommand{\predicateBitpreciseInterpolKindResultsRFInterpolPrecCorrectTrueCpuTimeSumPlain}{}
  \renewcommand{\predicateBitpreciseInterpolKindResultsRFInterpolPrecCorrectTrueCpuTimeSumPlain}{73970.66930792292\xspace}
\providecommand{\predicateBitpreciseInterpolKindResultsRFInterpolPrecCorrectTrueCpuTimeSumPlainHours}{}
  \renewcommand{\predicateBitpreciseInterpolKindResultsRFInterpolPrecCorrectTrueCpuTimeSumPlainHours}{20.547408141089697\xspace}

  % cpu-time-avg
\providecommand{\predicateBitpreciseInterpolKindResultsRFInterpolPrecCorrectTrueCpuTimeAvgPlain}{}
  \renewcommand{\predicateBitpreciseInterpolKindResultsRFInterpolPrecCorrectTrueCpuTimeAvgPlain}{55.45027684252093\xspace}
\providecommand{\predicateBitpreciseInterpolKindResultsRFInterpolPrecCorrectTrueCpuTimeAvgPlainHours}{}
  \renewcommand{\predicateBitpreciseInterpolKindResultsRFInterpolPrecCorrectTrueCpuTimeAvgPlainHours}{0.015402854678478036\xspace}

  % inv-succ
\providecommand{\predicateBitpreciseInterpolKindResultsRFInterpolPrecCorrectTrueInvSuccPlain}{}
  \renewcommand{\predicateBitpreciseInterpolKindResultsRFInterpolPrecCorrectTrueInvSuccPlain}{1144\xspace}

  % inv-tries
\providecommand{\predicateBitpreciseInterpolKindResultsRFInterpolPrecCorrectTrueInvTriesPlain}{}
  \renewcommand{\predicateBitpreciseInterpolKindResultsRFInterpolPrecCorrectTrueInvTriesPlain}{4148\xspace}

  % inv-time-sum
\providecommand{\predicateBitpreciseInterpolKindResultsRFInterpolPrecCorrectTrueInvTimeSumPlain}{}
  \renewcommand{\predicateBitpreciseInterpolKindResultsRFInterpolPrecCorrectTrueInvTimeSumPlain}{18248.083999999966\xspace}
\providecommand{\predicateBitpreciseInterpolKindResultsRFInterpolPrecCorrectTrueInvTimeSumPlainHours}{}
  \renewcommand{\predicateBitpreciseInterpolKindResultsRFInterpolPrecCorrectTrueInvTimeSumPlainHours}{5.068912222222213\xspace}

 %% incorrect-false %%
\providecommand{\predicateBitpreciseInterpolKindResultsRFInterpolPrecIncorrectFalsePlain}{}
  \renewcommand{\predicateBitpreciseInterpolKindResultsRFInterpolPrecIncorrectFalsePlain}{23\xspace}

  % cpu-time-sum
\providecommand{\predicateBitpreciseInterpolKindResultsRFInterpolPrecIncorrectFalseCpuTimeSumPlain}{}
  \renewcommand{\predicateBitpreciseInterpolKindResultsRFInterpolPrecIncorrectFalseCpuTimeSumPlain}{1057.0374651350003\xspace}
\providecommand{\predicateBitpreciseInterpolKindResultsRFInterpolPrecIncorrectFalseCpuTimeSumPlainHours}{}
  \renewcommand{\predicateBitpreciseInterpolKindResultsRFInterpolPrecIncorrectFalseCpuTimeSumPlainHours}{0.29362151809305564\xspace}

  % cpu-time-avg
\providecommand{\predicateBitpreciseInterpolKindResultsRFInterpolPrecIncorrectFalseCpuTimeAvgPlain}{}
  \renewcommand{\predicateBitpreciseInterpolKindResultsRFInterpolPrecIncorrectFalseCpuTimeAvgPlain}{45.95815065804349\xspace}
\providecommand{\predicateBitpreciseInterpolKindResultsRFInterpolPrecIncorrectFalseCpuTimeAvgPlainHours}{}
  \renewcommand{\predicateBitpreciseInterpolKindResultsRFInterpolPrecIncorrectFalseCpuTimeAvgPlainHours}{0.012766152960567637\xspace}

  % inv-succ
\providecommand{\predicateBitpreciseInterpolKindResultsRFInterpolPrecIncorrectFalseInvSuccPlain}{}
  \renewcommand{\predicateBitpreciseInterpolKindResultsRFInterpolPrecIncorrectFalseInvSuccPlain}{0\xspace}

  % inv-tries
\providecommand{\predicateBitpreciseInterpolKindResultsRFInterpolPrecIncorrectFalseInvTriesPlain}{}
  \renewcommand{\predicateBitpreciseInterpolKindResultsRFInterpolPrecIncorrectFalseInvTriesPlain}{70\xspace}

  % inv-time-sum
\providecommand{\predicateBitpreciseInterpolKindResultsRFInterpolPrecIncorrectFalseInvTimeSumPlain}{}
  \renewcommand{\predicateBitpreciseInterpolKindResultsRFInterpolPrecIncorrectFalseInvTimeSumPlain}{496.91499999999996\xspace}
\providecommand{\predicateBitpreciseInterpolKindResultsRFInterpolPrecIncorrectFalseInvTimeSumPlainHours}{}
  \renewcommand{\predicateBitpreciseInterpolKindResultsRFInterpolPrecIncorrectFalseInvTimeSumPlainHours}{0.13803194444444444\xspace}

 %% incorrect-true %%
\providecommand{\predicateBitpreciseInterpolKindResultsRFInterpolPrecIncorrectTruePlain}{}
  \renewcommand{\predicateBitpreciseInterpolKindResultsRFInterpolPrecIncorrectTruePlain}{0\xspace}

  % cpu-time-sum
\providecommand{\predicateBitpreciseInterpolKindResultsRFInterpolPrecIncorrectTrueCpuTimeSumPlain}{}
  \renewcommand{\predicateBitpreciseInterpolKindResultsRFInterpolPrecIncorrectTrueCpuTimeSumPlain}{0.0\xspace}
\providecommand{\predicateBitpreciseInterpolKindResultsRFInterpolPrecIncorrectTrueCpuTimeSumPlainHours}{}
  \renewcommand{\predicateBitpreciseInterpolKindResultsRFInterpolPrecIncorrectTrueCpuTimeSumPlainHours}{0.0\xspace}

  % cpu-time-avg
\providecommand{\predicateBitpreciseInterpolKindResultsRFInterpolPrecIncorrectTrueCpuTimeAvgPlain}{}
  \renewcommand{\predicateBitpreciseInterpolKindResultsRFInterpolPrecIncorrectTrueCpuTimeAvgPlain}{NaN\xspace}
\providecommand{\predicateBitpreciseInterpolKindResultsRFInterpolPrecIncorrectTrueCpuTimeAvgPlainHours}{}
  \renewcommand{\predicateBitpreciseInterpolKindResultsRFInterpolPrecIncorrectTrueCpuTimeAvgPlainHours}{NaN\xspace}

  % inv-succ
\providecommand{\predicateBitpreciseInterpolKindResultsRFInterpolPrecIncorrectTrueInvSuccPlain}{}
  \renewcommand{\predicateBitpreciseInterpolKindResultsRFInterpolPrecIncorrectTrueInvSuccPlain}{0\xspace}

  % inv-tries
\providecommand{\predicateBitpreciseInterpolKindResultsRFInterpolPrecIncorrectTrueInvTriesPlain}{}
  \renewcommand{\predicateBitpreciseInterpolKindResultsRFInterpolPrecIncorrectTrueInvTriesPlain}{0\xspace}

  % inv-time-sum
\providecommand{\predicateBitpreciseInterpolKindResultsRFInterpolPrecIncorrectTrueInvTimeSumPlain}{}
  \renewcommand{\predicateBitpreciseInterpolKindResultsRFInterpolPrecIncorrectTrueInvTimeSumPlain}{0.0\xspace}
\providecommand{\predicateBitpreciseInterpolKindResultsRFInterpolPrecIncorrectTrueInvTimeSumPlainHours}{}
  \renewcommand{\predicateBitpreciseInterpolKindResultsRFInterpolPrecIncorrectTrueInvTimeSumPlainHours}{0.0\xspace}

 %% all %%
\providecommand{\predicateBitpreciseInterpolKindResultsRFInterpolPrecAllPlain}{}
  \renewcommand{\predicateBitpreciseInterpolKindResultsRFInterpolPrecAllPlain}{3488\xspace}

  % cpu-time-sum
\providecommand{\predicateBitpreciseInterpolKindResultsRFInterpolPrecAllCpuTimeSumPlain}{}
  \renewcommand{\predicateBitpreciseInterpolKindResultsRFInterpolPrecAllCpuTimeSumPlain}{594843.8386216415\xspace}
\providecommand{\predicateBitpreciseInterpolKindResultsRFInterpolPrecAllCpuTimeSumPlainHours}{}
  \renewcommand{\predicateBitpreciseInterpolKindResultsRFInterpolPrecAllCpuTimeSumPlainHours}{165.23439961712265\xspace}

  % cpu-time-avg
\providecommand{\predicateBitpreciseInterpolKindResultsRFInterpolPrecAllCpuTimeAvgPlain}{}
  \renewcommand{\predicateBitpreciseInterpolKindResultsRFInterpolPrecAllCpuTimeAvgPlain}{170.54009134794768\xspace}
\providecommand{\predicateBitpreciseInterpolKindResultsRFInterpolPrecAllCpuTimeAvgPlainHours}{}
  \renewcommand{\predicateBitpreciseInterpolKindResultsRFInterpolPrecAllCpuTimeAvgPlainHours}{0.04737224759665214\xspace}

  % inv-succ
\providecommand{\predicateBitpreciseInterpolKindResultsRFInterpolPrecAllInvSuccPlain}{}
  \renewcommand{\predicateBitpreciseInterpolKindResultsRFInterpolPrecAllInvSuccPlain}{1580\xspace}

  % inv-tries
\providecommand{\predicateBitpreciseInterpolKindResultsRFInterpolPrecAllInvTriesPlain}{}
  \renewcommand{\predicateBitpreciseInterpolKindResultsRFInterpolPrecAllInvTriesPlain}{14336\xspace}

  % inv-time-sum
\providecommand{\predicateBitpreciseInterpolKindResultsRFInterpolPrecAllInvTimeSumPlain}{}
  \renewcommand{\predicateBitpreciseInterpolKindResultsRFInterpolPrecAllInvTimeSumPlain}{188246.12500000023\xspace}
\providecommand{\predicateBitpreciseInterpolKindResultsRFInterpolPrecAllInvTimeSumPlainHours}{}
  \renewcommand{\predicateBitpreciseInterpolKindResultsRFInterpolPrecAllInvTimeSumPlainHours}{52.290590277777845\xspace}

 %% equal-only %%
\providecommand{\predicateBitpreciseInterpolKindResultsRFInterpolPrecEqualOnlyPlain}{}
  \renewcommand{\predicateBitpreciseInterpolKindResultsRFInterpolPrecEqualOnlyPlain}{1754\xspace}

  % cpu-time-sum
\providecommand{\predicateBitpreciseInterpolKindResultsRFInterpolPrecEqualOnlyCpuTimeSumPlain}{}
  \renewcommand{\predicateBitpreciseInterpolKindResultsRFInterpolPrecEqualOnlyCpuTimeSumPlain}{99090.51664190197\xspace}
\providecommand{\predicateBitpreciseInterpolKindResultsRFInterpolPrecEqualOnlyCpuTimeSumPlainHours}{}
  \renewcommand{\predicateBitpreciseInterpolKindResultsRFInterpolPrecEqualOnlyCpuTimeSumPlainHours}{27.525143511639435\xspace}

  % cpu-time-avg
\providecommand{\predicateBitpreciseInterpolKindResultsRFInterpolPrecEqualOnlyCpuTimeAvgPlain}{}
  \renewcommand{\predicateBitpreciseInterpolKindResultsRFInterpolPrecEqualOnlyCpuTimeAvgPlain}{56.494023170981734\xspace}
\providecommand{\predicateBitpreciseInterpolKindResultsRFInterpolPrecEqualOnlyCpuTimeAvgPlainHours}{}
  \renewcommand{\predicateBitpreciseInterpolKindResultsRFInterpolPrecEqualOnlyCpuTimeAvgPlainHours}{0.015692784214161593\xspace}

  % inv-succ
\providecommand{\predicateBitpreciseInterpolKindResultsRFInterpolPrecEqualOnlyInvSuccPlain}{}
  \renewcommand{\predicateBitpreciseInterpolKindResultsRFInterpolPrecEqualOnlyInvSuccPlain}{1253\xspace}

  % inv-tries
\providecommand{\predicateBitpreciseInterpolKindResultsRFInterpolPrecEqualOnlyInvTriesPlain}{}
  \renewcommand{\predicateBitpreciseInterpolKindResultsRFInterpolPrecEqualOnlyInvTriesPlain}{5978\xspace}

  % inv-time-sum
\providecommand{\predicateBitpreciseInterpolKindResultsRFInterpolPrecEqualOnlyInvTimeSumPlain}{}
  \renewcommand{\predicateBitpreciseInterpolKindResultsRFInterpolPrecEqualOnlyInvTimeSumPlain}{23734.15599999999\xspace}
\providecommand{\predicateBitpreciseInterpolKindResultsRFInterpolPrecEqualOnlyInvTimeSumPlainHours}{}
  \renewcommand{\predicateBitpreciseInterpolKindResultsRFInterpolPrecEqualOnlyInvTimeSumPlainHours}{6.592821111111109\xspace}

%%% predicate_bitprecise_CNF_KIND.2016-09-04_1153.results.CNF_KIND-path %%%
 %% correct %%
\providecommand{\predicateBitpreciseCNFKINDResultsCNFKINDPathCorrectPlain}{}
  \renewcommand{\predicateBitpreciseCNFKINDResultsCNFKINDPathCorrectPlain}{1927\xspace}

  % cpu-time-sum
\providecommand{\predicateBitpreciseCNFKINDResultsCNFKINDPathCorrectCpuTimeSumPlain}{}
  \renewcommand{\predicateBitpreciseCNFKINDResultsCNFKINDPathCorrectCpuTimeSumPlain}{95438.06251107609\xspace}
\providecommand{\predicateBitpreciseCNFKINDResultsCNFKINDPathCorrectCpuTimeSumPlainHours}{}
  \renewcommand{\predicateBitpreciseCNFKINDResultsCNFKINDPathCorrectCpuTimeSumPlainHours}{26.510572919743357\xspace}

  % cpu-time-avg
\providecommand{\predicateBitpreciseCNFKINDResultsCNFKINDPathCorrectCpuTimeAvgPlain}{}
  \renewcommand{\predicateBitpreciseCNFKINDResultsCNFKINDPathCorrectCpuTimeAvgPlain}{49.526757919603575\xspace}
\providecommand{\predicateBitpreciseCNFKINDResultsCNFKINDPathCorrectCpuTimeAvgPlainHours}{}
  \renewcommand{\predicateBitpreciseCNFKINDResultsCNFKINDPathCorrectCpuTimeAvgPlainHours}{0.013757432755445437\xspace}

  % inv-succ
\providecommand{\predicateBitpreciseCNFKINDResultsCNFKINDPathCorrectInvSuccPlain}{}
  \renewcommand{\predicateBitpreciseCNFKINDResultsCNFKINDPathCorrectInvSuccPlain}{0\xspace}

  % inv-tries
\providecommand{\predicateBitpreciseCNFKINDResultsCNFKINDPathCorrectInvTriesPlain}{}
  \renewcommand{\predicateBitpreciseCNFKINDResultsCNFKINDPathCorrectInvTriesPlain}{6976\xspace}

  % inv-time-sum
\providecommand{\predicateBitpreciseCNFKINDResultsCNFKINDPathCorrectInvTimeSumPlain}{}
  \renewcommand{\predicateBitpreciseCNFKINDResultsCNFKINDPathCorrectInvTimeSumPlain}{3499.6579999999963\xspace}
\providecommand{\predicateBitpreciseCNFKINDResultsCNFKINDPathCorrectInvTimeSumPlainHours}{}
  \renewcommand{\predicateBitpreciseCNFKINDResultsCNFKINDPathCorrectInvTimeSumPlainHours}{0.9721272222222211\xspace}

 %% incorrect %%
\providecommand{\predicateBitpreciseCNFKINDResultsCNFKINDPathIncorrectPlain}{}
  \renewcommand{\predicateBitpreciseCNFKINDResultsCNFKINDPathIncorrectPlain}{27\xspace}

  % cpu-time-sum
\providecommand{\predicateBitpreciseCNFKINDResultsCNFKINDPathIncorrectCpuTimeSumPlain}{}
  \renewcommand{\predicateBitpreciseCNFKINDResultsCNFKINDPathIncorrectCpuTimeSumPlain}{782.7250891379998\xspace}
\providecommand{\predicateBitpreciseCNFKINDResultsCNFKINDPathIncorrectCpuTimeSumPlainHours}{}
  \renewcommand{\predicateBitpreciseCNFKINDResultsCNFKINDPathIncorrectCpuTimeSumPlainHours}{0.2174236358716666\xspace}

  % cpu-time-avg
\providecommand{\predicateBitpreciseCNFKINDResultsCNFKINDPathIncorrectCpuTimeAvgPlain}{}
  \renewcommand{\predicateBitpreciseCNFKINDResultsCNFKINDPathIncorrectCpuTimeAvgPlain}{28.989818116222214\xspace}
\providecommand{\predicateBitpreciseCNFKINDResultsCNFKINDPathIncorrectCpuTimeAvgPlainHours}{}
  \renewcommand{\predicateBitpreciseCNFKINDResultsCNFKINDPathIncorrectCpuTimeAvgPlainHours}{0.00805272725450617\xspace}

  % inv-succ
\providecommand{\predicateBitpreciseCNFKINDResultsCNFKINDPathIncorrectInvSuccPlain}{}
  \renewcommand{\predicateBitpreciseCNFKINDResultsCNFKINDPathIncorrectInvSuccPlain}{0\xspace}

  % inv-tries
\providecommand{\predicateBitpreciseCNFKINDResultsCNFKINDPathIncorrectInvTriesPlain}{}
  \renewcommand{\predicateBitpreciseCNFKINDResultsCNFKINDPathIncorrectInvTriesPlain}{87\xspace}

  % inv-time-sum
\providecommand{\predicateBitpreciseCNFKINDResultsCNFKINDPathIncorrectInvTimeSumPlain}{}
  \renewcommand{\predicateBitpreciseCNFKINDResultsCNFKINDPathIncorrectInvTimeSumPlain}{30.72\xspace}
\providecommand{\predicateBitpreciseCNFKINDResultsCNFKINDPathIncorrectInvTimeSumPlainHours}{}
  \renewcommand{\predicateBitpreciseCNFKINDResultsCNFKINDPathIncorrectInvTimeSumPlainHours}{0.008533333333333334\xspace}

 %% timeout %%
\providecommand{\predicateBitpreciseCNFKINDResultsCNFKINDPathTimeoutPlain}{}
  \renewcommand{\predicateBitpreciseCNFKINDResultsCNFKINDPathTimeoutPlain}{1369\xspace}

  % cpu-time-sum
\providecommand{\predicateBitpreciseCNFKINDResultsCNFKINDPathTimeoutCpuTimeSumPlain}{}
  \renewcommand{\predicateBitpreciseCNFKINDResultsCNFKINDPathTimeoutCpuTimeSumPlain}{419108.83451896417\xspace}
\providecommand{\predicateBitpreciseCNFKINDResultsCNFKINDPathTimeoutCpuTimeSumPlainHours}{}
  \renewcommand{\predicateBitpreciseCNFKINDResultsCNFKINDPathTimeoutCpuTimeSumPlainHours}{116.41912069971227\xspace}

  % cpu-time-avg
\providecommand{\predicateBitpreciseCNFKINDResultsCNFKINDPathTimeoutCpuTimeAvgPlain}{}
  \renewcommand{\predicateBitpreciseCNFKINDResultsCNFKINDPathTimeoutCpuTimeAvgPlain}{306.1423188597255\xspace}
\providecommand{\predicateBitpreciseCNFKINDResultsCNFKINDPathTimeoutCpuTimeAvgPlainHours}{}
  \renewcommand{\predicateBitpreciseCNFKINDResultsCNFKINDPathTimeoutCpuTimeAvgPlainHours}{0.08503953301659041\xspace}

  % inv-succ
\providecommand{\predicateBitpreciseCNFKINDResultsCNFKINDPathTimeoutInvSuccPlain}{}
  \renewcommand{\predicateBitpreciseCNFKINDResultsCNFKINDPathTimeoutInvSuccPlain}{0\xspace}

  % inv-tries
\providecommand{\predicateBitpreciseCNFKINDResultsCNFKINDPathTimeoutInvTriesPlain}{}
  \renewcommand{\predicateBitpreciseCNFKINDResultsCNFKINDPathTimeoutInvTriesPlain}{5224\xspace}

  % inv-time-sum
\providecommand{\predicateBitpreciseCNFKINDResultsCNFKINDPathTimeoutInvTimeSumPlain}{}
  \renewcommand{\predicateBitpreciseCNFKINDResultsCNFKINDPathTimeoutInvTimeSumPlain}{7599.274999999995\xspace}
\providecommand{\predicateBitpreciseCNFKINDResultsCNFKINDPathTimeoutInvTimeSumPlainHours}{}
  \renewcommand{\predicateBitpreciseCNFKINDResultsCNFKINDPathTimeoutInvTimeSumPlainHours}{2.110909722222221\xspace}

 %% unknown-or-category-error %%
\providecommand{\predicateBitpreciseCNFKINDResultsCNFKINDPathUnknownOrCategoryErrorPlain}{}
  \renewcommand{\predicateBitpreciseCNFKINDResultsCNFKINDPathUnknownOrCategoryErrorPlain}{1534\xspace}

  % cpu-time-sum
\providecommand{\predicateBitpreciseCNFKINDResultsCNFKINDPathUnknownOrCategoryErrorCpuTimeSumPlain}{}
  \renewcommand{\predicateBitpreciseCNFKINDResultsCNFKINDPathUnknownOrCategoryErrorCpuTimeSumPlain}{434766.1013655032\xspace}
\providecommand{\predicateBitpreciseCNFKINDResultsCNFKINDPathUnknownOrCategoryErrorCpuTimeSumPlainHours}{}
  \renewcommand{\predicateBitpreciseCNFKINDResultsCNFKINDPathUnknownOrCategoryErrorCpuTimeSumPlainHours}{120.76836149041756\xspace}

  % cpu-time-avg
\providecommand{\predicateBitpreciseCNFKINDResultsCNFKINDPathUnknownOrCategoryErrorCpuTimeAvgPlain}{}
  \renewcommand{\predicateBitpreciseCNFKINDResultsCNFKINDPathUnknownOrCategoryErrorCpuTimeAvgPlain}{283.4198835498717\xspace}
\providecommand{\predicateBitpreciseCNFKINDResultsCNFKINDPathUnknownOrCategoryErrorCpuTimeAvgPlainHours}{}
  \renewcommand{\predicateBitpreciseCNFKINDResultsCNFKINDPathUnknownOrCategoryErrorCpuTimeAvgPlainHours}{0.07872774543051993\xspace}

  % inv-succ
\providecommand{\predicateBitpreciseCNFKINDResultsCNFKINDPathUnknownOrCategoryErrorInvSuccPlain}{}
  \renewcommand{\predicateBitpreciseCNFKINDResultsCNFKINDPathUnknownOrCategoryErrorInvSuccPlain}{0\xspace}

  % inv-tries
\providecommand{\predicateBitpreciseCNFKINDResultsCNFKINDPathUnknownOrCategoryErrorInvTriesPlain}{}
  \renewcommand{\predicateBitpreciseCNFKINDResultsCNFKINDPathUnknownOrCategoryErrorInvTriesPlain}{5608\xspace}

  % inv-time-sum
\providecommand{\predicateBitpreciseCNFKINDResultsCNFKINDPathUnknownOrCategoryErrorInvTimeSumPlain}{}
  \renewcommand{\predicateBitpreciseCNFKINDResultsCNFKINDPathUnknownOrCategoryErrorInvTimeSumPlain}{8013.874999999995\xspace}
\providecommand{\predicateBitpreciseCNFKINDResultsCNFKINDPathUnknownOrCategoryErrorInvTimeSumPlainHours}{}
  \renewcommand{\predicateBitpreciseCNFKINDResultsCNFKINDPathUnknownOrCategoryErrorInvTimeSumPlainHours}{2.2260763888888877\xspace}

 %% correct-false %%
\providecommand{\predicateBitpreciseCNFKINDResultsCNFKINDPathCorrectFalsePlain}{}
  \renewcommand{\predicateBitpreciseCNFKINDResultsCNFKINDPathCorrectFalsePlain}{543\xspace}

  % cpu-time-sum
\providecommand{\predicateBitpreciseCNFKINDResultsCNFKINDPathCorrectFalseCpuTimeSumPlain}{}
  \renewcommand{\predicateBitpreciseCNFKINDResultsCNFKINDPathCorrectFalseCpuTimeSumPlain}{38107.530329915\xspace}
\providecommand{\predicateBitpreciseCNFKINDResultsCNFKINDPathCorrectFalseCpuTimeSumPlainHours}{}
  \renewcommand{\predicateBitpreciseCNFKINDResultsCNFKINDPathCorrectFalseCpuTimeSumPlainHours}{10.585425091643055\xspace}

  % cpu-time-avg
\providecommand{\predicateBitpreciseCNFKINDResultsCNFKINDPathCorrectFalseCpuTimeAvgPlain}{}
  \renewcommand{\predicateBitpreciseCNFKINDResultsCNFKINDPathCorrectFalseCpuTimeAvgPlain}{70.17961386724677\xspace}
\providecommand{\predicateBitpreciseCNFKINDResultsCNFKINDPathCorrectFalseCpuTimeAvgPlainHours}{}
  \renewcommand{\predicateBitpreciseCNFKINDResultsCNFKINDPathCorrectFalseCpuTimeAvgPlainHours}{0.019494337185346326\xspace}

  % inv-succ
\providecommand{\predicateBitpreciseCNFKINDResultsCNFKINDPathCorrectFalseInvSuccPlain}{}
  \renewcommand{\predicateBitpreciseCNFKINDResultsCNFKINDPathCorrectFalseInvSuccPlain}{0\xspace}

  % inv-tries
\providecommand{\predicateBitpreciseCNFKINDResultsCNFKINDPathCorrectFalseInvTriesPlain}{}
  \renewcommand{\predicateBitpreciseCNFKINDResultsCNFKINDPathCorrectFalseInvTriesPlain}{2508\xspace}

  % inv-time-sum
\providecommand{\predicateBitpreciseCNFKINDResultsCNFKINDPathCorrectFalseInvTimeSumPlain}{}
  \renewcommand{\predicateBitpreciseCNFKINDResultsCNFKINDPathCorrectFalseInvTimeSumPlain}{1472.6810000000005\xspace}
\providecommand{\predicateBitpreciseCNFKINDResultsCNFKINDPathCorrectFalseInvTimeSumPlainHours}{}
  \renewcommand{\predicateBitpreciseCNFKINDResultsCNFKINDPathCorrectFalseInvTimeSumPlainHours}{0.4090780555555557\xspace}

 %% correct-true %%
\providecommand{\predicateBitpreciseCNFKINDResultsCNFKINDPathCorrectTruePlain}{}
  \renewcommand{\predicateBitpreciseCNFKINDResultsCNFKINDPathCorrectTruePlain}{1384\xspace}

  % cpu-time-sum
\providecommand{\predicateBitpreciseCNFKINDResultsCNFKINDPathCorrectTrueCpuTimeSumPlain}{}
  \renewcommand{\predicateBitpreciseCNFKINDResultsCNFKINDPathCorrectTrueCpuTimeSumPlain}{57330.532181160976\xspace}
\providecommand{\predicateBitpreciseCNFKINDResultsCNFKINDPathCorrectTrueCpuTimeSumPlainHours}{}
  \renewcommand{\predicateBitpreciseCNFKINDResultsCNFKINDPathCorrectTrueCpuTimeSumPlainHours}{15.925147828100272\xspace}

  % cpu-time-avg
\providecommand{\predicateBitpreciseCNFKINDResultsCNFKINDPathCorrectTrueCpuTimeAvgPlain}{}
  \renewcommand{\predicateBitpreciseCNFKINDResultsCNFKINDPathCorrectTrueCpuTimeAvgPlain}{41.42379492858452\xspace}
\providecommand{\predicateBitpreciseCNFKINDResultsCNFKINDPathCorrectTrueCpuTimeAvgPlainHours}{}
  \renewcommand{\predicateBitpreciseCNFKINDResultsCNFKINDPathCorrectTrueCpuTimeAvgPlainHours}{0.011506609702384589\xspace}

  % inv-succ
\providecommand{\predicateBitpreciseCNFKINDResultsCNFKINDPathCorrectTrueInvSuccPlain}{}
  \renewcommand{\predicateBitpreciseCNFKINDResultsCNFKINDPathCorrectTrueInvSuccPlain}{0\xspace}

  % inv-tries
\providecommand{\predicateBitpreciseCNFKINDResultsCNFKINDPathCorrectTrueInvTriesPlain}{}
  \renewcommand{\predicateBitpreciseCNFKINDResultsCNFKINDPathCorrectTrueInvTriesPlain}{4468\xspace}

  % inv-time-sum
\providecommand{\predicateBitpreciseCNFKINDResultsCNFKINDPathCorrectTrueInvTimeSumPlain}{}
  \renewcommand{\predicateBitpreciseCNFKINDResultsCNFKINDPathCorrectTrueInvTimeSumPlain}{2026.9770000000003\xspace}
\providecommand{\predicateBitpreciseCNFKINDResultsCNFKINDPathCorrectTrueInvTimeSumPlainHours}{}
  \renewcommand{\predicateBitpreciseCNFKINDResultsCNFKINDPathCorrectTrueInvTimeSumPlainHours}{0.5630491666666667\xspace}

 %% incorrect-false %%
\providecommand{\predicateBitpreciseCNFKINDResultsCNFKINDPathIncorrectFalsePlain}{}
  \renewcommand{\predicateBitpreciseCNFKINDResultsCNFKINDPathIncorrectFalsePlain}{27\xspace}

  % cpu-time-sum
\providecommand{\predicateBitpreciseCNFKINDResultsCNFKINDPathIncorrectFalseCpuTimeSumPlain}{}
  \renewcommand{\predicateBitpreciseCNFKINDResultsCNFKINDPathIncorrectFalseCpuTimeSumPlain}{782.7250891379998\xspace}
\providecommand{\predicateBitpreciseCNFKINDResultsCNFKINDPathIncorrectFalseCpuTimeSumPlainHours}{}
  \renewcommand{\predicateBitpreciseCNFKINDResultsCNFKINDPathIncorrectFalseCpuTimeSumPlainHours}{0.2174236358716666\xspace}

  % cpu-time-avg
\providecommand{\predicateBitpreciseCNFKINDResultsCNFKINDPathIncorrectFalseCpuTimeAvgPlain}{}
  \renewcommand{\predicateBitpreciseCNFKINDResultsCNFKINDPathIncorrectFalseCpuTimeAvgPlain}{28.989818116222214\xspace}
\providecommand{\predicateBitpreciseCNFKINDResultsCNFKINDPathIncorrectFalseCpuTimeAvgPlainHours}{}
  \renewcommand{\predicateBitpreciseCNFKINDResultsCNFKINDPathIncorrectFalseCpuTimeAvgPlainHours}{0.00805272725450617\xspace}

  % inv-succ
\providecommand{\predicateBitpreciseCNFKINDResultsCNFKINDPathIncorrectFalseInvSuccPlain}{}
  \renewcommand{\predicateBitpreciseCNFKINDResultsCNFKINDPathIncorrectFalseInvSuccPlain}{0\xspace}

  % inv-tries
\providecommand{\predicateBitpreciseCNFKINDResultsCNFKINDPathIncorrectFalseInvTriesPlain}{}
  \renewcommand{\predicateBitpreciseCNFKINDResultsCNFKINDPathIncorrectFalseInvTriesPlain}{87\xspace}

  % inv-time-sum
\providecommand{\predicateBitpreciseCNFKINDResultsCNFKINDPathIncorrectFalseInvTimeSumPlain}{}
  \renewcommand{\predicateBitpreciseCNFKINDResultsCNFKINDPathIncorrectFalseInvTimeSumPlain}{30.72\xspace}
\providecommand{\predicateBitpreciseCNFKINDResultsCNFKINDPathIncorrectFalseInvTimeSumPlainHours}{}
  \renewcommand{\predicateBitpreciseCNFKINDResultsCNFKINDPathIncorrectFalseInvTimeSumPlainHours}{0.008533333333333334\xspace}

 %% incorrect-true %%
\providecommand{\predicateBitpreciseCNFKINDResultsCNFKINDPathIncorrectTruePlain}{}
  \renewcommand{\predicateBitpreciseCNFKINDResultsCNFKINDPathIncorrectTruePlain}{0\xspace}

  % cpu-time-sum
\providecommand{\predicateBitpreciseCNFKINDResultsCNFKINDPathIncorrectTrueCpuTimeSumPlain}{}
  \renewcommand{\predicateBitpreciseCNFKINDResultsCNFKINDPathIncorrectTrueCpuTimeSumPlain}{0.0\xspace}
\providecommand{\predicateBitpreciseCNFKINDResultsCNFKINDPathIncorrectTrueCpuTimeSumPlainHours}{}
  \renewcommand{\predicateBitpreciseCNFKINDResultsCNFKINDPathIncorrectTrueCpuTimeSumPlainHours}{0.0\xspace}

  % cpu-time-avg
\providecommand{\predicateBitpreciseCNFKINDResultsCNFKINDPathIncorrectTrueCpuTimeAvgPlain}{}
  \renewcommand{\predicateBitpreciseCNFKINDResultsCNFKINDPathIncorrectTrueCpuTimeAvgPlain}{NaN\xspace}
\providecommand{\predicateBitpreciseCNFKINDResultsCNFKINDPathIncorrectTrueCpuTimeAvgPlainHours}{}
  \renewcommand{\predicateBitpreciseCNFKINDResultsCNFKINDPathIncorrectTrueCpuTimeAvgPlainHours}{NaN\xspace}

  % inv-succ
\providecommand{\predicateBitpreciseCNFKINDResultsCNFKINDPathIncorrectTrueInvSuccPlain}{}
  \renewcommand{\predicateBitpreciseCNFKINDResultsCNFKINDPathIncorrectTrueInvSuccPlain}{0\xspace}

  % inv-tries
\providecommand{\predicateBitpreciseCNFKINDResultsCNFKINDPathIncorrectTrueInvTriesPlain}{}
  \renewcommand{\predicateBitpreciseCNFKINDResultsCNFKINDPathIncorrectTrueInvTriesPlain}{0\xspace}

  % inv-time-sum
\providecommand{\predicateBitpreciseCNFKINDResultsCNFKINDPathIncorrectTrueInvTimeSumPlain}{}
  \renewcommand{\predicateBitpreciseCNFKINDResultsCNFKINDPathIncorrectTrueInvTimeSumPlain}{0.0\xspace}
\providecommand{\predicateBitpreciseCNFKINDResultsCNFKINDPathIncorrectTrueInvTimeSumPlainHours}{}
  \renewcommand{\predicateBitpreciseCNFKINDResultsCNFKINDPathIncorrectTrueInvTimeSumPlainHours}{0.0\xspace}

 %% all %%
\providecommand{\predicateBitpreciseCNFKINDResultsCNFKINDPathAllPlain}{}
  \renewcommand{\predicateBitpreciseCNFKINDResultsCNFKINDPathAllPlain}{3488\xspace}

  % cpu-time-sum
\providecommand{\predicateBitpreciseCNFKINDResultsCNFKINDPathAllCpuTimeSumPlain}{}
  \renewcommand{\predicateBitpreciseCNFKINDResultsCNFKINDPathAllCpuTimeSumPlain}{530986.8889657179\xspace}
\providecommand{\predicateBitpreciseCNFKINDResultsCNFKINDPathAllCpuTimeSumPlainHours}{}
  \renewcommand{\predicateBitpreciseCNFKINDResultsCNFKINDPathAllCpuTimeSumPlainHours}{147.49635804603275\xspace}

  % cpu-time-avg
\providecommand{\predicateBitpreciseCNFKINDResultsCNFKINDPathAllCpuTimeAvgPlain}{}
  \renewcommand{\predicateBitpreciseCNFKINDResultsCNFKINDPathAllCpuTimeAvgPlain}{152.23247963466682\xspace}
\providecommand{\predicateBitpreciseCNFKINDResultsCNFKINDPathAllCpuTimeAvgPlainHours}{}
  \renewcommand{\predicateBitpreciseCNFKINDResultsCNFKINDPathAllCpuTimeAvgPlainHours}{0.04228679989851856\xspace}

  % inv-succ
\providecommand{\predicateBitpreciseCNFKINDResultsCNFKINDPathAllInvSuccPlain}{}
  \renewcommand{\predicateBitpreciseCNFKINDResultsCNFKINDPathAllInvSuccPlain}{0\xspace}

  % inv-tries
\providecommand{\predicateBitpreciseCNFKINDResultsCNFKINDPathAllInvTriesPlain}{}
  \renewcommand{\predicateBitpreciseCNFKINDResultsCNFKINDPathAllInvTriesPlain}{12671\xspace}

  % inv-time-sum
\providecommand{\predicateBitpreciseCNFKINDResultsCNFKINDPathAllInvTimeSumPlain}{}
  \renewcommand{\predicateBitpreciseCNFKINDResultsCNFKINDPathAllInvTimeSumPlain}{11544.253000000022\xspace}
\providecommand{\predicateBitpreciseCNFKINDResultsCNFKINDPathAllInvTimeSumPlainHours}{}
  \renewcommand{\predicateBitpreciseCNFKINDResultsCNFKINDPathAllInvTimeSumPlainHours}{3.206736944444451\xspace}

 %% equal-only %%
\providecommand{\predicateBitpreciseCNFKINDResultsCNFKINDPathEqualOnlyPlain}{}
  \renewcommand{\predicateBitpreciseCNFKINDResultsCNFKINDPathEqualOnlyPlain}{1754\xspace}

  % cpu-time-sum
\providecommand{\predicateBitpreciseCNFKINDResultsCNFKINDPathEqualOnlyCpuTimeSumPlain}{}
  \renewcommand{\predicateBitpreciseCNFKINDResultsCNFKINDPathEqualOnlyCpuTimeSumPlain}{68517.13307448516\xspace}
\providecommand{\predicateBitpreciseCNFKINDResultsCNFKINDPathEqualOnlyCpuTimeSumPlainHours}{}
  \renewcommand{\predicateBitpreciseCNFKINDResultsCNFKINDPathEqualOnlyCpuTimeSumPlainHours}{19.032536965134767\xspace}

  % cpu-time-avg
\providecommand{\predicateBitpreciseCNFKINDResultsCNFKINDPathEqualOnlyCpuTimeAvgPlain}{}
  \renewcommand{\predicateBitpreciseCNFKINDResultsCNFKINDPathEqualOnlyCpuTimeAvgPlain}{39.063359791610694\xspace}
\providecommand{\predicateBitpreciseCNFKINDResultsCNFKINDPathEqualOnlyCpuTimeAvgPlainHours}{}
  \renewcommand{\predicateBitpreciseCNFKINDResultsCNFKINDPathEqualOnlyCpuTimeAvgPlainHours}{0.010850933275447415\xspace}

  % inv-succ
\providecommand{\predicateBitpreciseCNFKINDResultsCNFKINDPathEqualOnlyInvSuccPlain}{}
  \renewcommand{\predicateBitpreciseCNFKINDResultsCNFKINDPathEqualOnlyInvSuccPlain}{0\xspace}

  % inv-tries
\providecommand{\predicateBitpreciseCNFKINDResultsCNFKINDPathEqualOnlyInvTriesPlain}{}
  \renewcommand{\predicateBitpreciseCNFKINDResultsCNFKINDPathEqualOnlyInvTriesPlain}{5920\xspace}

  % inv-time-sum
\providecommand{\predicateBitpreciseCNFKINDResultsCNFKINDPathEqualOnlyInvTimeSumPlain}{}
  \renewcommand{\predicateBitpreciseCNFKINDResultsCNFKINDPathEqualOnlyInvTimeSumPlain}{2425.4310000000014\xspace}
\providecommand{\predicateBitpreciseCNFKINDResultsCNFKINDPathEqualOnlyInvTimeSumPlainHours}{}
  \renewcommand{\predicateBitpreciseCNFKINDResultsCNFKINDPathEqualOnlyInvTimeSumPlainHours}{0.6737308333333337\xspace}

%%% predicate_bitprecise_weakening.2016-09-05_2021.results.Weakening-path %%%
 %% correct %%
\providecommand{\predicateBitpreciseWeakeningResultsWeakeningPathCorrectPlain}{}
  \renewcommand{\predicateBitpreciseWeakeningResultsWeakeningPathCorrectPlain}{1913\xspace}

  % cpu-time-sum
\providecommand{\predicateBitpreciseWeakeningResultsWeakeningPathCorrectCpuTimeSumPlain}{}
  \renewcommand{\predicateBitpreciseWeakeningResultsWeakeningPathCorrectCpuTimeSumPlain}{94240.38449697709\xspace}
\providecommand{\predicateBitpreciseWeakeningResultsWeakeningPathCorrectCpuTimeSumPlainHours}{}
  \renewcommand{\predicateBitpreciseWeakeningResultsWeakeningPathCorrectCpuTimeSumPlainHours}{26.177884582493636\xspace}

  % cpu-time-avg
\providecommand{\predicateBitpreciseWeakeningResultsWeakeningPathCorrectCpuTimeAvgPlain}{}
  \renewcommand{\predicateBitpreciseWeakeningResultsWeakeningPathCorrectCpuTimeAvgPlain}{49.26313878566497\xspace}
\providecommand{\predicateBitpreciseWeakeningResultsWeakeningPathCorrectCpuTimeAvgPlainHours}{}
  \renewcommand{\predicateBitpreciseWeakeningResultsWeakeningPathCorrectCpuTimeAvgPlainHours}{0.01368420521824027\xspace}

  % inv-succ
\providecommand{\predicateBitpreciseWeakeningResultsWeakeningPathCorrectInvSuccPlain}{}
  \renewcommand{\predicateBitpreciseWeakeningResultsWeakeningPathCorrectInvSuccPlain}{0\xspace}

  % inv-tries
\providecommand{\predicateBitpreciseWeakeningResultsWeakeningPathCorrectInvTriesPlain}{}
  \renewcommand{\predicateBitpreciseWeakeningResultsWeakeningPathCorrectInvTriesPlain}{7291\xspace}

  % inv-time-sum
\providecommand{\predicateBitpreciseWeakeningResultsWeakeningPathCorrectInvTimeSumPlain}{}
  \renewcommand{\predicateBitpreciseWeakeningResultsWeakeningPathCorrectInvTimeSumPlain}{7795.977000000009\xspace}
\providecommand{\predicateBitpreciseWeakeningResultsWeakeningPathCorrectInvTimeSumPlainHours}{}
  \renewcommand{\predicateBitpreciseWeakeningResultsWeakeningPathCorrectInvTimeSumPlainHours}{2.1655491666666693\xspace}

 %% incorrect %%
\providecommand{\predicateBitpreciseWeakeningResultsWeakeningPathIncorrectPlain}{}
  \renewcommand{\predicateBitpreciseWeakeningResultsWeakeningPathIncorrectPlain}{27\xspace}

  % cpu-time-sum
\providecommand{\predicateBitpreciseWeakeningResultsWeakeningPathIncorrectCpuTimeSumPlain}{}
  \renewcommand{\predicateBitpreciseWeakeningResultsWeakeningPathIncorrectCpuTimeSumPlain}{769.2952002560002\xspace}
\providecommand{\predicateBitpreciseWeakeningResultsWeakeningPathIncorrectCpuTimeSumPlainHours}{}
  \renewcommand{\predicateBitpreciseWeakeningResultsWeakeningPathIncorrectCpuTimeSumPlainHours}{0.21369311118222226\xspace}

  % cpu-time-avg
\providecommand{\predicateBitpreciseWeakeningResultsWeakeningPathIncorrectCpuTimeAvgPlain}{}
  \renewcommand{\predicateBitpreciseWeakeningResultsWeakeningPathIncorrectCpuTimeAvgPlain}{28.4924148242963\xspace}
\providecommand{\predicateBitpreciseWeakeningResultsWeakeningPathIncorrectCpuTimeAvgPlainHours}{}
  \renewcommand{\predicateBitpreciseWeakeningResultsWeakeningPathIncorrectCpuTimeAvgPlainHours}{0.007914559673415639\xspace}

  % inv-succ
\providecommand{\predicateBitpreciseWeakeningResultsWeakeningPathIncorrectInvSuccPlain}{}
  \renewcommand{\predicateBitpreciseWeakeningResultsWeakeningPathIncorrectInvSuccPlain}{0\xspace}

  % inv-tries
\providecommand{\predicateBitpreciseWeakeningResultsWeakeningPathIncorrectInvTriesPlain}{}
  \renewcommand{\predicateBitpreciseWeakeningResultsWeakeningPathIncorrectInvTriesPlain}{89\xspace}

  % inv-time-sum
\providecommand{\predicateBitpreciseWeakeningResultsWeakeningPathIncorrectInvTimeSumPlain}{}
  \renewcommand{\predicateBitpreciseWeakeningResultsWeakeningPathIncorrectInvTimeSumPlain}{35.729\xspace}
\providecommand{\predicateBitpreciseWeakeningResultsWeakeningPathIncorrectInvTimeSumPlainHours}{}
  \renewcommand{\predicateBitpreciseWeakeningResultsWeakeningPathIncorrectInvTimeSumPlainHours}{0.009924722222222222\xspace}

 %% timeout %%
\providecommand{\predicateBitpreciseWeakeningResultsWeakeningPathTimeoutPlain}{}
  \renewcommand{\predicateBitpreciseWeakeningResultsWeakeningPathTimeoutPlain}{1399\xspace}

  % cpu-time-sum
\providecommand{\predicateBitpreciseWeakeningResultsWeakeningPathTimeoutCpuTimeSumPlain}{}
  \renewcommand{\predicateBitpreciseWeakeningResultsWeakeningPathTimeoutCpuTimeSumPlain}{426212.02063801396\xspace}
\providecommand{\predicateBitpreciseWeakeningResultsWeakeningPathTimeoutCpuTimeSumPlainHours}{}
  \renewcommand{\predicateBitpreciseWeakeningResultsWeakeningPathTimeoutCpuTimeSumPlainHours}{118.39222795500388\xspace}

  % cpu-time-avg
\providecommand{\predicateBitpreciseWeakeningResultsWeakeningPathTimeoutCpuTimeAvgPlain}{}
  \renewcommand{\predicateBitpreciseWeakeningResultsWeakeningPathTimeoutCpuTimeAvgPlain}{304.654768147258\xspace}
\providecommand{\predicateBitpreciseWeakeningResultsWeakeningPathTimeoutCpuTimeAvgPlainHours}{}
  \renewcommand{\predicateBitpreciseWeakeningResultsWeakeningPathTimeoutCpuTimeAvgPlainHours}{0.08462632448534944\xspace}

  % inv-succ
\providecommand{\predicateBitpreciseWeakeningResultsWeakeningPathTimeoutInvSuccPlain}{}
  \renewcommand{\predicateBitpreciseWeakeningResultsWeakeningPathTimeoutInvSuccPlain}{0\xspace}

  % inv-tries
\providecommand{\predicateBitpreciseWeakeningResultsWeakeningPathTimeoutInvTriesPlain}{}
  \renewcommand{\predicateBitpreciseWeakeningResultsWeakeningPathTimeoutInvTriesPlain}{11521\xspace}

  % inv-time-sum
\providecommand{\predicateBitpreciseWeakeningResultsWeakeningPathTimeoutInvTimeSumPlain}{}
  \renewcommand{\predicateBitpreciseWeakeningResultsWeakeningPathTimeoutInvTimeSumPlain}{54769.06999999997\xspace}
\providecommand{\predicateBitpreciseWeakeningResultsWeakeningPathTimeoutInvTimeSumPlainHours}{}
  \renewcommand{\predicateBitpreciseWeakeningResultsWeakeningPathTimeoutInvTimeSumPlainHours}{15.213630555555547\xspace}

 %% unknown-or-category-error %%
\providecommand{\predicateBitpreciseWeakeningResultsWeakeningPathUnknownOrCategoryErrorPlain}{}
  \renewcommand{\predicateBitpreciseWeakeningResultsWeakeningPathUnknownOrCategoryErrorPlain}{1548\xspace}

  % cpu-time-sum
\providecommand{\predicateBitpreciseWeakeningResultsWeakeningPathUnknownOrCategoryErrorCpuTimeSumPlain}{}
  \renewcommand{\predicateBitpreciseWeakeningResultsWeakeningPathUnknownOrCategoryErrorCpuTimeSumPlain}{446836.61386010784\xspace}
\providecommand{\predicateBitpreciseWeakeningResultsWeakeningPathUnknownOrCategoryErrorCpuTimeSumPlainHours}{}
  \renewcommand{\predicateBitpreciseWeakeningResultsWeakeningPathUnknownOrCategoryErrorCpuTimeSumPlainHours}{124.12128162780773\xspace}

  % cpu-time-avg
\providecommand{\predicateBitpreciseWeakeningResultsWeakeningPathUnknownOrCategoryErrorCpuTimeAvgPlain}{}
  \renewcommand{\predicateBitpreciseWeakeningResultsWeakeningPathUnknownOrCategoryErrorCpuTimeAvgPlain}{288.6541433204831\xspace}
\providecommand{\predicateBitpreciseWeakeningResultsWeakeningPathUnknownOrCategoryErrorCpuTimeAvgPlainHours}{}
  \renewcommand{\predicateBitpreciseWeakeningResultsWeakeningPathUnknownOrCategoryErrorCpuTimeAvgPlainHours}{0.08018170647791198\xspace}

  % inv-succ
\providecommand{\predicateBitpreciseWeakeningResultsWeakeningPathUnknownOrCategoryErrorInvSuccPlain}{}
  \renewcommand{\predicateBitpreciseWeakeningResultsWeakeningPathUnknownOrCategoryErrorInvSuccPlain}{0\xspace}

  % inv-tries
\providecommand{\predicateBitpreciseWeakeningResultsWeakeningPathUnknownOrCategoryErrorInvTriesPlain}{}
  \renewcommand{\predicateBitpreciseWeakeningResultsWeakeningPathUnknownOrCategoryErrorInvTriesPlain}{11812\xspace}

  % inv-time-sum
\providecommand{\predicateBitpreciseWeakeningResultsWeakeningPathUnknownOrCategoryErrorInvTimeSumPlain}{}
  \renewcommand{\predicateBitpreciseWeakeningResultsWeakeningPathUnknownOrCategoryErrorInvTimeSumPlain}{58634.81699999998\xspace}
\providecommand{\predicateBitpreciseWeakeningResultsWeakeningPathUnknownOrCategoryErrorInvTimeSumPlainHours}{}
  \renewcommand{\predicateBitpreciseWeakeningResultsWeakeningPathUnknownOrCategoryErrorInvTimeSumPlainHours}{16.28744916666666\xspace}

 %% correct-false %%
\providecommand{\predicateBitpreciseWeakeningResultsWeakeningPathCorrectFalsePlain}{}
  \renewcommand{\predicateBitpreciseWeakeningResultsWeakeningPathCorrectFalsePlain}{534\xspace}

  % cpu-time-sum
\providecommand{\predicateBitpreciseWeakeningResultsWeakeningPathCorrectFalseCpuTimeSumPlain}{}
  \renewcommand{\predicateBitpreciseWeakeningResultsWeakeningPathCorrectFalseCpuTimeSumPlain}{35533.126862952995\xspace}
\providecommand{\predicateBitpreciseWeakeningResultsWeakeningPathCorrectFalseCpuTimeSumPlainHours}{}
  \renewcommand{\predicateBitpreciseWeakeningResultsWeakeningPathCorrectFalseCpuTimeSumPlainHours}{9.870313017486943\xspace}

  % cpu-time-avg
\providecommand{\predicateBitpreciseWeakeningResultsWeakeningPathCorrectFalseCpuTimeAvgPlain}{}
  \renewcommand{\predicateBitpreciseWeakeningResultsWeakeningPathCorrectFalseCpuTimeAvgPlain}{66.54143607294569\xspace}
\providecommand{\predicateBitpreciseWeakeningResultsWeakeningPathCorrectFalseCpuTimeAvgPlainHours}{}
  \renewcommand{\predicateBitpreciseWeakeningResultsWeakeningPathCorrectFalseCpuTimeAvgPlainHours}{0.018483732242484913\xspace}

  % inv-succ
\providecommand{\predicateBitpreciseWeakeningResultsWeakeningPathCorrectFalseInvSuccPlain}{}
  \renewcommand{\predicateBitpreciseWeakeningResultsWeakeningPathCorrectFalseInvSuccPlain}{0\xspace}

  % inv-tries
\providecommand{\predicateBitpreciseWeakeningResultsWeakeningPathCorrectFalseInvTriesPlain}{}
  \renewcommand{\predicateBitpreciseWeakeningResultsWeakeningPathCorrectFalseInvTriesPlain}{2521\xspace}

  % inv-time-sum
\providecommand{\predicateBitpreciseWeakeningResultsWeakeningPathCorrectFalseInvTimeSumPlain}{}
  \renewcommand{\predicateBitpreciseWeakeningResultsWeakeningPathCorrectFalseInvTimeSumPlain}{2631.9269999999988\xspace}
\providecommand{\predicateBitpreciseWeakeningResultsWeakeningPathCorrectFalseInvTimeSumPlainHours}{}
  \renewcommand{\predicateBitpreciseWeakeningResultsWeakeningPathCorrectFalseInvTimeSumPlainHours}{0.731090833333333\xspace}

 %% correct-true %%
\providecommand{\predicateBitpreciseWeakeningResultsWeakeningPathCorrectTruePlain}{}
  \renewcommand{\predicateBitpreciseWeakeningResultsWeakeningPathCorrectTruePlain}{1379\xspace}

  % cpu-time-sum
\providecommand{\predicateBitpreciseWeakeningResultsWeakeningPathCorrectTrueCpuTimeSumPlain}{}
  \renewcommand{\predicateBitpreciseWeakeningResultsWeakeningPathCorrectTrueCpuTimeSumPlain}{58707.25763402396\xspace}
\providecommand{\predicateBitpreciseWeakeningResultsWeakeningPathCorrectTrueCpuTimeSumPlainHours}{}
  \renewcommand{\predicateBitpreciseWeakeningResultsWeakeningPathCorrectTrueCpuTimeSumPlainHours}{16.307571565006654\xspace}

  % cpu-time-avg
\providecommand{\predicateBitpreciseWeakeningResultsWeakeningPathCorrectTrueCpuTimeAvgPlain}{}
  \renewcommand{\predicateBitpreciseWeakeningResultsWeakeningPathCorrectTrueCpuTimeAvgPlain}{42.572340561293665\xspace}
\providecommand{\predicateBitpreciseWeakeningResultsWeakeningPathCorrectTrueCpuTimeAvgPlainHours}{}
  \renewcommand{\predicateBitpreciseWeakeningResultsWeakeningPathCorrectTrueCpuTimeAvgPlainHours}{0.011825650155914906\xspace}

  % inv-succ
\providecommand{\predicateBitpreciseWeakeningResultsWeakeningPathCorrectTrueInvSuccPlain}{}
  \renewcommand{\predicateBitpreciseWeakeningResultsWeakeningPathCorrectTrueInvSuccPlain}{0\xspace}

  % inv-tries
\providecommand{\predicateBitpreciseWeakeningResultsWeakeningPathCorrectTrueInvTriesPlain}{}
  \renewcommand{\predicateBitpreciseWeakeningResultsWeakeningPathCorrectTrueInvTriesPlain}{4770\xspace}

  % inv-time-sum
\providecommand{\predicateBitpreciseWeakeningResultsWeakeningPathCorrectTrueInvTimeSumPlain}{}
  \renewcommand{\predicateBitpreciseWeakeningResultsWeakeningPathCorrectTrueInvTimeSumPlain}{5164.0500000000175\xspace}
\providecommand{\predicateBitpreciseWeakeningResultsWeakeningPathCorrectTrueInvTimeSumPlainHours}{}
  \renewcommand{\predicateBitpreciseWeakeningResultsWeakeningPathCorrectTrueInvTimeSumPlainHours}{1.4344583333333383\xspace}

 %% incorrect-false %%
\providecommand{\predicateBitpreciseWeakeningResultsWeakeningPathIncorrectFalsePlain}{}
  \renewcommand{\predicateBitpreciseWeakeningResultsWeakeningPathIncorrectFalsePlain}{27\xspace}

  % cpu-time-sum
\providecommand{\predicateBitpreciseWeakeningResultsWeakeningPathIncorrectFalseCpuTimeSumPlain}{}
  \renewcommand{\predicateBitpreciseWeakeningResultsWeakeningPathIncorrectFalseCpuTimeSumPlain}{769.2952002560002\xspace}
\providecommand{\predicateBitpreciseWeakeningResultsWeakeningPathIncorrectFalseCpuTimeSumPlainHours}{}
  \renewcommand{\predicateBitpreciseWeakeningResultsWeakeningPathIncorrectFalseCpuTimeSumPlainHours}{0.21369311118222226\xspace}

  % cpu-time-avg
\providecommand{\predicateBitpreciseWeakeningResultsWeakeningPathIncorrectFalseCpuTimeAvgPlain}{}
  \renewcommand{\predicateBitpreciseWeakeningResultsWeakeningPathIncorrectFalseCpuTimeAvgPlain}{28.4924148242963\xspace}
\providecommand{\predicateBitpreciseWeakeningResultsWeakeningPathIncorrectFalseCpuTimeAvgPlainHours}{}
  \renewcommand{\predicateBitpreciseWeakeningResultsWeakeningPathIncorrectFalseCpuTimeAvgPlainHours}{0.007914559673415639\xspace}

  % inv-succ
\providecommand{\predicateBitpreciseWeakeningResultsWeakeningPathIncorrectFalseInvSuccPlain}{}
  \renewcommand{\predicateBitpreciseWeakeningResultsWeakeningPathIncorrectFalseInvSuccPlain}{0\xspace}

  % inv-tries
\providecommand{\predicateBitpreciseWeakeningResultsWeakeningPathIncorrectFalseInvTriesPlain}{}
  \renewcommand{\predicateBitpreciseWeakeningResultsWeakeningPathIncorrectFalseInvTriesPlain}{89\xspace}

  % inv-time-sum
\providecommand{\predicateBitpreciseWeakeningResultsWeakeningPathIncorrectFalseInvTimeSumPlain}{}
  \renewcommand{\predicateBitpreciseWeakeningResultsWeakeningPathIncorrectFalseInvTimeSumPlain}{35.729\xspace}
\providecommand{\predicateBitpreciseWeakeningResultsWeakeningPathIncorrectFalseInvTimeSumPlainHours}{}
  \renewcommand{\predicateBitpreciseWeakeningResultsWeakeningPathIncorrectFalseInvTimeSumPlainHours}{0.009924722222222222\xspace}

 %% incorrect-true %%
\providecommand{\predicateBitpreciseWeakeningResultsWeakeningPathIncorrectTruePlain}{}
  \renewcommand{\predicateBitpreciseWeakeningResultsWeakeningPathIncorrectTruePlain}{0\xspace}

  % cpu-time-sum
\providecommand{\predicateBitpreciseWeakeningResultsWeakeningPathIncorrectTrueCpuTimeSumPlain}{}
  \renewcommand{\predicateBitpreciseWeakeningResultsWeakeningPathIncorrectTrueCpuTimeSumPlain}{0.0\xspace}
\providecommand{\predicateBitpreciseWeakeningResultsWeakeningPathIncorrectTrueCpuTimeSumPlainHours}{}
  \renewcommand{\predicateBitpreciseWeakeningResultsWeakeningPathIncorrectTrueCpuTimeSumPlainHours}{0.0\xspace}

  % cpu-time-avg
\providecommand{\predicateBitpreciseWeakeningResultsWeakeningPathIncorrectTrueCpuTimeAvgPlain}{}
  \renewcommand{\predicateBitpreciseWeakeningResultsWeakeningPathIncorrectTrueCpuTimeAvgPlain}{NaN\xspace}
\providecommand{\predicateBitpreciseWeakeningResultsWeakeningPathIncorrectTrueCpuTimeAvgPlainHours}{}
  \renewcommand{\predicateBitpreciseWeakeningResultsWeakeningPathIncorrectTrueCpuTimeAvgPlainHours}{NaN\xspace}

  % inv-succ
\providecommand{\predicateBitpreciseWeakeningResultsWeakeningPathIncorrectTrueInvSuccPlain}{}
  \renewcommand{\predicateBitpreciseWeakeningResultsWeakeningPathIncorrectTrueInvSuccPlain}{0\xspace}

  % inv-tries
\providecommand{\predicateBitpreciseWeakeningResultsWeakeningPathIncorrectTrueInvTriesPlain}{}
  \renewcommand{\predicateBitpreciseWeakeningResultsWeakeningPathIncorrectTrueInvTriesPlain}{0\xspace}

  % inv-time-sum
\providecommand{\predicateBitpreciseWeakeningResultsWeakeningPathIncorrectTrueInvTimeSumPlain}{}
  \renewcommand{\predicateBitpreciseWeakeningResultsWeakeningPathIncorrectTrueInvTimeSumPlain}{0.0\xspace}
\providecommand{\predicateBitpreciseWeakeningResultsWeakeningPathIncorrectTrueInvTimeSumPlainHours}{}
  \renewcommand{\predicateBitpreciseWeakeningResultsWeakeningPathIncorrectTrueInvTimeSumPlainHours}{0.0\xspace}

 %% all %%
\providecommand{\predicateBitpreciseWeakeningResultsWeakeningPathAllPlain}{}
  \renewcommand{\predicateBitpreciseWeakeningResultsWeakeningPathAllPlain}{3488\xspace}

  % cpu-time-sum
\providecommand{\predicateBitpreciseWeakeningResultsWeakeningPathAllCpuTimeSumPlain}{}
  \renewcommand{\predicateBitpreciseWeakeningResultsWeakeningPathAllCpuTimeSumPlain}{541846.2935573392\xspace}
\providecommand{\predicateBitpreciseWeakeningResultsWeakeningPathAllCpuTimeSumPlainHours}{}
  \renewcommand{\predicateBitpreciseWeakeningResultsWeakeningPathAllCpuTimeSumPlainHours}{150.51285932148312\xspace}

  % cpu-time-avg
\providecommand{\predicateBitpreciseWeakeningResultsWeakeningPathAllCpuTimeAvgPlain}{}
  \renewcommand{\predicateBitpreciseWeakeningResultsWeakeningPathAllCpuTimeAvgPlain}{155.34584104281515\xspace}
\providecommand{\predicateBitpreciseWeakeningResultsWeakeningPathAllCpuTimeAvgPlainHours}{}
  \renewcommand{\predicateBitpreciseWeakeningResultsWeakeningPathAllCpuTimeAvgPlainHours}{0.0431516225118931\xspace}

  % inv-succ
\providecommand{\predicateBitpreciseWeakeningResultsWeakeningPathAllInvSuccPlain}{}
  \renewcommand{\predicateBitpreciseWeakeningResultsWeakeningPathAllInvSuccPlain}{0\xspace}

  % inv-tries
\providecommand{\predicateBitpreciseWeakeningResultsWeakeningPathAllInvTriesPlain}{}
  \renewcommand{\predicateBitpreciseWeakeningResultsWeakeningPathAllInvTriesPlain}{19192\xspace}

  % inv-time-sum
\providecommand{\predicateBitpreciseWeakeningResultsWeakeningPathAllInvTimeSumPlain}{}
  \renewcommand{\predicateBitpreciseWeakeningResultsWeakeningPathAllInvTimeSumPlain}{66466.52300000002\xspace}
\providecommand{\predicateBitpreciseWeakeningResultsWeakeningPathAllInvTimeSumPlainHours}{}
  \renewcommand{\predicateBitpreciseWeakeningResultsWeakeningPathAllInvTimeSumPlainHours}{18.46292305555556\xspace}

 %% equal-only %%
\providecommand{\predicateBitpreciseWeakeningResultsWeakeningPathEqualOnlyPlain}{}
  \renewcommand{\predicateBitpreciseWeakeningResultsWeakeningPathEqualOnlyPlain}{1754\xspace}

  % cpu-time-sum
\providecommand{\predicateBitpreciseWeakeningResultsWeakeningPathEqualOnlyCpuTimeSumPlain}{}
  \renewcommand{\predicateBitpreciseWeakeningResultsWeakeningPathEqualOnlyCpuTimeSumPlain}{70747.71166591799\xspace}
\providecommand{\predicateBitpreciseWeakeningResultsWeakeningPathEqualOnlyCpuTimeSumPlainHours}{}
  \renewcommand{\predicateBitpreciseWeakeningResultsWeakeningPathEqualOnlyCpuTimeSumPlainHours}{19.652142129421662\xspace}

  % cpu-time-avg
\providecommand{\predicateBitpreciseWeakeningResultsWeakeningPathEqualOnlyCpuTimeAvgPlain}{}
  \renewcommand{\predicateBitpreciseWeakeningResultsWeakeningPathEqualOnlyCpuTimeAvgPlain}{40.335069364833515\xspace}
\providecommand{\predicateBitpreciseWeakeningResultsWeakeningPathEqualOnlyCpuTimeAvgPlainHours}{}
  \renewcommand{\predicateBitpreciseWeakeningResultsWeakeningPathEqualOnlyCpuTimeAvgPlainHours}{0.011204185934675976\xspace}

  % inv-succ
\providecommand{\predicateBitpreciseWeakeningResultsWeakeningPathEqualOnlyInvSuccPlain}{}
  \renewcommand{\predicateBitpreciseWeakeningResultsWeakeningPathEqualOnlyInvSuccPlain}{0\xspace}

  % inv-tries
\providecommand{\predicateBitpreciseWeakeningResultsWeakeningPathEqualOnlyInvTriesPlain}{}
  \renewcommand{\predicateBitpreciseWeakeningResultsWeakeningPathEqualOnlyInvTriesPlain}{5920\xspace}

  % inv-time-sum
\providecommand{\predicateBitpreciseWeakeningResultsWeakeningPathEqualOnlyInvTimeSumPlain}{}
  \renewcommand{\predicateBitpreciseWeakeningResultsWeakeningPathEqualOnlyInvTimeSumPlain}{6144.231000000008\xspace}
\providecommand{\predicateBitpreciseWeakeningResultsWeakeningPathEqualOnlyInvTimeSumPlainHours}{}
  \renewcommand{\predicateBitpreciseWeakeningResultsWeakeningPathEqualOnlyInvTimeSumPlainHours}{1.7067308333333355\xspace}

%%% predicate_bitprecise_pathinvariants.2016-09-10_1338.results.pathInvariants-policyCPA %%%
 %% correct %%
\providecommand{\predicateBitprecisePathinvariantsResultsPathInvariantsPolicyCPACorrectPlain}{}
  \renewcommand{\predicateBitprecisePathinvariantsResultsPathInvariantsPolicyCPACorrectPlain}{1866\xspace}

  % cpu-time-sum
\providecommand{\predicateBitprecisePathinvariantsResultsPathInvariantsPolicyCPACorrectCpuTimeSumPlain}{}
  \renewcommand{\predicateBitprecisePathinvariantsResultsPathInvariantsPolicyCPACorrectCpuTimeSumPlain}{113204.03672177006\xspace}
\providecommand{\predicateBitprecisePathinvariantsResultsPathInvariantsPolicyCPACorrectCpuTimeSumPlainHours}{}
  \renewcommand{\predicateBitprecisePathinvariantsResultsPathInvariantsPolicyCPACorrectCpuTimeSumPlainHours}{31.445565756047237\xspace}

  % cpu-time-avg
\providecommand{\predicateBitprecisePathinvariantsResultsPathInvariantsPolicyCPACorrectCpuTimeAvgPlain}{}
  \renewcommand{\predicateBitprecisePathinvariantsResultsPathInvariantsPolicyCPACorrectCpuTimeAvgPlain}{60.66668634607184\xspace}
\providecommand{\predicateBitprecisePathinvariantsResultsPathInvariantsPolicyCPACorrectCpuTimeAvgPlainHours}{}
  \renewcommand{\predicateBitprecisePathinvariantsResultsPathInvariantsPolicyCPACorrectCpuTimeAvgPlainHours}{0.01685185731835329\xspace}

  % inv-succ
\providecommand{\predicateBitprecisePathinvariantsResultsPathInvariantsPolicyCPACorrectInvSuccPlain}{}
  \renewcommand{\predicateBitprecisePathinvariantsResultsPathInvariantsPolicyCPACorrectInvSuccPlain}{1611\xspace}

  % inv-tries
\providecommand{\predicateBitprecisePathinvariantsResultsPathInvariantsPolicyCPACorrectInvTriesPlain}{}
  \renewcommand{\predicateBitprecisePathinvariantsResultsPathInvariantsPolicyCPACorrectInvTriesPlain}{4600\xspace}

  % inv-time-sum
\providecommand{\predicateBitprecisePathinvariantsResultsPathInvariantsPolicyCPACorrectInvTimeSumPlain}{}
  \renewcommand{\predicateBitprecisePathinvariantsResultsPathInvariantsPolicyCPACorrectInvTimeSumPlain}{13829.819000000007\xspace}
\providecommand{\predicateBitprecisePathinvariantsResultsPathInvariantsPolicyCPACorrectInvTimeSumPlainHours}{}
  \renewcommand{\predicateBitprecisePathinvariantsResultsPathInvariantsPolicyCPACorrectInvTimeSumPlainHours}{3.8416163888888906\xspace}

 %% incorrect %%
\providecommand{\predicateBitprecisePathinvariantsResultsPathInvariantsPolicyCPAIncorrectPlain}{}
  \renewcommand{\predicateBitprecisePathinvariantsResultsPathInvariantsPolicyCPAIncorrectPlain}{27\xspace}

  % cpu-time-sum
\providecommand{\predicateBitprecisePathinvariantsResultsPathInvariantsPolicyCPAIncorrectCpuTimeSumPlain}{}
  \renewcommand{\predicateBitprecisePathinvariantsResultsPathInvariantsPolicyCPAIncorrectCpuTimeSumPlain}{1057.8719591509998\xspace}
\providecommand{\predicateBitprecisePathinvariantsResultsPathInvariantsPolicyCPAIncorrectCpuTimeSumPlainHours}{}
  \renewcommand{\predicateBitprecisePathinvariantsResultsPathInvariantsPolicyCPAIncorrectCpuTimeSumPlainHours}{0.29385332198638886\xspace}

  % cpu-time-avg
\providecommand{\predicateBitprecisePathinvariantsResultsPathInvariantsPolicyCPAIncorrectCpuTimeAvgPlain}{}
  \renewcommand{\predicateBitprecisePathinvariantsResultsPathInvariantsPolicyCPAIncorrectCpuTimeAvgPlain}{39.18044293151851\xspace}
\providecommand{\predicateBitprecisePathinvariantsResultsPathInvariantsPolicyCPAIncorrectCpuTimeAvgPlainHours}{}
  \renewcommand{\predicateBitprecisePathinvariantsResultsPathInvariantsPolicyCPAIncorrectCpuTimeAvgPlainHours}{0.010883456369866254\xspace}

  % inv-succ
\providecommand{\predicateBitprecisePathinvariantsResultsPathInvariantsPolicyCPAIncorrectInvSuccPlain}{}
  \renewcommand{\predicateBitprecisePathinvariantsResultsPathInvariantsPolicyCPAIncorrectInvSuccPlain}{3\xspace}

  % inv-tries
\providecommand{\predicateBitprecisePathinvariantsResultsPathInvariantsPolicyCPAIncorrectInvTriesPlain}{}
  \renewcommand{\predicateBitprecisePathinvariantsResultsPathInvariantsPolicyCPAIncorrectInvTriesPlain}{60\xspace}

  % inv-time-sum
\providecommand{\predicateBitprecisePathinvariantsResultsPathInvariantsPolicyCPAIncorrectInvTimeSumPlain}{}
  \renewcommand{\predicateBitprecisePathinvariantsResultsPathInvariantsPolicyCPAIncorrectInvTimeSumPlain}{168.88299999999998\xspace}
\providecommand{\predicateBitprecisePathinvariantsResultsPathInvariantsPolicyCPAIncorrectInvTimeSumPlainHours}{}
  \renewcommand{\predicateBitprecisePathinvariantsResultsPathInvariantsPolicyCPAIncorrectInvTimeSumPlainHours}{0.04691194444444444\xspace}

 %% timeout %%
\providecommand{\predicateBitprecisePathinvariantsResultsPathInvariantsPolicyCPATimeoutPlain}{}
  \renewcommand{\predicateBitprecisePathinvariantsResultsPathInvariantsPolicyCPATimeoutPlain}{1485\xspace}

  % cpu-time-sum
\providecommand{\predicateBitprecisePathinvariantsResultsPathInvariantsPolicyCPATimeoutCpuTimeSumPlain}{}
  \renewcommand{\predicateBitprecisePathinvariantsResultsPathInvariantsPolicyCPATimeoutCpuTimeSumPlain}{453937.33718315023\xspace}
\providecommand{\predicateBitprecisePathinvariantsResultsPathInvariantsPolicyCPATimeoutCpuTimeSumPlainHours}{}
  \renewcommand{\predicateBitprecisePathinvariantsResultsPathInvariantsPolicyCPATimeoutCpuTimeSumPlainHours}{126.09370477309729\xspace}

  % cpu-time-avg
\providecommand{\predicateBitprecisePathinvariantsResultsPathInvariantsPolicyCPATimeoutCpuTimeAvgPlain}{}
  \renewcommand{\predicateBitprecisePathinvariantsResultsPathInvariantsPolicyCPATimeoutCpuTimeAvgPlain}{305.6817085408419\xspace}
\providecommand{\predicateBitprecisePathinvariantsResultsPathInvariantsPolicyCPATimeoutCpuTimeAvgPlainHours}{}
  \renewcommand{\predicateBitprecisePathinvariantsResultsPathInvariantsPolicyCPATimeoutCpuTimeAvgPlainHours}{0.08491158570578941\xspace}

  % inv-succ
\providecommand{\predicateBitprecisePathinvariantsResultsPathInvariantsPolicyCPATimeoutInvSuccPlain}{}
  \renewcommand{\predicateBitprecisePathinvariantsResultsPathInvariantsPolicyCPATimeoutInvSuccPlain}{51884\xspace}

  % inv-tries
\providecommand{\predicateBitprecisePathinvariantsResultsPathInvariantsPolicyCPATimeoutInvTriesPlain}{}
  \renewcommand{\predicateBitprecisePathinvariantsResultsPathInvariantsPolicyCPATimeoutInvTriesPlain}{57927\xspace}

  % inv-time-sum
\providecommand{\predicateBitprecisePathinvariantsResultsPathInvariantsPolicyCPATimeoutInvTimeSumPlain}{}
  \renewcommand{\predicateBitprecisePathinvariantsResultsPathInvariantsPolicyCPATimeoutInvTimeSumPlain}{17444.481\xspace}
\providecommand{\predicateBitprecisePathinvariantsResultsPathInvariantsPolicyCPATimeoutInvTimeSumPlainHours}{}
  \renewcommand{\predicateBitprecisePathinvariantsResultsPathInvariantsPolicyCPATimeoutInvTimeSumPlainHours}{4.845689166666666\xspace}

 %% unknown-or-category-error %%
\providecommand{\predicateBitprecisePathinvariantsResultsPathInvariantsPolicyCPAUnknownOrCategoryErrorPlain}{}
  \renewcommand{\predicateBitprecisePathinvariantsResultsPathInvariantsPolicyCPAUnknownOrCategoryErrorPlain}{1595\xspace}

  % cpu-time-sum
\providecommand{\predicateBitprecisePathinvariantsResultsPathInvariantsPolicyCPAUnknownOrCategoryErrorCpuTimeSumPlain}{}
  \renewcommand{\predicateBitprecisePathinvariantsResultsPathInvariantsPolicyCPAUnknownOrCategoryErrorCpuTimeSumPlain}{465644.634696504\xspace}
\providecommand{\predicateBitprecisePathinvariantsResultsPathInvariantsPolicyCPAUnknownOrCategoryErrorCpuTimeSumPlainHours}{}
  \renewcommand{\predicateBitprecisePathinvariantsResultsPathInvariantsPolicyCPAUnknownOrCategoryErrorCpuTimeSumPlainHours}{129.34573186014\xspace}

  % cpu-time-avg
\providecommand{\predicateBitprecisePathinvariantsResultsPathInvariantsPolicyCPAUnknownOrCategoryErrorCpuTimeAvgPlain}{}
  \renewcommand{\predicateBitprecisePathinvariantsResultsPathInvariantsPolicyCPAUnknownOrCategoryErrorCpuTimeAvgPlain}{291.94020984106834\xspace}
\providecommand{\predicateBitprecisePathinvariantsResultsPathInvariantsPolicyCPAUnknownOrCategoryErrorCpuTimeAvgPlainHours}{}
  \renewcommand{\predicateBitprecisePathinvariantsResultsPathInvariantsPolicyCPAUnknownOrCategoryErrorCpuTimeAvgPlainHours}{0.0810945027336301\xspace}

  % inv-succ
\providecommand{\predicateBitprecisePathinvariantsResultsPathInvariantsPolicyCPAUnknownOrCategoryErrorInvSuccPlain}{}
  \renewcommand{\predicateBitprecisePathinvariantsResultsPathInvariantsPolicyCPAUnknownOrCategoryErrorInvSuccPlain}{51890\xspace}

  % inv-tries
\providecommand{\predicateBitprecisePathinvariantsResultsPathInvariantsPolicyCPAUnknownOrCategoryErrorInvTriesPlain}{}
  \renewcommand{\predicateBitprecisePathinvariantsResultsPathInvariantsPolicyCPAUnknownOrCategoryErrorInvTriesPlain}{58259\xspace}

  % inv-time-sum
\providecommand{\predicateBitprecisePathinvariantsResultsPathInvariantsPolicyCPAUnknownOrCategoryErrorInvTimeSumPlain}{}
  \renewcommand{\predicateBitprecisePathinvariantsResultsPathInvariantsPolicyCPAUnknownOrCategoryErrorInvTimeSumPlain}{18697.801000000014\xspace}
\providecommand{\predicateBitprecisePathinvariantsResultsPathInvariantsPolicyCPAUnknownOrCategoryErrorInvTimeSumPlainHours}{}
  \renewcommand{\predicateBitprecisePathinvariantsResultsPathInvariantsPolicyCPAUnknownOrCategoryErrorInvTimeSumPlainHours}{5.193833611111115\xspace}

 %% correct-false %%
\providecommand{\predicateBitprecisePathinvariantsResultsPathInvariantsPolicyCPACorrectFalsePlain}{}
  \renewcommand{\predicateBitprecisePathinvariantsResultsPathInvariantsPolicyCPACorrectFalsePlain}{529\xspace}

  % cpu-time-sum
\providecommand{\predicateBitprecisePathinvariantsResultsPathInvariantsPolicyCPACorrectFalseCpuTimeSumPlain}{}
  \renewcommand{\predicateBitprecisePathinvariantsResultsPathInvariantsPolicyCPACorrectFalseCpuTimeSumPlain}{39744.338430942014\xspace}
\providecommand{\predicateBitprecisePathinvariantsResultsPathInvariantsPolicyCPACorrectFalseCpuTimeSumPlainHours}{}
  \renewcommand{\predicateBitprecisePathinvariantsResultsPathInvariantsPolicyCPACorrectFalseCpuTimeSumPlainHours}{11.040094008595004\xspace}

  % cpu-time-avg
\providecommand{\predicateBitprecisePathinvariantsResultsPathInvariantsPolicyCPACorrectFalseCpuTimeAvgPlain}{}
  \renewcommand{\predicateBitprecisePathinvariantsResultsPathInvariantsPolicyCPACorrectFalseCpuTimeAvgPlain}{75.13107453864275\xspace}
\providecommand{\predicateBitprecisePathinvariantsResultsPathInvariantsPolicyCPACorrectFalseCpuTimeAvgPlainHours}{}
  \renewcommand{\predicateBitprecisePathinvariantsResultsPathInvariantsPolicyCPACorrectFalseCpuTimeAvgPlainHours}{0.020869742927400764\xspace}

  % inv-succ
\providecommand{\predicateBitprecisePathinvariantsResultsPathInvariantsPolicyCPACorrectFalseInvSuccPlain}{}
  \renewcommand{\predicateBitprecisePathinvariantsResultsPathInvariantsPolicyCPACorrectFalseInvSuccPlain}{440\xspace}

  % inv-tries
\providecommand{\predicateBitprecisePathinvariantsResultsPathInvariantsPolicyCPACorrectFalseInvTriesPlain}{}
  \renewcommand{\predicateBitprecisePathinvariantsResultsPathInvariantsPolicyCPACorrectFalseInvTriesPlain}{1527\xspace}

  % inv-time-sum
\providecommand{\predicateBitprecisePathinvariantsResultsPathInvariantsPolicyCPACorrectFalseInvTimeSumPlain}{}
  \renewcommand{\predicateBitprecisePathinvariantsResultsPathInvariantsPolicyCPACorrectFalseInvTimeSumPlain}{3416.9859999999962\xspace}
\providecommand{\predicateBitprecisePathinvariantsResultsPathInvariantsPolicyCPACorrectFalseInvTimeSumPlainHours}{}
  \renewcommand{\predicateBitprecisePathinvariantsResultsPathInvariantsPolicyCPACorrectFalseInvTimeSumPlainHours}{0.9491627777777767\xspace}

 %% correct-true %%
\providecommand{\predicateBitprecisePathinvariantsResultsPathInvariantsPolicyCPACorrectTruePlain}{}
  \renewcommand{\predicateBitprecisePathinvariantsResultsPathInvariantsPolicyCPACorrectTruePlain}{1337\xspace}

  % cpu-time-sum
\providecommand{\predicateBitprecisePathinvariantsResultsPathInvariantsPolicyCPACorrectTrueCpuTimeSumPlain}{}
  \renewcommand{\predicateBitprecisePathinvariantsResultsPathInvariantsPolicyCPACorrectTrueCpuTimeSumPlain}{73459.6982908281\xspace}
\providecommand{\predicateBitprecisePathinvariantsResultsPathInvariantsPolicyCPACorrectTrueCpuTimeSumPlainHours}{}
  \renewcommand{\predicateBitprecisePathinvariantsResultsPathInvariantsPolicyCPACorrectTrueCpuTimeSumPlainHours}{20.405471747452253\xspace}

  % cpu-time-avg
\providecommand{\predicateBitprecisePathinvariantsResultsPathInvariantsPolicyCPACorrectTrueCpuTimeAvgPlain}{}
  \renewcommand{\predicateBitprecisePathinvariantsResultsPathInvariantsPolicyCPACorrectTrueCpuTimeAvgPlain}{54.94367860196567\xspace}
\providecommand{\predicateBitprecisePathinvariantsResultsPathInvariantsPolicyCPACorrectTrueCpuTimeAvgPlainHours}{}
  \renewcommand{\predicateBitprecisePathinvariantsResultsPathInvariantsPolicyCPACorrectTrueCpuTimeAvgPlainHours}{0.015262132944990464\xspace}

  % inv-succ
\providecommand{\predicateBitprecisePathinvariantsResultsPathInvariantsPolicyCPACorrectTrueInvSuccPlain}{}
  \renewcommand{\predicateBitprecisePathinvariantsResultsPathInvariantsPolicyCPACorrectTrueInvSuccPlain}{1171\xspace}

  % inv-tries
\providecommand{\predicateBitprecisePathinvariantsResultsPathInvariantsPolicyCPACorrectTrueInvTriesPlain}{}
  \renewcommand{\predicateBitprecisePathinvariantsResultsPathInvariantsPolicyCPACorrectTrueInvTriesPlain}{3073\xspace}

  % inv-time-sum
\providecommand{\predicateBitprecisePathinvariantsResultsPathInvariantsPolicyCPACorrectTrueInvTimeSumPlain}{}
  \renewcommand{\predicateBitprecisePathinvariantsResultsPathInvariantsPolicyCPACorrectTrueInvTimeSumPlain}{10412.833000000002\xspace}
\providecommand{\predicateBitprecisePathinvariantsResultsPathInvariantsPolicyCPACorrectTrueInvTimeSumPlainHours}{}
  \renewcommand{\predicateBitprecisePathinvariantsResultsPathInvariantsPolicyCPACorrectTrueInvTimeSumPlainHours}{2.8924536111111117\xspace}

 %% incorrect-false %%
\providecommand{\predicateBitprecisePathinvariantsResultsPathInvariantsPolicyCPAIncorrectFalsePlain}{}
  \renewcommand{\predicateBitprecisePathinvariantsResultsPathInvariantsPolicyCPAIncorrectFalsePlain}{27\xspace}

  % cpu-time-sum
\providecommand{\predicateBitprecisePathinvariantsResultsPathInvariantsPolicyCPAIncorrectFalseCpuTimeSumPlain}{}
  \renewcommand{\predicateBitprecisePathinvariantsResultsPathInvariantsPolicyCPAIncorrectFalseCpuTimeSumPlain}{1057.8719591509998\xspace}
\providecommand{\predicateBitprecisePathinvariantsResultsPathInvariantsPolicyCPAIncorrectFalseCpuTimeSumPlainHours}{}
  \renewcommand{\predicateBitprecisePathinvariantsResultsPathInvariantsPolicyCPAIncorrectFalseCpuTimeSumPlainHours}{0.29385332198638886\xspace}

  % cpu-time-avg
\providecommand{\predicateBitprecisePathinvariantsResultsPathInvariantsPolicyCPAIncorrectFalseCpuTimeAvgPlain}{}
  \renewcommand{\predicateBitprecisePathinvariantsResultsPathInvariantsPolicyCPAIncorrectFalseCpuTimeAvgPlain}{39.18044293151851\xspace}
\providecommand{\predicateBitprecisePathinvariantsResultsPathInvariantsPolicyCPAIncorrectFalseCpuTimeAvgPlainHours}{}
  \renewcommand{\predicateBitprecisePathinvariantsResultsPathInvariantsPolicyCPAIncorrectFalseCpuTimeAvgPlainHours}{0.010883456369866254\xspace}

  % inv-succ
\providecommand{\predicateBitprecisePathinvariantsResultsPathInvariantsPolicyCPAIncorrectFalseInvSuccPlain}{}
  \renewcommand{\predicateBitprecisePathinvariantsResultsPathInvariantsPolicyCPAIncorrectFalseInvSuccPlain}{3\xspace}

  % inv-tries
\providecommand{\predicateBitprecisePathinvariantsResultsPathInvariantsPolicyCPAIncorrectFalseInvTriesPlain}{}
  \renewcommand{\predicateBitprecisePathinvariantsResultsPathInvariantsPolicyCPAIncorrectFalseInvTriesPlain}{60\xspace}

  % inv-time-sum
\providecommand{\predicateBitprecisePathinvariantsResultsPathInvariantsPolicyCPAIncorrectFalseInvTimeSumPlain}{}
  \renewcommand{\predicateBitprecisePathinvariantsResultsPathInvariantsPolicyCPAIncorrectFalseInvTimeSumPlain}{168.88299999999998\xspace}
\providecommand{\predicateBitprecisePathinvariantsResultsPathInvariantsPolicyCPAIncorrectFalseInvTimeSumPlainHours}{}
  \renewcommand{\predicateBitprecisePathinvariantsResultsPathInvariantsPolicyCPAIncorrectFalseInvTimeSumPlainHours}{0.04691194444444444\xspace}

 %% incorrect-true %%
\providecommand{\predicateBitprecisePathinvariantsResultsPathInvariantsPolicyCPAIncorrectTruePlain}{}
  \renewcommand{\predicateBitprecisePathinvariantsResultsPathInvariantsPolicyCPAIncorrectTruePlain}{0\xspace}

  % cpu-time-sum
\providecommand{\predicateBitprecisePathinvariantsResultsPathInvariantsPolicyCPAIncorrectTrueCpuTimeSumPlain}{}
  \renewcommand{\predicateBitprecisePathinvariantsResultsPathInvariantsPolicyCPAIncorrectTrueCpuTimeSumPlain}{0.0\xspace}
\providecommand{\predicateBitprecisePathinvariantsResultsPathInvariantsPolicyCPAIncorrectTrueCpuTimeSumPlainHours}{}
  \renewcommand{\predicateBitprecisePathinvariantsResultsPathInvariantsPolicyCPAIncorrectTrueCpuTimeSumPlainHours}{0.0\xspace}

  % cpu-time-avg
\providecommand{\predicateBitprecisePathinvariantsResultsPathInvariantsPolicyCPAIncorrectTrueCpuTimeAvgPlain}{}
  \renewcommand{\predicateBitprecisePathinvariantsResultsPathInvariantsPolicyCPAIncorrectTrueCpuTimeAvgPlain}{NaN\xspace}
\providecommand{\predicateBitprecisePathinvariantsResultsPathInvariantsPolicyCPAIncorrectTrueCpuTimeAvgPlainHours}{}
  \renewcommand{\predicateBitprecisePathinvariantsResultsPathInvariantsPolicyCPAIncorrectTrueCpuTimeAvgPlainHours}{NaN\xspace}

  % inv-succ
\providecommand{\predicateBitprecisePathinvariantsResultsPathInvariantsPolicyCPAIncorrectTrueInvSuccPlain}{}
  \renewcommand{\predicateBitprecisePathinvariantsResultsPathInvariantsPolicyCPAIncorrectTrueInvSuccPlain}{0\xspace}

  % inv-tries
\providecommand{\predicateBitprecisePathinvariantsResultsPathInvariantsPolicyCPAIncorrectTrueInvTriesPlain}{}
  \renewcommand{\predicateBitprecisePathinvariantsResultsPathInvariantsPolicyCPAIncorrectTrueInvTriesPlain}{0\xspace}

  % inv-time-sum
\providecommand{\predicateBitprecisePathinvariantsResultsPathInvariantsPolicyCPAIncorrectTrueInvTimeSumPlain}{}
  \renewcommand{\predicateBitprecisePathinvariantsResultsPathInvariantsPolicyCPAIncorrectTrueInvTimeSumPlain}{0.0\xspace}
\providecommand{\predicateBitprecisePathinvariantsResultsPathInvariantsPolicyCPAIncorrectTrueInvTimeSumPlainHours}{}
  \renewcommand{\predicateBitprecisePathinvariantsResultsPathInvariantsPolicyCPAIncorrectTrueInvTimeSumPlainHours}{0.0\xspace}

 %% all %%
\providecommand{\predicateBitprecisePathinvariantsResultsPathInvariantsPolicyCPAAllPlain}{}
  \renewcommand{\predicateBitprecisePathinvariantsResultsPathInvariantsPolicyCPAAllPlain}{3488\xspace}

  % cpu-time-sum
\providecommand{\predicateBitprecisePathinvariantsResultsPathInvariantsPolicyCPAAllCpuTimeSumPlain}{}
  \renewcommand{\predicateBitprecisePathinvariantsResultsPathInvariantsPolicyCPAAllCpuTimeSumPlain}{579906.5433774246\xspace}
\providecommand{\predicateBitprecisePathinvariantsResultsPathInvariantsPolicyCPAAllCpuTimeSumPlainHours}{}
  \renewcommand{\predicateBitprecisePathinvariantsResultsPathInvariantsPolicyCPAAllCpuTimeSumPlainHours}{161.08515093817348\xspace}

  % cpu-time-avg
\providecommand{\predicateBitprecisePathinvariantsResultsPathInvariantsPolicyCPAAllCpuTimeAvgPlain}{}
  \renewcommand{\predicateBitprecisePathinvariantsResultsPathInvariantsPolicyCPAAllCpuTimeAvgPlain}{166.25760991325245\xspace}
\providecommand{\predicateBitprecisePathinvariantsResultsPathInvariantsPolicyCPAAllCpuTimeAvgPlainHours}{}
  \renewcommand{\predicateBitprecisePathinvariantsResultsPathInvariantsPolicyCPAAllCpuTimeAvgPlainHours}{0.046182669420347905\xspace}

  % inv-succ
\providecommand{\predicateBitprecisePathinvariantsResultsPathInvariantsPolicyCPAAllInvSuccPlain}{}
  \renewcommand{\predicateBitprecisePathinvariantsResultsPathInvariantsPolicyCPAAllInvSuccPlain}{53504\xspace}

  % inv-tries
\providecommand{\predicateBitprecisePathinvariantsResultsPathInvariantsPolicyCPAAllInvTriesPlain}{}
  \renewcommand{\predicateBitprecisePathinvariantsResultsPathInvariantsPolicyCPAAllInvTriesPlain}{62919\xspace}

  % inv-time-sum
\providecommand{\predicateBitprecisePathinvariantsResultsPathInvariantsPolicyCPAAllInvTimeSumPlain}{}
  \renewcommand{\predicateBitprecisePathinvariantsResultsPathInvariantsPolicyCPAAllInvTimeSumPlain}{32696.50300000008\xspace}
\providecommand{\predicateBitprecisePathinvariantsResultsPathInvariantsPolicyCPAAllInvTimeSumPlainHours}{}
  \renewcommand{\predicateBitprecisePathinvariantsResultsPathInvariantsPolicyCPAAllInvTimeSumPlainHours}{9.082361944444466\xspace}

 %% equal-only %%
\providecommand{\predicateBitprecisePathinvariantsResultsPathInvariantsPolicyCPAEqualOnlyPlain}{}
  \renewcommand{\predicateBitprecisePathinvariantsResultsPathInvariantsPolicyCPAEqualOnlyPlain}{1754\xspace}

  % cpu-time-sum
\providecommand{\predicateBitprecisePathinvariantsResultsPathInvariantsPolicyCPAEqualOnlyCpuTimeSumPlain}{}
  \renewcommand{\predicateBitprecisePathinvariantsResultsPathInvariantsPolicyCPAEqualOnlyCpuTimeSumPlain}{92467.98912765415\xspace}
\providecommand{\predicateBitprecisePathinvariantsResultsPathInvariantsPolicyCPAEqualOnlyCpuTimeSumPlainHours}{}
  \renewcommand{\predicateBitprecisePathinvariantsResultsPathInvariantsPolicyCPAEqualOnlyCpuTimeSumPlainHours}{25.685552535459486\xspace}

  % cpu-time-avg
\providecommand{\predicateBitprecisePathinvariantsResultsPathInvariantsPolicyCPAEqualOnlyCpuTimeAvgPlain}{}
  \renewcommand{\predicateBitprecisePathinvariantsResultsPathInvariantsPolicyCPAEqualOnlyCpuTimeAvgPlain}{52.7183518401677\xspace}
\providecommand{\predicateBitprecisePathinvariantsResultsPathInvariantsPolicyCPAEqualOnlyCpuTimeAvgPlainHours}{}
  \renewcommand{\predicateBitprecisePathinvariantsResultsPathInvariantsPolicyCPAEqualOnlyCpuTimeAvgPlainHours}{0.014643986622268805\xspace}

  % inv-succ
\providecommand{\predicateBitprecisePathinvariantsResultsPathInvariantsPolicyCPAEqualOnlyInvSuccPlain}{}
  \renewcommand{\predicateBitprecisePathinvariantsResultsPathInvariantsPolicyCPAEqualOnlyInvSuccPlain}{1498\xspace}

  % inv-tries
\providecommand{\predicateBitprecisePathinvariantsResultsPathInvariantsPolicyCPAEqualOnlyInvTriesPlain}{}
  \renewcommand{\predicateBitprecisePathinvariantsResultsPathInvariantsPolicyCPAEqualOnlyInvTriesPlain}{3950\xspace}

  % inv-time-sum
\providecommand{\predicateBitprecisePathinvariantsResultsPathInvariantsPolicyCPAEqualOnlyInvTimeSumPlain}{}
  \renewcommand{\predicateBitprecisePathinvariantsResultsPathInvariantsPolicyCPAEqualOnlyInvTimeSumPlain}{12683.69300000001\xspace}
\providecommand{\predicateBitprecisePathinvariantsResultsPathInvariantsPolicyCPAEqualOnlyInvTimeSumPlainHours}{}
  \renewcommand{\predicateBitprecisePathinvariantsResultsPathInvariantsPolicyCPAEqualOnlyInvTimeSumPlainHours}{3.5232480555555585\xspace}







\begin{table}
 \caption{Details on analyses using lightweight heuristics for generating auxiliary invariants and their baseline}
 \label{table:lightweight_overall}
\begin{adjustbox}{max width=\textwidth}
  \begin{tabular}{l
                  S[table-format=4.0, round-mode=off, round-precision=3]
                  S[table-format=3.0, round-mode=off, round-precision=3]
                  S[table-format=2.0, round-mode=off, round-precision=3]
                  S[table-format=1.3, round-mode=figures, round-precision=3]
                  S[table-format=4.0, round-mode=off, round-precision=3]
                  S[table-format=4.0, round-mode=off, round-precision=3]
                  S[table-format=3.0, round-mode=figures, round-precision=3]
                  S[table-format=2.1, round-mode=figures, round-precision=3]
                  S[table-format=2.1, round-mode=figures, round-precision=3]}
\toprule
 & \multicolumn{2}{c}{\textbf{correct}} & \multicolumn{1}{c}{\textbf{wrong}} & \multicolumn{3}{c}{\textbf{Invariants (equal)}} & \multicolumn{3}{c}{\textbf{CPU time (h)}}\\
 & \multicolumn{1}{c}{proof} & \multicolumn{1}{c}{alarm} & \multicolumn{1}{c}{alarm} & \multicolumn{1}{c}{time (h)} & \multicolumn{1}{c}{attempts} & \multicolumn{1}{c}{succ} & \multicolumn{1}{c}{all} & \multicolumn{1}{c}{correct} & \multicolumn{1}{c}{equal} \\
\cmidrule(lr){2-3}\cmidrule(lr){4-4}\cmidrule(lr){5-7}\cmidrule(lr){8-10}


\textbf{base300} & \predicateBaseResultsPredBitvectorsCorrectTruePlain & \predicateBaseResultsPredBitvectorsCorrectFalsePlain & \predicateBaseResultsPredBitvectorsIncorrectFalsePlain &  &  &  & \predicateBaseResultsPredBitvectorsAllCpuTimeSumPlainHours & \predicateBaseResultsPredBitvectorsCorrectCpuTimeSumPlainHours & \predicateBaseResultsPredBitvectorsEqualOnlyCpuTimeSumPlainHours \\
\textbf{weakening-path} & \predicateBitpreciseWeakeningResultsWeakeningPathCorrectTruePlain & \predicateBitpreciseWeakeningResultsWeakeningPathCorrectFalsePlain & \predicateBitpreciseWeakeningResultsWeakeningPathIncorrectFalsePlain & \predicateBitpreciseWeakeningResultsWeakeningPathEqualOnlyInvTimeSumPlainHours & \predicateBitpreciseWeakeningResultsWeakeningPathEqualOnlyInvTriesPlain & \predicateBitpreciseWeakeningResultsWeakeningPathEqualOnlyInvSuccPlain & \predicateBitpreciseWeakeningResultsWeakeningPathAllCpuTimeSumPlainHours & \predicateBitpreciseWeakeningResultsWeakeningPathCorrectCpuTimeSumPlainHours & \predicateBitpreciseWeakeningResultsWeakeningPathEqualOnlyCpuTimeSumPlainHours \\
\textbf{path-policy} & \predicateBitprecisePathinvariantsResultsPathInvariantsPolicyCPACorrectTruePlain & \predicateBitprecisePathinvariantsResultsPathInvariantsPolicyCPACorrectFalsePlain & \predicateBitprecisePathinvariantsResultsPathInvariantsPolicyCPAIncorrectFalsePlain & \predicateBitprecisePathinvariantsResultsPathInvariantsPolicyCPAEqualOnlyInvTimeSumPlainHours & \predicateBitprecisePathinvariantsResultsPathInvariantsPolicyCPAEqualOnlyInvTriesPlain & \predicateBitprecisePathinvariantsResultsPathInvariantsPolicyCPAEqualOnlyInvSuccPlain & \predicateBitprecisePathinvariantsResultsPathInvariantsPolicyCPAAllCpuTimeSumPlainHours & \predicateBitprecisePathinvariantsResultsPathInvariantsPolicyCPACorrectCpuTimeSumPlainHours & \predicateBitprecisePathinvariantsResultsPathInvariantsPolicyCPAEqualOnlyCpuTimeSumPlainHours \\
\textbf{int-check-prec} & \predicateBitpreciseInterpolKindResultsRFInterpolPrecCorrectTruePlain & \predicateBitpreciseInterpolKindResultsRFInterpolPrecCorrectFalsePlain & \predicateBitpreciseInterpolKindResultsRFInterpolPrecIncorrectFalsePlain & \predicateBitpreciseInterpolKindResultsRFInterpolPrecEqualOnlyInvTimeSumPlainHours & \predicateBitpreciseInterpolKindResultsRFInterpolPrecEqualOnlyInvTriesPlain & \predicateBitpreciseInterpolKindResultsRFInterpolPrecEqualOnlyInvSuccPlain & \predicateBitpreciseInterpolKindResultsRFInterpolPrecAllCpuTimeSumPlainHours & \predicateBitpreciseInterpolKindResultsRFInterpolPrecCorrectCpuTimeSumPlainHours & \predicateBitpreciseInterpolKindResultsRFInterpolPrecEqualOnlyCpuTimeSumPlainHours \\
\textbf{conj-check-path} & \predicateBitpreciseCNFKINDResultsCNFKINDPathCorrectTruePlain & \predicateBitpreciseCNFKINDResultsCNFKINDPathCorrectFalsePlain & \predicateBitpreciseCNFKINDResultsCNFKINDPathIncorrectFalsePlain & \predicateBitpreciseCNFKINDResultsCNFKINDPathEqualOnlyInvTimeSumPlainHours & \predicateBitpreciseCNFKINDResultsCNFKINDPathEqualOnlyInvTriesPlain & \predicateBitpreciseCNFKINDResultsCNFKINDPathEqualOnlyInvSuccPlain & \predicateBitpreciseCNFKINDResultsCNFKINDPathAllCpuTimeSumPlainHours & \predicateBitpreciseCNFKINDResultsCNFKINDPathCorrectCpuTimeSumPlainHours & \predicateBitpreciseCNFKINDResultsCNFKINDPathEqualOnlyCpuTimeSumPlainHours \\




\bottomrule
 \end{tabular}
 \end{adjustbox}
\end{table}


In \autoref{table:lightweight_overall} the best configuration for each of the heuristics can be seen\,\sidenote{The values in the \emph{Invariants} column are extracted from
logfiles, so they are only available for the verification tasks not running into timeouts or experiencing other errors. Additionally they are measured by \CPAchecker{} itself,
so they are not as reliable as the CPU time which is measured by BenchExec.}. The columns show the number of correctly analyzed programs divided into found proofs and alarms.
Additionally the wrong results are displayed. As only safe programs were erroneously treated as unsafe programs the other column was left out. The statistics about
invariants show the \textbf{time} and the amount of invariant generation \textbf{attempts} as well as the amount of successful invariant generations (\textbf{succ}) for all equal
and correct verification runs. The last part of the table are the statistics about the CPU time.

The table shows that all lightweight invariant-generation approaches take too much
time away from the main analysis, and furthermore, in this time not enough, or not the required invariants are found. The approaches of weakening path formulas or checking the conjuncts of
path formulas transformed into a \ac{CNF} were used about \num{6000} times per configuration over all correctly analyzed programs, but not a single valid invariant could be generated. This
can also be seen from the CPU time taken by the equal verification runs: The time for analyzing these programs increased approximately by the time needed for the invariant generation
tries. Due to the higher time consumption also fewer programs could be analyzed in time.

Path invariants and checking interpolants on invariance lead to even worse results. While these configurations are able to generate invariants --- and both add the invariants to the precision ---
they are taking even more time and therefore the performance suffers. For \textbf{path-policy} and \textbf{base300} the difference in the CPU time is over \SI{4}{\hour} higher than the time for the invariant
generation. This leads to the conclusion that the invariants have a negative impact on the performance, which could be the case for example, by adding predicates
to the precision which force that loops have to be unrolled, an issue we wanted to overcome with these approaches\,\sidenote{\Eg, the predicate \mbox{\texttt{i = i + 1}} with \texttt{i} being a loop counter could cause this issue. An example where such formulas are found as interpolant, and the invariant is helpful follows in the section about the results with path invariants.}.
More insights into the four heuristics are given in the following sections.

\subsubsection*{Weakening of Path Formulas}\label{title:evalWeakening}
Weakening path formulas up to the point where the remaining formula is an invariant did not work as expected. As can be seen in \autoref{table:lightweight_overall}, this approach did not find any 
invariants. Without invariants all configurations we have tested are the same, as they only differ in where the invariants would be added. So besides minor changes to the results, which are caused by tasks
that can be analyzed in approximately \SI{300}{\second} and therefore time out in some configurations but are successfully analyzed in other configurations, there is no difference. Upon further investigation 
we found several bugs in the usage and implementation of the reduced \ac{CNF} conversion, they are fixed on later revisions of \CPAchecker{} than the evaluation was made on. An additional limitation to 
solvers that support quantifiers was necessary. Quantification is used for removing variables not having the most up to date SSA index in the process of converting a path formula to a reduced \ac{CNF}. The 
combination of bitvectors and quantifiers is only possible with the \ac{SMT} solver \ZT{}. Unfortunately the integration of this solver in \CPAchecker{} is not perfect, and the results are not comparable to 
analyses with \MathSAT{}.


\begin{table}
\centering
\caption{Details on analysis using weakening or checking path formula conjuncts with \ZT{} instead of \MathSAT{}}
\label{table:heurZ3}
\begin{adjustbox}{max width=.8\textwidth}
\begin{tabular}{l
                  S[table-format=4.0, round-mode=off, round-precision=3]
                  S[table-format=3.0, round-mode=off, round-precision=3]
                  S[table-format=1.0, round-mode=off, round-precision=3]
                  S[table-format=2.0, round-mode=off, round-precision=3]
                  S[table-format=4.0, round-mode=off, round-precision=3]
                  S[table-format=4.0, round-mode=off, round-precision=3]}
\toprule
 & \multicolumn{2}{c}{\textbf{correct}} & \multicolumn{2}{c}{\textbf{wrong}} &  \multicolumn{2}{c}{\textbf{Invariants (all)}}\\              
 & \multicolumn{1}{c}{proof} & \multicolumn{1}{c}{alarm} & \multicolumn{1}{c}{proof} & \multicolumn{1}{c}{alarm} & \multicolumn{1}{c}{attempts} & \multicolumn{1}{c}{succ} \\
\cmidrule(lr){2-3}\cmidrule(lr){4-5}\cmidrule(lr){6-7}

\textbf{z3-base300} & 1155 & 297 & 0 & 21 & & \\
\textbf{z3-weakening-abs} & 1047 & 249 & 5 & 25 & 7802 & 5523 \\
\textbf{z3-weakening-prec} & 965 & 250 & 0 & 20 & 8442 & 6026 \\
\textbf{z3-int-check-abs} & 1111 & 254 & 0 & 18 & 6976 & 1648 \\
\textbf{z3-int-check-prec} & 1093 & 249 & 0 & 21 & 7117 & 1792 \\
\bottomrule
\end{tabular}
\end{adjustbox}
\end{table}

\autoref{table:heurZ3} shows some experimental results made with revision 23206~(\emph{trunk}). Instead of \MathSAT{}, \ZT{} was used. This leads to a drastic performance decrease. By comparing 
\textbf{base300} with \textbf{z3-base300} we can see that 492 fewer tasks can be verified successfully. When using weakening of path formulas for invariant generation, the number of successfully analyzed 
tasks decreases even further, although the ratio of invariant generation attempts to successful invariant generations is higher than for the other lightweight heuristics.

\subsubsection*{Checking Conjuncts of Path Formulas on Invariance}
This approach transforms a path formula to a reduced conjunctive normal form and checks the conjuncts on invariance with $k$-induction. As explained in \autoref{title:evalWeakening} the conversion of
formulas to reduced conjunctive normal forms does not work as expected with \MathSAT{}. Some experimental results with \ZT{} can be found in \autoref{table:heurZ3}. While this approach is strictly better
than weakening path formulas, it is still not able to correctly analyze as many tasks as \textbf{z3-base300} does.

The main difference of this approach and the weakening of path formulas is how the conjuncts are checked on invariance. Here we use $k$-induction as a separate analysis for finding 1-
inductive invariants. In contrast, weakening uses counterexamples to remove conjuncts which cannot be part of the final invariant~\cite{Karpenkov:Slicing}. Both approaches can currently
only be used with \ZT{} and therefore suffer from a worse performance than analyses using \MathSAT{}. Additionally it seems that generating invariants with these approaches has no beneficial influence on the 
analyses. What can be seen, is that \textbf{-abs} configurations perform better than \textbf{-prec} configurations, which is caused by the computational overhead of adding invariants to the precision. This 
observation can be made for all invariant generation approaches we have evaluated.

\subsubsection*{Checking Interpolants on Invariance}
%%% predicate_bitprecise_interpol_kind.2016-09-04_2044.results.RF_interpol-prec %%%
 %% correct %%
\providecommand{\predicateBitpreciseInterpolKindResultsRFInterpolPrecCorrectPlain}{}
  \renewcommand{\predicateBitpreciseInterpolKindResultsRFInterpolPrecCorrectPlain}{1817\xspace}

  % cpu-time-sum
\providecommand{\predicateBitpreciseInterpolKindResultsRFInterpolPrecCorrectCpuTimeSumPlain}{}
  \renewcommand{\predicateBitpreciseInterpolKindResultsRFInterpolPrecCorrectCpuTimeSumPlain}{105972.85310964301\xspace}
\providecommand{\predicateBitpreciseInterpolKindResultsRFInterpolPrecCorrectCpuTimeSumPlainHours}{}
  \renewcommand{\predicateBitpreciseInterpolKindResultsRFInterpolPrecCorrectCpuTimeSumPlainHours}{29.436903641567504\xspace}

  % cpu-time-avg
\providecommand{\predicateBitpreciseInterpolKindResultsRFInterpolPrecCorrectCpuTimeAvgPlain}{}
  \renewcommand{\predicateBitpreciseInterpolKindResultsRFInterpolPrecCorrectCpuTimeAvgPlain}{58.32297914674904\xspace}
\providecommand{\predicateBitpreciseInterpolKindResultsRFInterpolPrecCorrectCpuTimeAvgPlainHours}{}
  \renewcommand{\predicateBitpreciseInterpolKindResultsRFInterpolPrecCorrectCpuTimeAvgPlainHours}{0.016200827540763622\xspace}

  % inv-succ
\providecommand{\predicateBitpreciseInterpolKindResultsRFInterpolPrecCorrectInvSuccPlain}{}
  \renewcommand{\predicateBitpreciseInterpolKindResultsRFInterpolPrecCorrectInvSuccPlain}{1272\xspace}

  % inv-tries
\providecommand{\predicateBitpreciseInterpolKindResultsRFInterpolPrecCorrectInvTriesPlain}{}
  \renewcommand{\predicateBitpreciseInterpolKindResultsRFInterpolPrecCorrectInvTriesPlain}{6455\xspace}

  % inv-time-sum
\providecommand{\predicateBitpreciseInterpolKindResultsRFInterpolPrecCorrectInvTimeSumPlain}{}
  \renewcommand{\predicateBitpreciseInterpolKindResultsRFInterpolPrecCorrectInvTimeSumPlain}{25538.078999999994\xspace}
\providecommand{\predicateBitpreciseInterpolKindResultsRFInterpolPrecCorrectInvTimeSumPlainHours}{}
  \renewcommand{\predicateBitpreciseInterpolKindResultsRFInterpolPrecCorrectInvTimeSumPlainHours}{7.093910833333331\xspace}

 %% incorrect %%
\providecommand{\predicateBitpreciseInterpolKindResultsRFInterpolPrecIncorrectPlain}{}
  \renewcommand{\predicateBitpreciseInterpolKindResultsRFInterpolPrecIncorrectPlain}{23\xspace}

  % cpu-time-sum
\providecommand{\predicateBitpreciseInterpolKindResultsRFInterpolPrecIncorrectCpuTimeSumPlain}{}
  \renewcommand{\predicateBitpreciseInterpolKindResultsRFInterpolPrecIncorrectCpuTimeSumPlain}{1057.0374651350003\xspace}
\providecommand{\predicateBitpreciseInterpolKindResultsRFInterpolPrecIncorrectCpuTimeSumPlainHours}{}
  \renewcommand{\predicateBitpreciseInterpolKindResultsRFInterpolPrecIncorrectCpuTimeSumPlainHours}{0.29362151809305564\xspace}

  % cpu-time-avg
\providecommand{\predicateBitpreciseInterpolKindResultsRFInterpolPrecIncorrectCpuTimeAvgPlain}{}
  \renewcommand{\predicateBitpreciseInterpolKindResultsRFInterpolPrecIncorrectCpuTimeAvgPlain}{45.95815065804349\xspace}
\providecommand{\predicateBitpreciseInterpolKindResultsRFInterpolPrecIncorrectCpuTimeAvgPlainHours}{}
  \renewcommand{\predicateBitpreciseInterpolKindResultsRFInterpolPrecIncorrectCpuTimeAvgPlainHours}{0.012766152960567637\xspace}

  % inv-succ
\providecommand{\predicateBitpreciseInterpolKindResultsRFInterpolPrecIncorrectInvSuccPlain}{}
  \renewcommand{\predicateBitpreciseInterpolKindResultsRFInterpolPrecIncorrectInvSuccPlain}{0\xspace}

  % inv-tries
\providecommand{\predicateBitpreciseInterpolKindResultsRFInterpolPrecIncorrectInvTriesPlain}{}
  \renewcommand{\predicateBitpreciseInterpolKindResultsRFInterpolPrecIncorrectInvTriesPlain}{70\xspace}

  % inv-time-sum
\providecommand{\predicateBitpreciseInterpolKindResultsRFInterpolPrecIncorrectInvTimeSumPlain}{}
  \renewcommand{\predicateBitpreciseInterpolKindResultsRFInterpolPrecIncorrectInvTimeSumPlain}{496.91499999999996\xspace}
\providecommand{\predicateBitpreciseInterpolKindResultsRFInterpolPrecIncorrectInvTimeSumPlainHours}{}
  \renewcommand{\predicateBitpreciseInterpolKindResultsRFInterpolPrecIncorrectInvTimeSumPlainHours}{0.13803194444444444\xspace}

 %% timeout %%
\providecommand{\predicateBitpreciseInterpolKindResultsRFInterpolPrecTimeoutPlain}{}
  \renewcommand{\predicateBitpreciseInterpolKindResultsRFInterpolPrecTimeoutPlain}{1565\xspace}

  % cpu-time-sum
\providecommand{\predicateBitpreciseInterpolKindResultsRFInterpolPrecTimeoutCpuTimeSumPlain}{}
  \renewcommand{\predicateBitpreciseInterpolKindResultsRFInterpolPrecTimeoutCpuTimeSumPlain}{479053.91880363267\xspace}
\providecommand{\predicateBitpreciseInterpolKindResultsRFInterpolPrecTimeoutCpuTimeSumPlainHours}{}
  \renewcommand{\predicateBitpreciseInterpolKindResultsRFInterpolPrecTimeoutCpuTimeSumPlainHours}{133.07053300100907\xspace}

  % cpu-time-avg
\providecommand{\predicateBitpreciseInterpolKindResultsRFInterpolPrecTimeoutCpuTimeAvgPlain}{}
  \renewcommand{\predicateBitpreciseInterpolKindResultsRFInterpolPrecTimeoutCpuTimeAvgPlain}{306.1047404496055\xspace}
\providecommand{\predicateBitpreciseInterpolKindResultsRFInterpolPrecTimeoutCpuTimeAvgPlainHours}{}
  \renewcommand{\predicateBitpreciseInterpolKindResultsRFInterpolPrecTimeoutCpuTimeAvgPlainHours}{0.08502909456933487\xspace}

  % inv-succ
\providecommand{\predicateBitpreciseInterpolKindResultsRFInterpolPrecTimeoutInvSuccPlain}{}
  \renewcommand{\predicateBitpreciseInterpolKindResultsRFInterpolPrecTimeoutInvSuccPlain}{304\xspace}

  % inv-tries
\providecommand{\predicateBitpreciseInterpolKindResultsRFInterpolPrecTimeoutInvTriesPlain}{}
  \renewcommand{\predicateBitpreciseInterpolKindResultsRFInterpolPrecTimeoutInvTriesPlain}{7580\xspace}

  % inv-time-sum
\providecommand{\predicateBitpreciseInterpolKindResultsRFInterpolPrecTimeoutInvTimeSumPlain}{}
  \renewcommand{\predicateBitpreciseInterpolKindResultsRFInterpolPrecTimeoutInvTimeSumPlain}{159550.94300000003\xspace}
\providecommand{\predicateBitpreciseInterpolKindResultsRFInterpolPrecTimeoutInvTimeSumPlainHours}{}
  \renewcommand{\predicateBitpreciseInterpolKindResultsRFInterpolPrecTimeoutInvTimeSumPlainHours}{44.319706388888896\xspace}

 %% unknown-or-category-error %%
\providecommand{\predicateBitpreciseInterpolKindResultsRFInterpolPrecUnknownOrCategoryErrorPlain}{}
  \renewcommand{\predicateBitpreciseInterpolKindResultsRFInterpolPrecUnknownOrCategoryErrorPlain}{1648\xspace}

  % cpu-time-sum
\providecommand{\predicateBitpreciseInterpolKindResultsRFInterpolPrecUnknownOrCategoryErrorCpuTimeSumPlain}{}
  \renewcommand{\predicateBitpreciseInterpolKindResultsRFInterpolPrecUnknownOrCategoryErrorCpuTimeSumPlain}{487813.94804686395\xspace}
\providecommand{\predicateBitpreciseInterpolKindResultsRFInterpolPrecUnknownOrCategoryErrorCpuTimeSumPlainHours}{}
  \renewcommand{\predicateBitpreciseInterpolKindResultsRFInterpolPrecUnknownOrCategoryErrorCpuTimeSumPlainHours}{135.50387445746222\xspace}

  % cpu-time-avg
\providecommand{\predicateBitpreciseInterpolKindResultsRFInterpolPrecUnknownOrCategoryErrorCpuTimeAvgPlain}{}
  \renewcommand{\predicateBitpreciseInterpolKindResultsRFInterpolPrecUnknownOrCategoryErrorCpuTimeAvgPlain}{296.0036092517378\xspace}
\providecommand{\predicateBitpreciseInterpolKindResultsRFInterpolPrecUnknownOrCategoryErrorCpuTimeAvgPlainHours}{}
  \renewcommand{\predicateBitpreciseInterpolKindResultsRFInterpolPrecUnknownOrCategoryErrorCpuTimeAvgPlainHours}{0.0822232247921494\xspace}

  % inv-succ
\providecommand{\predicateBitpreciseInterpolKindResultsRFInterpolPrecUnknownOrCategoryErrorInvSuccPlain}{}
  \renewcommand{\predicateBitpreciseInterpolKindResultsRFInterpolPrecUnknownOrCategoryErrorInvSuccPlain}{308\xspace}

  % inv-tries
\providecommand{\predicateBitpreciseInterpolKindResultsRFInterpolPrecUnknownOrCategoryErrorInvTriesPlain}{}
  \renewcommand{\predicateBitpreciseInterpolKindResultsRFInterpolPrecUnknownOrCategoryErrorInvTriesPlain}{7811\xspace}

  % inv-time-sum
\providecommand{\predicateBitpreciseInterpolKindResultsRFInterpolPrecUnknownOrCategoryErrorInvTimeSumPlain}{}
  \renewcommand{\predicateBitpreciseInterpolKindResultsRFInterpolPrecUnknownOrCategoryErrorInvTimeSumPlain}{162211.13099999996\xspace}
\providecommand{\predicateBitpreciseInterpolKindResultsRFInterpolPrecUnknownOrCategoryErrorInvTimeSumPlainHours}{}
  \renewcommand{\predicateBitpreciseInterpolKindResultsRFInterpolPrecUnknownOrCategoryErrorInvTimeSumPlainHours}{45.05864749999999\xspace}

 %% correct-false %%
\providecommand{\predicateBitpreciseInterpolKindResultsRFInterpolPrecCorrectFalsePlain}{}
  \renewcommand{\predicateBitpreciseInterpolKindResultsRFInterpolPrecCorrectFalsePlain}{483\xspace}

  % cpu-time-sum
\providecommand{\predicateBitpreciseInterpolKindResultsRFInterpolPrecCorrectFalseCpuTimeSumPlain}{}
  \renewcommand{\predicateBitpreciseInterpolKindResultsRFInterpolPrecCorrectFalseCpuTimeSumPlain}{32002.18380171998\xspace}
\providecommand{\predicateBitpreciseInterpolKindResultsRFInterpolPrecCorrectFalseCpuTimeSumPlainHours}{}
  \renewcommand{\predicateBitpreciseInterpolKindResultsRFInterpolPrecCorrectFalseCpuTimeSumPlainHours}{8.889495500477773\xspace}

  % cpu-time-avg
\providecommand{\predicateBitpreciseInterpolKindResultsRFInterpolPrecCorrectFalseCpuTimeAvgPlain}{}
  \renewcommand{\predicateBitpreciseInterpolKindResultsRFInterpolPrecCorrectFalseCpuTimeAvgPlain}{66.25710932033122\xspace}
\providecommand{\predicateBitpreciseInterpolKindResultsRFInterpolPrecCorrectFalseCpuTimeAvgPlainHours}{}
  \renewcommand{\predicateBitpreciseInterpolKindResultsRFInterpolPrecCorrectFalseCpuTimeAvgPlainHours}{0.018404752588980895\xspace}

  % inv-succ
\providecommand{\predicateBitpreciseInterpolKindResultsRFInterpolPrecCorrectFalseInvSuccPlain}{}
  \renewcommand{\predicateBitpreciseInterpolKindResultsRFInterpolPrecCorrectFalseInvSuccPlain}{128\xspace}

  % inv-tries
\providecommand{\predicateBitpreciseInterpolKindResultsRFInterpolPrecCorrectFalseInvTriesPlain}{}
  \renewcommand{\predicateBitpreciseInterpolKindResultsRFInterpolPrecCorrectFalseInvTriesPlain}{2307\xspace}

  % inv-time-sum
\providecommand{\predicateBitpreciseInterpolKindResultsRFInterpolPrecCorrectFalseInvTimeSumPlain}{}
  \renewcommand{\predicateBitpreciseInterpolKindResultsRFInterpolPrecCorrectFalseInvTimeSumPlain}{7289.994999999995\xspace}
\providecommand{\predicateBitpreciseInterpolKindResultsRFInterpolPrecCorrectFalseInvTimeSumPlainHours}{}
  \renewcommand{\predicateBitpreciseInterpolKindResultsRFInterpolPrecCorrectFalseInvTimeSumPlainHours}{2.02499861111111\xspace}

 %% correct-true %%
\providecommand{\predicateBitpreciseInterpolKindResultsRFInterpolPrecCorrectTruePlain}{}
  \renewcommand{\predicateBitpreciseInterpolKindResultsRFInterpolPrecCorrectTruePlain}{1334\xspace}

  % cpu-time-sum
\providecommand{\predicateBitpreciseInterpolKindResultsRFInterpolPrecCorrectTrueCpuTimeSumPlain}{}
  \renewcommand{\predicateBitpreciseInterpolKindResultsRFInterpolPrecCorrectTrueCpuTimeSumPlain}{73970.66930792292\xspace}
\providecommand{\predicateBitpreciseInterpolKindResultsRFInterpolPrecCorrectTrueCpuTimeSumPlainHours}{}
  \renewcommand{\predicateBitpreciseInterpolKindResultsRFInterpolPrecCorrectTrueCpuTimeSumPlainHours}{20.547408141089697\xspace}

  % cpu-time-avg
\providecommand{\predicateBitpreciseInterpolKindResultsRFInterpolPrecCorrectTrueCpuTimeAvgPlain}{}
  \renewcommand{\predicateBitpreciseInterpolKindResultsRFInterpolPrecCorrectTrueCpuTimeAvgPlain}{55.45027684252093\xspace}
\providecommand{\predicateBitpreciseInterpolKindResultsRFInterpolPrecCorrectTrueCpuTimeAvgPlainHours}{}
  \renewcommand{\predicateBitpreciseInterpolKindResultsRFInterpolPrecCorrectTrueCpuTimeAvgPlainHours}{0.015402854678478036\xspace}

  % inv-succ
\providecommand{\predicateBitpreciseInterpolKindResultsRFInterpolPrecCorrectTrueInvSuccPlain}{}
  \renewcommand{\predicateBitpreciseInterpolKindResultsRFInterpolPrecCorrectTrueInvSuccPlain}{1144\xspace}

  % inv-tries
\providecommand{\predicateBitpreciseInterpolKindResultsRFInterpolPrecCorrectTrueInvTriesPlain}{}
  \renewcommand{\predicateBitpreciseInterpolKindResultsRFInterpolPrecCorrectTrueInvTriesPlain}{4148\xspace}

  % inv-time-sum
\providecommand{\predicateBitpreciseInterpolKindResultsRFInterpolPrecCorrectTrueInvTimeSumPlain}{}
  \renewcommand{\predicateBitpreciseInterpolKindResultsRFInterpolPrecCorrectTrueInvTimeSumPlain}{18248.083999999966\xspace}
\providecommand{\predicateBitpreciseInterpolKindResultsRFInterpolPrecCorrectTrueInvTimeSumPlainHours}{}
  \renewcommand{\predicateBitpreciseInterpolKindResultsRFInterpolPrecCorrectTrueInvTimeSumPlainHours}{5.068912222222213\xspace}

 %% incorrect-false %%
\providecommand{\predicateBitpreciseInterpolKindResultsRFInterpolPrecIncorrectFalsePlain}{}
  \renewcommand{\predicateBitpreciseInterpolKindResultsRFInterpolPrecIncorrectFalsePlain}{23\xspace}

  % cpu-time-sum
\providecommand{\predicateBitpreciseInterpolKindResultsRFInterpolPrecIncorrectFalseCpuTimeSumPlain}{}
  \renewcommand{\predicateBitpreciseInterpolKindResultsRFInterpolPrecIncorrectFalseCpuTimeSumPlain}{1057.0374651350003\xspace}
\providecommand{\predicateBitpreciseInterpolKindResultsRFInterpolPrecIncorrectFalseCpuTimeSumPlainHours}{}
  \renewcommand{\predicateBitpreciseInterpolKindResultsRFInterpolPrecIncorrectFalseCpuTimeSumPlainHours}{0.29362151809305564\xspace}

  % cpu-time-avg
\providecommand{\predicateBitpreciseInterpolKindResultsRFInterpolPrecIncorrectFalseCpuTimeAvgPlain}{}
  \renewcommand{\predicateBitpreciseInterpolKindResultsRFInterpolPrecIncorrectFalseCpuTimeAvgPlain}{45.95815065804349\xspace}
\providecommand{\predicateBitpreciseInterpolKindResultsRFInterpolPrecIncorrectFalseCpuTimeAvgPlainHours}{}
  \renewcommand{\predicateBitpreciseInterpolKindResultsRFInterpolPrecIncorrectFalseCpuTimeAvgPlainHours}{0.012766152960567637\xspace}

  % inv-succ
\providecommand{\predicateBitpreciseInterpolKindResultsRFInterpolPrecIncorrectFalseInvSuccPlain}{}
  \renewcommand{\predicateBitpreciseInterpolKindResultsRFInterpolPrecIncorrectFalseInvSuccPlain}{0\xspace}

  % inv-tries
\providecommand{\predicateBitpreciseInterpolKindResultsRFInterpolPrecIncorrectFalseInvTriesPlain}{}
  \renewcommand{\predicateBitpreciseInterpolKindResultsRFInterpolPrecIncorrectFalseInvTriesPlain}{70\xspace}

  % inv-time-sum
\providecommand{\predicateBitpreciseInterpolKindResultsRFInterpolPrecIncorrectFalseInvTimeSumPlain}{}
  \renewcommand{\predicateBitpreciseInterpolKindResultsRFInterpolPrecIncorrectFalseInvTimeSumPlain}{496.91499999999996\xspace}
\providecommand{\predicateBitpreciseInterpolKindResultsRFInterpolPrecIncorrectFalseInvTimeSumPlainHours}{}
  \renewcommand{\predicateBitpreciseInterpolKindResultsRFInterpolPrecIncorrectFalseInvTimeSumPlainHours}{0.13803194444444444\xspace}

 %% incorrect-true %%
\providecommand{\predicateBitpreciseInterpolKindResultsRFInterpolPrecIncorrectTruePlain}{}
  \renewcommand{\predicateBitpreciseInterpolKindResultsRFInterpolPrecIncorrectTruePlain}{0\xspace}

  % cpu-time-sum
\providecommand{\predicateBitpreciseInterpolKindResultsRFInterpolPrecIncorrectTrueCpuTimeSumPlain}{}
  \renewcommand{\predicateBitpreciseInterpolKindResultsRFInterpolPrecIncorrectTrueCpuTimeSumPlain}{0.0\xspace}
\providecommand{\predicateBitpreciseInterpolKindResultsRFInterpolPrecIncorrectTrueCpuTimeSumPlainHours}{}
  \renewcommand{\predicateBitpreciseInterpolKindResultsRFInterpolPrecIncorrectTrueCpuTimeSumPlainHours}{0.0\xspace}

  % cpu-time-avg
\providecommand{\predicateBitpreciseInterpolKindResultsRFInterpolPrecIncorrectTrueCpuTimeAvgPlain}{}
  \renewcommand{\predicateBitpreciseInterpolKindResultsRFInterpolPrecIncorrectTrueCpuTimeAvgPlain}{NaN\xspace}
\providecommand{\predicateBitpreciseInterpolKindResultsRFInterpolPrecIncorrectTrueCpuTimeAvgPlainHours}{}
  \renewcommand{\predicateBitpreciseInterpolKindResultsRFInterpolPrecIncorrectTrueCpuTimeAvgPlainHours}{NaN\xspace}

  % inv-succ
\providecommand{\predicateBitpreciseInterpolKindResultsRFInterpolPrecIncorrectTrueInvSuccPlain}{}
  \renewcommand{\predicateBitpreciseInterpolKindResultsRFInterpolPrecIncorrectTrueInvSuccPlain}{0\xspace}

  % inv-tries
\providecommand{\predicateBitpreciseInterpolKindResultsRFInterpolPrecIncorrectTrueInvTriesPlain}{}
  \renewcommand{\predicateBitpreciseInterpolKindResultsRFInterpolPrecIncorrectTrueInvTriesPlain}{0\xspace}

  % inv-time-sum
\providecommand{\predicateBitpreciseInterpolKindResultsRFInterpolPrecIncorrectTrueInvTimeSumPlain}{}
  \renewcommand{\predicateBitpreciseInterpolKindResultsRFInterpolPrecIncorrectTrueInvTimeSumPlain}{0.0\xspace}
\providecommand{\predicateBitpreciseInterpolKindResultsRFInterpolPrecIncorrectTrueInvTimeSumPlainHours}{}
  \renewcommand{\predicateBitpreciseInterpolKindResultsRFInterpolPrecIncorrectTrueInvTimeSumPlainHours}{0.0\xspace}

 %% all %%
\providecommand{\predicateBitpreciseInterpolKindResultsRFInterpolPrecAllPlain}{}
  \renewcommand{\predicateBitpreciseInterpolKindResultsRFInterpolPrecAllPlain}{3488\xspace}

  % cpu-time-sum
\providecommand{\predicateBitpreciseInterpolKindResultsRFInterpolPrecAllCpuTimeSumPlain}{}
  \renewcommand{\predicateBitpreciseInterpolKindResultsRFInterpolPrecAllCpuTimeSumPlain}{594843.8386216415\xspace}
\providecommand{\predicateBitpreciseInterpolKindResultsRFInterpolPrecAllCpuTimeSumPlainHours}{}
  \renewcommand{\predicateBitpreciseInterpolKindResultsRFInterpolPrecAllCpuTimeSumPlainHours}{165.23439961712265\xspace}

  % cpu-time-avg
\providecommand{\predicateBitpreciseInterpolKindResultsRFInterpolPrecAllCpuTimeAvgPlain}{}
  \renewcommand{\predicateBitpreciseInterpolKindResultsRFInterpolPrecAllCpuTimeAvgPlain}{170.54009134794768\xspace}
\providecommand{\predicateBitpreciseInterpolKindResultsRFInterpolPrecAllCpuTimeAvgPlainHours}{}
  \renewcommand{\predicateBitpreciseInterpolKindResultsRFInterpolPrecAllCpuTimeAvgPlainHours}{0.04737224759665214\xspace}

  % inv-succ
\providecommand{\predicateBitpreciseInterpolKindResultsRFInterpolPrecAllInvSuccPlain}{}
  \renewcommand{\predicateBitpreciseInterpolKindResultsRFInterpolPrecAllInvSuccPlain}{1580\xspace}

  % inv-tries
\providecommand{\predicateBitpreciseInterpolKindResultsRFInterpolPrecAllInvTriesPlain}{}
  \renewcommand{\predicateBitpreciseInterpolKindResultsRFInterpolPrecAllInvTriesPlain}{14336\xspace}

  % inv-time-sum
\providecommand{\predicateBitpreciseInterpolKindResultsRFInterpolPrecAllInvTimeSumPlain}{}
  \renewcommand{\predicateBitpreciseInterpolKindResultsRFInterpolPrecAllInvTimeSumPlain}{188246.12500000023\xspace}
\providecommand{\predicateBitpreciseInterpolKindResultsRFInterpolPrecAllInvTimeSumPlainHours}{}
  \renewcommand{\predicateBitpreciseInterpolKindResultsRFInterpolPrecAllInvTimeSumPlainHours}{52.290590277777845\xspace}

 %% equal-only %%
\providecommand{\predicateBitpreciseInterpolKindResultsRFInterpolPrecEqualOnlyPlain}{}
  \renewcommand{\predicateBitpreciseInterpolKindResultsRFInterpolPrecEqualOnlyPlain}{1793\xspace}

  % cpu-time-sum
\providecommand{\predicateBitpreciseInterpolKindResultsRFInterpolPrecEqualOnlyCpuTimeSumPlain}{}
  \renewcommand{\predicateBitpreciseInterpolKindResultsRFInterpolPrecEqualOnlyCpuTimeSumPlain}{100404.61593234699\xspace}
\providecommand{\predicateBitpreciseInterpolKindResultsRFInterpolPrecEqualOnlyCpuTimeSumPlainHours}{}
  \renewcommand{\predicateBitpreciseInterpolKindResultsRFInterpolPrecEqualOnlyCpuTimeSumPlainHours}{27.89017109231861\xspace}

  % cpu-time-avg
\providecommand{\predicateBitpreciseInterpolKindResultsRFInterpolPrecEqualOnlyCpuTimeAvgPlain}{}
  \renewcommand{\predicateBitpreciseInterpolKindResultsRFInterpolPrecEqualOnlyCpuTimeAvgPlain}{55.99811262261405\xspace}
\providecommand{\predicateBitpreciseInterpolKindResultsRFInterpolPrecEqualOnlyCpuTimeAvgPlainHours}{}
  \renewcommand{\predicateBitpreciseInterpolKindResultsRFInterpolPrecEqualOnlyCpuTimeAvgPlainHours}{0.015555031284059459\xspace}

  % inv-succ
\providecommand{\predicateBitpreciseInterpolKindResultsRFInterpolPrecEqualOnlyInvSuccPlain}{}
  \renewcommand{\predicateBitpreciseInterpolKindResultsRFInterpolPrecEqualOnlyInvSuccPlain}{1258\xspace}

  % inv-tries
\providecommand{\predicateBitpreciseInterpolKindResultsRFInterpolPrecEqualOnlyInvTriesPlain}{}
  \renewcommand{\predicateBitpreciseInterpolKindResultsRFInterpolPrecEqualOnlyInvTriesPlain}{6352\xspace}

  % inv-time-sum
\providecommand{\predicateBitpreciseInterpolKindResultsRFInterpolPrecEqualOnlyInvTimeSumPlain}{}
  \renewcommand{\predicateBitpreciseInterpolKindResultsRFInterpolPrecEqualOnlyInvTimeSumPlain}{24413.673999999995\xspace}
\providecommand{\predicateBitpreciseInterpolKindResultsRFInterpolPrecEqualOnlyInvTimeSumPlainHours}{}
  \renewcommand{\predicateBitpreciseInterpolKindResultsRFInterpolPrecEqualOnlyInvTimeSumPlainHours}{6.78157611111111\xspace}

%%% predicate_bitprecise_interpol_kind.2016-09-04_2044.results.RF_interpol-prec-pf %%%
 %% correct %%
\providecommand{\predicateBitpreciseInterpolKindResultsRFInterpolPrecPfCorrectPlain}{}
  \renewcommand{\predicateBitpreciseInterpolKindResultsRFInterpolPrecPfCorrectPlain}{1820\xspace}

  % cpu-time-sum
\providecommand{\predicateBitpreciseInterpolKindResultsRFInterpolPrecPfCorrectCpuTimeSumPlain}{}
  \renewcommand{\predicateBitpreciseInterpolKindResultsRFInterpolPrecPfCorrectCpuTimeSumPlain}{104229.60893102917\xspace}
\providecommand{\predicateBitpreciseInterpolKindResultsRFInterpolPrecPfCorrectCpuTimeSumPlainHours}{}
  \renewcommand{\predicateBitpreciseInterpolKindResultsRFInterpolPrecPfCorrectCpuTimeSumPlainHours}{28.952669147508104\xspace}

  % cpu-time-avg
\providecommand{\predicateBitpreciseInterpolKindResultsRFInterpolPrecPfCorrectCpuTimeAvgPlain}{}
  \renewcommand{\predicateBitpreciseInterpolKindResultsRFInterpolPrecPfCorrectCpuTimeAvgPlain}{57.269015896169876\xspace}
\providecommand{\predicateBitpreciseInterpolKindResultsRFInterpolPrecPfCorrectCpuTimeAvgPlainHours}{}
  \renewcommand{\predicateBitpreciseInterpolKindResultsRFInterpolPrecPfCorrectCpuTimeAvgPlainHours}{0.0159080599711583\xspace}

  % inv-succ
\providecommand{\predicateBitpreciseInterpolKindResultsRFInterpolPrecPfCorrectInvSuccPlain}{}
  \renewcommand{\predicateBitpreciseInterpolKindResultsRFInterpolPrecPfCorrectInvSuccPlain}{1172\xspace}

  % inv-tries
\providecommand{\predicateBitpreciseInterpolKindResultsRFInterpolPrecPfCorrectInvTriesPlain}{}
  \renewcommand{\predicateBitpreciseInterpolKindResultsRFInterpolPrecPfCorrectInvTriesPlain}{6348\xspace}

  % inv-time-sum
\providecommand{\predicateBitpreciseInterpolKindResultsRFInterpolPrecPfCorrectInvTimeSumPlain}{}
  \renewcommand{\predicateBitpreciseInterpolKindResultsRFInterpolPrecPfCorrectInvTimeSumPlain}{24191.552000000054\xspace}
\providecommand{\predicateBitpreciseInterpolKindResultsRFInterpolPrecPfCorrectInvTimeSumPlainHours}{}
  \renewcommand{\predicateBitpreciseInterpolKindResultsRFInterpolPrecPfCorrectInvTimeSumPlainHours}{6.7198755555555705\xspace}

 %% incorrect %%
\providecommand{\predicateBitpreciseInterpolKindResultsRFInterpolPrecPfIncorrectPlain}{}
  \renewcommand{\predicateBitpreciseInterpolKindResultsRFInterpolPrecPfIncorrectPlain}{24\xspace}

  % cpu-time-sum
\providecommand{\predicateBitpreciseInterpolKindResultsRFInterpolPrecPfIncorrectCpuTimeSumPlain}{}
  \renewcommand{\predicateBitpreciseInterpolKindResultsRFInterpolPrecPfIncorrectCpuTimeSumPlain}{1225.4014701649996\xspace}
\providecommand{\predicateBitpreciseInterpolKindResultsRFInterpolPrecPfIncorrectCpuTimeSumPlainHours}{}
  \renewcommand{\predicateBitpreciseInterpolKindResultsRFInterpolPrecPfIncorrectCpuTimeSumPlainHours}{0.34038929726805545\xspace}

  % cpu-time-avg
\providecommand{\predicateBitpreciseInterpolKindResultsRFInterpolPrecPfIncorrectCpuTimeAvgPlain}{}
  \renewcommand{\predicateBitpreciseInterpolKindResultsRFInterpolPrecPfIncorrectCpuTimeAvgPlain}{51.058394590208316\xspace}
\providecommand{\predicateBitpreciseInterpolKindResultsRFInterpolPrecPfIncorrectCpuTimeAvgPlainHours}{}
  \renewcommand{\predicateBitpreciseInterpolKindResultsRFInterpolPrecPfIncorrectCpuTimeAvgPlainHours}{0.014182887386168977\xspace}

  % inv-succ
\providecommand{\predicateBitpreciseInterpolKindResultsRFInterpolPrecPfIncorrectInvSuccPlain}{}
  \renewcommand{\predicateBitpreciseInterpolKindResultsRFInterpolPrecPfIncorrectInvSuccPlain}{4\xspace}

  % inv-tries
\providecommand{\predicateBitpreciseInterpolKindResultsRFInterpolPrecPfIncorrectInvTriesPlain}{}
  \renewcommand{\predicateBitpreciseInterpolKindResultsRFInterpolPrecPfIncorrectInvTriesPlain}{82\xspace}

  % inv-time-sum
\providecommand{\predicateBitpreciseInterpolKindResultsRFInterpolPrecPfIncorrectInvTimeSumPlain}{}
  \renewcommand{\predicateBitpreciseInterpolKindResultsRFInterpolPrecPfIncorrectInvTimeSumPlain}{597.3420000000001\xspace}
\providecommand{\predicateBitpreciseInterpolKindResultsRFInterpolPrecPfIncorrectInvTimeSumPlainHours}{}
  \renewcommand{\predicateBitpreciseInterpolKindResultsRFInterpolPrecPfIncorrectInvTimeSumPlainHours}{0.16592833333333337\xspace}

 %% timeout %%
\providecommand{\predicateBitpreciseInterpolKindResultsRFInterpolPrecPfTimeoutPlain}{}
  \renewcommand{\predicateBitpreciseInterpolKindResultsRFInterpolPrecPfTimeoutPlain}{1559\xspace}

  % cpu-time-sum
\providecommand{\predicateBitpreciseInterpolKindResultsRFInterpolPrecPfTimeoutCpuTimeSumPlain}{}
  \renewcommand{\predicateBitpreciseInterpolKindResultsRFInterpolPrecPfTimeoutCpuTimeSumPlain}{477278.55607602466\xspace}
\providecommand{\predicateBitpreciseInterpolKindResultsRFInterpolPrecPfTimeoutCpuTimeSumPlainHours}{}
  \renewcommand{\predicateBitpreciseInterpolKindResultsRFInterpolPrecPfTimeoutCpuTimeSumPlainHours}{132.57737668778464\xspace}

  % cpu-time-avg
\providecommand{\predicateBitpreciseInterpolKindResultsRFInterpolPrecPfTimeoutCpuTimeAvgPlain}{}
  \renewcommand{\predicateBitpreciseInterpolKindResultsRFInterpolPrecPfTimeoutCpuTimeAvgPlain}{306.14403853497413\xspace}
\providecommand{\predicateBitpreciseInterpolKindResultsRFInterpolPrecPfTimeoutCpuTimeAvgPlainHours}{}
  \renewcommand{\predicateBitpreciseInterpolKindResultsRFInterpolPrecPfTimeoutCpuTimeAvgPlainHours}{0.08504001070415948\xspace}

  % inv-succ
\providecommand{\predicateBitpreciseInterpolKindResultsRFInterpolPrecPfTimeoutInvSuccPlain}{}
  \renewcommand{\predicateBitpreciseInterpolKindResultsRFInterpolPrecPfTimeoutInvSuccPlain}{229\xspace}

  % inv-tries
\providecommand{\predicateBitpreciseInterpolKindResultsRFInterpolPrecPfTimeoutInvTriesPlain}{}
  \renewcommand{\predicateBitpreciseInterpolKindResultsRFInterpolPrecPfTimeoutInvTriesPlain}{7446\xspace}

  % inv-time-sum
\providecommand{\predicateBitpreciseInterpolKindResultsRFInterpolPrecPfTimeoutInvTimeSumPlain}{}
  \renewcommand{\predicateBitpreciseInterpolKindResultsRFInterpolPrecPfTimeoutInvTimeSumPlain}{157760.849\xspace}
\providecommand{\predicateBitpreciseInterpolKindResultsRFInterpolPrecPfTimeoutInvTimeSumPlainHours}{}
  \renewcommand{\predicateBitpreciseInterpolKindResultsRFInterpolPrecPfTimeoutInvTimeSumPlainHours}{43.82245805555555\xspace}

 %% unknown-or-category-error %%
\providecommand{\predicateBitpreciseInterpolKindResultsRFInterpolPrecPfUnknownOrCategoryErrorPlain}{}
  \renewcommand{\predicateBitpreciseInterpolKindResultsRFInterpolPrecPfUnknownOrCategoryErrorPlain}{1644\xspace}

  % cpu-time-sum
\providecommand{\predicateBitpreciseInterpolKindResultsRFInterpolPrecPfUnknownOrCategoryErrorCpuTimeSumPlain}{}
  \renewcommand{\predicateBitpreciseInterpolKindResultsRFInterpolPrecPfUnknownOrCategoryErrorCpuTimeSumPlain}{486686.88058949844\xspace}
\providecommand{\predicateBitpreciseInterpolKindResultsRFInterpolPrecPfUnknownOrCategoryErrorCpuTimeSumPlainHours}{}
  \renewcommand{\predicateBitpreciseInterpolKindResultsRFInterpolPrecPfUnknownOrCategoryErrorCpuTimeSumPlainHours}{135.19080016374957\xspace}

  % cpu-time-avg
\providecommand{\predicateBitpreciseInterpolKindResultsRFInterpolPrecPfUnknownOrCategoryErrorCpuTimeAvgPlain}{}
  \renewcommand{\predicateBitpreciseInterpolKindResultsRFInterpolPrecPfUnknownOrCategoryErrorCpuTimeAvgPlain}{296.0382485337582\xspace}
\providecommand{\predicateBitpreciseInterpolKindResultsRFInterpolPrecPfUnknownOrCategoryErrorCpuTimeAvgPlainHours}{}
  \renewcommand{\predicateBitpreciseInterpolKindResultsRFInterpolPrecPfUnknownOrCategoryErrorCpuTimeAvgPlainHours}{0.08223284681493283\xspace}

  % inv-succ
\providecommand{\predicateBitpreciseInterpolKindResultsRFInterpolPrecPfUnknownOrCategoryErrorInvSuccPlain}{}
  \renewcommand{\predicateBitpreciseInterpolKindResultsRFInterpolPrecPfUnknownOrCategoryErrorInvSuccPlain}{233\xspace}

  % inv-tries
\providecommand{\predicateBitpreciseInterpolKindResultsRFInterpolPrecPfUnknownOrCategoryErrorInvTriesPlain}{}
  \renewcommand{\predicateBitpreciseInterpolKindResultsRFInterpolPrecPfUnknownOrCategoryErrorInvTriesPlain}{7685\xspace}

  % inv-time-sum
\providecommand{\predicateBitpreciseInterpolKindResultsRFInterpolPrecPfUnknownOrCategoryErrorInvTimeSumPlain}{}
  \renewcommand{\predicateBitpreciseInterpolKindResultsRFInterpolPrecPfUnknownOrCategoryErrorInvTimeSumPlain}{160533.59700000004\xspace}
\providecommand{\predicateBitpreciseInterpolKindResultsRFInterpolPrecPfUnknownOrCategoryErrorInvTimeSumPlainHours}{}
  \renewcommand{\predicateBitpreciseInterpolKindResultsRFInterpolPrecPfUnknownOrCategoryErrorInvTimeSumPlainHours}{44.59266583333334\xspace}

 %% correct-false %%
\providecommand{\predicateBitpreciseInterpolKindResultsRFInterpolPrecPfCorrectFalsePlain}{}
  \renewcommand{\predicateBitpreciseInterpolKindResultsRFInterpolPrecPfCorrectFalsePlain}{485\xspace}

  % cpu-time-sum
\providecommand{\predicateBitpreciseInterpolKindResultsRFInterpolPrecPfCorrectFalseCpuTimeSumPlain}{}
  \renewcommand{\predicateBitpreciseInterpolKindResultsRFInterpolPrecPfCorrectFalseCpuTimeSumPlain}{32346.225872882987\xspace}
\providecommand{\predicateBitpreciseInterpolKindResultsRFInterpolPrecPfCorrectFalseCpuTimeSumPlainHours}{}
  \renewcommand{\predicateBitpreciseInterpolKindResultsRFInterpolPrecPfCorrectFalseCpuTimeSumPlainHours}{8.985062742467496\xspace}

  % cpu-time-avg
\providecommand{\predicateBitpreciseInterpolKindResultsRFInterpolPrecPfCorrectFalseCpuTimeAvgPlain}{}
  \renewcommand{\predicateBitpreciseInterpolKindResultsRFInterpolPrecPfCorrectFalseCpuTimeAvgPlain}{66.69324922243915\xspace}
\providecommand{\predicateBitpreciseInterpolKindResultsRFInterpolPrecPfCorrectFalseCpuTimeAvgPlainHours}{}
  \renewcommand{\predicateBitpreciseInterpolKindResultsRFInterpolPrecPfCorrectFalseCpuTimeAvgPlainHours}{0.018525902561788654\xspace}

  % inv-succ
\providecommand{\predicateBitpreciseInterpolKindResultsRFInterpolPrecPfCorrectFalseInvSuccPlain}{}
  \renewcommand{\predicateBitpreciseInterpolKindResultsRFInterpolPrecPfCorrectFalseInvSuccPlain}{131\xspace}

  % inv-tries
\providecommand{\predicateBitpreciseInterpolKindResultsRFInterpolPrecPfCorrectFalseInvTriesPlain}{}
  \renewcommand{\predicateBitpreciseInterpolKindResultsRFInterpolPrecPfCorrectFalseInvTriesPlain}{2328\xspace}

  % inv-time-sum
\providecommand{\predicateBitpreciseInterpolKindResultsRFInterpolPrecPfCorrectFalseInvTimeSumPlain}{}
  \renewcommand{\predicateBitpreciseInterpolKindResultsRFInterpolPrecPfCorrectFalseInvTimeSumPlain}{7151.249999999994\xspace}
\providecommand{\predicateBitpreciseInterpolKindResultsRFInterpolPrecPfCorrectFalseInvTimeSumPlainHours}{}
  \renewcommand{\predicateBitpreciseInterpolKindResultsRFInterpolPrecPfCorrectFalseInvTimeSumPlainHours}{1.9864583333333317\xspace}

 %% correct-true %%
\providecommand{\predicateBitpreciseInterpolKindResultsRFInterpolPrecPfCorrectTruePlain}{}
  \renewcommand{\predicateBitpreciseInterpolKindResultsRFInterpolPrecPfCorrectTruePlain}{1335\xspace}

  % cpu-time-sum
\providecommand{\predicateBitpreciseInterpolKindResultsRFInterpolPrecPfCorrectTrueCpuTimeSumPlain}{}
  \renewcommand{\predicateBitpreciseInterpolKindResultsRFInterpolPrecPfCorrectTrueCpuTimeSumPlain}{71883.38305814598\xspace}
\providecommand{\predicateBitpreciseInterpolKindResultsRFInterpolPrecPfCorrectTrueCpuTimeSumPlainHours}{}
  \renewcommand{\predicateBitpreciseInterpolKindResultsRFInterpolPrecPfCorrectTrueCpuTimeSumPlainHours}{19.96760640504055\xspace}

  % cpu-time-avg
\providecommand{\predicateBitpreciseInterpolKindResultsRFInterpolPrecPfCorrectTrueCpuTimeAvgPlain}{}
  \renewcommand{\predicateBitpreciseInterpolKindResultsRFInterpolPrecPfCorrectTrueCpuTimeAvgPlain}{53.845230755165524\xspace}
\providecommand{\predicateBitpreciseInterpolKindResultsRFInterpolPrecPfCorrectTrueCpuTimeAvgPlainHours}{}
  \renewcommand{\predicateBitpreciseInterpolKindResultsRFInterpolPrecPfCorrectTrueCpuTimeAvgPlainHours}{0.014957008543101535\xspace}

  % inv-succ
\providecommand{\predicateBitpreciseInterpolKindResultsRFInterpolPrecPfCorrectTrueInvSuccPlain}{}
  \renewcommand{\predicateBitpreciseInterpolKindResultsRFInterpolPrecPfCorrectTrueInvSuccPlain}{1041\xspace}

  % inv-tries
\providecommand{\predicateBitpreciseInterpolKindResultsRFInterpolPrecPfCorrectTrueInvTriesPlain}{}
  \renewcommand{\predicateBitpreciseInterpolKindResultsRFInterpolPrecPfCorrectTrueInvTriesPlain}{4020\xspace}

  % inv-time-sum
\providecommand{\predicateBitpreciseInterpolKindResultsRFInterpolPrecPfCorrectTrueInvTimeSumPlain}{}
  \renewcommand{\predicateBitpreciseInterpolKindResultsRFInterpolPrecPfCorrectTrueInvTimeSumPlain}{17040.30199999999\xspace}
\providecommand{\predicateBitpreciseInterpolKindResultsRFInterpolPrecPfCorrectTrueInvTimeSumPlainHours}{}
  \renewcommand{\predicateBitpreciseInterpolKindResultsRFInterpolPrecPfCorrectTrueInvTimeSumPlainHours}{4.7334172222222195\xspace}

 %% incorrect-false %%
\providecommand{\predicateBitpreciseInterpolKindResultsRFInterpolPrecPfIncorrectFalsePlain}{}
  \renewcommand{\predicateBitpreciseInterpolKindResultsRFInterpolPrecPfIncorrectFalsePlain}{23\xspace}

  % cpu-time-sum
\providecommand{\predicateBitpreciseInterpolKindResultsRFInterpolPrecPfIncorrectFalseCpuTimeSumPlain}{}
  \renewcommand{\predicateBitpreciseInterpolKindResultsRFInterpolPrecPfIncorrectFalseCpuTimeSumPlain}{1117.3380084979997\xspace}
\providecommand{\predicateBitpreciseInterpolKindResultsRFInterpolPrecPfIncorrectFalseCpuTimeSumPlainHours}{}
  \renewcommand{\predicateBitpreciseInterpolKindResultsRFInterpolPrecPfIncorrectFalseCpuTimeSumPlainHours}{0.31037166902722213\xspace}

  % cpu-time-avg
\providecommand{\predicateBitpreciseInterpolKindResultsRFInterpolPrecPfIncorrectFalseCpuTimeAvgPlain}{}
  \renewcommand{\predicateBitpreciseInterpolKindResultsRFInterpolPrecPfIncorrectFalseCpuTimeAvgPlain}{48.57991341295651\xspace}
\providecommand{\predicateBitpreciseInterpolKindResultsRFInterpolPrecPfIncorrectFalseCpuTimeAvgPlainHours}{}
  \renewcommand{\predicateBitpreciseInterpolKindResultsRFInterpolPrecPfIncorrectFalseCpuTimeAvgPlainHours}{0.013494420392487919\xspace}

  % inv-succ
\providecommand{\predicateBitpreciseInterpolKindResultsRFInterpolPrecPfIncorrectFalseInvSuccPlain}{}
  \renewcommand{\predicateBitpreciseInterpolKindResultsRFInterpolPrecPfIncorrectFalseInvSuccPlain}{0\xspace}

  % inv-tries
\providecommand{\predicateBitpreciseInterpolKindResultsRFInterpolPrecPfIncorrectFalseInvTriesPlain}{}
  \renewcommand{\predicateBitpreciseInterpolKindResultsRFInterpolPrecPfIncorrectFalseInvTriesPlain}{70\xspace}

  % inv-time-sum
\providecommand{\predicateBitpreciseInterpolKindResultsRFInterpolPrecPfIncorrectFalseInvTimeSumPlain}{}
  \renewcommand{\predicateBitpreciseInterpolKindResultsRFInterpolPrecPfIncorrectFalseInvTimeSumPlain}{543.614\xspace}
\providecommand{\predicateBitpreciseInterpolKindResultsRFInterpolPrecPfIncorrectFalseInvTimeSumPlainHours}{}
  \renewcommand{\predicateBitpreciseInterpolKindResultsRFInterpolPrecPfIncorrectFalseInvTimeSumPlainHours}{0.1510038888888889\xspace}

 %% incorrect-true %%
\providecommand{\predicateBitpreciseInterpolKindResultsRFInterpolPrecPfIncorrectTruePlain}{}
  \renewcommand{\predicateBitpreciseInterpolKindResultsRFInterpolPrecPfIncorrectTruePlain}{1\xspace}

  % cpu-time-sum
\providecommand{\predicateBitpreciseInterpolKindResultsRFInterpolPrecPfIncorrectTrueCpuTimeSumPlain}{}
  \renewcommand{\predicateBitpreciseInterpolKindResultsRFInterpolPrecPfIncorrectTrueCpuTimeSumPlain}{108.063461667\xspace}
\providecommand{\predicateBitpreciseInterpolKindResultsRFInterpolPrecPfIncorrectTrueCpuTimeSumPlainHours}{}
  \renewcommand{\predicateBitpreciseInterpolKindResultsRFInterpolPrecPfIncorrectTrueCpuTimeSumPlainHours}{0.030017628240833334\xspace}

  % cpu-time-avg
\providecommand{\predicateBitpreciseInterpolKindResultsRFInterpolPrecPfIncorrectTrueCpuTimeAvgPlain}{}
  \renewcommand{\predicateBitpreciseInterpolKindResultsRFInterpolPrecPfIncorrectTrueCpuTimeAvgPlain}{108.063461667\xspace}
\providecommand{\predicateBitpreciseInterpolKindResultsRFInterpolPrecPfIncorrectTrueCpuTimeAvgPlainHours}{}
  \renewcommand{\predicateBitpreciseInterpolKindResultsRFInterpolPrecPfIncorrectTrueCpuTimeAvgPlainHours}{0.030017628240833334\xspace}

  % inv-succ
\providecommand{\predicateBitpreciseInterpolKindResultsRFInterpolPrecPfIncorrectTrueInvSuccPlain}{}
  \renewcommand{\predicateBitpreciseInterpolKindResultsRFInterpolPrecPfIncorrectTrueInvSuccPlain}{4\xspace}

  % inv-tries
\providecommand{\predicateBitpreciseInterpolKindResultsRFInterpolPrecPfIncorrectTrueInvTriesPlain}{}
  \renewcommand{\predicateBitpreciseInterpolKindResultsRFInterpolPrecPfIncorrectTrueInvTriesPlain}{12\xspace}

  % inv-time-sum
\providecommand{\predicateBitpreciseInterpolKindResultsRFInterpolPrecPfIncorrectTrueInvTimeSumPlain}{}
  \renewcommand{\predicateBitpreciseInterpolKindResultsRFInterpolPrecPfIncorrectTrueInvTimeSumPlain}{53.728\xspace}
\providecommand{\predicateBitpreciseInterpolKindResultsRFInterpolPrecPfIncorrectTrueInvTimeSumPlainHours}{}
  \renewcommand{\predicateBitpreciseInterpolKindResultsRFInterpolPrecPfIncorrectTrueInvTimeSumPlainHours}{0.014924444444444445\xspace}

 %% all %%
\providecommand{\predicateBitpreciseInterpolKindResultsRFInterpolPrecPfAllPlain}{}
  \renewcommand{\predicateBitpreciseInterpolKindResultsRFInterpolPrecPfAllPlain}{3488\xspace}

  % cpu-time-sum
\providecommand{\predicateBitpreciseInterpolKindResultsRFInterpolPrecPfAllCpuTimeSumPlain}{}
  \renewcommand{\predicateBitpreciseInterpolKindResultsRFInterpolPrecPfAllCpuTimeSumPlain}{592141.8909906944\xspace}
\providecommand{\predicateBitpreciseInterpolKindResultsRFInterpolPrecPfAllCpuTimeSumPlainHours}{}
  \renewcommand{\predicateBitpreciseInterpolKindResultsRFInterpolPrecPfAllCpuTimeSumPlainHours}{164.48385860852622\xspace}

  % cpu-time-avg
\providecommand{\predicateBitpreciseInterpolKindResultsRFInterpolPrecPfAllCpuTimeAvgPlain}{}
  \renewcommand{\predicateBitpreciseInterpolKindResultsRFInterpolPrecPfAllCpuTimeAvgPlain}{169.76545039870825\xspace}
\providecommand{\predicateBitpreciseInterpolKindResultsRFInterpolPrecPfAllCpuTimeAvgPlainHours}{}
  \renewcommand{\predicateBitpreciseInterpolKindResultsRFInterpolPrecPfAllCpuTimeAvgPlainHours}{0.04715706955519674\xspace}

  % inv-succ
\providecommand{\predicateBitpreciseInterpolKindResultsRFInterpolPrecPfAllInvSuccPlain}{}
  \renewcommand{\predicateBitpreciseInterpolKindResultsRFInterpolPrecPfAllInvSuccPlain}{1409\xspace}

  % inv-tries
\providecommand{\predicateBitpreciseInterpolKindResultsRFInterpolPrecPfAllInvTriesPlain}{}
  \renewcommand{\predicateBitpreciseInterpolKindResultsRFInterpolPrecPfAllInvTriesPlain}{14115\xspace}

  % inv-time-sum
\providecommand{\predicateBitpreciseInterpolKindResultsRFInterpolPrecPfAllInvTimeSumPlain}{}
  \renewcommand{\predicateBitpreciseInterpolKindResultsRFInterpolPrecPfAllInvTimeSumPlain}{185322.49100000013\xspace}
\providecommand{\predicateBitpreciseInterpolKindResultsRFInterpolPrecPfAllInvTimeSumPlainHours}{}
  \renewcommand{\predicateBitpreciseInterpolKindResultsRFInterpolPrecPfAllInvTimeSumPlainHours}{51.47846972222226\xspace}

 %% equal-only %%
\providecommand{\predicateBitpreciseInterpolKindResultsRFInterpolPrecPfEqualOnlyPlain}{}
  \renewcommand{\predicateBitpreciseInterpolKindResultsRFInterpolPrecPfEqualOnlyPlain}{1793\xspace}

  % cpu-time-sum
\providecommand{\predicateBitpreciseInterpolKindResultsRFInterpolPrecPfEqualOnlyCpuTimeSumPlain}{}
  \renewcommand{\predicateBitpreciseInterpolKindResultsRFInterpolPrecPfEqualOnlyCpuTimeSumPlain}{99123.50466524516\xspace}
\providecommand{\predicateBitpreciseInterpolKindResultsRFInterpolPrecPfEqualOnlyCpuTimeSumPlainHours}{}
  \renewcommand{\predicateBitpreciseInterpolKindResultsRFInterpolPrecPfEqualOnlyCpuTimeSumPlainHours}{27.53430685145699\xspace}

  % cpu-time-avg
\providecommand{\predicateBitpreciseInterpolKindResultsRFInterpolPrecPfEqualOnlyCpuTimeAvgPlain}{}
  \renewcommand{\predicateBitpreciseInterpolKindResultsRFInterpolPrecPfEqualOnlyCpuTimeAvgPlain}{55.28360550208877\xspace}
\providecommand{\predicateBitpreciseInterpolKindResultsRFInterpolPrecPfEqualOnlyCpuTimeAvgPlainHours}{}
  \renewcommand{\predicateBitpreciseInterpolKindResultsRFInterpolPrecPfEqualOnlyCpuTimeAvgPlainHours}{0.015356557083913546\xspace}

  % inv-succ
\providecommand{\predicateBitpreciseInterpolKindResultsRFInterpolPrecPfEqualOnlyInvSuccPlain}{}
  \renewcommand{\predicateBitpreciseInterpolKindResultsRFInterpolPrecPfEqualOnlyInvSuccPlain}{1146\xspace}

  % inv-tries
\providecommand{\predicateBitpreciseInterpolKindResultsRFInterpolPrecPfEqualOnlyInvTriesPlain}{}
  \renewcommand{\predicateBitpreciseInterpolKindResultsRFInterpolPrecPfEqualOnlyInvTriesPlain}{6209\xspace}

  % inv-time-sum
\providecommand{\predicateBitpreciseInterpolKindResultsRFInterpolPrecPfEqualOnlyInvTimeSumPlain}{}
  \renewcommand{\predicateBitpreciseInterpolKindResultsRFInterpolPrecPfEqualOnlyInvTimeSumPlain}{23097.20200000006\xspace}
\providecommand{\predicateBitpreciseInterpolKindResultsRFInterpolPrecPfEqualOnlyInvTimeSumPlainHours}{}
  \renewcommand{\predicateBitpreciseInterpolKindResultsRFInterpolPrecPfEqualOnlyInvTimeSumPlainHours}{6.415889444444461\xspace}

%%% predicate_bitprecise_interpol_kind.2016-09-04_2044.results.RF_interpol-pf %%%
 %% correct %%
\providecommand{\predicateBitpreciseInterpolKindResultsRFInterpolPfCorrectPlain}{}
  \renewcommand{\predicateBitpreciseInterpolKindResultsRFInterpolPfCorrectPlain}{1831\xspace}

  % cpu-time-sum
\providecommand{\predicateBitpreciseInterpolKindResultsRFInterpolPfCorrectCpuTimeSumPlain}{}
  \renewcommand{\predicateBitpreciseInterpolKindResultsRFInterpolPfCorrectCpuTimeSumPlain}{104964.98008863088\xspace}
\providecommand{\predicateBitpreciseInterpolKindResultsRFInterpolPfCorrectCpuTimeSumPlainHours}{}
  \renewcommand{\predicateBitpreciseInterpolKindResultsRFInterpolPfCorrectCpuTimeSumPlainHours}{29.15693891350858\xspace}

  % cpu-time-avg
\providecommand{\predicateBitpreciseInterpolKindResultsRFInterpolPfCorrectCpuTimeAvgPlain}{}
  \renewcommand{\predicateBitpreciseInterpolKindResultsRFInterpolPfCorrectCpuTimeAvgPlain}{57.326586613124455\xspace}
\providecommand{\predicateBitpreciseInterpolKindResultsRFInterpolPfCorrectCpuTimeAvgPlainHours}{}
  \renewcommand{\predicateBitpreciseInterpolKindResultsRFInterpolPfCorrectCpuTimeAvgPlainHours}{0.015924051836979015\xspace}

  % inv-succ
\providecommand{\predicateBitpreciseInterpolKindResultsRFInterpolPfCorrectInvSuccPlain}{}
  \renewcommand{\predicateBitpreciseInterpolKindResultsRFInterpolPfCorrectInvSuccPlain}{1007\xspace}

  % inv-tries
\providecommand{\predicateBitpreciseInterpolKindResultsRFInterpolPfCorrectInvTriesPlain}{}
  \renewcommand{\predicateBitpreciseInterpolKindResultsRFInterpolPfCorrectInvTriesPlain}{6218\xspace}

  % inv-time-sum
\providecommand{\predicateBitpreciseInterpolKindResultsRFInterpolPfCorrectInvTimeSumPlain}{}
  \renewcommand{\predicateBitpreciseInterpolKindResultsRFInterpolPfCorrectInvTimeSumPlain}{24339.989000000016\xspace}
\providecommand{\predicateBitpreciseInterpolKindResultsRFInterpolPfCorrectInvTimeSumPlainHours}{}
  \renewcommand{\predicateBitpreciseInterpolKindResultsRFInterpolPfCorrectInvTimeSumPlainHours}{6.76110805555556\xspace}

 %% incorrect %%
\providecommand{\predicateBitpreciseInterpolKindResultsRFInterpolPfIncorrectPlain}{}
  \renewcommand{\predicateBitpreciseInterpolKindResultsRFInterpolPfIncorrectPlain}{24\xspace}

  % cpu-time-sum
\providecommand{\predicateBitpreciseInterpolKindResultsRFInterpolPfIncorrectCpuTimeSumPlain}{}
  \renewcommand{\predicateBitpreciseInterpolKindResultsRFInterpolPfIncorrectCpuTimeSumPlain}{1164.026001707\xspace}
\providecommand{\predicateBitpreciseInterpolKindResultsRFInterpolPfIncorrectCpuTimeSumPlainHours}{}
  \renewcommand{\predicateBitpreciseInterpolKindResultsRFInterpolPfIncorrectCpuTimeSumPlainHours}{0.3233405560297222\xspace}

  % cpu-time-avg
\providecommand{\predicateBitpreciseInterpolKindResultsRFInterpolPfIncorrectCpuTimeAvgPlain}{}
  \renewcommand{\predicateBitpreciseInterpolKindResultsRFInterpolPfIncorrectCpuTimeAvgPlain}{48.501083404458335\xspace}
\providecommand{\predicateBitpreciseInterpolKindResultsRFInterpolPfIncorrectCpuTimeAvgPlainHours}{}
  \renewcommand{\predicateBitpreciseInterpolKindResultsRFInterpolPfIncorrectCpuTimeAvgPlainHours}{0.013472523167905093\xspace}

  % inv-succ
\providecommand{\predicateBitpreciseInterpolKindResultsRFInterpolPfIncorrectInvSuccPlain}{}
  \renewcommand{\predicateBitpreciseInterpolKindResultsRFInterpolPfIncorrectInvSuccPlain}{4\xspace}

  % inv-tries
\providecommand{\predicateBitpreciseInterpolKindResultsRFInterpolPfIncorrectInvTriesPlain}{}
  \renewcommand{\predicateBitpreciseInterpolKindResultsRFInterpolPfIncorrectInvTriesPlain}{80\xspace}

  % inv-time-sum
\providecommand{\predicateBitpreciseInterpolKindResultsRFInterpolPfIncorrectInvTimeSumPlain}{}
  \renewcommand{\predicateBitpreciseInterpolKindResultsRFInterpolPfIncorrectInvTimeSumPlain}{572.184\xspace}
\providecommand{\predicateBitpreciseInterpolKindResultsRFInterpolPfIncorrectInvTimeSumPlainHours}{}
  \renewcommand{\predicateBitpreciseInterpolKindResultsRFInterpolPfIncorrectInvTimeSumPlainHours}{0.15894\xspace}

 %% timeout %%
\providecommand{\predicateBitpreciseInterpolKindResultsRFInterpolPfTimeoutPlain}{}
  \renewcommand{\predicateBitpreciseInterpolKindResultsRFInterpolPfTimeoutPlain}{1547\xspace}

  % cpu-time-sum
\providecommand{\predicateBitpreciseInterpolKindResultsRFInterpolPfTimeoutCpuTimeSumPlain}{}
  \renewcommand{\predicateBitpreciseInterpolKindResultsRFInterpolPfTimeoutCpuTimeSumPlain}{473634.604936016\xspace}
\providecommand{\predicateBitpreciseInterpolKindResultsRFInterpolPfTimeoutCpuTimeSumPlainHours}{}
  \renewcommand{\predicateBitpreciseInterpolKindResultsRFInterpolPfTimeoutCpuTimeSumPlainHours}{131.56516803778223\xspace}

  % cpu-time-avg
\providecommand{\predicateBitpreciseInterpolKindResultsRFInterpolPfTimeoutCpuTimeAvgPlain}{}
  \renewcommand{\predicateBitpreciseInterpolKindResultsRFInterpolPfTimeoutCpuTimeAvgPlain}{306.1632869657505\xspace}
\providecommand{\predicateBitpreciseInterpolKindResultsRFInterpolPfTimeoutCpuTimeAvgPlainHours}{}
  \renewcommand{\predicateBitpreciseInterpolKindResultsRFInterpolPfTimeoutCpuTimeAvgPlainHours}{0.08504535749048625\xspace}

  % inv-succ
\providecommand{\predicateBitpreciseInterpolKindResultsRFInterpolPfTimeoutInvSuccPlain}{}
  \renewcommand{\predicateBitpreciseInterpolKindResultsRFInterpolPfTimeoutInvSuccPlain}{201\xspace}

  % inv-tries
\providecommand{\predicateBitpreciseInterpolKindResultsRFInterpolPfTimeoutInvTriesPlain}{}
  \renewcommand{\predicateBitpreciseInterpolKindResultsRFInterpolPfTimeoutInvTriesPlain}{7470\xspace}

  % inv-time-sum
\providecommand{\predicateBitpreciseInterpolKindResultsRFInterpolPfTimeoutInvTimeSumPlain}{}
  \renewcommand{\predicateBitpreciseInterpolKindResultsRFInterpolPfTimeoutInvTimeSumPlain}{159344.4389999998\xspace}
\providecommand{\predicateBitpreciseInterpolKindResultsRFInterpolPfTimeoutInvTimeSumPlainHours}{}
  \renewcommand{\predicateBitpreciseInterpolKindResultsRFInterpolPfTimeoutInvTimeSumPlainHours}{44.262344166666615\xspace}

 %% unknown-or-category-error %%
\providecommand{\predicateBitpreciseInterpolKindResultsRFInterpolPfUnknownOrCategoryErrorPlain}{}
  \renewcommand{\predicateBitpreciseInterpolKindResultsRFInterpolPfUnknownOrCategoryErrorPlain}{1633\xspace}

  % cpu-time-sum
\providecommand{\predicateBitpreciseInterpolKindResultsRFInterpolPfUnknownOrCategoryErrorCpuTimeSumPlain}{}
  \renewcommand{\predicateBitpreciseInterpolKindResultsRFInterpolPfUnknownOrCategoryErrorCpuTimeSumPlain}{483236.5632334168\xspace}
\providecommand{\predicateBitpreciseInterpolKindResultsRFInterpolPfUnknownOrCategoryErrorCpuTimeSumPlainHours}{}
  \renewcommand{\predicateBitpreciseInterpolKindResultsRFInterpolPfUnknownOrCategoryErrorCpuTimeSumPlainHours}{134.23237867594912\xspace}

  % cpu-time-avg
\providecommand{\predicateBitpreciseInterpolKindResultsRFInterpolPfUnknownOrCategoryErrorCpuTimeAvgPlain}{}
  \renewcommand{\predicateBitpreciseInterpolKindResultsRFInterpolPfUnknownOrCategoryErrorCpuTimeAvgPlain}{295.9195120841499\xspace}
\providecommand{\predicateBitpreciseInterpolKindResultsRFInterpolPfUnknownOrCategoryErrorCpuTimeAvgPlainHours}{}
  \renewcommand{\predicateBitpreciseInterpolKindResultsRFInterpolPfUnknownOrCategoryErrorCpuTimeAvgPlainHours}{0.08219986446781942\xspace}

  % inv-succ
\providecommand{\predicateBitpreciseInterpolKindResultsRFInterpolPfUnknownOrCategoryErrorInvSuccPlain}{}
  \renewcommand{\predicateBitpreciseInterpolKindResultsRFInterpolPfUnknownOrCategoryErrorInvSuccPlain}{204\xspace}

  % inv-tries
\providecommand{\predicateBitpreciseInterpolKindResultsRFInterpolPfUnknownOrCategoryErrorInvTriesPlain}{}
  \renewcommand{\predicateBitpreciseInterpolKindResultsRFInterpolPfUnknownOrCategoryErrorInvTriesPlain}{7710\xspace}

  % inv-time-sum
\providecommand{\predicateBitpreciseInterpolKindResultsRFInterpolPfUnknownOrCategoryErrorInvTimeSumPlain}{}
  \renewcommand{\predicateBitpreciseInterpolKindResultsRFInterpolPfUnknownOrCategoryErrorInvTimeSumPlain}{162096.24099999972\xspace}
\providecommand{\predicateBitpreciseInterpolKindResultsRFInterpolPfUnknownOrCategoryErrorInvTimeSumPlainHours}{}
  \renewcommand{\predicateBitpreciseInterpolKindResultsRFInterpolPfUnknownOrCategoryErrorInvTimeSumPlainHours}{45.026733611111034\xspace}

 %% correct-false %%
\providecommand{\predicateBitpreciseInterpolKindResultsRFInterpolPfCorrectFalsePlain}{}
  \renewcommand{\predicateBitpreciseInterpolKindResultsRFInterpolPfCorrectFalsePlain}{490\xspace}

  % cpu-time-sum
\providecommand{\predicateBitpreciseInterpolKindResultsRFInterpolPfCorrectFalseCpuTimeSumPlain}{}
  \renewcommand{\predicateBitpreciseInterpolKindResultsRFInterpolPfCorrectFalseCpuTimeSumPlain}{33269.556898617964\xspace}
\providecommand{\predicateBitpreciseInterpolKindResultsRFInterpolPfCorrectFalseCpuTimeSumPlainHours}{}
  \renewcommand{\predicateBitpreciseInterpolKindResultsRFInterpolPfCorrectFalseCpuTimeSumPlainHours}{9.241543582949435\xspace}

  % cpu-time-avg
\providecommand{\predicateBitpreciseInterpolKindResultsRFInterpolPfCorrectFalseCpuTimeAvgPlain}{}
  \renewcommand{\predicateBitpreciseInterpolKindResultsRFInterpolPfCorrectFalseCpuTimeAvgPlain}{67.89705489513871\xspace}
\providecommand{\predicateBitpreciseInterpolKindResultsRFInterpolPfCorrectFalseCpuTimeAvgPlainHours}{}
  \renewcommand{\predicateBitpreciseInterpolKindResultsRFInterpolPfCorrectFalseCpuTimeAvgPlainHours}{0.01886029302642742\xspace}

  % inv-succ
\providecommand{\predicateBitpreciseInterpolKindResultsRFInterpolPfCorrectFalseInvSuccPlain}{}
  \renewcommand{\predicateBitpreciseInterpolKindResultsRFInterpolPfCorrectFalseInvSuccPlain}{96\xspace}

  % inv-tries
\providecommand{\predicateBitpreciseInterpolKindResultsRFInterpolPfCorrectFalseInvTriesPlain}{}
  \renewcommand{\predicateBitpreciseInterpolKindResultsRFInterpolPfCorrectFalseInvTriesPlain}{2308\xspace}

  % inv-time-sum
\providecommand{\predicateBitpreciseInterpolKindResultsRFInterpolPfCorrectFalseInvTimeSumPlain}{}
  \renewcommand{\predicateBitpreciseInterpolKindResultsRFInterpolPfCorrectFalseInvTimeSumPlain}{7452.449999999993\xspace}
\providecommand{\predicateBitpreciseInterpolKindResultsRFInterpolPfCorrectFalseInvTimeSumPlainHours}{}
  \renewcommand{\predicateBitpreciseInterpolKindResultsRFInterpolPfCorrectFalseInvTimeSumPlainHours}{2.070124999999998\xspace}

 %% correct-true %%
\providecommand{\predicateBitpreciseInterpolKindResultsRFInterpolPfCorrectTruePlain}{}
  \renewcommand{\predicateBitpreciseInterpolKindResultsRFInterpolPfCorrectTruePlain}{1341\xspace}

  % cpu-time-sum
\providecommand{\predicateBitpreciseInterpolKindResultsRFInterpolPfCorrectTrueCpuTimeSumPlain}{}
  \renewcommand{\predicateBitpreciseInterpolKindResultsRFInterpolPfCorrectTrueCpuTimeSumPlain}{71695.42319001298\xspace}
\providecommand{\predicateBitpreciseInterpolKindResultsRFInterpolPfCorrectTrueCpuTimeSumPlainHours}{}
  \renewcommand{\predicateBitpreciseInterpolKindResultsRFInterpolPfCorrectTrueCpuTimeSumPlainHours}{19.91539533055916\xspace}

  % cpu-time-avg
\providecommand{\predicateBitpreciseInterpolKindResultsRFInterpolPfCorrectTrueCpuTimeAvgPlain}{}
  \renewcommand{\predicateBitpreciseInterpolKindResultsRFInterpolPfCorrectTrueCpuTimeAvgPlain}{53.46414853841386\xspace}
\providecommand{\predicateBitpreciseInterpolKindResultsRFInterpolPfCorrectTrueCpuTimeAvgPlainHours}{}
  \renewcommand{\predicateBitpreciseInterpolKindResultsRFInterpolPfCorrectTrueCpuTimeAvgPlainHours}{0.014851152371781627\xspace}

  % inv-succ
\providecommand{\predicateBitpreciseInterpolKindResultsRFInterpolPfCorrectTrueInvSuccPlain}{}
  \renewcommand{\predicateBitpreciseInterpolKindResultsRFInterpolPfCorrectTrueInvSuccPlain}{911\xspace}

  % inv-tries
\providecommand{\predicateBitpreciseInterpolKindResultsRFInterpolPfCorrectTrueInvTriesPlain}{}
  \renewcommand{\predicateBitpreciseInterpolKindResultsRFInterpolPfCorrectTrueInvTriesPlain}{3910\xspace}

  % inv-time-sum
\providecommand{\predicateBitpreciseInterpolKindResultsRFInterpolPfCorrectTrueInvTimeSumPlain}{}
  \renewcommand{\predicateBitpreciseInterpolKindResultsRFInterpolPfCorrectTrueInvTimeSumPlain}{16887.53899999998\xspace}
\providecommand{\predicateBitpreciseInterpolKindResultsRFInterpolPfCorrectTrueInvTimeSumPlainHours}{}
  \renewcommand{\predicateBitpreciseInterpolKindResultsRFInterpolPfCorrectTrueInvTimeSumPlainHours}{4.69098305555555\xspace}

 %% incorrect-false %%
\providecommand{\predicateBitpreciseInterpolKindResultsRFInterpolPfIncorrectFalsePlain}{}
  \renewcommand{\predicateBitpreciseInterpolKindResultsRFInterpolPfIncorrectFalsePlain}{23\xspace}

  % cpu-time-sum
\providecommand{\predicateBitpreciseInterpolKindResultsRFInterpolPfIncorrectFalseCpuTimeSumPlain}{}
  \renewcommand{\predicateBitpreciseInterpolKindResultsRFInterpolPfIncorrectFalseCpuTimeSumPlain}{1066.767974097\xspace}
\providecommand{\predicateBitpreciseInterpolKindResultsRFInterpolPfIncorrectFalseCpuTimeSumPlainHours}{}
  \renewcommand{\predicateBitpreciseInterpolKindResultsRFInterpolPfIncorrectFalseCpuTimeSumPlainHours}{0.2963244372491667\xspace}

  % cpu-time-avg
\providecommand{\predicateBitpreciseInterpolKindResultsRFInterpolPfIncorrectFalseCpuTimeAvgPlain}{}
  \renewcommand{\predicateBitpreciseInterpolKindResultsRFInterpolPfIncorrectFalseCpuTimeAvgPlain}{46.381216265086955\xspace}
\providecommand{\predicateBitpreciseInterpolKindResultsRFInterpolPfIncorrectFalseCpuTimeAvgPlainHours}{}
  \renewcommand{\predicateBitpreciseInterpolKindResultsRFInterpolPfIncorrectFalseCpuTimeAvgPlainHours}{0.012883671184746377\xspace}

  % inv-succ
\providecommand{\predicateBitpreciseInterpolKindResultsRFInterpolPfIncorrectFalseInvSuccPlain}{}
  \renewcommand{\predicateBitpreciseInterpolKindResultsRFInterpolPfIncorrectFalseInvSuccPlain}{0\xspace}

  % inv-tries
\providecommand{\predicateBitpreciseInterpolKindResultsRFInterpolPfIncorrectFalseInvTriesPlain}{}
  \renewcommand{\predicateBitpreciseInterpolKindResultsRFInterpolPfIncorrectFalseInvTriesPlain}{70\xspace}

  % inv-time-sum
\providecommand{\predicateBitpreciseInterpolKindResultsRFInterpolPfIncorrectFalseInvTimeSumPlain}{}
  \renewcommand{\predicateBitpreciseInterpolKindResultsRFInterpolPfIncorrectFalseInvTimeSumPlain}{525.104\xspace}
\providecommand{\predicateBitpreciseInterpolKindResultsRFInterpolPfIncorrectFalseInvTimeSumPlainHours}{}
  \renewcommand{\predicateBitpreciseInterpolKindResultsRFInterpolPfIncorrectFalseInvTimeSumPlainHours}{0.14586222222222223\xspace}

 %% incorrect-true %%
\providecommand{\predicateBitpreciseInterpolKindResultsRFInterpolPfIncorrectTruePlain}{}
  \renewcommand{\predicateBitpreciseInterpolKindResultsRFInterpolPfIncorrectTruePlain}{1\xspace}

  % cpu-time-sum
\providecommand{\predicateBitpreciseInterpolKindResultsRFInterpolPfIncorrectTrueCpuTimeSumPlain}{}
  \renewcommand{\predicateBitpreciseInterpolKindResultsRFInterpolPfIncorrectTrueCpuTimeSumPlain}{97.25802761\xspace}
\providecommand{\predicateBitpreciseInterpolKindResultsRFInterpolPfIncorrectTrueCpuTimeSumPlainHours}{}
  \renewcommand{\predicateBitpreciseInterpolKindResultsRFInterpolPfIncorrectTrueCpuTimeSumPlainHours}{0.027016118780555556\xspace}

  % cpu-time-avg
\providecommand{\predicateBitpreciseInterpolKindResultsRFInterpolPfIncorrectTrueCpuTimeAvgPlain}{}
  \renewcommand{\predicateBitpreciseInterpolKindResultsRFInterpolPfIncorrectTrueCpuTimeAvgPlain}{97.25802761\xspace}
\providecommand{\predicateBitpreciseInterpolKindResultsRFInterpolPfIncorrectTrueCpuTimeAvgPlainHours}{}
  \renewcommand{\predicateBitpreciseInterpolKindResultsRFInterpolPfIncorrectTrueCpuTimeAvgPlainHours}{0.027016118780555556\xspace}

  % inv-succ
\providecommand{\predicateBitpreciseInterpolKindResultsRFInterpolPfIncorrectTrueInvSuccPlain}{}
  \renewcommand{\predicateBitpreciseInterpolKindResultsRFInterpolPfIncorrectTrueInvSuccPlain}{4\xspace}

  % inv-tries
\providecommand{\predicateBitpreciseInterpolKindResultsRFInterpolPfIncorrectTrueInvTriesPlain}{}
  \renewcommand{\predicateBitpreciseInterpolKindResultsRFInterpolPfIncorrectTrueInvTriesPlain}{10\xspace}

  % inv-time-sum
\providecommand{\predicateBitpreciseInterpolKindResultsRFInterpolPfIncorrectTrueInvTimeSumPlain}{}
  \renewcommand{\predicateBitpreciseInterpolKindResultsRFInterpolPfIncorrectTrueInvTimeSumPlain}{47.08\xspace}
\providecommand{\predicateBitpreciseInterpolKindResultsRFInterpolPfIncorrectTrueInvTimeSumPlainHours}{}
  \renewcommand{\predicateBitpreciseInterpolKindResultsRFInterpolPfIncorrectTrueInvTimeSumPlainHours}{0.013077777777777777\xspace}

 %% all %%
\providecommand{\predicateBitpreciseInterpolKindResultsRFInterpolPfAllPlain}{}
  \renewcommand{\predicateBitpreciseInterpolKindResultsRFInterpolPfAllPlain}{3488\xspace}

  % cpu-time-sum
\providecommand{\predicateBitpreciseInterpolKindResultsRFInterpolPfAllCpuTimeSumPlain}{}
  \renewcommand{\predicateBitpreciseInterpolKindResultsRFInterpolPfAllCpuTimeSumPlain}{589365.5693237552\xspace}
\providecommand{\predicateBitpreciseInterpolKindResultsRFInterpolPfAllCpuTimeSumPlainHours}{}
  \renewcommand{\predicateBitpreciseInterpolKindResultsRFInterpolPfAllCpuTimeSumPlainHours}{163.71265814548755\xspace}

  % cpu-time-avg
\providecommand{\predicateBitpreciseInterpolKindResultsRFInterpolPfAllCpuTimeAvgPlain}{}
  \renewcommand{\predicateBitpreciseInterpolKindResultsRFInterpolPfAllCpuTimeAvgPlain}{168.96948661804907\xspace}
\providecommand{\predicateBitpreciseInterpolKindResultsRFInterpolPfAllCpuTimeAvgPlainHours}{}
  \renewcommand{\predicateBitpreciseInterpolKindResultsRFInterpolPfAllCpuTimeAvgPlainHours}{0.04693596850501363\xspace}

  % inv-succ
\providecommand{\predicateBitpreciseInterpolKindResultsRFInterpolPfAllInvSuccPlain}{}
  \renewcommand{\predicateBitpreciseInterpolKindResultsRFInterpolPfAllInvSuccPlain}{1215\xspace}

  % inv-tries
\providecommand{\predicateBitpreciseInterpolKindResultsRFInterpolPfAllInvTriesPlain}{}
  \renewcommand{\predicateBitpreciseInterpolKindResultsRFInterpolPfAllInvTriesPlain}{14008\xspace}

  % inv-time-sum
\providecommand{\predicateBitpreciseInterpolKindResultsRFInterpolPfAllInvTimeSumPlain}{}
  \renewcommand{\predicateBitpreciseInterpolKindResultsRFInterpolPfAllInvTimeSumPlain}{187008.41399999976\xspace}
\providecommand{\predicateBitpreciseInterpolKindResultsRFInterpolPfAllInvTimeSumPlainHours}{}
  \renewcommand{\predicateBitpreciseInterpolKindResultsRFInterpolPfAllInvTimeSumPlainHours}{51.9467816666666\xspace}

 %% equal-only %%
\providecommand{\predicateBitpreciseInterpolKindResultsRFInterpolPfEqualOnlyPlain}{}
  \renewcommand{\predicateBitpreciseInterpolKindResultsRFInterpolPfEqualOnlyPlain}{1793\xspace}

  % cpu-time-sum
\providecommand{\predicateBitpreciseInterpolKindResultsRFInterpolPfEqualOnlyCpuTimeSumPlain}{}
  \renewcommand{\predicateBitpreciseInterpolKindResultsRFInterpolPfEqualOnlyCpuTimeSumPlain}{97476.66686951877\xspace}
\providecommand{\predicateBitpreciseInterpolKindResultsRFInterpolPfEqualOnlyCpuTimeSumPlainHours}{}
  \renewcommand{\predicateBitpreciseInterpolKindResultsRFInterpolPfEqualOnlyCpuTimeSumPlainHours}{27.076851908199657\xspace}

  % cpu-time-avg
\providecommand{\predicateBitpreciseInterpolKindResultsRFInterpolPfEqualOnlyCpuTimeAvgPlain}{}
  \renewcommand{\predicateBitpreciseInterpolKindResultsRFInterpolPfEqualOnlyCpuTimeAvgPlain}{54.36512374206289\xspace}
\providecommand{\predicateBitpreciseInterpolKindResultsRFInterpolPfEqualOnlyCpuTimeAvgPlainHours}{}
  \renewcommand{\predicateBitpreciseInterpolKindResultsRFInterpolPfEqualOnlyCpuTimeAvgPlainHours}{0.015101423261684137\xspace}

  % inv-succ
\providecommand{\predicateBitpreciseInterpolKindResultsRFInterpolPfEqualOnlyInvSuccPlain}{}
  \renewcommand{\predicateBitpreciseInterpolKindResultsRFInterpolPfEqualOnlyInvSuccPlain}{975\xspace}

  % inv-tries
\providecommand{\predicateBitpreciseInterpolKindResultsRFInterpolPfEqualOnlyInvTriesPlain}{}
  \renewcommand{\predicateBitpreciseInterpolKindResultsRFInterpolPfEqualOnlyInvTriesPlain}{6020\xspace}

  % inv-time-sum
\providecommand{\predicateBitpreciseInterpolKindResultsRFInterpolPfEqualOnlyInvTimeSumPlain}{}
  \renewcommand{\predicateBitpreciseInterpolKindResultsRFInterpolPfEqualOnlyInvTimeSumPlain}{22824.718000000037\xspace}
\providecommand{\predicateBitpreciseInterpolKindResultsRFInterpolPfEqualOnlyInvTimeSumPlainHours}{}
  \renewcommand{\predicateBitpreciseInterpolKindResultsRFInterpolPfEqualOnlyInvTimeSumPlainHours}{6.340199444444455\xspace}

%%% predicate_bitprecise_interpol_kind.2016-09-04_2044.results.RF_interpol-abs %%%
 %% correct %%
\providecommand{\predicateBitpreciseInterpolKindResultsRFInterpolAbsCorrectPlain}{}
  \renewcommand{\predicateBitpreciseInterpolKindResultsRFInterpolAbsCorrectPlain}{1821\xspace}

  % cpu-time-sum
\providecommand{\predicateBitpreciseInterpolKindResultsRFInterpolAbsCorrectCpuTimeSumPlain}{}
  \renewcommand{\predicateBitpreciseInterpolKindResultsRFInterpolAbsCorrectCpuTimeSumPlain}{103113.56238654381\xspace}
\providecommand{\predicateBitpreciseInterpolKindResultsRFInterpolAbsCorrectCpuTimeSumPlainHours}{}
  \renewcommand{\predicateBitpreciseInterpolKindResultsRFInterpolAbsCorrectCpuTimeSumPlainHours}{28.64265621848439\xspace}

  % cpu-time-avg
\providecommand{\predicateBitpreciseInterpolKindResultsRFInterpolAbsCorrectCpuTimeAvgPlain}{}
  \renewcommand{\predicateBitpreciseInterpolKindResultsRFInterpolAbsCorrectCpuTimeAvgPlain}{56.624691041484795\xspace}
\providecommand{\predicateBitpreciseInterpolKindResultsRFInterpolAbsCorrectCpuTimeAvgPlainHours}{}
  \renewcommand{\predicateBitpreciseInterpolKindResultsRFInterpolAbsCorrectCpuTimeAvgPlainHours}{0.01572908084485689\xspace}

  % inv-succ
\providecommand{\predicateBitpreciseInterpolKindResultsRFInterpolAbsCorrectInvSuccPlain}{}
  \renewcommand{\predicateBitpreciseInterpolKindResultsRFInterpolAbsCorrectInvSuccPlain}{1011\xspace}

  % inv-tries
\providecommand{\predicateBitpreciseInterpolKindResultsRFInterpolAbsCorrectInvTriesPlain}{}
  \renewcommand{\predicateBitpreciseInterpolKindResultsRFInterpolAbsCorrectInvTriesPlain}{6197\xspace}

  % inv-time-sum
\providecommand{\predicateBitpreciseInterpolKindResultsRFInterpolAbsCorrectInvTimeSumPlain}{}
  \renewcommand{\predicateBitpreciseInterpolKindResultsRFInterpolAbsCorrectInvTimeSumPlain}{24320.70199999998\xspace}
\providecommand{\predicateBitpreciseInterpolKindResultsRFInterpolAbsCorrectInvTimeSumPlainHours}{}
  \renewcommand{\predicateBitpreciseInterpolKindResultsRFInterpolAbsCorrectInvTimeSumPlainHours}{6.75575055555555\xspace}

 %% incorrect %%
\providecommand{\predicateBitpreciseInterpolKindResultsRFInterpolAbsIncorrectPlain}{}
  \renewcommand{\predicateBitpreciseInterpolKindResultsRFInterpolAbsIncorrectPlain}{24\xspace}

  % cpu-time-sum
\providecommand{\predicateBitpreciseInterpolKindResultsRFInterpolAbsIncorrectCpuTimeSumPlain}{}
  \renewcommand{\predicateBitpreciseInterpolKindResultsRFInterpolAbsIncorrectCpuTimeSumPlain}{1143.9101750309999\xspace}
\providecommand{\predicateBitpreciseInterpolKindResultsRFInterpolAbsIncorrectCpuTimeSumPlainHours}{}
  \renewcommand{\predicateBitpreciseInterpolKindResultsRFInterpolAbsIncorrectCpuTimeSumPlainHours}{0.31775282639749997\xspace}

  % cpu-time-avg
\providecommand{\predicateBitpreciseInterpolKindResultsRFInterpolAbsIncorrectCpuTimeAvgPlain}{}
  \renewcommand{\predicateBitpreciseInterpolKindResultsRFInterpolAbsIncorrectCpuTimeAvgPlain}{47.662923959625\xspace}
\providecommand{\predicateBitpreciseInterpolKindResultsRFInterpolAbsIncorrectCpuTimeAvgPlainHours}{}
  \renewcommand{\predicateBitpreciseInterpolKindResultsRFInterpolAbsIncorrectCpuTimeAvgPlainHours}{0.013239701099895832\xspace}

  % inv-succ
\providecommand{\predicateBitpreciseInterpolKindResultsRFInterpolAbsIncorrectInvSuccPlain}{}
  \renewcommand{\predicateBitpreciseInterpolKindResultsRFInterpolAbsIncorrectInvSuccPlain}{3\xspace}

  % inv-tries
\providecommand{\predicateBitpreciseInterpolKindResultsRFInterpolAbsIncorrectInvTriesPlain}{}
  \renewcommand{\predicateBitpreciseInterpolKindResultsRFInterpolAbsIncorrectInvTriesPlain}{77\xspace}

  % inv-time-sum
\providecommand{\predicateBitpreciseInterpolKindResultsRFInterpolAbsIncorrectInvTimeSumPlain}{}
  \renewcommand{\predicateBitpreciseInterpolKindResultsRFInterpolAbsIncorrectInvTimeSumPlain}{556.9110000000001\xspace}
\providecommand{\predicateBitpreciseInterpolKindResultsRFInterpolAbsIncorrectInvTimeSumPlainHours}{}
  \renewcommand{\predicateBitpreciseInterpolKindResultsRFInterpolAbsIncorrectInvTimeSumPlainHours}{0.15469750000000002\xspace}

 %% timeout %%
\providecommand{\predicateBitpreciseInterpolKindResultsRFInterpolAbsTimeoutPlain}{}
  \renewcommand{\predicateBitpreciseInterpolKindResultsRFInterpolAbsTimeoutPlain}{1559\xspace}

  % cpu-time-sum
\providecommand{\predicateBitpreciseInterpolKindResultsRFInterpolAbsTimeoutCpuTimeSumPlain}{}
  \renewcommand{\predicateBitpreciseInterpolKindResultsRFInterpolAbsTimeoutCpuTimeSumPlain}{477326.66322944494\xspace}
\providecommand{\predicateBitpreciseInterpolKindResultsRFInterpolAbsTimeoutCpuTimeSumPlainHours}{}
  \renewcommand{\predicateBitpreciseInterpolKindResultsRFInterpolAbsTimeoutCpuTimeSumPlainHours}{132.5907397859569\xspace}

  % cpu-time-avg
\providecommand{\predicateBitpreciseInterpolKindResultsRFInterpolAbsTimeoutCpuTimeAvgPlain}{}
  \renewcommand{\predicateBitpreciseInterpolKindResultsRFInterpolAbsTimeoutCpuTimeAvgPlain}{306.17489623440986\xspace}
\providecommand{\predicateBitpreciseInterpolKindResultsRFInterpolAbsTimeoutCpuTimeAvgPlainHours}{}
  \renewcommand{\predicateBitpreciseInterpolKindResultsRFInterpolAbsTimeoutCpuTimeAvgPlainHours}{0.08504858228733607\xspace}

  % inv-succ
\providecommand{\predicateBitpreciseInterpolKindResultsRFInterpolAbsTimeoutInvSuccPlain}{}
  \renewcommand{\predicateBitpreciseInterpolKindResultsRFInterpolAbsTimeoutInvSuccPlain}{206\xspace}

  % inv-tries
\providecommand{\predicateBitpreciseInterpolKindResultsRFInterpolAbsTimeoutInvTriesPlain}{}
  \renewcommand{\predicateBitpreciseInterpolKindResultsRFInterpolAbsTimeoutInvTriesPlain}{7461\xspace}

  % inv-time-sum
\providecommand{\predicateBitpreciseInterpolKindResultsRFInterpolAbsTimeoutInvTimeSumPlain}{}
  \renewcommand{\predicateBitpreciseInterpolKindResultsRFInterpolAbsTimeoutInvTimeSumPlain}{159277.78499999992\xspace}
\providecommand{\predicateBitpreciseInterpolKindResultsRFInterpolAbsTimeoutInvTimeSumPlainHours}{}
  \renewcommand{\predicateBitpreciseInterpolKindResultsRFInterpolAbsTimeoutInvTimeSumPlainHours}{44.24382916666664\xspace}

 %% unknown-or-category-error %%
\providecommand{\predicateBitpreciseInterpolKindResultsRFInterpolAbsUnknownOrCategoryErrorPlain}{}
  \renewcommand{\predicateBitpreciseInterpolKindResultsRFInterpolAbsUnknownOrCategoryErrorPlain}{1643\xspace}

  % cpu-time-sum
\providecommand{\predicateBitpreciseInterpolKindResultsRFInterpolAbsUnknownOrCategoryErrorCpuTimeSumPlain}{}
  \renewcommand{\predicateBitpreciseInterpolKindResultsRFInterpolAbsUnknownOrCategoryErrorCpuTimeSumPlain}{486319.3685770872\xspace}
\providecommand{\predicateBitpreciseInterpolKindResultsRFInterpolAbsUnknownOrCategoryErrorCpuTimeSumPlainHours}{}
  \renewcommand{\predicateBitpreciseInterpolKindResultsRFInterpolAbsUnknownOrCategoryErrorCpuTimeSumPlainHours}{135.08871349363534\xspace}

  % cpu-time-avg
\providecommand{\predicateBitpreciseInterpolKindResultsRFInterpolAbsUnknownOrCategoryErrorCpuTimeAvgPlain}{}
  \renewcommand{\predicateBitpreciseInterpolKindResultsRFInterpolAbsUnknownOrCategoryErrorCpuTimeAvgPlain}{295.99474654722286\xspace}
\providecommand{\predicateBitpreciseInterpolKindResultsRFInterpolAbsUnknownOrCategoryErrorCpuTimeAvgPlainHours}{}
  \renewcommand{\predicateBitpreciseInterpolKindResultsRFInterpolAbsUnknownOrCategoryErrorCpuTimeAvgPlainHours}{0.08222076292978413\xspace}

  % inv-succ
\providecommand{\predicateBitpreciseInterpolKindResultsRFInterpolAbsUnknownOrCategoryErrorInvSuccPlain}{}
  \renewcommand{\predicateBitpreciseInterpolKindResultsRFInterpolAbsUnknownOrCategoryErrorInvSuccPlain}{209\xspace}

  % inv-tries
\providecommand{\predicateBitpreciseInterpolKindResultsRFInterpolAbsUnknownOrCategoryErrorInvTriesPlain}{}
  \renewcommand{\predicateBitpreciseInterpolKindResultsRFInterpolAbsUnknownOrCategoryErrorInvTriesPlain}{7694\xspace}

  % inv-time-sum
\providecommand{\predicateBitpreciseInterpolKindResultsRFInterpolAbsUnknownOrCategoryErrorInvTimeSumPlain}{}
  \renewcommand{\predicateBitpreciseInterpolKindResultsRFInterpolAbsUnknownOrCategoryErrorInvTimeSumPlain}{161974.90399999992\xspace}
\providecommand{\predicateBitpreciseInterpolKindResultsRFInterpolAbsUnknownOrCategoryErrorInvTimeSumPlainHours}{}
  \renewcommand{\predicateBitpreciseInterpolKindResultsRFInterpolAbsUnknownOrCategoryErrorInvTimeSumPlainHours}{44.993028888888865\xspace}

 %% correct-false %%
\providecommand{\predicateBitpreciseInterpolKindResultsRFInterpolAbsCorrectFalsePlain}{}
  \renewcommand{\predicateBitpreciseInterpolKindResultsRFInterpolAbsCorrectFalsePlain}{482\xspace}

  % cpu-time-sum
\providecommand{\predicateBitpreciseInterpolKindResultsRFInterpolAbsCorrectFalseCpuTimeSumPlain}{}
  \renewcommand{\predicateBitpreciseInterpolKindResultsRFInterpolAbsCorrectFalseCpuTimeSumPlain}{31248.686382940024\xspace}
\providecommand{\predicateBitpreciseInterpolKindResultsRFInterpolAbsCorrectFalseCpuTimeSumPlainHours}{}
  \renewcommand{\predicateBitpreciseInterpolKindResultsRFInterpolAbsCorrectFalseCpuTimeSumPlainHours}{8.680190661927785\xspace}

  % cpu-time-avg
\providecommand{\predicateBitpreciseInterpolKindResultsRFInterpolAbsCorrectFalseCpuTimeAvgPlain}{}
  \renewcommand{\predicateBitpreciseInterpolKindResultsRFInterpolAbsCorrectFalseCpuTimeAvgPlain}{64.8312995496681\xspace}
\providecommand{\predicateBitpreciseInterpolKindResultsRFInterpolAbsCorrectFalseCpuTimeAvgPlainHours}{}
  \renewcommand{\predicateBitpreciseInterpolKindResultsRFInterpolAbsCorrectFalseCpuTimeAvgPlainHours}{0.01800869431935225\xspace}

  % inv-succ
\providecommand{\predicateBitpreciseInterpolKindResultsRFInterpolAbsCorrectFalseInvSuccPlain}{}
  \renewcommand{\predicateBitpreciseInterpolKindResultsRFInterpolAbsCorrectFalseInvSuccPlain}{91\xspace}

  % inv-tries
\providecommand{\predicateBitpreciseInterpolKindResultsRFInterpolAbsCorrectFalseInvTriesPlain}{}
  \renewcommand{\predicateBitpreciseInterpolKindResultsRFInterpolAbsCorrectFalseInvTriesPlain}{2265\xspace}

  % inv-time-sum
\providecommand{\predicateBitpreciseInterpolKindResultsRFInterpolAbsCorrectFalseInvTimeSumPlain}{}
  \renewcommand{\predicateBitpreciseInterpolKindResultsRFInterpolAbsCorrectFalseInvTimeSumPlain}{7155.508000000002\xspace}
\providecommand{\predicateBitpreciseInterpolKindResultsRFInterpolAbsCorrectFalseInvTimeSumPlainHours}{}
  \renewcommand{\predicateBitpreciseInterpolKindResultsRFInterpolAbsCorrectFalseInvTimeSumPlainHours}{1.9876411111111116\xspace}

 %% correct-true %%
\providecommand{\predicateBitpreciseInterpolKindResultsRFInterpolAbsCorrectTruePlain}{}
  \renewcommand{\predicateBitpreciseInterpolKindResultsRFInterpolAbsCorrectTruePlain}{1339\xspace}

  % cpu-time-sum
\providecommand{\predicateBitpreciseInterpolKindResultsRFInterpolAbsCorrectTrueCpuTimeSumPlain}{}
  \renewcommand{\predicateBitpreciseInterpolKindResultsRFInterpolAbsCorrectTrueCpuTimeSumPlain}{71864.87600360386\xspace}
\providecommand{\predicateBitpreciseInterpolKindResultsRFInterpolAbsCorrectTrueCpuTimeSumPlainHours}{}
  \renewcommand{\predicateBitpreciseInterpolKindResultsRFInterpolAbsCorrectTrueCpuTimeSumPlainHours}{19.962465556556626\xspace}

  % cpu-time-avg
\providecommand{\predicateBitpreciseInterpolKindResultsRFInterpolAbsCorrectTrueCpuTimeAvgPlain}{}
  \renewcommand{\predicateBitpreciseInterpolKindResultsRFInterpolAbsCorrectTrueCpuTimeAvgPlain}{53.67055713487966\xspace}
\providecommand{\predicateBitpreciseInterpolKindResultsRFInterpolAbsCorrectTrueCpuTimeAvgPlainHours}{}
  \renewcommand{\predicateBitpreciseInterpolKindResultsRFInterpolAbsCorrectTrueCpuTimeAvgPlainHours}{0.014908488093022127\xspace}

  % inv-succ
\providecommand{\predicateBitpreciseInterpolKindResultsRFInterpolAbsCorrectTrueInvSuccPlain}{}
  \renewcommand{\predicateBitpreciseInterpolKindResultsRFInterpolAbsCorrectTrueInvSuccPlain}{920\xspace}

  % inv-tries
\providecommand{\predicateBitpreciseInterpolKindResultsRFInterpolAbsCorrectTrueInvTriesPlain}{}
  \renewcommand{\predicateBitpreciseInterpolKindResultsRFInterpolAbsCorrectTrueInvTriesPlain}{3932\xspace}

  % inv-time-sum
\providecommand{\predicateBitpreciseInterpolKindResultsRFInterpolAbsCorrectTrueInvTimeSumPlain}{}
  \renewcommand{\predicateBitpreciseInterpolKindResultsRFInterpolAbsCorrectTrueInvTimeSumPlain}{17165.193999999985\xspace}
\providecommand{\predicateBitpreciseInterpolKindResultsRFInterpolAbsCorrectTrueInvTimeSumPlainHours}{}
  \renewcommand{\predicateBitpreciseInterpolKindResultsRFInterpolAbsCorrectTrueInvTimeSumPlainHours}{4.76810944444444\xspace}

 %% incorrect-false %%
\providecommand{\predicateBitpreciseInterpolKindResultsRFInterpolAbsIncorrectFalsePlain}{}
  \renewcommand{\predicateBitpreciseInterpolKindResultsRFInterpolAbsIncorrectFalsePlain}{23\xspace}

  % cpu-time-sum
\providecommand{\predicateBitpreciseInterpolKindResultsRFInterpolAbsIncorrectFalseCpuTimeSumPlain}{}
  \renewcommand{\predicateBitpreciseInterpolKindResultsRFInterpolAbsIncorrectFalseCpuTimeSumPlain}{1061.397500699\xspace}
\providecommand{\predicateBitpreciseInterpolKindResultsRFInterpolAbsIncorrectFalseCpuTimeSumPlainHours}{}
  \renewcommand{\predicateBitpreciseInterpolKindResultsRFInterpolAbsIncorrectFalseCpuTimeSumPlainHours}{0.2948326390830555\xspace}

  % cpu-time-avg
\providecommand{\predicateBitpreciseInterpolKindResultsRFInterpolAbsIncorrectFalseCpuTimeAvgPlain}{}
  \renewcommand{\predicateBitpreciseInterpolKindResultsRFInterpolAbsIncorrectFalseCpuTimeAvgPlain}{46.147717421695646\xspace}
\providecommand{\predicateBitpreciseInterpolKindResultsRFInterpolAbsIncorrectFalseCpuTimeAvgPlainHours}{}
  \renewcommand{\predicateBitpreciseInterpolKindResultsRFInterpolAbsIncorrectFalseCpuTimeAvgPlainHours}{0.012818810394915458\xspace}

  % inv-succ
\providecommand{\predicateBitpreciseInterpolKindResultsRFInterpolAbsIncorrectFalseInvSuccPlain}{}
  \renewcommand{\predicateBitpreciseInterpolKindResultsRFInterpolAbsIncorrectFalseInvSuccPlain}{0\xspace}

  % inv-tries
\providecommand{\predicateBitpreciseInterpolKindResultsRFInterpolAbsIncorrectFalseInvTriesPlain}{}
  \renewcommand{\predicateBitpreciseInterpolKindResultsRFInterpolAbsIncorrectFalseInvTriesPlain}{70\xspace}

  % inv-time-sum
\providecommand{\predicateBitpreciseInterpolKindResultsRFInterpolAbsIncorrectFalseInvTimeSumPlain}{}
  \renewcommand{\predicateBitpreciseInterpolKindResultsRFInterpolAbsIncorrectFalseInvTimeSumPlain}{522.5840000000001\xspace}
\providecommand{\predicateBitpreciseInterpolKindResultsRFInterpolAbsIncorrectFalseInvTimeSumPlainHours}{}
  \renewcommand{\predicateBitpreciseInterpolKindResultsRFInterpolAbsIncorrectFalseInvTimeSumPlainHours}{0.14516222222222225\xspace}

 %% incorrect-true %%
\providecommand{\predicateBitpreciseInterpolKindResultsRFInterpolAbsIncorrectTruePlain}{}
  \renewcommand{\predicateBitpreciseInterpolKindResultsRFInterpolAbsIncorrectTruePlain}{1\xspace}

  % cpu-time-sum
\providecommand{\predicateBitpreciseInterpolKindResultsRFInterpolAbsIncorrectTrueCpuTimeSumPlain}{}
  \renewcommand{\predicateBitpreciseInterpolKindResultsRFInterpolAbsIncorrectTrueCpuTimeSumPlain}{82.512674332\xspace}
\providecommand{\predicateBitpreciseInterpolKindResultsRFInterpolAbsIncorrectTrueCpuTimeSumPlainHours}{}
  \renewcommand{\predicateBitpreciseInterpolKindResultsRFInterpolAbsIncorrectTrueCpuTimeSumPlainHours}{0.022920187314444444\xspace}

  % cpu-time-avg
\providecommand{\predicateBitpreciseInterpolKindResultsRFInterpolAbsIncorrectTrueCpuTimeAvgPlain}{}
  \renewcommand{\predicateBitpreciseInterpolKindResultsRFInterpolAbsIncorrectTrueCpuTimeAvgPlain}{82.512674332\xspace}
\providecommand{\predicateBitpreciseInterpolKindResultsRFInterpolAbsIncorrectTrueCpuTimeAvgPlainHours}{}
  \renewcommand{\predicateBitpreciseInterpolKindResultsRFInterpolAbsIncorrectTrueCpuTimeAvgPlainHours}{0.022920187314444444\xspace}

  % inv-succ
\providecommand{\predicateBitpreciseInterpolKindResultsRFInterpolAbsIncorrectTrueInvSuccPlain}{}
  \renewcommand{\predicateBitpreciseInterpolKindResultsRFInterpolAbsIncorrectTrueInvSuccPlain}{3\xspace}

  % inv-tries
\providecommand{\predicateBitpreciseInterpolKindResultsRFInterpolAbsIncorrectTrueInvTriesPlain}{}
  \renewcommand{\predicateBitpreciseInterpolKindResultsRFInterpolAbsIncorrectTrueInvTriesPlain}{7\xspace}

  % inv-time-sum
\providecommand{\predicateBitpreciseInterpolKindResultsRFInterpolAbsIncorrectTrueInvTimeSumPlain}{}
  \renewcommand{\predicateBitpreciseInterpolKindResultsRFInterpolAbsIncorrectTrueInvTimeSumPlain}{34.327\xspace}
\providecommand{\predicateBitpreciseInterpolKindResultsRFInterpolAbsIncorrectTrueInvTimeSumPlainHours}{}
  \renewcommand{\predicateBitpreciseInterpolKindResultsRFInterpolAbsIncorrectTrueInvTimeSumPlainHours}{0.009535277777777778\xspace}

 %% all %%
\providecommand{\predicateBitpreciseInterpolKindResultsRFInterpolAbsAllPlain}{}
  \renewcommand{\predicateBitpreciseInterpolKindResultsRFInterpolAbsAllPlain}{3488\xspace}

  % cpu-time-sum
\providecommand{\predicateBitpreciseInterpolKindResultsRFInterpolAbsAllCpuTimeSumPlain}{}
  \renewcommand{\predicateBitpreciseInterpolKindResultsRFInterpolAbsAllCpuTimeSumPlain}{590576.841138662\xspace}
\providecommand{\predicateBitpreciseInterpolKindResultsRFInterpolAbsAllCpuTimeSumPlainHours}{}
  \renewcommand{\predicateBitpreciseInterpolKindResultsRFInterpolAbsAllCpuTimeSumPlainHours}{164.0491225385172\xspace}

  % cpu-time-avg
\providecommand{\predicateBitpreciseInterpolKindResultsRFInterpolAbsAllCpuTimeAvgPlain}{}
  \renewcommand{\predicateBitpreciseInterpolKindResultsRFInterpolAbsAllCpuTimeAvgPlain}{169.3167549136072\xspace}
\providecommand{\predicateBitpreciseInterpolKindResultsRFInterpolAbsAllCpuTimeAvgPlainHours}{}
  \renewcommand{\predicateBitpreciseInterpolKindResultsRFInterpolAbsAllCpuTimeAvgPlainHours}{0.04703243192044645\xspace}

  % inv-succ
\providecommand{\predicateBitpreciseInterpolKindResultsRFInterpolAbsAllInvSuccPlain}{}
  \renewcommand{\predicateBitpreciseInterpolKindResultsRFInterpolAbsAllInvSuccPlain}{1223\xspace}

  % inv-tries
\providecommand{\predicateBitpreciseInterpolKindResultsRFInterpolAbsAllInvTriesPlain}{}
  \renewcommand{\predicateBitpreciseInterpolKindResultsRFInterpolAbsAllInvTriesPlain}{13968\xspace}

  % inv-time-sum
\providecommand{\predicateBitpreciseInterpolKindResultsRFInterpolAbsAllInvTimeSumPlain}{}
  \renewcommand{\predicateBitpreciseInterpolKindResultsRFInterpolAbsAllInvTimeSumPlain}{186852.51699999964\xspace}
\providecommand{\predicateBitpreciseInterpolKindResultsRFInterpolAbsAllInvTimeSumPlainHours}{}
  \renewcommand{\predicateBitpreciseInterpolKindResultsRFInterpolAbsAllInvTimeSumPlainHours}{51.90347694444434\xspace}

 %% equal-only %%
\providecommand{\predicateBitpreciseInterpolKindResultsRFInterpolAbsEqualOnlyPlain}{}
  \renewcommand{\predicateBitpreciseInterpolKindResultsRFInterpolAbsEqualOnlyPlain}{1793\xspace}

  % cpu-time-sum
\providecommand{\predicateBitpreciseInterpolKindResultsRFInterpolAbsEqualOnlyCpuTimeSumPlain}{}
  \renewcommand{\predicateBitpreciseInterpolKindResultsRFInterpolAbsEqualOnlyCpuTimeSumPlain}{97044.03145303077\xspace}
\providecommand{\predicateBitpreciseInterpolKindResultsRFInterpolAbsEqualOnlyCpuTimeSumPlainHours}{}
  \renewcommand{\predicateBitpreciseInterpolKindResultsRFInterpolAbsEqualOnlyCpuTimeSumPlainHours}{26.956675403619656\xspace}

  % cpu-time-avg
\providecommand{\predicateBitpreciseInterpolKindResultsRFInterpolAbsEqualOnlyCpuTimeAvgPlain}{}
  \renewcommand{\predicateBitpreciseInterpolKindResultsRFInterpolAbsEqualOnlyCpuTimeAvgPlain}{54.123832377596635\xspace}
\providecommand{\predicateBitpreciseInterpolKindResultsRFInterpolAbsEqualOnlyCpuTimeAvgPlainHours}{}
  \renewcommand{\predicateBitpreciseInterpolKindResultsRFInterpolAbsEqualOnlyCpuTimeAvgPlainHours}{0.015034397882665732\xspace}

  % inv-succ
\providecommand{\predicateBitpreciseInterpolKindResultsRFInterpolAbsEqualOnlyInvSuccPlain}{}
  \renewcommand{\predicateBitpreciseInterpolKindResultsRFInterpolAbsEqualOnlyInvSuccPlain}{992\xspace}

  % inv-tries
\providecommand{\predicateBitpreciseInterpolKindResultsRFInterpolAbsEqualOnlyInvTriesPlain}{}
  \renewcommand{\predicateBitpreciseInterpolKindResultsRFInterpolAbsEqualOnlyInvTriesPlain}{6045\xspace}

  % inv-time-sum
\providecommand{\predicateBitpreciseInterpolKindResultsRFInterpolAbsEqualOnlyInvTimeSumPlain}{}
  \renewcommand{\predicateBitpreciseInterpolKindResultsRFInterpolAbsEqualOnlyInvTimeSumPlain}{22818.124999999985\xspace}
\providecommand{\predicateBitpreciseInterpolKindResultsRFInterpolAbsEqualOnlyInvTimeSumPlainHours}{}
  \renewcommand{\predicateBitpreciseInterpolKindResultsRFInterpolAbsEqualOnlyInvTimeSumPlainHours}{6.338368055555551\xspace}

%%% predicate_bitprecise_interpol_kind.2016-09-04_2044.results.RF_interpol-abs-prec %%%
 %% correct %%
\providecommand{\predicateBitpreciseInterpolKindResultsRFInterpolAbsPrecCorrectPlain}{}
  \renewcommand{\predicateBitpreciseInterpolKindResultsRFInterpolAbsPrecCorrectPlain}{1814\xspace}

  % cpu-time-sum
\providecommand{\predicateBitpreciseInterpolKindResultsRFInterpolAbsPrecCorrectCpuTimeSumPlain}{}
  \renewcommand{\predicateBitpreciseInterpolKindResultsRFInterpolAbsPrecCorrectCpuTimeSumPlain}{102643.43669037503\xspace}
\providecommand{\predicateBitpreciseInterpolKindResultsRFInterpolAbsPrecCorrectCpuTimeSumPlainHours}{}
  \renewcommand{\predicateBitpreciseInterpolKindResultsRFInterpolAbsPrecCorrectCpuTimeSumPlainHours}{28.5120657473264\xspace}

  % cpu-time-avg
\providecommand{\predicateBitpreciseInterpolKindResultsRFInterpolAbsPrecCorrectCpuTimeAvgPlain}{}
  \renewcommand{\predicateBitpreciseInterpolKindResultsRFInterpolAbsPrecCorrectCpuTimeAvgPlain}{56.58403345665658\xspace}
\providecommand{\predicateBitpreciseInterpolKindResultsRFInterpolAbsPrecCorrectCpuTimeAvgPlainHours}{}
  \renewcommand{\predicateBitpreciseInterpolKindResultsRFInterpolAbsPrecCorrectCpuTimeAvgPlainHours}{0.015717787071293492\xspace}

  % inv-succ
\providecommand{\predicateBitpreciseInterpolKindResultsRFInterpolAbsPrecCorrectInvSuccPlain}{}
  \renewcommand{\predicateBitpreciseInterpolKindResultsRFInterpolAbsPrecCorrectInvSuccPlain}{1162\xspace}

  % inv-tries
\providecommand{\predicateBitpreciseInterpolKindResultsRFInterpolAbsPrecCorrectInvTriesPlain}{}
  \renewcommand{\predicateBitpreciseInterpolKindResultsRFInterpolAbsPrecCorrectInvTriesPlain}{6293\xspace}

  % inv-time-sum
\providecommand{\predicateBitpreciseInterpolKindResultsRFInterpolAbsPrecCorrectInvTimeSumPlain}{}
  \renewcommand{\predicateBitpreciseInterpolKindResultsRFInterpolAbsPrecCorrectInvTimeSumPlain}{24129.38400000005\xspace}
\providecommand{\predicateBitpreciseInterpolKindResultsRFInterpolAbsPrecCorrectInvTimeSumPlainHours}{}
  \renewcommand{\predicateBitpreciseInterpolKindResultsRFInterpolAbsPrecCorrectInvTimeSumPlainHours}{6.70260666666668\xspace}

 %% incorrect %%
\providecommand{\predicateBitpreciseInterpolKindResultsRFInterpolAbsPrecIncorrectPlain}{}
  \renewcommand{\predicateBitpreciseInterpolKindResultsRFInterpolAbsPrecIncorrectPlain}{24\xspace}

  % cpu-time-sum
\providecommand{\predicateBitpreciseInterpolKindResultsRFInterpolAbsPrecIncorrectCpuTimeSumPlain}{}
  \renewcommand{\predicateBitpreciseInterpolKindResultsRFInterpolAbsPrecIncorrectCpuTimeSumPlain}{1180.6085590460002\xspace}
\providecommand{\predicateBitpreciseInterpolKindResultsRFInterpolAbsPrecIncorrectCpuTimeSumPlainHours}{}
  \renewcommand{\predicateBitpreciseInterpolKindResultsRFInterpolAbsPrecIncorrectCpuTimeSumPlainHours}{0.3279468219572223\xspace}

  % cpu-time-avg
\providecommand{\predicateBitpreciseInterpolKindResultsRFInterpolAbsPrecIncorrectCpuTimeAvgPlain}{}
  \renewcommand{\predicateBitpreciseInterpolKindResultsRFInterpolAbsPrecIncorrectCpuTimeAvgPlain}{49.19202329358334\xspace}
\providecommand{\predicateBitpreciseInterpolKindResultsRFInterpolAbsPrecIncorrectCpuTimeAvgPlainHours}{}
  \renewcommand{\predicateBitpreciseInterpolKindResultsRFInterpolAbsPrecIncorrectCpuTimeAvgPlainHours}{0.01366445091488426\xspace}

  % inv-succ
\providecommand{\predicateBitpreciseInterpolKindResultsRFInterpolAbsPrecIncorrectInvSuccPlain}{}
  \renewcommand{\predicateBitpreciseInterpolKindResultsRFInterpolAbsPrecIncorrectInvSuccPlain}{4\xspace}

  % inv-tries
\providecommand{\predicateBitpreciseInterpolKindResultsRFInterpolAbsPrecIncorrectInvTriesPlain}{}
  \renewcommand{\predicateBitpreciseInterpolKindResultsRFInterpolAbsPrecIncorrectInvTriesPlain}{79\xspace}

  % inv-time-sum
\providecommand{\predicateBitpreciseInterpolKindResultsRFInterpolAbsPrecIncorrectInvTimeSumPlain}{}
  \renewcommand{\predicateBitpreciseInterpolKindResultsRFInterpolAbsPrecIncorrectInvTimeSumPlain}{583.0209999999998\xspace}
\providecommand{\predicateBitpreciseInterpolKindResultsRFInterpolAbsPrecIncorrectInvTimeSumPlainHours}{}
  \renewcommand{\predicateBitpreciseInterpolKindResultsRFInterpolAbsPrecIncorrectInvTimeSumPlainHours}{0.16195027777777773\xspace}

 %% timeout %%
\providecommand{\predicateBitpreciseInterpolKindResultsRFInterpolAbsPrecTimeoutPlain}{}
  \renewcommand{\predicateBitpreciseInterpolKindResultsRFInterpolAbsPrecTimeoutPlain}{1561\xspace}

  % cpu-time-sum
\providecommand{\predicateBitpreciseInterpolKindResultsRFInterpolAbsPrecTimeoutCpuTimeSumPlain}{}
  \renewcommand{\predicateBitpreciseInterpolKindResultsRFInterpolAbsPrecTimeoutCpuTimeSumPlain}{477821.35815596714\xspace}
\providecommand{\predicateBitpreciseInterpolKindResultsRFInterpolAbsPrecTimeoutCpuTimeSumPlainHours}{}
  \renewcommand{\predicateBitpreciseInterpolKindResultsRFInterpolAbsPrecTimeoutCpuTimeSumPlainHours}{132.7281550433242\xspace}

  % cpu-time-avg
\providecommand{\predicateBitpreciseInterpolKindResultsRFInterpolAbsPrecTimeoutCpuTimeAvgPlain}{}
  \renewcommand{\predicateBitpreciseInterpolKindResultsRFInterpolAbsPrecTimeoutCpuTimeAvgPlain}{306.099524763592\xspace}
\providecommand{\predicateBitpreciseInterpolKindResultsRFInterpolAbsPrecTimeoutCpuTimeAvgPlainHours}{}
  \renewcommand{\predicateBitpreciseInterpolKindResultsRFInterpolAbsPrecTimeoutCpuTimeAvgPlainHours}{0.08502764576766444\xspace}

  % inv-succ
\providecommand{\predicateBitpreciseInterpolKindResultsRFInterpolAbsPrecTimeoutInvSuccPlain}{}
  \renewcommand{\predicateBitpreciseInterpolKindResultsRFInterpolAbsPrecTimeoutInvSuccPlain}{229\xspace}

  % inv-tries
\providecommand{\predicateBitpreciseInterpolKindResultsRFInterpolAbsPrecTimeoutInvTriesPlain}{}
  \renewcommand{\predicateBitpreciseInterpolKindResultsRFInterpolAbsPrecTimeoutInvTriesPlain}{7421\xspace}

  % inv-time-sum
\providecommand{\predicateBitpreciseInterpolKindResultsRFInterpolAbsPrecTimeoutInvTimeSumPlain}{}
  \renewcommand{\predicateBitpreciseInterpolKindResultsRFInterpolAbsPrecTimeoutInvTimeSumPlain}{157970.36299999998\xspace}
\providecommand{\predicateBitpreciseInterpolKindResultsRFInterpolAbsPrecTimeoutInvTimeSumPlainHours}{}
  \renewcommand{\predicateBitpreciseInterpolKindResultsRFInterpolAbsPrecTimeoutInvTimeSumPlainHours}{43.88065638888889\xspace}

 %% unknown-or-category-error %%
\providecommand{\predicateBitpreciseInterpolKindResultsRFInterpolAbsPrecUnknownOrCategoryErrorPlain}{}
  \renewcommand{\predicateBitpreciseInterpolKindResultsRFInterpolAbsPrecUnknownOrCategoryErrorPlain}{1650\xspace}

  % cpu-time-sum
\providecommand{\predicateBitpreciseInterpolKindResultsRFInterpolAbsPrecUnknownOrCategoryErrorCpuTimeSumPlain}{}
  \renewcommand{\predicateBitpreciseInterpolKindResultsRFInterpolAbsPrecUnknownOrCategoryErrorCpuTimeSumPlain}{488423.3743851963\xspace}
\providecommand{\predicateBitpreciseInterpolKindResultsRFInterpolAbsPrecUnknownOrCategoryErrorCpuTimeSumPlainHours}{}
  \renewcommand{\predicateBitpreciseInterpolKindResultsRFInterpolAbsPrecUnknownOrCategoryErrorCpuTimeSumPlainHours}{135.6731595514434\xspace}

  % cpu-time-avg
\providecommand{\predicateBitpreciseInterpolKindResultsRFInterpolAbsPrecUnknownOrCategoryErrorCpuTimeAvgPlain}{}
  \renewcommand{\predicateBitpreciseInterpolKindResultsRFInterpolAbsPrecUnknownOrCategoryErrorCpuTimeAvgPlain}{296.0141662940584\xspace}
\providecommand{\predicateBitpreciseInterpolKindResultsRFInterpolAbsPrecUnknownOrCategoryErrorCpuTimeAvgPlainHours}{}
  \renewcommand{\predicateBitpreciseInterpolKindResultsRFInterpolAbsPrecUnknownOrCategoryErrorCpuTimeAvgPlainHours}{0.0822261573039051\xspace}

  % inv-succ
\providecommand{\predicateBitpreciseInterpolKindResultsRFInterpolAbsPrecUnknownOrCategoryErrorInvSuccPlain}{}
  \renewcommand{\predicateBitpreciseInterpolKindResultsRFInterpolAbsPrecUnknownOrCategoryErrorInvSuccPlain}{233\xspace}

  % inv-tries
\providecommand{\predicateBitpreciseInterpolKindResultsRFInterpolAbsPrecUnknownOrCategoryErrorInvTriesPlain}{}
  \renewcommand{\predicateBitpreciseInterpolKindResultsRFInterpolAbsPrecUnknownOrCategoryErrorInvTriesPlain}{7675\xspace}

  % inv-time-sum
\providecommand{\predicateBitpreciseInterpolKindResultsRFInterpolAbsPrecUnknownOrCategoryErrorInvTimeSumPlain}{}
  \renewcommand{\predicateBitpreciseInterpolKindResultsRFInterpolAbsPrecUnknownOrCategoryErrorInvTimeSumPlain}{160929.04100000003\xspace}
\providecommand{\predicateBitpreciseInterpolKindResultsRFInterpolAbsPrecUnknownOrCategoryErrorInvTimeSumPlainHours}{}
  \renewcommand{\predicateBitpreciseInterpolKindResultsRFInterpolAbsPrecUnknownOrCategoryErrorInvTimeSumPlainHours}{44.702511388888894\xspace}

 %% correct-false %%
\providecommand{\predicateBitpreciseInterpolKindResultsRFInterpolAbsPrecCorrectFalsePlain}{}
  \renewcommand{\predicateBitpreciseInterpolKindResultsRFInterpolAbsPrecCorrectFalsePlain}{481\xspace}

  % cpu-time-sum
\providecommand{\predicateBitpreciseInterpolKindResultsRFInterpolAbsPrecCorrectFalseCpuTimeSumPlain}{}
  \renewcommand{\predicateBitpreciseInterpolKindResultsRFInterpolAbsPrecCorrectFalseCpuTimeSumPlain}{31305.48169731799\xspace}
\providecommand{\predicateBitpreciseInterpolKindResultsRFInterpolAbsPrecCorrectFalseCpuTimeSumPlainHours}{}
  \renewcommand{\predicateBitpreciseInterpolKindResultsRFInterpolAbsPrecCorrectFalseCpuTimeSumPlainHours}{8.695967138143887\xspace}

  % cpu-time-avg
\providecommand{\predicateBitpreciseInterpolKindResultsRFInterpolAbsPrecCorrectFalseCpuTimeAvgPlain}{}
  \renewcommand{\predicateBitpreciseInterpolKindResultsRFInterpolAbsPrecCorrectFalseCpuTimeAvgPlain}{65.08416153288563\xspace}
\providecommand{\predicateBitpreciseInterpolKindResultsRFInterpolAbsPrecCorrectFalseCpuTimeAvgPlainHours}{}
  \renewcommand{\predicateBitpreciseInterpolKindResultsRFInterpolAbsPrecCorrectFalseCpuTimeAvgPlainHours}{0.018078933759134896\xspace}

  % inv-succ
\providecommand{\predicateBitpreciseInterpolKindResultsRFInterpolAbsPrecCorrectFalseInvSuccPlain}{}
  \renewcommand{\predicateBitpreciseInterpolKindResultsRFInterpolAbsPrecCorrectFalseInvSuccPlain}{128\xspace}

  % inv-tries
\providecommand{\predicateBitpreciseInterpolKindResultsRFInterpolAbsPrecCorrectFalseInvTriesPlain}{}
  \renewcommand{\predicateBitpreciseInterpolKindResultsRFInterpolAbsPrecCorrectFalseInvTriesPlain}{2299\xspace}

  % inv-time-sum
\providecommand{\predicateBitpreciseInterpolKindResultsRFInterpolAbsPrecCorrectFalseInvTimeSumPlain}{}
  \renewcommand{\predicateBitpreciseInterpolKindResultsRFInterpolAbsPrecCorrectFalseInvTimeSumPlain}{7160.593999999996\xspace}
\providecommand{\predicateBitpreciseInterpolKindResultsRFInterpolAbsPrecCorrectFalseInvTimeSumPlainHours}{}
  \renewcommand{\predicateBitpreciseInterpolKindResultsRFInterpolAbsPrecCorrectFalseInvTimeSumPlainHours}{1.9890538888888878\xspace}

 %% correct-true %%
\providecommand{\predicateBitpreciseInterpolKindResultsRFInterpolAbsPrecCorrectTruePlain}{}
  \renewcommand{\predicateBitpreciseInterpolKindResultsRFInterpolAbsPrecCorrectTruePlain}{1333\xspace}

  % cpu-time-sum
\providecommand{\predicateBitpreciseInterpolKindResultsRFInterpolAbsPrecCorrectTrueCpuTimeSumPlain}{}
  \renewcommand{\predicateBitpreciseInterpolKindResultsRFInterpolAbsPrecCorrectTrueCpuTimeSumPlain}{71337.9549930569\xspace}
\providecommand{\predicateBitpreciseInterpolKindResultsRFInterpolAbsPrecCorrectTrueCpuTimeSumPlainHours}{}
  \renewcommand{\predicateBitpreciseInterpolKindResultsRFInterpolAbsPrecCorrectTrueCpuTimeSumPlainHours}{19.816098609182472\xspace}

  % cpu-time-avg
\providecommand{\predicateBitpreciseInterpolKindResultsRFInterpolAbsPrecCorrectTrueCpuTimeAvgPlain}{}
  \renewcommand{\predicateBitpreciseInterpolKindResultsRFInterpolAbsPrecCorrectTrueCpuTimeAvgPlain}{53.516845456156716\xspace}
\providecommand{\predicateBitpreciseInterpolKindResultsRFInterpolAbsPrecCorrectTrueCpuTimeAvgPlainHours}{}
  \renewcommand{\predicateBitpreciseInterpolKindResultsRFInterpolAbsPrecCorrectTrueCpuTimeAvgPlainHours}{0.014865790404487976\xspace}

  % inv-succ
\providecommand{\predicateBitpreciseInterpolKindResultsRFInterpolAbsPrecCorrectTrueInvSuccPlain}{}
  \renewcommand{\predicateBitpreciseInterpolKindResultsRFInterpolAbsPrecCorrectTrueInvSuccPlain}{1034\xspace}

  % inv-tries
\providecommand{\predicateBitpreciseInterpolKindResultsRFInterpolAbsPrecCorrectTrueInvTriesPlain}{}
  \renewcommand{\predicateBitpreciseInterpolKindResultsRFInterpolAbsPrecCorrectTrueInvTriesPlain}{3994\xspace}

  % inv-time-sum
\providecommand{\predicateBitpreciseInterpolKindResultsRFInterpolAbsPrecCorrectTrueInvTimeSumPlain}{}
  \renewcommand{\predicateBitpreciseInterpolKindResultsRFInterpolAbsPrecCorrectTrueInvTimeSumPlain}{16968.79000000001\xspace}
\providecommand{\predicateBitpreciseInterpolKindResultsRFInterpolAbsPrecCorrectTrueInvTimeSumPlainHours}{}
  \renewcommand{\predicateBitpreciseInterpolKindResultsRFInterpolAbsPrecCorrectTrueInvTimeSumPlainHours}{4.713552777777781\xspace}

 %% incorrect-false %%
\providecommand{\predicateBitpreciseInterpolKindResultsRFInterpolAbsPrecIncorrectFalsePlain}{}
  \renewcommand{\predicateBitpreciseInterpolKindResultsRFInterpolAbsPrecIncorrectFalsePlain}{23\xspace}

  % cpu-time-sum
\providecommand{\predicateBitpreciseInterpolKindResultsRFInterpolAbsPrecIncorrectFalseCpuTimeSumPlain}{}
  \renewcommand{\predicateBitpreciseInterpolKindResultsRFInterpolAbsPrecIncorrectFalseCpuTimeSumPlain}{1085.2662659830003\xspace}
\providecommand{\predicateBitpreciseInterpolKindResultsRFInterpolAbsPrecIncorrectFalseCpuTimeSumPlainHours}{}
  \renewcommand{\predicateBitpreciseInterpolKindResultsRFInterpolAbsPrecIncorrectFalseCpuTimeSumPlainHours}{0.30146285166194453\xspace}

  % cpu-time-avg
\providecommand{\predicateBitpreciseInterpolKindResultsRFInterpolAbsPrecIncorrectFalseCpuTimeAvgPlain}{}
  \renewcommand{\predicateBitpreciseInterpolKindResultsRFInterpolAbsPrecIncorrectFalseCpuTimeAvgPlain}{47.18548982534784\xspace}
\providecommand{\predicateBitpreciseInterpolKindResultsRFInterpolAbsPrecIncorrectFalseCpuTimeAvgPlainHours}{}
  \renewcommand{\predicateBitpreciseInterpolKindResultsRFInterpolAbsPrecIncorrectFalseCpuTimeAvgPlainHours}{0.013107080507041066\xspace}

  % inv-succ
\providecommand{\predicateBitpreciseInterpolKindResultsRFInterpolAbsPrecIncorrectFalseInvSuccPlain}{}
  \renewcommand{\predicateBitpreciseInterpolKindResultsRFInterpolAbsPrecIncorrectFalseInvSuccPlain}{0\xspace}

  % inv-tries
\providecommand{\predicateBitpreciseInterpolKindResultsRFInterpolAbsPrecIncorrectFalseInvTriesPlain}{}
  \renewcommand{\predicateBitpreciseInterpolKindResultsRFInterpolAbsPrecIncorrectFalseInvTriesPlain}{70\xspace}

  % inv-time-sum
\providecommand{\predicateBitpreciseInterpolKindResultsRFInterpolAbsPrecIncorrectFalseInvTimeSumPlain}{}
  \renewcommand{\predicateBitpreciseInterpolKindResultsRFInterpolAbsPrecIncorrectFalseInvTimeSumPlain}{537.8379999999999\xspace}
\providecommand{\predicateBitpreciseInterpolKindResultsRFInterpolAbsPrecIncorrectFalseInvTimeSumPlainHours}{}
  \renewcommand{\predicateBitpreciseInterpolKindResultsRFInterpolAbsPrecIncorrectFalseInvTimeSumPlainHours}{0.1493994444444444\xspace}

 %% incorrect-true %%
\providecommand{\predicateBitpreciseInterpolKindResultsRFInterpolAbsPrecIncorrectTruePlain}{}
  \renewcommand{\predicateBitpreciseInterpolKindResultsRFInterpolAbsPrecIncorrectTruePlain}{1\xspace}

  % cpu-time-sum
\providecommand{\predicateBitpreciseInterpolKindResultsRFInterpolAbsPrecIncorrectTrueCpuTimeSumPlain}{}
  \renewcommand{\predicateBitpreciseInterpolKindResultsRFInterpolAbsPrecIncorrectTrueCpuTimeSumPlain}{95.342293063\xspace}
\providecommand{\predicateBitpreciseInterpolKindResultsRFInterpolAbsPrecIncorrectTrueCpuTimeSumPlainHours}{}
  \renewcommand{\predicateBitpreciseInterpolKindResultsRFInterpolAbsPrecIncorrectTrueCpuTimeSumPlainHours}{0.026483970295277777\xspace}

  % cpu-time-avg
\providecommand{\predicateBitpreciseInterpolKindResultsRFInterpolAbsPrecIncorrectTrueCpuTimeAvgPlain}{}
  \renewcommand{\predicateBitpreciseInterpolKindResultsRFInterpolAbsPrecIncorrectTrueCpuTimeAvgPlain}{95.342293063\xspace}
\providecommand{\predicateBitpreciseInterpolKindResultsRFInterpolAbsPrecIncorrectTrueCpuTimeAvgPlainHours}{}
  \renewcommand{\predicateBitpreciseInterpolKindResultsRFInterpolAbsPrecIncorrectTrueCpuTimeAvgPlainHours}{0.026483970295277777\xspace}

  % inv-succ
\providecommand{\predicateBitpreciseInterpolKindResultsRFInterpolAbsPrecIncorrectTrueInvSuccPlain}{}
  \renewcommand{\predicateBitpreciseInterpolKindResultsRFInterpolAbsPrecIncorrectTrueInvSuccPlain}{4\xspace}

  % inv-tries
\providecommand{\predicateBitpreciseInterpolKindResultsRFInterpolAbsPrecIncorrectTrueInvTriesPlain}{}
  \renewcommand{\predicateBitpreciseInterpolKindResultsRFInterpolAbsPrecIncorrectTrueInvTriesPlain}{9\xspace}

  % inv-time-sum
\providecommand{\predicateBitpreciseInterpolKindResultsRFInterpolAbsPrecIncorrectTrueInvTimeSumPlain}{}
  \renewcommand{\predicateBitpreciseInterpolKindResultsRFInterpolAbsPrecIncorrectTrueInvTimeSumPlain}{45.183\xspace}
\providecommand{\predicateBitpreciseInterpolKindResultsRFInterpolAbsPrecIncorrectTrueInvTimeSumPlainHours}{}
  \renewcommand{\predicateBitpreciseInterpolKindResultsRFInterpolAbsPrecIncorrectTrueInvTimeSumPlainHours}{0.012550833333333334\xspace}

 %% all %%
\providecommand{\predicateBitpreciseInterpolKindResultsRFInterpolAbsPrecAllPlain}{}
  \renewcommand{\predicateBitpreciseInterpolKindResultsRFInterpolAbsPrecAllPlain}{3488\xspace}

  % cpu-time-sum
\providecommand{\predicateBitpreciseInterpolKindResultsRFInterpolAbsPrecAllCpuTimeSumPlain}{}
  \renewcommand{\predicateBitpreciseInterpolKindResultsRFInterpolAbsPrecAllCpuTimeSumPlain}{592247.419634618\xspace}
\providecommand{\predicateBitpreciseInterpolKindResultsRFInterpolAbsPrecAllCpuTimeSumPlainHours}{}
  \renewcommand{\predicateBitpreciseInterpolKindResultsRFInterpolAbsPrecAllCpuTimeSumPlainHours}{164.5131721207272\xspace}

  % cpu-time-avg
\providecommand{\predicateBitpreciseInterpolKindResultsRFInterpolAbsPrecAllCpuTimeAvgPlain}{}
  \renewcommand{\predicateBitpreciseInterpolKindResultsRFInterpolAbsPrecAllCpuTimeAvgPlain}{169.79570517047534\xspace}
\providecommand{\predicateBitpreciseInterpolKindResultsRFInterpolAbsPrecAllCpuTimeAvgPlainHours}{}
  \renewcommand{\predicateBitpreciseInterpolKindResultsRFInterpolAbsPrecAllCpuTimeAvgPlainHours}{0.04716547365846537\xspace}

  % inv-succ
\providecommand{\predicateBitpreciseInterpolKindResultsRFInterpolAbsPrecAllInvSuccPlain}{}
  \renewcommand{\predicateBitpreciseInterpolKindResultsRFInterpolAbsPrecAllInvSuccPlain}{1399\xspace}

  % inv-tries
\providecommand{\predicateBitpreciseInterpolKindResultsRFInterpolAbsPrecAllInvTriesPlain}{}
  \renewcommand{\predicateBitpreciseInterpolKindResultsRFInterpolAbsPrecAllInvTriesPlain}{14047\xspace}

  % inv-time-sum
\providecommand{\predicateBitpreciseInterpolKindResultsRFInterpolAbsPrecAllInvTimeSumPlain}{}
  \renewcommand{\predicateBitpreciseInterpolKindResultsRFInterpolAbsPrecAllInvTimeSumPlain}{185641.4459999999\xspace}
\providecommand{\predicateBitpreciseInterpolKindResultsRFInterpolAbsPrecAllInvTimeSumPlainHours}{}
  \renewcommand{\predicateBitpreciseInterpolKindResultsRFInterpolAbsPrecAllInvTimeSumPlainHours}{51.56706833333331\xspace}

 %% equal-only %%
\providecommand{\predicateBitpreciseInterpolKindResultsRFInterpolAbsPrecEqualOnlyPlain}{}
  \renewcommand{\predicateBitpreciseInterpolKindResultsRFInterpolAbsPrecEqualOnlyPlain}{1793\xspace}

  % cpu-time-sum
\providecommand{\predicateBitpreciseInterpolKindResultsRFInterpolAbsPrecEqualOnlyCpuTimeSumPlain}{}
  \renewcommand{\predicateBitpreciseInterpolKindResultsRFInterpolAbsPrecEqualOnlyCpuTimeSumPlain}{98463.57439198413\xspace}
\providecommand{\predicateBitpreciseInterpolKindResultsRFInterpolAbsPrecEqualOnlyCpuTimeSumPlainHours}{}
  \renewcommand{\predicateBitpreciseInterpolKindResultsRFInterpolAbsPrecEqualOnlyCpuTimeSumPlainHours}{27.350992886662258\xspace}

  % cpu-time-avg
\providecommand{\predicateBitpreciseInterpolKindResultsRFInterpolAbsPrecEqualOnlyCpuTimeAvgPlain}{}
  \renewcommand{\predicateBitpreciseInterpolKindResultsRFInterpolAbsPrecEqualOnlyCpuTimeAvgPlain}{54.915546230889085\xspace}
\providecommand{\predicateBitpreciseInterpolKindResultsRFInterpolAbsPrecEqualOnlyCpuTimeAvgPlainHours}{}
  \renewcommand{\predicateBitpreciseInterpolKindResultsRFInterpolAbsPrecEqualOnlyCpuTimeAvgPlainHours}{0.01525431839746919\xspace}

  % inv-succ
\providecommand{\predicateBitpreciseInterpolKindResultsRFInterpolAbsPrecEqualOnlyInvSuccPlain}{}
  \renewcommand{\predicateBitpreciseInterpolKindResultsRFInterpolAbsPrecEqualOnlyInvSuccPlain}{1146\xspace}

  % inv-tries
\providecommand{\predicateBitpreciseInterpolKindResultsRFInterpolAbsPrecEqualOnlyInvTriesPlain}{}
  \renewcommand{\predicateBitpreciseInterpolKindResultsRFInterpolAbsPrecEqualOnlyInvTriesPlain}{6201\xspace}

  % inv-time-sum
\providecommand{\predicateBitpreciseInterpolKindResultsRFInterpolAbsPrecEqualOnlyInvTimeSumPlain}{}
  \renewcommand{\predicateBitpreciseInterpolKindResultsRFInterpolAbsPrecEqualOnlyInvTimeSumPlain}{23130.133000000023\xspace}
\providecommand{\predicateBitpreciseInterpolKindResultsRFInterpolAbsPrecEqualOnlyInvTimeSumPlainHours}{}
  \renewcommand{\predicateBitpreciseInterpolKindResultsRFInterpolAbsPrecEqualOnlyInvTimeSumPlainHours}{6.425036944444451\xspace}

%%% predicate_bitprecise_interpol_kind.2016-09-04_2044.results.RF_interpol-abs-prec-pf %%%
 %% correct %%
\providecommand{\predicateBitpreciseInterpolKindResultsRFInterpolAbsPrecPfCorrectPlain}{}
  \renewcommand{\predicateBitpreciseInterpolKindResultsRFInterpolAbsPrecPfCorrectPlain}{1822\xspace}

  % cpu-time-sum
\providecommand{\predicateBitpreciseInterpolKindResultsRFInterpolAbsPrecPfCorrectCpuTimeSumPlain}{}
  \renewcommand{\predicateBitpreciseInterpolKindResultsRFInterpolAbsPrecPfCorrectCpuTimeSumPlain}{105018.09969337902\xspace}
\providecommand{\predicateBitpreciseInterpolKindResultsRFInterpolAbsPrecPfCorrectCpuTimeSumPlainHours}{}
  \renewcommand{\predicateBitpreciseInterpolKindResultsRFInterpolAbsPrecPfCorrectCpuTimeSumPlainHours}{29.17169435927195\xspace}

  % cpu-time-avg
\providecommand{\predicateBitpreciseInterpolKindResultsRFInterpolAbsPrecPfCorrectCpuTimeAvgPlain}{}
  \renewcommand{\predicateBitpreciseInterpolKindResultsRFInterpolAbsPrecPfCorrectCpuTimeAvgPlain}{57.638913113819434\xspace}
\providecommand{\predicateBitpreciseInterpolKindResultsRFInterpolAbsPrecPfCorrectCpuTimeAvgPlainHours}{}
  \renewcommand{\predicateBitpreciseInterpolKindResultsRFInterpolAbsPrecPfCorrectCpuTimeAvgPlainHours}{0.016010809198283175\xspace}

  % inv-succ
\providecommand{\predicateBitpreciseInterpolKindResultsRFInterpolAbsPrecPfCorrectInvSuccPlain}{}
  \renewcommand{\predicateBitpreciseInterpolKindResultsRFInterpolAbsPrecPfCorrectInvSuccPlain}{1174\xspace}

  % inv-tries
\providecommand{\predicateBitpreciseInterpolKindResultsRFInterpolAbsPrecPfCorrectInvTriesPlain}{}
  \renewcommand{\predicateBitpreciseInterpolKindResultsRFInterpolAbsPrecPfCorrectInvTriesPlain}{6364\xspace}

  % inv-time-sum
\providecommand{\predicateBitpreciseInterpolKindResultsRFInterpolAbsPrecPfCorrectInvTimeSumPlain}{}
  \renewcommand{\predicateBitpreciseInterpolKindResultsRFInterpolAbsPrecPfCorrectInvTimeSumPlain}{24085.983999999997\xspace}
\providecommand{\predicateBitpreciseInterpolKindResultsRFInterpolAbsPrecPfCorrectInvTimeSumPlainHours}{}
  \renewcommand{\predicateBitpreciseInterpolKindResultsRFInterpolAbsPrecPfCorrectInvTimeSumPlainHours}{6.69055111111111\xspace}

 %% incorrect %%
\providecommand{\predicateBitpreciseInterpolKindResultsRFInterpolAbsPrecPfIncorrectPlain}{}
  \renewcommand{\predicateBitpreciseInterpolKindResultsRFInterpolAbsPrecPfIncorrectPlain}{24\xspace}

  % cpu-time-sum
\providecommand{\predicateBitpreciseInterpolKindResultsRFInterpolAbsPrecPfIncorrectCpuTimeSumPlain}{}
  \renewcommand{\predicateBitpreciseInterpolKindResultsRFInterpolAbsPrecPfIncorrectCpuTimeSumPlain}{1191.5065502559999\xspace}
\providecommand{\predicateBitpreciseInterpolKindResultsRFInterpolAbsPrecPfIncorrectCpuTimeSumPlainHours}{}
  \renewcommand{\predicateBitpreciseInterpolKindResultsRFInterpolAbsPrecPfIncorrectCpuTimeSumPlainHours}{0.3309740417377777\xspace}

  % cpu-time-avg
\providecommand{\predicateBitpreciseInterpolKindResultsRFInterpolAbsPrecPfIncorrectCpuTimeAvgPlain}{}
  \renewcommand{\predicateBitpreciseInterpolKindResultsRFInterpolAbsPrecPfIncorrectCpuTimeAvgPlain}{49.64610626066666\xspace}
\providecommand{\predicateBitpreciseInterpolKindResultsRFInterpolAbsPrecPfIncorrectCpuTimeAvgPlainHours}{}
  \renewcommand{\predicateBitpreciseInterpolKindResultsRFInterpolAbsPrecPfIncorrectCpuTimeAvgPlainHours}{0.013790585072407406\xspace}

  % inv-succ
\providecommand{\predicateBitpreciseInterpolKindResultsRFInterpolAbsPrecPfIncorrectInvSuccPlain}{}
  \renewcommand{\predicateBitpreciseInterpolKindResultsRFInterpolAbsPrecPfIncorrectInvSuccPlain}{4\xspace}

  % inv-tries
\providecommand{\predicateBitpreciseInterpolKindResultsRFInterpolAbsPrecPfIncorrectInvTriesPlain}{}
  \renewcommand{\predicateBitpreciseInterpolKindResultsRFInterpolAbsPrecPfIncorrectInvTriesPlain}{82\xspace}

  % inv-time-sum
\providecommand{\predicateBitpreciseInterpolKindResultsRFInterpolAbsPrecPfIncorrectInvTimeSumPlain}{}
  \renewcommand{\predicateBitpreciseInterpolKindResultsRFInterpolAbsPrecPfIncorrectInvTimeSumPlain}{589.808\xspace}
\providecommand{\predicateBitpreciseInterpolKindResultsRFInterpolAbsPrecPfIncorrectInvTimeSumPlainHours}{}
  \renewcommand{\predicateBitpreciseInterpolKindResultsRFInterpolAbsPrecPfIncorrectInvTimeSumPlainHours}{0.16383555555555554\xspace}

 %% timeout %%
\providecommand{\predicateBitpreciseInterpolKindResultsRFInterpolAbsPrecPfTimeoutPlain}{}
  \renewcommand{\predicateBitpreciseInterpolKindResultsRFInterpolAbsPrecPfTimeoutPlain}{1556\xspace}

  % cpu-time-sum
\providecommand{\predicateBitpreciseInterpolKindResultsRFInterpolAbsPrecPfTimeoutCpuTimeSumPlain}{}
  \renewcommand{\predicateBitpreciseInterpolKindResultsRFInterpolAbsPrecPfTimeoutCpuTimeSumPlain}{476324.66836053657\xspace}
\providecommand{\predicateBitpreciseInterpolKindResultsRFInterpolAbsPrecPfTimeoutCpuTimeSumPlainHours}{}
  \renewcommand{\predicateBitpreciseInterpolKindResultsRFInterpolAbsPrecPfTimeoutCpuTimeSumPlainHours}{132.31240787792683\xspace}

  % cpu-time-avg
\providecommand{\predicateBitpreciseInterpolKindResultsRFInterpolAbsPrecPfTimeoutCpuTimeAvgPlain}{}
  \renewcommand{\predicateBitpreciseInterpolKindResultsRFInterpolAbsPrecPfTimeoutCpuTimeAvgPlain}{306.1212521597279\xspace}
\providecommand{\predicateBitpreciseInterpolKindResultsRFInterpolAbsPrecPfTimeoutCpuTimeAvgPlainHours}{}
  \renewcommand{\predicateBitpreciseInterpolKindResultsRFInterpolAbsPrecPfTimeoutCpuTimeAvgPlainHours}{0.08503368115547996\xspace}

  % inv-succ
\providecommand{\predicateBitpreciseInterpolKindResultsRFInterpolAbsPrecPfTimeoutInvSuccPlain}{}
  \renewcommand{\predicateBitpreciseInterpolKindResultsRFInterpolAbsPrecPfTimeoutInvSuccPlain}{227\xspace}

  % inv-tries
\providecommand{\predicateBitpreciseInterpolKindResultsRFInterpolAbsPrecPfTimeoutInvTriesPlain}{}
  \renewcommand{\predicateBitpreciseInterpolKindResultsRFInterpolAbsPrecPfTimeoutInvTriesPlain}{7459\xspace}

  % inv-time-sum
\providecommand{\predicateBitpreciseInterpolKindResultsRFInterpolAbsPrecPfTimeoutInvTimeSumPlain}{}
  \renewcommand{\predicateBitpreciseInterpolKindResultsRFInterpolAbsPrecPfTimeoutInvTimeSumPlain}{159019.50700000004\xspace}
\providecommand{\predicateBitpreciseInterpolKindResultsRFInterpolAbsPrecPfTimeoutInvTimeSumPlainHours}{}
  \renewcommand{\predicateBitpreciseInterpolKindResultsRFInterpolAbsPrecPfTimeoutInvTimeSumPlainHours}{44.17208527777779\xspace}

 %% unknown-or-category-error %%
\providecommand{\predicateBitpreciseInterpolKindResultsRFInterpolAbsPrecPfUnknownOrCategoryErrorPlain}{}
  \renewcommand{\predicateBitpreciseInterpolKindResultsRFInterpolAbsPrecPfUnknownOrCategoryErrorPlain}{1642\xspace}

  % cpu-time-sum
\providecommand{\predicateBitpreciseInterpolKindResultsRFInterpolAbsPrecPfUnknownOrCategoryErrorCpuTimeSumPlain}{}
  \renewcommand{\predicateBitpreciseInterpolKindResultsRFInterpolAbsPrecPfUnknownOrCategoryErrorCpuTimeSumPlain}{486019.9874459076\xspace}
\providecommand{\predicateBitpreciseInterpolKindResultsRFInterpolAbsPrecPfUnknownOrCategoryErrorCpuTimeSumPlainHours}{}
  \renewcommand{\predicateBitpreciseInterpolKindResultsRFInterpolAbsPrecPfUnknownOrCategoryErrorCpuTimeSumPlainHours}{135.00555206830768\xspace}

  % cpu-time-avg
\providecommand{\predicateBitpreciseInterpolKindResultsRFInterpolAbsPrecPfUnknownOrCategoryErrorCpuTimeAvgPlain}{}
  \renewcommand{\predicateBitpreciseInterpolKindResultsRFInterpolAbsPrecPfUnknownOrCategoryErrorCpuTimeAvgPlain}{295.99268419361\xspace}
\providecommand{\predicateBitpreciseInterpolKindResultsRFInterpolAbsPrecPfUnknownOrCategoryErrorCpuTimeAvgPlainHours}{}
  \renewcommand{\predicateBitpreciseInterpolKindResultsRFInterpolAbsPrecPfUnknownOrCategoryErrorCpuTimeAvgPlainHours}{0.08222019005378055\xspace}

  % inv-succ
\providecommand{\predicateBitpreciseInterpolKindResultsRFInterpolAbsPrecPfUnknownOrCategoryErrorInvSuccPlain}{}
  \renewcommand{\predicateBitpreciseInterpolKindResultsRFInterpolAbsPrecPfUnknownOrCategoryErrorInvSuccPlain}{231\xspace}

  % inv-tries
\providecommand{\predicateBitpreciseInterpolKindResultsRFInterpolAbsPrecPfUnknownOrCategoryErrorInvTriesPlain}{}
  \renewcommand{\predicateBitpreciseInterpolKindResultsRFInterpolAbsPrecPfUnknownOrCategoryErrorInvTriesPlain}{7700\xspace}

  % inv-time-sum
\providecommand{\predicateBitpreciseInterpolKindResultsRFInterpolAbsPrecPfUnknownOrCategoryErrorInvTimeSumPlain}{}
  \renewcommand{\predicateBitpreciseInterpolKindResultsRFInterpolAbsPrecPfUnknownOrCategoryErrorInvTimeSumPlain}{161821.25000000012\xspace}
\providecommand{\predicateBitpreciseInterpolKindResultsRFInterpolAbsPrecPfUnknownOrCategoryErrorInvTimeSumPlainHours}{}
  \renewcommand{\predicateBitpreciseInterpolKindResultsRFInterpolAbsPrecPfUnknownOrCategoryErrorInvTimeSumPlainHours}{44.950347222222256\xspace}

 %% correct-false %%
\providecommand{\predicateBitpreciseInterpolKindResultsRFInterpolAbsPrecPfCorrectFalsePlain}{}
  \renewcommand{\predicateBitpreciseInterpolKindResultsRFInterpolAbsPrecPfCorrectFalsePlain}{486\xspace}

  % cpu-time-sum
\providecommand{\predicateBitpreciseInterpolKindResultsRFInterpolAbsPrecPfCorrectFalseCpuTimeSumPlain}{}
  \renewcommand{\predicateBitpreciseInterpolKindResultsRFInterpolAbsPrecPfCorrectFalseCpuTimeSumPlain}{32775.93221478502\xspace}
\providecommand{\predicateBitpreciseInterpolKindResultsRFInterpolAbsPrecPfCorrectFalseCpuTimeSumPlainHours}{}
  \renewcommand{\predicateBitpreciseInterpolKindResultsRFInterpolAbsPrecPfCorrectFalseCpuTimeSumPlainHours}{9.10442561521806\xspace}

  % cpu-time-avg
\providecommand{\predicateBitpreciseInterpolKindResultsRFInterpolAbsPrecPfCorrectFalseCpuTimeAvgPlain}{}
  \renewcommand{\predicateBitpreciseInterpolKindResultsRFInterpolAbsPrecPfCorrectFalseCpuTimeAvgPlain}{67.440189742356\xspace}
\providecommand{\predicateBitpreciseInterpolKindResultsRFInterpolAbsPrecPfCorrectFalseCpuTimeAvgPlainHours}{}
  \renewcommand{\predicateBitpreciseInterpolKindResultsRFInterpolAbsPrecPfCorrectFalseCpuTimeAvgPlainHours}{0.01873338603954333\xspace}

  % inv-succ
\providecommand{\predicateBitpreciseInterpolKindResultsRFInterpolAbsPrecPfCorrectFalseInvSuccPlain}{}
  \renewcommand{\predicateBitpreciseInterpolKindResultsRFInterpolAbsPrecPfCorrectFalseInvSuccPlain}{131\xspace}

  % inv-tries
\providecommand{\predicateBitpreciseInterpolKindResultsRFInterpolAbsPrecPfCorrectFalseInvTriesPlain}{}
  \renewcommand{\predicateBitpreciseInterpolKindResultsRFInterpolAbsPrecPfCorrectFalseInvTriesPlain}{2331\xspace}

  % inv-time-sum
\providecommand{\predicateBitpreciseInterpolKindResultsRFInterpolAbsPrecPfCorrectFalseInvTimeSumPlain}{}
  \renewcommand{\predicateBitpreciseInterpolKindResultsRFInterpolAbsPrecPfCorrectFalseInvTimeSumPlain}{7259.041000000013\xspace}
\providecommand{\predicateBitpreciseInterpolKindResultsRFInterpolAbsPrecPfCorrectFalseInvTimeSumPlainHours}{}
  \renewcommand{\predicateBitpreciseInterpolKindResultsRFInterpolAbsPrecPfCorrectFalseInvTimeSumPlainHours}{2.016400277777781\xspace}

 %% correct-true %%
\providecommand{\predicateBitpreciseInterpolKindResultsRFInterpolAbsPrecPfCorrectTruePlain}{}
  \renewcommand{\predicateBitpreciseInterpolKindResultsRFInterpolAbsPrecPfCorrectTruePlain}{1336\xspace}

  % cpu-time-sum
\providecommand{\predicateBitpreciseInterpolKindResultsRFInterpolAbsPrecPfCorrectTrueCpuTimeSumPlain}{}
  \renewcommand{\predicateBitpreciseInterpolKindResultsRFInterpolAbsPrecPfCorrectTrueCpuTimeSumPlain}{72242.16747859413\xspace}
\providecommand{\predicateBitpreciseInterpolKindResultsRFInterpolAbsPrecPfCorrectTrueCpuTimeSumPlainHours}{}
  \renewcommand{\predicateBitpreciseInterpolKindResultsRFInterpolAbsPrecPfCorrectTrueCpuTimeSumPlainHours}{20.067268744053926\xspace}

  % cpu-time-avg
\providecommand{\predicateBitpreciseInterpolKindResultsRFInterpolAbsPrecPfCorrectTrueCpuTimeAvgPlain}{}
  \renewcommand{\predicateBitpreciseInterpolKindResultsRFInterpolAbsPrecPfCorrectTrueCpuTimeAvgPlain}{54.07347865164231\xspace}
\providecommand{\predicateBitpreciseInterpolKindResultsRFInterpolAbsPrecPfCorrectTrueCpuTimeAvgPlainHours}{}
  \renewcommand{\predicateBitpreciseInterpolKindResultsRFInterpolAbsPrecPfCorrectTrueCpuTimeAvgPlainHours}{0.015020410736567309\xspace}

  % inv-succ
\providecommand{\predicateBitpreciseInterpolKindResultsRFInterpolAbsPrecPfCorrectTrueInvSuccPlain}{}
  \renewcommand{\predicateBitpreciseInterpolKindResultsRFInterpolAbsPrecPfCorrectTrueInvSuccPlain}{1043\xspace}

  % inv-tries
\providecommand{\predicateBitpreciseInterpolKindResultsRFInterpolAbsPrecPfCorrectTrueInvTriesPlain}{}
  \renewcommand{\predicateBitpreciseInterpolKindResultsRFInterpolAbsPrecPfCorrectTrueInvTriesPlain}{4033\xspace}

  % inv-time-sum
\providecommand{\predicateBitpreciseInterpolKindResultsRFInterpolAbsPrecPfCorrectTrueInvTimeSumPlain}{}
  \renewcommand{\predicateBitpreciseInterpolKindResultsRFInterpolAbsPrecPfCorrectTrueInvTimeSumPlain}{16826.942999999996\xspace}
\providecommand{\predicateBitpreciseInterpolKindResultsRFInterpolAbsPrecPfCorrectTrueInvTimeSumPlainHours}{}
  \renewcommand{\predicateBitpreciseInterpolKindResultsRFInterpolAbsPrecPfCorrectTrueInvTimeSumPlainHours}{4.674150833333332\xspace}

 %% incorrect-false %%
\providecommand{\predicateBitpreciseInterpolKindResultsRFInterpolAbsPrecPfIncorrectFalsePlain}{}
  \renewcommand{\predicateBitpreciseInterpolKindResultsRFInterpolAbsPrecPfIncorrectFalsePlain}{23\xspace}

  % cpu-time-sum
\providecommand{\predicateBitpreciseInterpolKindResultsRFInterpolAbsPrecPfIncorrectFalseCpuTimeSumPlain}{}
  \renewcommand{\predicateBitpreciseInterpolKindResultsRFInterpolAbsPrecPfIncorrectFalseCpuTimeSumPlain}{1083.29649311\xspace}
\providecommand{\predicateBitpreciseInterpolKindResultsRFInterpolAbsPrecPfIncorrectFalseCpuTimeSumPlainHours}{}
  \renewcommand{\predicateBitpreciseInterpolKindResultsRFInterpolAbsPrecPfIncorrectFalseCpuTimeSumPlainHours}{0.30091569253055556\xspace}

  % cpu-time-avg
\providecommand{\predicateBitpreciseInterpolKindResultsRFInterpolAbsPrecPfIncorrectFalseCpuTimeAvgPlain}{}
  \renewcommand{\predicateBitpreciseInterpolKindResultsRFInterpolAbsPrecPfIncorrectFalseCpuTimeAvgPlain}{47.09984752652174\xspace}
\providecommand{\predicateBitpreciseInterpolKindResultsRFInterpolAbsPrecPfIncorrectFalseCpuTimeAvgPlainHours}{}
  \renewcommand{\predicateBitpreciseInterpolKindResultsRFInterpolAbsPrecPfIncorrectFalseCpuTimeAvgPlainHours}{0.013083290979589372\xspace}

  % inv-succ
\providecommand{\predicateBitpreciseInterpolKindResultsRFInterpolAbsPrecPfIncorrectFalseInvSuccPlain}{}
  \renewcommand{\predicateBitpreciseInterpolKindResultsRFInterpolAbsPrecPfIncorrectFalseInvSuccPlain}{0\xspace}

  % inv-tries
\providecommand{\predicateBitpreciseInterpolKindResultsRFInterpolAbsPrecPfIncorrectFalseInvTriesPlain}{}
  \renewcommand{\predicateBitpreciseInterpolKindResultsRFInterpolAbsPrecPfIncorrectFalseInvTriesPlain}{70\xspace}

  % inv-time-sum
\providecommand{\predicateBitpreciseInterpolKindResultsRFInterpolAbsPrecPfIncorrectFalseInvTimeSumPlain}{}
  \renewcommand{\predicateBitpreciseInterpolKindResultsRFInterpolAbsPrecPfIncorrectFalseInvTimeSumPlain}{535.809\xspace}
\providecommand{\predicateBitpreciseInterpolKindResultsRFInterpolAbsPrecPfIncorrectFalseInvTimeSumPlainHours}{}
  \renewcommand{\predicateBitpreciseInterpolKindResultsRFInterpolAbsPrecPfIncorrectFalseInvTimeSumPlainHours}{0.14883583333333333\xspace}

 %% incorrect-true %%
\providecommand{\predicateBitpreciseInterpolKindResultsRFInterpolAbsPrecPfIncorrectTruePlain}{}
  \renewcommand{\predicateBitpreciseInterpolKindResultsRFInterpolAbsPrecPfIncorrectTruePlain}{1\xspace}

  % cpu-time-sum
\providecommand{\predicateBitpreciseInterpolKindResultsRFInterpolAbsPrecPfIncorrectTrueCpuTimeSumPlain}{}
  \renewcommand{\predicateBitpreciseInterpolKindResultsRFInterpolAbsPrecPfIncorrectTrueCpuTimeSumPlain}{108.210057146\xspace}
\providecommand{\predicateBitpreciseInterpolKindResultsRFInterpolAbsPrecPfIncorrectTrueCpuTimeSumPlainHours}{}
  \renewcommand{\predicateBitpreciseInterpolKindResultsRFInterpolAbsPrecPfIncorrectTrueCpuTimeSumPlainHours}{0.03005834920722222\xspace}

  % cpu-time-avg
\providecommand{\predicateBitpreciseInterpolKindResultsRFInterpolAbsPrecPfIncorrectTrueCpuTimeAvgPlain}{}
  \renewcommand{\predicateBitpreciseInterpolKindResultsRFInterpolAbsPrecPfIncorrectTrueCpuTimeAvgPlain}{108.210057146\xspace}
\providecommand{\predicateBitpreciseInterpolKindResultsRFInterpolAbsPrecPfIncorrectTrueCpuTimeAvgPlainHours}{}
  \renewcommand{\predicateBitpreciseInterpolKindResultsRFInterpolAbsPrecPfIncorrectTrueCpuTimeAvgPlainHours}{0.03005834920722222\xspace}

  % inv-succ
\providecommand{\predicateBitpreciseInterpolKindResultsRFInterpolAbsPrecPfIncorrectTrueInvSuccPlain}{}
  \renewcommand{\predicateBitpreciseInterpolKindResultsRFInterpolAbsPrecPfIncorrectTrueInvSuccPlain}{4\xspace}

  % inv-tries
\providecommand{\predicateBitpreciseInterpolKindResultsRFInterpolAbsPrecPfIncorrectTrueInvTriesPlain}{}
  \renewcommand{\predicateBitpreciseInterpolKindResultsRFInterpolAbsPrecPfIncorrectTrueInvTriesPlain}{12\xspace}

  % inv-time-sum
\providecommand{\predicateBitpreciseInterpolKindResultsRFInterpolAbsPrecPfIncorrectTrueInvTimeSumPlain}{}
  \renewcommand{\predicateBitpreciseInterpolKindResultsRFInterpolAbsPrecPfIncorrectTrueInvTimeSumPlain}{53.999\xspace}
\providecommand{\predicateBitpreciseInterpolKindResultsRFInterpolAbsPrecPfIncorrectTrueInvTimeSumPlainHours}{}
  \renewcommand{\predicateBitpreciseInterpolKindResultsRFInterpolAbsPrecPfIncorrectTrueInvTimeSumPlainHours}{0.014999722222222223\xspace}

 %% all %%
\providecommand{\predicateBitpreciseInterpolKindResultsRFInterpolAbsPrecPfAllPlain}{}
  \renewcommand{\predicateBitpreciseInterpolKindResultsRFInterpolAbsPrecPfAllPlain}{3488\xspace}

  % cpu-time-sum
\providecommand{\predicateBitpreciseInterpolKindResultsRFInterpolAbsPrecPfAllCpuTimeSumPlain}{}
  \renewcommand{\predicateBitpreciseInterpolKindResultsRFInterpolAbsPrecPfAllCpuTimeSumPlain}{592229.5936895409\xspace}
\providecommand{\predicateBitpreciseInterpolKindResultsRFInterpolAbsPrecPfAllCpuTimeSumPlainHours}{}
  \renewcommand{\predicateBitpreciseInterpolKindResultsRFInterpolAbsPrecPfAllCpuTimeSumPlainHours}{164.5082204693169\xspace}

  % cpu-time-avg
\providecommand{\predicateBitpreciseInterpolKindResultsRFInterpolAbsPrecPfAllCpuTimeAvgPlain}{}
  \renewcommand{\predicateBitpreciseInterpolKindResultsRFInterpolAbsPrecPfAllCpuTimeAvgPlain}{169.79059452108396\xspace}
\providecommand{\predicateBitpreciseInterpolKindResultsRFInterpolAbsPrecPfAllCpuTimeAvgPlainHours}{}
  \renewcommand{\predicateBitpreciseInterpolKindResultsRFInterpolAbsPrecPfAllCpuTimeAvgPlainHours}{0.04716405403363443\xspace}

  % inv-succ
\providecommand{\predicateBitpreciseInterpolKindResultsRFInterpolAbsPrecPfAllInvSuccPlain}{}
  \renewcommand{\predicateBitpreciseInterpolKindResultsRFInterpolAbsPrecPfAllInvSuccPlain}{1409\xspace}

  % inv-tries
\providecommand{\predicateBitpreciseInterpolKindResultsRFInterpolAbsPrecPfAllInvTriesPlain}{}
  \renewcommand{\predicateBitpreciseInterpolKindResultsRFInterpolAbsPrecPfAllInvTriesPlain}{14146\xspace}

  % inv-time-sum
\providecommand{\predicateBitpreciseInterpolKindResultsRFInterpolAbsPrecPfAllInvTimeSumPlain}{}
  \renewcommand{\predicateBitpreciseInterpolKindResultsRFInterpolAbsPrecPfAllInvTimeSumPlain}{186497.04200000004\xspace}
\providecommand{\predicateBitpreciseInterpolKindResultsRFInterpolAbsPrecPfAllInvTimeSumPlainHours}{}
  \renewcommand{\predicateBitpreciseInterpolKindResultsRFInterpolAbsPrecPfAllInvTimeSumPlainHours}{51.804733888888904\xspace}

 %% equal-only %%
\providecommand{\predicateBitpreciseInterpolKindResultsRFInterpolAbsPrecPfEqualOnlyPlain}{}
  \renewcommand{\predicateBitpreciseInterpolKindResultsRFInterpolAbsPrecPfEqualOnlyPlain}{1793\xspace}

  % cpu-time-sum
\providecommand{\predicateBitpreciseInterpolKindResultsRFInterpolAbsPrecPfEqualOnlyCpuTimeSumPlain}{}
  \renewcommand{\predicateBitpreciseInterpolKindResultsRFInterpolAbsPrecPfEqualOnlyCpuTimeSumPlain}{99074.69342504606\xspace}
\providecommand{\predicateBitpreciseInterpolKindResultsRFInterpolAbsPrecPfEqualOnlyCpuTimeSumPlainHours}{}
  \renewcommand{\predicateBitpreciseInterpolKindResultsRFInterpolAbsPrecPfEqualOnlyCpuTimeSumPlainHours}{27.520748173623907\xspace}

  % cpu-time-avg
\providecommand{\predicateBitpreciseInterpolKindResultsRFInterpolAbsPrecPfEqualOnlyCpuTimeAvgPlain}{}
  \renewcommand{\predicateBitpreciseInterpolKindResultsRFInterpolAbsPrecPfEqualOnlyCpuTimeAvgPlain}{55.25638227833021\xspace}
\providecommand{\predicateBitpreciseInterpolKindResultsRFInterpolAbsPrecPfEqualOnlyCpuTimeAvgPlainHours}{}
  \renewcommand{\predicateBitpreciseInterpolKindResultsRFInterpolAbsPrecPfEqualOnlyCpuTimeAvgPlainHours}{0.015348995077313946\xspace}

  % inv-succ
\providecommand{\predicateBitpreciseInterpolKindResultsRFInterpolAbsPrecPfEqualOnlyInvSuccPlain}{}
  \renewcommand{\predicateBitpreciseInterpolKindResultsRFInterpolAbsPrecPfEqualOnlyInvSuccPlain}{1146\xspace}

  % inv-tries
\providecommand{\predicateBitpreciseInterpolKindResultsRFInterpolAbsPrecPfEqualOnlyInvTriesPlain}{}
  \renewcommand{\predicateBitpreciseInterpolKindResultsRFInterpolAbsPrecPfEqualOnlyInvTriesPlain}{6209\xspace}

  % inv-time-sum
\providecommand{\predicateBitpreciseInterpolKindResultsRFInterpolAbsPrecPfEqualOnlyInvTimeSumPlain}{}
  \renewcommand{\predicateBitpreciseInterpolKindResultsRFInterpolAbsPrecPfEqualOnlyInvTimeSumPlain}{23052.766999999993\xspace}
\providecommand{\predicateBitpreciseInterpolKindResultsRFInterpolAbsPrecPfEqualOnlyInvTimeSumPlainHours}{}
  \renewcommand{\predicateBitpreciseInterpolKindResultsRFInterpolAbsPrecPfEqualOnlyInvTimeSumPlainHours}{6.403546388888887\xspace}

%%% predicate_bitprecise_interpol_kind.2016-09-04_2044.results.RF_interpol-abs-pf %%%
 %% correct %%
\providecommand{\predicateBitpreciseInterpolKindResultsRFInterpolAbsPfCorrectPlain}{}
  \renewcommand{\predicateBitpreciseInterpolKindResultsRFInterpolAbsPfCorrectPlain}{1831\xspace}

  % cpu-time-sum
\providecommand{\predicateBitpreciseInterpolKindResultsRFInterpolAbsPfCorrectCpuTimeSumPlain}{}
  \renewcommand{\predicateBitpreciseInterpolKindResultsRFInterpolAbsPfCorrectCpuTimeSumPlain}{104985.804432024\xspace}
\providecommand{\predicateBitpreciseInterpolKindResultsRFInterpolAbsPfCorrectCpuTimeSumPlainHours}{}
  \renewcommand{\predicateBitpreciseInterpolKindResultsRFInterpolAbsPfCorrectCpuTimeSumPlainHours}{29.16272345334\xspace}

  % cpu-time-avg
\providecommand{\predicateBitpreciseInterpolKindResultsRFInterpolAbsPfCorrectCpuTimeAvgPlain}{}
  \renewcommand{\predicateBitpreciseInterpolKindResultsRFInterpolAbsPfCorrectCpuTimeAvgPlain}{57.33795982087602\xspace}
\providecommand{\predicateBitpreciseInterpolKindResultsRFInterpolAbsPfCorrectCpuTimeAvgPlainHours}{}
  \renewcommand{\predicateBitpreciseInterpolKindResultsRFInterpolAbsPfCorrectCpuTimeAvgPlainHours}{0.01592721106135445\xspace}

  % inv-succ
\providecommand{\predicateBitpreciseInterpolKindResultsRFInterpolAbsPfCorrectInvSuccPlain}{}
  \renewcommand{\predicateBitpreciseInterpolKindResultsRFInterpolAbsPfCorrectInvSuccPlain}{1011\xspace}

  % inv-tries
\providecommand{\predicateBitpreciseInterpolKindResultsRFInterpolAbsPfCorrectInvTriesPlain}{}
  \renewcommand{\predicateBitpreciseInterpolKindResultsRFInterpolAbsPfCorrectInvTriesPlain}{6229\xspace}

  % inv-time-sum
\providecommand{\predicateBitpreciseInterpolKindResultsRFInterpolAbsPfCorrectInvTimeSumPlain}{}
  \renewcommand{\predicateBitpreciseInterpolKindResultsRFInterpolAbsPfCorrectInvTimeSumPlain}{24471.352000000006\xspace}
\providecommand{\predicateBitpreciseInterpolKindResultsRFInterpolAbsPfCorrectInvTimeSumPlainHours}{}
  \renewcommand{\predicateBitpreciseInterpolKindResultsRFInterpolAbsPfCorrectInvTimeSumPlainHours}{6.79759777777778\xspace}

 %% incorrect %%
\providecommand{\predicateBitpreciseInterpolKindResultsRFInterpolAbsPfIncorrectPlain}{}
  \renewcommand{\predicateBitpreciseInterpolKindResultsRFInterpolAbsPfIncorrectPlain}{24\xspace}

  % cpu-time-sum
\providecommand{\predicateBitpreciseInterpolKindResultsRFInterpolAbsPfIncorrectCpuTimeSumPlain}{}
  \renewcommand{\predicateBitpreciseInterpolKindResultsRFInterpolAbsPfIncorrectCpuTimeSumPlain}{1172.0556684769997\xspace}
\providecommand{\predicateBitpreciseInterpolKindResultsRFInterpolAbsPfIncorrectCpuTimeSumPlainHours}{}
  \renewcommand{\predicateBitpreciseInterpolKindResultsRFInterpolAbsPfIncorrectCpuTimeSumPlainHours}{0.3255710190213888\xspace}

  % cpu-time-avg
\providecommand{\predicateBitpreciseInterpolKindResultsRFInterpolAbsPfIncorrectCpuTimeAvgPlain}{}
  \renewcommand{\predicateBitpreciseInterpolKindResultsRFInterpolAbsPfIncorrectCpuTimeAvgPlain}{48.83565285320832\xspace}
\providecommand{\predicateBitpreciseInterpolKindResultsRFInterpolAbsPfIncorrectCpuTimeAvgPlainHours}{}
  \renewcommand{\predicateBitpreciseInterpolKindResultsRFInterpolAbsPfIncorrectCpuTimeAvgPlainHours}{0.0135654591258912\xspace}

  % inv-succ
\providecommand{\predicateBitpreciseInterpolKindResultsRFInterpolAbsPfIncorrectInvSuccPlain}{}
  \renewcommand{\predicateBitpreciseInterpolKindResultsRFInterpolAbsPfIncorrectInvSuccPlain}{4\xspace}

  % inv-tries
\providecommand{\predicateBitpreciseInterpolKindResultsRFInterpolAbsPfIncorrectInvTriesPlain}{}
  \renewcommand{\predicateBitpreciseInterpolKindResultsRFInterpolAbsPfIncorrectInvTriesPlain}{80\xspace}

  % inv-time-sum
\providecommand{\predicateBitpreciseInterpolKindResultsRFInterpolAbsPfIncorrectInvTimeSumPlain}{}
  \renewcommand{\predicateBitpreciseInterpolKindResultsRFInterpolAbsPfIncorrectInvTimeSumPlain}{578.77\xspace}
\providecommand{\predicateBitpreciseInterpolKindResultsRFInterpolAbsPfIncorrectInvTimeSumPlainHours}{}
  \renewcommand{\predicateBitpreciseInterpolKindResultsRFInterpolAbsPfIncorrectInvTimeSumPlainHours}{0.16076944444444444\xspace}

 %% timeout %%
\providecommand{\predicateBitpreciseInterpolKindResultsRFInterpolAbsPfTimeoutPlain}{}
  \renewcommand{\predicateBitpreciseInterpolKindResultsRFInterpolAbsPfTimeoutPlain}{1549\xspace}

  % cpu-time-sum
\providecommand{\predicateBitpreciseInterpolKindResultsRFInterpolAbsPfTimeoutCpuTimeSumPlain}{}
  \renewcommand{\predicateBitpreciseInterpolKindResultsRFInterpolAbsPfTimeoutCpuTimeSumPlain}{474178.9879227428\xspace}
\providecommand{\predicateBitpreciseInterpolKindResultsRFInterpolAbsPfTimeoutCpuTimeSumPlainHours}{}
  \renewcommand{\predicateBitpreciseInterpolKindResultsRFInterpolAbsPfTimeoutCpuTimeSumPlainHours}{131.71638553409522\xspace}

  % cpu-time-avg
\providecommand{\predicateBitpreciseInterpolKindResultsRFInterpolAbsPfTimeoutCpuTimeAvgPlain}{}
  \renewcommand{\predicateBitpreciseInterpolKindResultsRFInterpolAbsPfTimeoutCpuTimeAvgPlain}{306.11942409473386\xspace}
\providecommand{\predicateBitpreciseInterpolKindResultsRFInterpolAbsPfTimeoutCpuTimeAvgPlainHours}{}
  \renewcommand{\predicateBitpreciseInterpolKindResultsRFInterpolAbsPfTimeoutCpuTimeAvgPlainHours}{0.0850331733596483\xspace}

  % inv-succ
\providecommand{\predicateBitpreciseInterpolKindResultsRFInterpolAbsPfTimeoutInvSuccPlain}{}
  \renewcommand{\predicateBitpreciseInterpolKindResultsRFInterpolAbsPfTimeoutInvSuccPlain}{193\xspace}

  % inv-tries
\providecommand{\predicateBitpreciseInterpolKindResultsRFInterpolAbsPfTimeoutInvTriesPlain}{}
  \renewcommand{\predicateBitpreciseInterpolKindResultsRFInterpolAbsPfTimeoutInvTriesPlain}{7485\xspace}

  % inv-time-sum
\providecommand{\predicateBitpreciseInterpolKindResultsRFInterpolAbsPfTimeoutInvTimeSumPlain}{}
  \renewcommand{\predicateBitpreciseInterpolKindResultsRFInterpolAbsPfTimeoutInvTimeSumPlain}{160235.9989999999\xspace}
\providecommand{\predicateBitpreciseInterpolKindResultsRFInterpolAbsPfTimeoutInvTimeSumPlainHours}{}
  \renewcommand{\predicateBitpreciseInterpolKindResultsRFInterpolAbsPfTimeoutInvTimeSumPlainHours}{44.50999972222219\xspace}

 %% unknown-or-category-error %%
\providecommand{\predicateBitpreciseInterpolKindResultsRFInterpolAbsPfUnknownOrCategoryErrorPlain}{}
  \renewcommand{\predicateBitpreciseInterpolKindResultsRFInterpolAbsPfUnknownOrCategoryErrorPlain}{1633\xspace}

  % cpu-time-sum
\providecommand{\predicateBitpreciseInterpolKindResultsRFInterpolAbsPfUnknownOrCategoryErrorCpuTimeSumPlain}{}
  \renewcommand{\predicateBitpreciseInterpolKindResultsRFInterpolAbsPfUnknownOrCategoryErrorCpuTimeSumPlain}{483132.7169542861\xspace}
\providecommand{\predicateBitpreciseInterpolKindResultsRFInterpolAbsPfUnknownOrCategoryErrorCpuTimeSumPlainHours}{}
  \renewcommand{\predicateBitpreciseInterpolKindResultsRFInterpolAbsPfUnknownOrCategoryErrorCpuTimeSumPlainHours}{134.2035324873017\xspace}

  % cpu-time-avg
\providecommand{\predicateBitpreciseInterpolKindResultsRFInterpolAbsPfUnknownOrCategoryErrorCpuTimeAvgPlain}{}
  \renewcommand{\predicateBitpreciseInterpolKindResultsRFInterpolAbsPfUnknownOrCategoryErrorCpuTimeAvgPlain}{295.855919751553\xspace}
\providecommand{\predicateBitpreciseInterpolKindResultsRFInterpolAbsPfUnknownOrCategoryErrorCpuTimeAvgPlainHours}{}
  \renewcommand{\predicateBitpreciseInterpolKindResultsRFInterpolAbsPfUnknownOrCategoryErrorCpuTimeAvgPlainHours}{0.08218219993098695\xspace}

  % inv-succ
\providecommand{\predicateBitpreciseInterpolKindResultsRFInterpolAbsPfUnknownOrCategoryErrorInvSuccPlain}{}
  \renewcommand{\predicateBitpreciseInterpolKindResultsRFInterpolAbsPfUnknownOrCategoryErrorInvSuccPlain}{196\xspace}

  % inv-tries
\providecommand{\predicateBitpreciseInterpolKindResultsRFInterpolAbsPfUnknownOrCategoryErrorInvTriesPlain}{}
  \renewcommand{\predicateBitpreciseInterpolKindResultsRFInterpolAbsPfUnknownOrCategoryErrorInvTriesPlain}{7721\xspace}

  % inv-time-sum
\providecommand{\predicateBitpreciseInterpolKindResultsRFInterpolAbsPfUnknownOrCategoryErrorInvTimeSumPlain}{}
  \renewcommand{\predicateBitpreciseInterpolKindResultsRFInterpolAbsPfUnknownOrCategoryErrorInvTimeSumPlain}{162856.7029999998\xspace}
\providecommand{\predicateBitpreciseInterpolKindResultsRFInterpolAbsPfUnknownOrCategoryErrorInvTimeSumPlainHours}{}
  \renewcommand{\predicateBitpreciseInterpolKindResultsRFInterpolAbsPfUnknownOrCategoryErrorInvTimeSumPlainHours}{45.2379730555555\xspace}

 %% correct-false %%
\providecommand{\predicateBitpreciseInterpolKindResultsRFInterpolAbsPfCorrectFalsePlain}{}
  \renewcommand{\predicateBitpreciseInterpolKindResultsRFInterpolAbsPfCorrectFalsePlain}{489\xspace}

  % cpu-time-sum
\providecommand{\predicateBitpreciseInterpolKindResultsRFInterpolAbsPfCorrectFalseCpuTimeSumPlain}{}
  \renewcommand{\predicateBitpreciseInterpolKindResultsRFInterpolAbsPfCorrectFalseCpuTimeSumPlain}{33079.25492993101\xspace}
\providecommand{\predicateBitpreciseInterpolKindResultsRFInterpolAbsPfCorrectFalseCpuTimeSumPlainHours}{}
  \renewcommand{\predicateBitpreciseInterpolKindResultsRFInterpolAbsPfCorrectFalseCpuTimeSumPlainHours}{9.188681924980838\xspace}

  % cpu-time-avg
\providecommand{\predicateBitpreciseInterpolKindResultsRFInterpolAbsPfCorrectFalseCpuTimeAvgPlain}{}
  \renewcommand{\predicateBitpreciseInterpolKindResultsRFInterpolAbsPfCorrectFalseCpuTimeAvgPlain}{67.64673809801843\xspace}
\providecommand{\predicateBitpreciseInterpolKindResultsRFInterpolAbsPfCorrectFalseCpuTimeAvgPlainHours}{}
  \renewcommand{\predicateBitpreciseInterpolKindResultsRFInterpolAbsPfCorrectFalseCpuTimeAvgPlainHours}{0.0187907605827829\xspace}

  % inv-succ
\providecommand{\predicateBitpreciseInterpolKindResultsRFInterpolAbsPfCorrectFalseInvSuccPlain}{}
  \renewcommand{\predicateBitpreciseInterpolKindResultsRFInterpolAbsPfCorrectFalseInvSuccPlain}{94\xspace}

  % inv-tries
\providecommand{\predicateBitpreciseInterpolKindResultsRFInterpolAbsPfCorrectFalseInvTriesPlain}{}
  \renewcommand{\predicateBitpreciseInterpolKindResultsRFInterpolAbsPfCorrectFalseInvTriesPlain}{2305\xspace}

  % inv-time-sum
\providecommand{\predicateBitpreciseInterpolKindResultsRFInterpolAbsPfCorrectFalseInvTimeSumPlain}{}
  \renewcommand{\predicateBitpreciseInterpolKindResultsRFInterpolAbsPfCorrectFalseInvTimeSumPlain}{7408.825999999998\xspace}
\providecommand{\predicateBitpreciseInterpolKindResultsRFInterpolAbsPfCorrectFalseInvTimeSumPlainHours}{}
  \renewcommand{\predicateBitpreciseInterpolKindResultsRFInterpolAbsPfCorrectFalseInvTimeSumPlainHours}{2.058007222222222\xspace}

 %% correct-true %%
\providecommand{\predicateBitpreciseInterpolKindResultsRFInterpolAbsPfCorrectTruePlain}{}
  \renewcommand{\predicateBitpreciseInterpolKindResultsRFInterpolAbsPfCorrectTruePlain}{1342\xspace}

  % cpu-time-sum
\providecommand{\predicateBitpreciseInterpolKindResultsRFInterpolAbsPfCorrectTrueCpuTimeSumPlain}{}
  \renewcommand{\predicateBitpreciseInterpolKindResultsRFInterpolAbsPfCorrectTrueCpuTimeSumPlain}{71906.54950209285\xspace}
\providecommand{\predicateBitpreciseInterpolKindResultsRFInterpolAbsPfCorrectTrueCpuTimeSumPlainHours}{}
  \renewcommand{\predicateBitpreciseInterpolKindResultsRFInterpolAbsPfCorrectTrueCpuTimeSumPlainHours}{19.974041528359123\xspace}

  % cpu-time-avg
\providecommand{\predicateBitpreciseInterpolKindResultsRFInterpolAbsPfCorrectTrueCpuTimeAvgPlain}{}
  \renewcommand{\predicateBitpreciseInterpolKindResultsRFInterpolAbsPfCorrectTrueCpuTimeAvgPlain}{53.58163152167872\xspace}
\providecommand{\predicateBitpreciseInterpolKindResultsRFInterpolAbsPfCorrectTrueCpuTimeAvgPlainHours}{}
  \renewcommand{\predicateBitpreciseInterpolKindResultsRFInterpolAbsPfCorrectTrueCpuTimeAvgPlainHours}{0.014883786533799645\xspace}

  % inv-succ
\providecommand{\predicateBitpreciseInterpolKindResultsRFInterpolAbsPfCorrectTrueInvSuccPlain}{}
  \renewcommand{\predicateBitpreciseInterpolKindResultsRFInterpolAbsPfCorrectTrueInvSuccPlain}{917\xspace}

  % inv-tries
\providecommand{\predicateBitpreciseInterpolKindResultsRFInterpolAbsPfCorrectTrueInvTriesPlain}{}
  \renewcommand{\predicateBitpreciseInterpolKindResultsRFInterpolAbsPfCorrectTrueInvTriesPlain}{3924\xspace}

  % inv-time-sum
\providecommand{\predicateBitpreciseInterpolKindResultsRFInterpolAbsPfCorrectTrueInvTimeSumPlain}{}
  \renewcommand{\predicateBitpreciseInterpolKindResultsRFInterpolAbsPfCorrectTrueInvTimeSumPlain}{17062.526000000013\xspace}
\providecommand{\predicateBitpreciseInterpolKindResultsRFInterpolAbsPfCorrectTrueInvTimeSumPlainHours}{}
  \renewcommand{\predicateBitpreciseInterpolKindResultsRFInterpolAbsPfCorrectTrueInvTimeSumPlainHours}{4.739590555555559\xspace}

 %% incorrect-false %%
\providecommand{\predicateBitpreciseInterpolKindResultsRFInterpolAbsPfIncorrectFalsePlain}{}
  \renewcommand{\predicateBitpreciseInterpolKindResultsRFInterpolAbsPfIncorrectFalsePlain}{23\xspace}

  % cpu-time-sum
\providecommand{\predicateBitpreciseInterpolKindResultsRFInterpolAbsPfIncorrectFalseCpuTimeSumPlain}{}
  \renewcommand{\predicateBitpreciseInterpolKindResultsRFInterpolAbsPfIncorrectFalseCpuTimeSumPlain}{1077.7426449579998\xspace}
\providecommand{\predicateBitpreciseInterpolKindResultsRFInterpolAbsPfIncorrectFalseCpuTimeSumPlainHours}{}
  \renewcommand{\predicateBitpreciseInterpolKindResultsRFInterpolAbsPfIncorrectFalseCpuTimeSumPlainHours}{0.2993729569327777\xspace}

  % cpu-time-avg
\providecommand{\predicateBitpreciseInterpolKindResultsRFInterpolAbsPfIncorrectFalseCpuTimeAvgPlain}{}
  \renewcommand{\predicateBitpreciseInterpolKindResultsRFInterpolAbsPfIncorrectFalseCpuTimeAvgPlain}{46.85837586773912\xspace}
\providecommand{\predicateBitpreciseInterpolKindResultsRFInterpolAbsPfIncorrectFalseCpuTimeAvgPlainHours}{}
  \renewcommand{\predicateBitpreciseInterpolKindResultsRFInterpolAbsPfIncorrectFalseCpuTimeAvgPlainHours}{0.013016215518816421\xspace}

  % inv-succ
\providecommand{\predicateBitpreciseInterpolKindResultsRFInterpolAbsPfIncorrectFalseInvSuccPlain}{}
  \renewcommand{\predicateBitpreciseInterpolKindResultsRFInterpolAbsPfIncorrectFalseInvSuccPlain}{0\xspace}

  % inv-tries
\providecommand{\predicateBitpreciseInterpolKindResultsRFInterpolAbsPfIncorrectFalseInvTriesPlain}{}
  \renewcommand{\predicateBitpreciseInterpolKindResultsRFInterpolAbsPfIncorrectFalseInvTriesPlain}{70\xspace}

  % inv-time-sum
\providecommand{\predicateBitpreciseInterpolKindResultsRFInterpolAbsPfIncorrectFalseInvTimeSumPlain}{}
  \renewcommand{\predicateBitpreciseInterpolKindResultsRFInterpolAbsPfIncorrectFalseInvTimeSumPlain}{534.3639999999999\xspace}
\providecommand{\predicateBitpreciseInterpolKindResultsRFInterpolAbsPfIncorrectFalseInvTimeSumPlainHours}{}
  \renewcommand{\predicateBitpreciseInterpolKindResultsRFInterpolAbsPfIncorrectFalseInvTimeSumPlainHours}{0.14843444444444442\xspace}

 %% incorrect-true %%
\providecommand{\predicateBitpreciseInterpolKindResultsRFInterpolAbsPfIncorrectTruePlain}{}
  \renewcommand{\predicateBitpreciseInterpolKindResultsRFInterpolAbsPfIncorrectTruePlain}{1\xspace}

  % cpu-time-sum
\providecommand{\predicateBitpreciseInterpolKindResultsRFInterpolAbsPfIncorrectTrueCpuTimeSumPlain}{}
  \renewcommand{\predicateBitpreciseInterpolKindResultsRFInterpolAbsPfIncorrectTrueCpuTimeSumPlain}{94.313023519\xspace}
\providecommand{\predicateBitpreciseInterpolKindResultsRFInterpolAbsPfIncorrectTrueCpuTimeSumPlainHours}{}
  \renewcommand{\predicateBitpreciseInterpolKindResultsRFInterpolAbsPfIncorrectTrueCpuTimeSumPlainHours}{0.02619806208861111\xspace}

  % cpu-time-avg
\providecommand{\predicateBitpreciseInterpolKindResultsRFInterpolAbsPfIncorrectTrueCpuTimeAvgPlain}{}
  \renewcommand{\predicateBitpreciseInterpolKindResultsRFInterpolAbsPfIncorrectTrueCpuTimeAvgPlain}{94.313023519\xspace}
\providecommand{\predicateBitpreciseInterpolKindResultsRFInterpolAbsPfIncorrectTrueCpuTimeAvgPlainHours}{}
  \renewcommand{\predicateBitpreciseInterpolKindResultsRFInterpolAbsPfIncorrectTrueCpuTimeAvgPlainHours}{0.02619806208861111\xspace}

  % inv-succ
\providecommand{\predicateBitpreciseInterpolKindResultsRFInterpolAbsPfIncorrectTrueInvSuccPlain}{}
  \renewcommand{\predicateBitpreciseInterpolKindResultsRFInterpolAbsPfIncorrectTrueInvSuccPlain}{4\xspace}

  % inv-tries
\providecommand{\predicateBitpreciseInterpolKindResultsRFInterpolAbsPfIncorrectTrueInvTriesPlain}{}
  \renewcommand{\predicateBitpreciseInterpolKindResultsRFInterpolAbsPfIncorrectTrueInvTriesPlain}{10\xspace}

  % inv-time-sum
\providecommand{\predicateBitpreciseInterpolKindResultsRFInterpolAbsPfIncorrectTrueInvTimeSumPlain}{}
  \renewcommand{\predicateBitpreciseInterpolKindResultsRFInterpolAbsPfIncorrectTrueInvTimeSumPlain}{44.406\xspace}
\providecommand{\predicateBitpreciseInterpolKindResultsRFInterpolAbsPfIncorrectTrueInvTimeSumPlainHours}{}
  \renewcommand{\predicateBitpreciseInterpolKindResultsRFInterpolAbsPfIncorrectTrueInvTimeSumPlainHours}{0.012335\xspace}

 %% all %%
\providecommand{\predicateBitpreciseInterpolKindResultsRFInterpolAbsPfAllPlain}{}
  \renewcommand{\predicateBitpreciseInterpolKindResultsRFInterpolAbsPfAllPlain}{3488\xspace}

  % cpu-time-sum
\providecommand{\predicateBitpreciseInterpolKindResultsRFInterpolAbsPfAllCpuTimeSumPlain}{}
  \renewcommand{\predicateBitpreciseInterpolKindResultsRFInterpolAbsPfAllCpuTimeSumPlain}{589290.5770547877\xspace}
\providecommand{\predicateBitpreciseInterpolKindResultsRFInterpolAbsPfAllCpuTimeSumPlainHours}{}
  \renewcommand{\predicateBitpreciseInterpolKindResultsRFInterpolAbsPfAllCpuTimeSumPlainHours}{163.69182695966325\xspace}

  % cpu-time-avg
\providecommand{\predicateBitpreciseInterpolKindResultsRFInterpolAbsPfAllCpuTimeAvgPlain}{}
  \renewcommand{\predicateBitpreciseInterpolKindResultsRFInterpolAbsPfAllCpuTimeAvgPlain}{168.94798654093682\xspace}
\providecommand{\predicateBitpreciseInterpolKindResultsRFInterpolAbsPfAllCpuTimeAvgPlainHours}{}
  \renewcommand{\predicateBitpreciseInterpolKindResultsRFInterpolAbsPfAllCpuTimeAvgPlainHours}{0.04692999626137134\xspace}

  % inv-succ
\providecommand{\predicateBitpreciseInterpolKindResultsRFInterpolAbsPfAllInvSuccPlain}{}
  \renewcommand{\predicateBitpreciseInterpolKindResultsRFInterpolAbsPfAllInvSuccPlain}{1211\xspace}

  % inv-tries
\providecommand{\predicateBitpreciseInterpolKindResultsRFInterpolAbsPfAllInvTriesPlain}{}
  \renewcommand{\predicateBitpreciseInterpolKindResultsRFInterpolAbsPfAllInvTriesPlain}{14030\xspace}

  % inv-time-sum
\providecommand{\predicateBitpreciseInterpolKindResultsRFInterpolAbsPfAllInvTimeSumPlain}{}
  \renewcommand{\predicateBitpreciseInterpolKindResultsRFInterpolAbsPfAllInvTimeSumPlain}{187906.82499999975\xspace}
\providecommand{\predicateBitpreciseInterpolKindResultsRFInterpolAbsPfAllInvTimeSumPlainHours}{}
  \renewcommand{\predicateBitpreciseInterpolKindResultsRFInterpolAbsPfAllInvTimeSumPlainHours}{52.19634027777771\xspace}

 %% equal-only %%
\providecommand{\predicateBitpreciseInterpolKindResultsRFInterpolAbsPfEqualOnlyPlain}{}
  \renewcommand{\predicateBitpreciseInterpolKindResultsRFInterpolAbsPfEqualOnlyPlain}{1793\xspace}

  % cpu-time-sum
\providecommand{\predicateBitpreciseInterpolKindResultsRFInterpolAbsPfEqualOnlyCpuTimeSumPlain}{}
  \renewcommand{\predicateBitpreciseInterpolKindResultsRFInterpolAbsPfEqualOnlyCpuTimeSumPlain}{97438.23700635\xspace}
\providecommand{\predicateBitpreciseInterpolKindResultsRFInterpolAbsPfEqualOnlyCpuTimeSumPlainHours}{}
  \renewcommand{\predicateBitpreciseInterpolKindResultsRFInterpolAbsPfEqualOnlyCpuTimeSumPlainHours}{27.066176946208333\xspace}

  % cpu-time-avg
\providecommand{\predicateBitpreciseInterpolKindResultsRFInterpolAbsPfEqualOnlyCpuTimeAvgPlain}{}
  \renewcommand{\predicateBitpreciseInterpolKindResultsRFInterpolAbsPfEqualOnlyCpuTimeAvgPlain}{54.34369046645287\xspace}
\providecommand{\predicateBitpreciseInterpolKindResultsRFInterpolAbsPfEqualOnlyCpuTimeAvgPlainHours}{}
  \renewcommand{\predicateBitpreciseInterpolKindResultsRFInterpolAbsPfEqualOnlyCpuTimeAvgPlainHours}{0.015095469574014687\xspace}

  % inv-succ
\providecommand{\predicateBitpreciseInterpolKindResultsRFInterpolAbsPfEqualOnlyInvSuccPlain}{}
  \renewcommand{\predicateBitpreciseInterpolKindResultsRFInterpolAbsPfEqualOnlyInvSuccPlain}{977\xspace}

  % inv-tries
\providecommand{\predicateBitpreciseInterpolKindResultsRFInterpolAbsPfEqualOnlyInvTriesPlain}{}
  \renewcommand{\predicateBitpreciseInterpolKindResultsRFInterpolAbsPfEqualOnlyInvTriesPlain}{6022\xspace}

  % inv-time-sum
\providecommand{\predicateBitpreciseInterpolKindResultsRFInterpolAbsPfEqualOnlyInvTimeSumPlain}{}
  \renewcommand{\predicateBitpreciseInterpolKindResultsRFInterpolAbsPfEqualOnlyInvTimeSumPlain}{22869.83700000002\xspace}
\providecommand{\predicateBitpreciseInterpolKindResultsRFInterpolAbsPfEqualOnlyInvTimeSumPlainHours}{}
  \renewcommand{\predicateBitpreciseInterpolKindResultsRFInterpolAbsPfEqualOnlyInvTimeSumPlainHours}{6.352732500000006\xspace}

%%% predicate_base.2016-09-03_1927.results.pred-bitvectors %%%
 %% correct %%
\providecommand{\predicateBaseResultsPredBitvectorsCorrectPlain}{}
  \renewcommand{\predicateBaseResultsPredBitvectorsCorrectPlain}{1944\xspace}

  % cpu-time-sum
\providecommand{\predicateBaseResultsPredBitvectorsCorrectCpuTimeSumPlain}{}
  \renewcommand{\predicateBaseResultsPredBitvectorsCorrectCpuTimeSumPlain}{93675.79971751993\xspace}
\providecommand{\predicateBaseResultsPredBitvectorsCorrectCpuTimeSumPlainHours}{}
  \renewcommand{\predicateBaseResultsPredBitvectorsCorrectCpuTimeSumPlainHours}{26.02105547708887\xspace}

  % cpu-time-avg
\providecommand{\predicateBaseResultsPredBitvectorsCorrectCpuTimeAvgPlain}{}
  \renewcommand{\predicateBaseResultsPredBitvectorsCorrectCpuTimeAvgPlain}{48.18713977238679\xspace}
\providecommand{\predicateBaseResultsPredBitvectorsCorrectCpuTimeAvgPlainHours}{}
  \renewcommand{\predicateBaseResultsPredBitvectorsCorrectCpuTimeAvgPlainHours}{0.013385316603440776\xspace}

  % inv-succ
\providecommand{\predicateBaseResultsPredBitvectorsCorrectInvSuccPlain}{}
  \renewcommand{\predicateBaseResultsPredBitvectorsCorrectInvSuccPlain}{0\xspace}

  % inv-tries
\providecommand{\predicateBaseResultsPredBitvectorsCorrectInvTriesPlain}{}
  \renewcommand{\predicateBaseResultsPredBitvectorsCorrectInvTriesPlain}{0\xspace}

  % inv-time-sum
\providecommand{\predicateBaseResultsPredBitvectorsCorrectInvTimeSumPlain}{}
  \renewcommand{\predicateBaseResultsPredBitvectorsCorrectInvTimeSumPlain}{0.0\xspace}
\providecommand{\predicateBaseResultsPredBitvectorsCorrectInvTimeSumPlainHours}{}
  \renewcommand{\predicateBaseResultsPredBitvectorsCorrectInvTimeSumPlainHours}{0.0\xspace}

 %% incorrect %%
\providecommand{\predicateBaseResultsPredBitvectorsIncorrectPlain}{}
  \renewcommand{\predicateBaseResultsPredBitvectorsIncorrectPlain}{27\xspace}

  % cpu-time-sum
\providecommand{\predicateBaseResultsPredBitvectorsIncorrectCpuTimeSumPlain}{}
  \renewcommand{\predicateBaseResultsPredBitvectorsIncorrectCpuTimeSumPlain}{712.5640386970001\xspace}
\providecommand{\predicateBaseResultsPredBitvectorsIncorrectCpuTimeSumPlainHours}{}
  \renewcommand{\predicateBaseResultsPredBitvectorsIncorrectCpuTimeSumPlainHours}{0.19793445519361114\xspace}

  % cpu-time-avg
\providecommand{\predicateBaseResultsPredBitvectorsIncorrectCpuTimeAvgPlain}{}
  \renewcommand{\predicateBaseResultsPredBitvectorsIncorrectCpuTimeAvgPlain}{26.391260692481485\xspace}
\providecommand{\predicateBaseResultsPredBitvectorsIncorrectCpuTimeAvgPlainHours}{}
  \renewcommand{\predicateBaseResultsPredBitvectorsIncorrectCpuTimeAvgPlainHours}{0.007330905747911523\xspace}

  % inv-succ
\providecommand{\predicateBaseResultsPredBitvectorsIncorrectInvSuccPlain}{}
  \renewcommand{\predicateBaseResultsPredBitvectorsIncorrectInvSuccPlain}{0\xspace}

  % inv-tries
\providecommand{\predicateBaseResultsPredBitvectorsIncorrectInvTriesPlain}{}
  \renewcommand{\predicateBaseResultsPredBitvectorsIncorrectInvTriesPlain}{0\xspace}

  % inv-time-sum
\providecommand{\predicateBaseResultsPredBitvectorsIncorrectInvTimeSumPlain}{}
  \renewcommand{\predicateBaseResultsPredBitvectorsIncorrectInvTimeSumPlain}{0.0\xspace}
\providecommand{\predicateBaseResultsPredBitvectorsIncorrectInvTimeSumPlainHours}{}
  \renewcommand{\predicateBaseResultsPredBitvectorsIncorrectInvTimeSumPlainHours}{0.0\xspace}

 %% timeout %%
\providecommand{\predicateBaseResultsPredBitvectorsTimeoutPlain}{}
  \renewcommand{\predicateBaseResultsPredBitvectorsTimeoutPlain}{1414\xspace}

  % cpu-time-sum
\providecommand{\predicateBaseResultsPredBitvectorsTimeoutCpuTimeSumPlain}{}
  \renewcommand{\predicateBaseResultsPredBitvectorsTimeoutCpuTimeSumPlain}{432542.40992774273\xspace}
\providecommand{\predicateBaseResultsPredBitvectorsTimeoutCpuTimeSumPlainHours}{}
  \renewcommand{\predicateBaseResultsPredBitvectorsTimeoutCpuTimeSumPlainHours}{120.15066942437298\xspace}

  % cpu-time-avg
\providecommand{\predicateBaseResultsPredBitvectorsTimeoutCpuTimeAvgPlain}{}
  \renewcommand{\predicateBaseResultsPredBitvectorsTimeoutCpuTimeAvgPlain}{305.8998655783188\xspace}
\providecommand{\predicateBaseResultsPredBitvectorsTimeoutCpuTimeAvgPlainHours}{}
  \renewcommand{\predicateBaseResultsPredBitvectorsTimeoutCpuTimeAvgPlainHours}{0.08497218488286633\xspace}

  % inv-succ
\providecommand{\predicateBaseResultsPredBitvectorsTimeoutInvSuccPlain}{}
  \renewcommand{\predicateBaseResultsPredBitvectorsTimeoutInvSuccPlain}{0\xspace}

  % inv-tries
\providecommand{\predicateBaseResultsPredBitvectorsTimeoutInvTriesPlain}{}
  \renewcommand{\predicateBaseResultsPredBitvectorsTimeoutInvTriesPlain}{0\xspace}

  % inv-time-sum
\providecommand{\predicateBaseResultsPredBitvectorsTimeoutInvTimeSumPlain}{}
  \renewcommand{\predicateBaseResultsPredBitvectorsTimeoutInvTimeSumPlain}{0.0\xspace}
\providecommand{\predicateBaseResultsPredBitvectorsTimeoutInvTimeSumPlainHours}{}
  \renewcommand{\predicateBaseResultsPredBitvectorsTimeoutInvTimeSumPlainHours}{0.0\xspace}

 %% unknown-or-category-error %%
\providecommand{\predicateBaseResultsPredBitvectorsUnknownOrCategoryErrorPlain}{}
  \renewcommand{\predicateBaseResultsPredBitvectorsUnknownOrCategoryErrorPlain}{1517\xspace}

  % cpu-time-sum
\providecommand{\predicateBaseResultsPredBitvectorsUnknownOrCategoryErrorCpuTimeSumPlain}{}
  \renewcommand{\predicateBaseResultsPredBitvectorsUnknownOrCategoryErrorCpuTimeSumPlain}{440619.1379288969\xspace}
\providecommand{\predicateBaseResultsPredBitvectorsUnknownOrCategoryErrorCpuTimeSumPlainHours}{}
  \renewcommand{\predicateBaseResultsPredBitvectorsUnknownOrCategoryErrorCpuTimeSumPlainHours}{122.39420498024913\xspace}

  % cpu-time-avg
\providecommand{\predicateBaseResultsPredBitvectorsUnknownOrCategoryErrorCpuTimeAvgPlain}{}
  \renewcommand{\predicateBaseResultsPredBitvectorsUnknownOrCategoryErrorCpuTimeAvgPlain}{290.4542768153572\xspace}
\providecommand{\predicateBaseResultsPredBitvectorsUnknownOrCategoryErrorCpuTimeAvgPlainHours}{}
  \renewcommand{\predicateBaseResultsPredBitvectorsUnknownOrCategoryErrorCpuTimeAvgPlainHours}{0.08068174355982145\xspace}

  % inv-succ
\providecommand{\predicateBaseResultsPredBitvectorsUnknownOrCategoryErrorInvSuccPlain}{}
  \renewcommand{\predicateBaseResultsPredBitvectorsUnknownOrCategoryErrorInvSuccPlain}{0\xspace}

  % inv-tries
\providecommand{\predicateBaseResultsPredBitvectorsUnknownOrCategoryErrorInvTriesPlain}{}
  \renewcommand{\predicateBaseResultsPredBitvectorsUnknownOrCategoryErrorInvTriesPlain}{0\xspace}

  % inv-time-sum
\providecommand{\predicateBaseResultsPredBitvectorsUnknownOrCategoryErrorInvTimeSumPlain}{}
  \renewcommand{\predicateBaseResultsPredBitvectorsUnknownOrCategoryErrorInvTimeSumPlain}{0.0\xspace}
\providecommand{\predicateBaseResultsPredBitvectorsUnknownOrCategoryErrorInvTimeSumPlainHours}{}
  \renewcommand{\predicateBaseResultsPredBitvectorsUnknownOrCategoryErrorInvTimeSumPlainHours}{0.0\xspace}

 %% correct-false %%
\providecommand{\predicateBaseResultsPredBitvectorsCorrectFalsePlain}{}
  \renewcommand{\predicateBaseResultsPredBitvectorsCorrectFalsePlain}{553\xspace}

  % cpu-time-sum
\providecommand{\predicateBaseResultsPredBitvectorsCorrectFalseCpuTimeSumPlain}{}
  \renewcommand{\predicateBaseResultsPredBitvectorsCorrectFalseCpuTimeSumPlain}{38293.01043458401\xspace}
\providecommand{\predicateBaseResultsPredBitvectorsCorrectFalseCpuTimeSumPlainHours}{}
  \renewcommand{\predicateBaseResultsPredBitvectorsCorrectFalseCpuTimeSumPlainHours}{10.636947342940003\xspace}

  % cpu-time-avg
\providecommand{\predicateBaseResultsPredBitvectorsCorrectFalseCpuTimeAvgPlain}{}
  \renewcommand{\predicateBaseResultsPredBitvectorsCorrectFalseCpuTimeAvgPlain}{69.2459501529548\xspace}
\providecommand{\predicateBaseResultsPredBitvectorsCorrectFalseCpuTimeAvgPlainHours}{}
  \renewcommand{\predicateBaseResultsPredBitvectorsCorrectFalseCpuTimeAvgPlainHours}{0.019234986153598557\xspace}

  % inv-succ
\providecommand{\predicateBaseResultsPredBitvectorsCorrectFalseInvSuccPlain}{}
  \renewcommand{\predicateBaseResultsPredBitvectorsCorrectFalseInvSuccPlain}{0\xspace}

  % inv-tries
\providecommand{\predicateBaseResultsPredBitvectorsCorrectFalseInvTriesPlain}{}
  \renewcommand{\predicateBaseResultsPredBitvectorsCorrectFalseInvTriesPlain}{0\xspace}

  % inv-time-sum
\providecommand{\predicateBaseResultsPredBitvectorsCorrectFalseInvTimeSumPlain}{}
  \renewcommand{\predicateBaseResultsPredBitvectorsCorrectFalseInvTimeSumPlain}{0.0\xspace}
\providecommand{\predicateBaseResultsPredBitvectorsCorrectFalseInvTimeSumPlainHours}{}
  \renewcommand{\predicateBaseResultsPredBitvectorsCorrectFalseInvTimeSumPlainHours}{0.0\xspace}

 %% correct-true %%
\providecommand{\predicateBaseResultsPredBitvectorsCorrectTruePlain}{}
  \renewcommand{\predicateBaseResultsPredBitvectorsCorrectTruePlain}{1391\xspace}

  % cpu-time-sum
\providecommand{\predicateBaseResultsPredBitvectorsCorrectTrueCpuTimeSumPlain}{}
  \renewcommand{\predicateBaseResultsPredBitvectorsCorrectTrueCpuTimeSumPlain}{55382.789282936064\xspace}
\providecommand{\predicateBaseResultsPredBitvectorsCorrectTrueCpuTimeSumPlainHours}{}
  \renewcommand{\predicateBaseResultsPredBitvectorsCorrectTrueCpuTimeSumPlainHours}{15.384108134148907\xspace}

  % cpu-time-avg
\providecommand{\predicateBaseResultsPredBitvectorsCorrectTrueCpuTimeAvgPlain}{}
  \renewcommand{\predicateBaseResultsPredBitvectorsCorrectTrueCpuTimeAvgPlain}{39.81508934790515\xspace}
\providecommand{\predicateBaseResultsPredBitvectorsCorrectTrueCpuTimeAvgPlainHours}{}
  \renewcommand{\predicateBaseResultsPredBitvectorsCorrectTrueCpuTimeAvgPlainHours}{0.011059747041084764\xspace}

  % inv-succ
\providecommand{\predicateBaseResultsPredBitvectorsCorrectTrueInvSuccPlain}{}
  \renewcommand{\predicateBaseResultsPredBitvectorsCorrectTrueInvSuccPlain}{0\xspace}

  % inv-tries
\providecommand{\predicateBaseResultsPredBitvectorsCorrectTrueInvTriesPlain}{}
  \renewcommand{\predicateBaseResultsPredBitvectorsCorrectTrueInvTriesPlain}{0\xspace}

  % inv-time-sum
\providecommand{\predicateBaseResultsPredBitvectorsCorrectTrueInvTimeSumPlain}{}
  \renewcommand{\predicateBaseResultsPredBitvectorsCorrectTrueInvTimeSumPlain}{0.0\xspace}
\providecommand{\predicateBaseResultsPredBitvectorsCorrectTrueInvTimeSumPlainHours}{}
  \renewcommand{\predicateBaseResultsPredBitvectorsCorrectTrueInvTimeSumPlainHours}{0.0\xspace}

 %% incorrect-false %%
\providecommand{\predicateBaseResultsPredBitvectorsIncorrectFalsePlain}{}
  \renewcommand{\predicateBaseResultsPredBitvectorsIncorrectFalsePlain}{27\xspace}

  % cpu-time-sum
\providecommand{\predicateBaseResultsPredBitvectorsIncorrectFalseCpuTimeSumPlain}{}
  \renewcommand{\predicateBaseResultsPredBitvectorsIncorrectFalseCpuTimeSumPlain}{712.5640386970001\xspace}
\providecommand{\predicateBaseResultsPredBitvectorsIncorrectFalseCpuTimeSumPlainHours}{}
  \renewcommand{\predicateBaseResultsPredBitvectorsIncorrectFalseCpuTimeSumPlainHours}{0.19793445519361114\xspace}

  % cpu-time-avg
\providecommand{\predicateBaseResultsPredBitvectorsIncorrectFalseCpuTimeAvgPlain}{}
  \renewcommand{\predicateBaseResultsPredBitvectorsIncorrectFalseCpuTimeAvgPlain}{26.391260692481485\xspace}
\providecommand{\predicateBaseResultsPredBitvectorsIncorrectFalseCpuTimeAvgPlainHours}{}
  \renewcommand{\predicateBaseResultsPredBitvectorsIncorrectFalseCpuTimeAvgPlainHours}{0.007330905747911523\xspace}

  % inv-succ
\providecommand{\predicateBaseResultsPredBitvectorsIncorrectFalseInvSuccPlain}{}
  \renewcommand{\predicateBaseResultsPredBitvectorsIncorrectFalseInvSuccPlain}{0\xspace}

  % inv-tries
\providecommand{\predicateBaseResultsPredBitvectorsIncorrectFalseInvTriesPlain}{}
  \renewcommand{\predicateBaseResultsPredBitvectorsIncorrectFalseInvTriesPlain}{0\xspace}

  % inv-time-sum
\providecommand{\predicateBaseResultsPredBitvectorsIncorrectFalseInvTimeSumPlain}{}
  \renewcommand{\predicateBaseResultsPredBitvectorsIncorrectFalseInvTimeSumPlain}{0.0\xspace}
\providecommand{\predicateBaseResultsPredBitvectorsIncorrectFalseInvTimeSumPlainHours}{}
  \renewcommand{\predicateBaseResultsPredBitvectorsIncorrectFalseInvTimeSumPlainHours}{0.0\xspace}

 %% incorrect-true %%
\providecommand{\predicateBaseResultsPredBitvectorsIncorrectTruePlain}{}
  \renewcommand{\predicateBaseResultsPredBitvectorsIncorrectTruePlain}{0\xspace}

  % cpu-time-sum
\providecommand{\predicateBaseResultsPredBitvectorsIncorrectTrueCpuTimeSumPlain}{}
  \renewcommand{\predicateBaseResultsPredBitvectorsIncorrectTrueCpuTimeSumPlain}{0.0\xspace}
\providecommand{\predicateBaseResultsPredBitvectorsIncorrectTrueCpuTimeSumPlainHours}{}
  \renewcommand{\predicateBaseResultsPredBitvectorsIncorrectTrueCpuTimeSumPlainHours}{0.0\xspace}

  % cpu-time-avg
\providecommand{\predicateBaseResultsPredBitvectorsIncorrectTrueCpuTimeAvgPlain}{}
  \renewcommand{\predicateBaseResultsPredBitvectorsIncorrectTrueCpuTimeAvgPlain}{NaN\xspace}
\providecommand{\predicateBaseResultsPredBitvectorsIncorrectTrueCpuTimeAvgPlainHours}{}
  \renewcommand{\predicateBaseResultsPredBitvectorsIncorrectTrueCpuTimeAvgPlainHours}{NaN\xspace}

  % inv-succ
\providecommand{\predicateBaseResultsPredBitvectorsIncorrectTrueInvSuccPlain}{}
  \renewcommand{\predicateBaseResultsPredBitvectorsIncorrectTrueInvSuccPlain}{0\xspace}

  % inv-tries
\providecommand{\predicateBaseResultsPredBitvectorsIncorrectTrueInvTriesPlain}{}
  \renewcommand{\predicateBaseResultsPredBitvectorsIncorrectTrueInvTriesPlain}{0\xspace}

  % inv-time-sum
\providecommand{\predicateBaseResultsPredBitvectorsIncorrectTrueInvTimeSumPlain}{}
  \renewcommand{\predicateBaseResultsPredBitvectorsIncorrectTrueInvTimeSumPlain}{0.0\xspace}
\providecommand{\predicateBaseResultsPredBitvectorsIncorrectTrueInvTimeSumPlainHours}{}
  \renewcommand{\predicateBaseResultsPredBitvectorsIncorrectTrueInvTimeSumPlainHours}{0.0\xspace}

 %% all %%
\providecommand{\predicateBaseResultsPredBitvectorsAllPlain}{}
  \renewcommand{\predicateBaseResultsPredBitvectorsAllPlain}{3488\xspace}

  % cpu-time-sum
\providecommand{\predicateBaseResultsPredBitvectorsAllCpuTimeSumPlain}{}
  \renewcommand{\predicateBaseResultsPredBitvectorsAllCpuTimeSumPlain}{535007.5016851136\xspace}
\providecommand{\predicateBaseResultsPredBitvectorsAllCpuTimeSumPlainHours}{}
  \renewcommand{\predicateBaseResultsPredBitvectorsAllCpuTimeSumPlainHours}{148.61319491253155\xspace}

  % cpu-time-avg
\providecommand{\predicateBaseResultsPredBitvectorsAllCpuTimeAvgPlain}{}
  \renewcommand{\predicateBaseResultsPredBitvectorsAllCpuTimeAvgPlain}{153.38517823541102\xspace}
\providecommand{\predicateBaseResultsPredBitvectorsAllCpuTimeAvgPlainHours}{}
  \renewcommand{\predicateBaseResultsPredBitvectorsAllCpuTimeAvgPlainHours}{0.04260699395428084\xspace}

  % inv-succ
\providecommand{\predicateBaseResultsPredBitvectorsAllInvSuccPlain}{}
  \renewcommand{\predicateBaseResultsPredBitvectorsAllInvSuccPlain}{0\xspace}

  % inv-tries
\providecommand{\predicateBaseResultsPredBitvectorsAllInvTriesPlain}{}
  \renewcommand{\predicateBaseResultsPredBitvectorsAllInvTriesPlain}{0\xspace}

  % inv-time-sum
\providecommand{\predicateBaseResultsPredBitvectorsAllInvTimeSumPlain}{}
  \renewcommand{\predicateBaseResultsPredBitvectorsAllInvTimeSumPlain}{0.0\xspace}
\providecommand{\predicateBaseResultsPredBitvectorsAllInvTimeSumPlainHours}{}
  \renewcommand{\predicateBaseResultsPredBitvectorsAllInvTimeSumPlainHours}{0.0\xspace}

 %% equal-only %%
\providecommand{\predicateBaseResultsPredBitvectorsEqualOnlyPlain}{}
  \renewcommand{\predicateBaseResultsPredBitvectorsEqualOnlyPlain}{1793\xspace}

  % cpu-time-sum
\providecommand{\predicateBaseResultsPredBitvectorsEqualOnlyCpuTimeSumPlain}{}
  \renewcommand{\predicateBaseResultsPredBitvectorsEqualOnlyCpuTimeSumPlain}{64177.80902456609\xspace}
\providecommand{\predicateBaseResultsPredBitvectorsEqualOnlyCpuTimeSumPlainHours}{}
  \renewcommand{\predicateBaseResultsPredBitvectorsEqualOnlyCpuTimeSumPlainHours}{17.82716917349058\xspace}

  % cpu-time-avg
\providecommand{\predicateBaseResultsPredBitvectorsEqualOnlyCpuTimeAvgPlain}{}
  \renewcommand{\predicateBaseResultsPredBitvectorsEqualOnlyCpuTimeAvgPlain}{35.79353542920585\xspace}
\providecommand{\predicateBaseResultsPredBitvectorsEqualOnlyCpuTimeAvgPlainHours}{}
  \renewcommand{\predicateBaseResultsPredBitvectorsEqualOnlyCpuTimeAvgPlainHours}{0.00994264873033496\xspace}

  % inv-succ
\providecommand{\predicateBaseResultsPredBitvectorsEqualOnlyInvSuccPlain}{}
  \renewcommand{\predicateBaseResultsPredBitvectorsEqualOnlyInvSuccPlain}{0\xspace}

  % inv-tries
\providecommand{\predicateBaseResultsPredBitvectorsEqualOnlyInvTriesPlain}{}
  \renewcommand{\predicateBaseResultsPredBitvectorsEqualOnlyInvTriesPlain}{0\xspace}

  % inv-time-sum
\providecommand{\predicateBaseResultsPredBitvectorsEqualOnlyInvTimeSumPlain}{}
  \renewcommand{\predicateBaseResultsPredBitvectorsEqualOnlyInvTimeSumPlain}{0.0\xspace}
\providecommand{\predicateBaseResultsPredBitvectorsEqualOnlyInvTimeSumPlainHours}{}
  \renewcommand{\predicateBaseResultsPredBitvectorsEqualOnlyInvTimeSumPlainHours}{0.0\xspace}



\begin{table}
 \caption{Details on analyses using checking interpolants on invariance and their baseline}
 \label{table:interpKind}
\begin{adjustbox}{max width=\textwidth}
  \begin{tabular}{l
                  S[table-format=4.0, round-mode=off, round-precision=3]
                  S[table-format=3.0, round-mode=off, round-precision=3]
                  S[table-format=1.0, round-mode=off, round-precision=3]
                  S[table-format=2.0, round-mode=off, round-precision=3]
                  S[table-format=1.3, round-mode=figures, round-precision=3]
                  S[table-format=4.0, round-mode=off, round-precision=3]
                  S[table-format=4.0, round-mode=off, round-precision=3]
                  S[table-format=3.0, round-mode=figures, round-precision=3]
                  S[table-format=2.1, round-mode=figures, round-precision=3]
                  S[table-format=2.1, round-mode=figures, round-precision=3]}
\toprule
 & \multicolumn{2}{c}{\textbf{correct}} & \multicolumn{2}{c}{\textbf{wrong}} & \multicolumn{3}{c}{\textbf{Invariants (equal)}} & \multicolumn{3}{c}{\textbf{CPU time (h)}}\\
 \textbf{int-check-}& \multicolumn{1}{c}{proof} & \multicolumn{1}{c}{alarm} & \multicolumn{1}{c}{proof} & \multicolumn{1}{c}{alarm} & \multicolumn{1}{c}{time (h)} & \multicolumn{1}{c}{tries} & \multicolumn{1}{c}{succ} & \multicolumn{1}{c}{all} & \multicolumn{1}{c}{correct} & \multicolumn{1}{c}{equal} \\
\cmidrule(lr){1-1}\cmidrule(lr){2-3}\cmidrule(lr){4-5}\cmidrule(lr){6-8}\cmidrule(lr){9-11}

\textbf{base300} & \predicateBaseResultsPredBitvectorsCorrectTruePlain & \predicateBaseResultsPredBitvectorsCorrectFalsePlain & \predicateBaseResultsPredBitvectorsIncorrectTruePlain & \predicateBaseResultsPredBitvectorsIncorrectFalsePlain &  &  &  & \predicateBaseResultsPredBitvectorsAllCpuTimeSumPlainHours & \predicateBaseResultsPredBitvectorsCorrectCpuTimeSumPlainHours & \predicateBaseResultsPredBitvectorsEqualOnlyCpuTimeSumPlainHours \\
\textbf{abs} & \predicateBitpreciseInterpolKindResultsRFInterpolAbsCorrectTruePlain & \predicateBitpreciseInterpolKindResultsRFInterpolAbsCorrectFalsePlain & \predicateBitpreciseInterpolKindResultsRFInterpolAbsIncorrectTruePlain & \predicateBitpreciseInterpolKindResultsRFInterpolAbsIncorrectFalsePlain & \predicateBitpreciseInterpolKindResultsRFInterpolAbsCorrectInvTimeSumPlainHours & \predicateBitpreciseInterpolKindResultsRFInterpolAbsCorrectInvTriesPlain & \predicateBitpreciseInterpolKindResultsRFInterpolAbsCorrectInvSuccPlain & \predicateBitpreciseInterpolKindResultsRFInterpolAbsAllCpuTimeSumPlainHours & \predicateBitpreciseInterpolKindResultsRFInterpolAbsCorrectCpuTimeSumPlainHours & \predicateBitpreciseInterpolKindResultsRFInterpolAbsEqualOnlyCpuTimeSumPlainHours \\
\textbf{path} & \predicateBitpreciseInterpolKindResultsRFInterpolPfCorrectTruePlain & \predicateBitpreciseInterpolKindResultsRFInterpolPfCorrectFalsePlain & \predicateBitpreciseInterpolKindResultsRFInterpolPfIncorrectTruePlain & \predicateBitpreciseInterpolKindResultsRFInterpolPfIncorrectFalsePlain & \predicateBitpreciseInterpolKindResultsRFInterpolPfCorrectInvTimeSumPlainHours & \predicateBitpreciseInterpolKindResultsRFInterpolPfCorrectInvTriesPlain & \predicateBitpreciseInterpolKindResultsRFInterpolPfCorrectInvSuccPlain & \predicateBitpreciseInterpolKindResultsRFInterpolPfAllCpuTimeSumPlainHours & \predicateBitpreciseInterpolKindResultsRFInterpolPfCorrectCpuTimeSumPlainHours & \predicateBitpreciseInterpolKindResultsRFInterpolPfEqualOnlyCpuTimeSumPlainHours \\
\textbf{prec} & \predicateBitpreciseInterpolKindResultsRFInterpolPrecCorrectTruePlain & \predicateBitpreciseInterpolKindResultsRFInterpolPrecCorrectFalsePlain & \predicateBitpreciseInterpolKindResultsRFInterpolPrecIncorrectTruePlain & \predicateBitpreciseInterpolKindResultsRFInterpolPrecIncorrectFalsePlain & \predicateBitpreciseInterpolKindResultsRFInterpolPrecCorrectInvTimeSumPlainHours & \predicateBitpreciseInterpolKindResultsRFInterpolPrecCorrectInvTriesPlain & \predicateBitpreciseInterpolKindResultsRFInterpolPrecCorrectInvSuccPlain & \predicateBitpreciseInterpolKindResultsRFInterpolPrecAllCpuTimeSumPlainHours & \predicateBitpreciseInterpolKindResultsRFInterpolPrecCorrectCpuTimeSumPlainHours & \predicateBitpreciseInterpolKindResultsRFInterpolPrecEqualOnlyCpuTimeSumPlainHours \\
\textbf{prec-path} & \predicateBitpreciseInterpolKindResultsRFInterpolPrecPfCorrectTruePlain & \predicateBitpreciseInterpolKindResultsRFInterpolPrecPfCorrectFalsePlain & \predicateBitpreciseInterpolKindResultsRFInterpolPrecPfIncorrectTruePlain & \predicateBitpreciseInterpolKindResultsRFInterpolPrecPfIncorrectFalsePlain & \predicateBitpreciseInterpolKindResultsRFInterpolPrecPfCorrectInvTimeSumPlainHours & \predicateBitpreciseInterpolKindResultsRFInterpolPrecPfCorrectInvTriesPlain & \predicateBitpreciseInterpolKindResultsRFInterpolPrecPfCorrectInvSuccPlain & \predicateBitpreciseInterpolKindResultsRFInterpolPrecPfAllCpuTimeSumPlainHours & \predicateBitpreciseInterpolKindResultsRFInterpolPrecPfCorrectCpuTimeSumPlainHours & \predicateBitpreciseInterpolKindResultsRFInterpolPrecPfEqualOnlyCpuTimeSumPlainHours \\
\textbf{abs-path} & \predicateBitpreciseInterpolKindResultsRFInterpolAbsPfCorrectTruePlain & \predicateBitpreciseInterpolKindResultsRFInterpolAbsPfCorrectFalsePlain & \predicateBitpreciseInterpolKindResultsRFInterpolAbsPfIncorrectTruePlain & \predicateBitpreciseInterpolKindResultsRFInterpolAbsPfIncorrectFalsePlain & \predicateBitpreciseInterpolKindResultsRFInterpolAbsPfCorrectInvTimeSumPlainHours & \predicateBitpreciseInterpolKindResultsRFInterpolAbsPfCorrectInvTriesPlain & \predicateBitpreciseInterpolKindResultsRFInterpolAbsPfCorrectInvSuccPlain & \predicateBitpreciseInterpolKindResultsRFInterpolAbsPfAllCpuTimeSumPlainHours & \predicateBitpreciseInterpolKindResultsRFInterpolAbsPfCorrectCpuTimeSumPlainHours & \predicateBitpreciseInterpolKindResultsRFInterpolAbsPfEqualOnlyCpuTimeSumPlainHours \\
\textbf{prec-abs} & \predicateBitpreciseInterpolKindResultsRFInterpolAbsPrecCorrectTruePlain & \predicateBitpreciseInterpolKindResultsRFInterpolAbsPrecCorrectFalsePlain & \predicateBitpreciseInterpolKindResultsRFInterpolAbsPrecIncorrectTruePlain & \predicateBitpreciseInterpolKindResultsRFInterpolAbsPrecIncorrectFalsePlain & \predicateBitpreciseInterpolKindResultsRFInterpolAbsPrecCorrectInvTimeSumPlainHours & \predicateBitpreciseInterpolKindResultsRFInterpolAbsPrecCorrectInvTriesPlain & \predicateBitpreciseInterpolKindResultsRFInterpolAbsPrecCorrectInvSuccPlain & \predicateBitpreciseInterpolKindResultsRFInterpolAbsPrecAllCpuTimeSumPlainHours & \predicateBitpreciseInterpolKindResultsRFInterpolAbsPrecCorrectCpuTimeSumPlainHours & \predicateBitpreciseInterpolKindResultsRFInterpolAbsPrecEqualOnlyCpuTimeSumPlainHours \\
\textbf{prec-abs-path} & \predicateBitpreciseInterpolKindResultsRFInterpolAbsPrecPfCorrectTruePlain & \predicateBitpreciseInterpolKindResultsRFInterpolAbsPrecPfCorrectFalsePlain & \predicateBitpreciseInterpolKindResultsRFInterpolAbsPrecPfIncorrectTruePlain & \predicateBitpreciseInterpolKindResultsRFInterpolAbsPrecPfIncorrectFalsePlain & \predicateBitpreciseInterpolKindResultsRFInterpolAbsPrecPfCorrectInvTimeSumPlainHours & \predicateBitpreciseInterpolKindResultsRFInterpolAbsPrecPfCorrectInvTriesPlain & \predicateBitpreciseInterpolKindResultsRFInterpolAbsPrecPfCorrectInvSuccPlain & \predicateBitpreciseInterpolKindResultsRFInterpolAbsPrecPfAllCpuTimeSumPlainHours & \predicateBitpreciseInterpolKindResultsRFInterpolAbsPrecPfCorrectCpuTimeSumPlainHours & \predicateBitpreciseInterpolKindResultsRFInterpolAbsPrecPfEqualOnlyCpuTimeSumPlainHours \\



\bottomrule
 \end{tabular}
 \end{adjustbox}
\end{table}

Checking interpolants on invariance with $k$-induction is assumed to be a rather lightweight-invariant generation approach. Slicing infeasible counterexample paths into several infeasible prefixes
and choosing one of them for interpolant computation~\cite{Beyer:RefinementSelection} is a technique that tries to guide the refinement such that the found predicates have a more positive impact on the
analysis than choosing another infeasible prefix would have. We extend this approach to not only search for one infeasible prefix that is used for interpolant computation, but instead we compute interpolants for each of the infeasible prefixes and afterwards check the computed interpolants on 1-inductivity.

\autoref{table:interpKind} shows the results for computing invariants with that strategy combined with all usage strategies we introduced earlier. All configurations using invariants are strictly worse
than \textbf{base300}. Fewer verification tasks --- safe and unsafe --- could be analyzed successfully, and the overall CPU time increases from \SI{149}{\hour} to over \SI{160}{\hour}.
When looking only at the equal and correct tasks, the difference is growing to almost \SI{10}{\hour}, an increase in the time spent of over \SI{50}{\percent}.
Most of the additional time, about \SI{7}{\hour}, is spent by trying to generate invariants, which is
successful in approximately 1 out of 6 cases. The time for invariant generation is however not measured as CPU time but as wall time, such that the comparison of these times makes not much sense.

%The other 3\,h are needed additionally for the main analysis which is quite unexpected as the invariants should be used for speeding up the analysis and not for slowing it down.

%base300
\providecommand{\predicateBaseResultsPredBitvectorsInvFailCpuTimeSumPlainHours}{}
\renewcommand{\predicateBaseResultsPredBitvectorsInvFailCpuTimeSumPlainHours}{13.90788057995389\xspace}
\providecommand{\predicateBaseResultsPredBitvectorsInvFailWallTimeSumPlainHours}{}
  \renewcommand{\predicateBaseResultsPredBitvectorsInvFailWallTimeSumPlainHours}{9.189025214115876\xspace}

\providecommand{\predicateBaseResultsPredBitvectorsInvSuccCpuTimeSumPlainHours}{}
\renewcommand{\predicateBaseResultsPredBitvectorsInvSuccCpuTimeSumPlainHours}{3.9192885935366655\xspace}
\providecommand{\predicateBaseResultsPredBitvectorsInvSuccWallTimeSumPlainHours}{}
  \renewcommand{\predicateBaseResultsPredBitvectorsInvSuccWallTimeSumPlainHours}{2.0585675003798523\xspace}


%prec-abs-path
\providecommand{\predicateBitpreciseInterpolKindResultsRFInterpolAbsPrecPfInvFailCpuTimeSumPlainHours}{}
\renewcommand{\predicateBitpreciseInterpolKindResultsRFInterpolAbsPrecPfInvFailCpuTimeSumPlainHours}{20.671640237939172\xspace}
\providecommand{\predicateBitpreciseInterpolKindResultsRFInterpolAbsPrecPfInvFailInvTimeSumPlainHours}{}
\renewcommand{\predicateBitpreciseInterpolKindResultsRFInterpolAbsPrecPfInvFailInvTimeSumPlainHours}{4.997347500000001\xspace}
\providecommand{\predicateBitpreciseInterpolKindResultsRFInterpolAbsPrecPfInvFailWallTimeSumPlainHours}{}
\renewcommand{\predicateBitpreciseInterpolKindResultsRFInterpolAbsPrecPfInvFailWallTimeSumPlainHours}{13.99464856121093\xspace}

\providecommand{\predicateBitpreciseInterpolKindResultsRFInterpolAbsPrecPfInvSuccCpuTimeSumPlainHours}{}
\renewcommand{\predicateBitpreciseInterpolKindResultsRFInterpolAbsPrecPfInvSuccCpuTimeSumPlainHours}{6.849107935684721\xspace}
\providecommand{\predicateBitpreciseInterpolKindResultsRFInterpolAbsPrecPfInvSuccInvTimeSumPlainHours}{}
\renewcommand{\predicateBitpreciseInterpolKindResultsRFInterpolAbsPrecPfInvSuccInvTimeSumPlainHours}{1.4061988888888906\xspace}
\providecommand{\predicateBitpreciseInterpolKindResultsRFInterpolAbsPrecPfInvSuccWallTimeSumPlainHours}{}
  \renewcommand{\predicateBitpreciseInterpolKindResultsRFInterpolAbsPrecPfInvSuccWallTimeSumPlainHours}{3.422355886101887\xspace}


%prec
\providecommand{\predicateBitpreciseInterpolKindResultsRFInterpolPrecInvFailCpuTimeSumPlainHours}{}
\renewcommand{\predicateBitpreciseInterpolKindResultsRFInterpolPrecInvFailCpuTimeSumPlainHours}{20.73301050581781\xspace}
\providecommand{\predicateBitpreciseInterpolKindResultsRFInterpolPrecInvFailInvTimeSumPlainHours}{}
\renewcommand{\predicateBitpreciseInterpolKindResultsRFInterpolPrecInvFailInvTimeSumPlainHours}{5.042057500000004\xspace}
\providecommand{\predicateBitpreciseInterpolKindResultsRFInterpolPrecInvFailWallTimeSumPlainHours}{}
  \renewcommand{\predicateBitpreciseInterpolKindResultsRFInterpolPrecInvFailWallTimeSumPlainHours}{14.035632757212799\xspace}

\providecommand{\predicateBitpreciseInterpolKindResultsRFInterpolPrecInvSuccCpuTimeSumPlainHours}{}
\renewcommand{\predicateBitpreciseInterpolKindResultsRFInterpolPrecInvSuccCpuTimeSumPlainHours}{7.1571605865008365\xspace}
\providecommand{\predicateBitpreciseInterpolKindResultsRFInterpolPrecInvSuccInvTimeSumPlainHours}{}
\renewcommand{\predicateBitpreciseInterpolKindResultsRFInterpolPrecInvSuccInvTimeSumPlainHours}{1.7395186111111103\xspace}
\providecommand{\predicateBitpreciseInterpolKindResultsRFInterpolPrecInvSuccWallTimeSumPlainHours}{}
  \renewcommand{\predicateBitpreciseInterpolKindResultsRFInterpolPrecInvSuccWallTimeSumPlainHours}{3.6776945145271145\xspace}


%abs
\providecommand{\predicateBitpreciseInterpolKindResultsRFInterpolAbsInvFailCpuTimeSumPlainHours}{}
\renewcommand{\predicateBitpreciseInterpolKindResultsRFInterpolAbsInvFailCpuTimeSumPlainHours}{20.4187654216922\xspace}
\providecommand{\predicateBitpreciseInterpolKindResultsRFInterpolAbsInvFailInvTimeSumPlainHours}{}
\renewcommand{\predicateBitpreciseInterpolKindResultsRFInterpolAbsInvFailInvTimeSumPlainHours}{4.992174444444435\xspace}
\providecommand{\predicateBitpreciseInterpolKindResultsRFInterpolAbsInvFailWallTimeSumPlainHours}{}
  \renewcommand{\predicateBitpreciseInterpolKindResultsRFInterpolAbsInvFailWallTimeSumPlainHours}{13.737679721648568\xspace}

\providecommand{\predicateBitpreciseInterpolKindResultsRFInterpolAbsInvSuccCpuTimeSumPlainHours}{}
\renewcommand{\predicateBitpreciseInterpolKindResultsRFInterpolAbsInvSuccCpuTimeSumPlainHours}{6.5379099819275\xspace}
\providecommand{\predicateBitpreciseInterpolKindResultsRFInterpolAbsInvSuccInvTimeSumPlainHours}{}
\renewcommand{\predicateBitpreciseInterpolKindResultsRFInterpolAbsInvSuccInvTimeSumPlainHours}{1.346193611111113\xspace}
\providecommand{\predicateBitpreciseInterpolKindResultsRFInterpolAbsInvSuccWallTimeSumPlainHours}{}
  \renewcommand{\predicateBitpreciseInterpolKindResultsRFInterpolAbsInvSuccWallTimeSumPlainHours}{3.190095941290905\xspace}


\begin{table}
\centering
 \caption{Drastic increase of CPU time for analyses succeeding in using invariants computed by checking interpolants}
  \label{table:interpKind_detail}
\begin{adjustbox}{max width=\textwidth}
  \begin{tabular}{lr
                  S[table-format=2.2, round-mode=figures, round-precision=3]
                  S[table-format=2.2, round-mode=figures, round-precision=3]
                  S[table-format=2.2, round-mode=figures, round-precision=3]
                  S[table-format=2.2, round-mode=figures, round-precision=3]
                  S[table-format=1.2, round-mode=figures, round-precision=3]
                  S[table-format=2.2, round-mode=figures, round-precision=3]
                  S[table-format=2.2, round-mode=figures, round-precision=3]
                  S[table-format=2.2, round-mode=figures, round-precision=3]}
\toprule
 && \multicolumn{4}{c}{invariant generation failed} & \multicolumn{4}{c}{invariant generation succeeded} \\
  &&  \multicolumn{1}{c}{\textbf{base300}} & \multicolumn{1}{c}{\textbf{prec-abs-pf}} & \multicolumn{1}{c}{\textbf{prec}} & \multicolumn{1}{c}{\textbf{abs}} & \multicolumn{1}{c}{\textbf{base300}} & \multicolumn{1}{c}{\textbf{prec-abs-pf}} & \multicolumn{1}{c}{\textbf{prec}} & \multicolumn{1}{c}{\textbf{abs}}\\
\cmidrule(lr){3-6}\cmidrule(lr){7-10}
CPU time (h) && \predicateBaseResultsPredBitvectorsInvFailCpuTimeSumPlainHours & \predicateBitpreciseInterpolKindResultsRFInterpolAbsPrecPfInvFailCpuTimeSumPlainHours & \predicateBitpreciseInterpolKindResultsRFInterpolPrecInvFailCpuTimeSumPlainHours & \predicateBitpreciseInterpolKindResultsRFInterpolAbsInvFailCpuTimeSumPlainHours & \predicateBaseResultsPredBitvectorsInvSuccCpuTimeSumPlainHours & \predicateBitpreciseInterpolKindResultsRFInterpolAbsPrecPfInvSuccCpuTimeSumPlainHours & \predicateBitpreciseInterpolKindResultsRFInterpolPrecInvSuccCpuTimeSumPlainHours & \predicateBitpreciseInterpolKindResultsRFInterpolAbsInvSuccCpuTimeSumPlainHours\\
inv time (h) && 0 &  \predicateBitpreciseInterpolKindResultsRFInterpolAbsPrecPfInvFailInvTimeSumPlainHours & \predicateBitpreciseInterpolKindResultsRFInterpolPrecInvFailInvTimeSumPlainHours & \predicateBitpreciseInterpolKindResultsRFInterpolAbsInvFailInvTimeSumPlainHours & 0 & \predicateBitpreciseInterpolKindResultsRFInterpolAbsPrecPfInvSuccInvTimeSumPlainHours & \predicateBitpreciseInterpolKindResultsRFInterpolPrecInvSuccInvTimeSumPlainHours & \predicateBitpreciseInterpolKindResultsRFInterpolAbsInvSuccInvTimeSumPlainHours\\
\cmidrule(lr){3-6}\cmidrule(lr){7-10}
%& \predicateBaseResultsPredBitvectorsInvFailCpuTimeSumPlainHours & 16.249387 & \predicateBaseResultsPredBitvectorsInvSuccCpuTimeSumPlainHours & 5.533062845\\
 &\textbf{-}& \predicateBaseResultsPredBitvectorsInvFailCpuTimeSumPlainHours & 15.6743 & 15.69096 & 15.426595 & \predicateBaseResultsPredBitvectorsInvSuccCpuTimeSumPlainHours & 5.44291 & 5.41765 & 5.1917163\\[.5ex]
\textbf{increase} (\%) & & & 12.95 & 12.95 & 10.79 & & 38.78 & 38.27 & 32.4\\[1ex]
\midrule
Wall time (h) && \predicateBaseResultsPredBitvectorsInvFailWallTimeSumPlainHours & \predicateBitpreciseInterpolKindResultsRFInterpolAbsPrecPfInvFailWallTimeSumPlainHours & \predicateBitpreciseInterpolKindResultsRFInterpolPrecInvFailWallTimeSumPlainHours & \predicateBitpreciseInterpolKindResultsRFInterpolAbsInvFailWallTimeSumPlainHours & \predicateBaseResultsPredBitvectorsInvSuccWallTimeSumPlainHours & \predicateBitpreciseInterpolKindResultsRFInterpolAbsPrecPfInvSuccWallTimeSumPlainHours & \predicateBitpreciseInterpolKindResultsRFInterpolPrecInvSuccWallTimeSumPlainHours & \predicateBitpreciseInterpolKindResultsRFInterpolAbsInvSuccWallTimeSumPlainHours\\
inv time (h) && 0 &  \predicateBitpreciseInterpolKindResultsRFInterpolAbsPrecPfInvFailInvTimeSumPlainHours & \predicateBitpreciseInterpolKindResultsRFInterpolPrecInvFailInvTimeSumPlainHours & \predicateBitpreciseInterpolKindResultsRFInterpolAbsInvFailInvTimeSumPlainHours & 0 & \predicateBitpreciseInterpolKindResultsRFInterpolAbsPrecPfInvSuccInvTimeSumPlainHours & \predicateBitpreciseInterpolKindResultsRFInterpolPrecInvSuccInvTimeSumPlainHours & \predicateBitpreciseInterpolKindResultsRFInterpolAbsInvSuccInvTimeSumPlainHours\\
\cmidrule(lr){3-6}\cmidrule(lr){7-10}
%& \predicateBaseResultsPredBitvectorsInvFailCpuTimeSumPlainHours & 16.249387 & \predicateBaseResultsPredBitvectorsInvSuccCpuTimeSumPlainHours & 5.533062845\\
 &\textbf{-}& \predicateBaseResultsPredBitvectorsInvFailWallTimeSumPlainHours & 8.9973 & 8.99358 & 8.7455 & \predicateBaseResultsPredBitvectorsInvSuccWallTimeSumPlainHours & 2.016165 & 1.9381845 & 1.8439023\\[.5ex]
\textbf{decrease} (\%) & & & 2.07 & 2.18 & 4.79 & & 1.94 & 5.83 & 10.68\\

\bottomrule
 \end{tabular}
 \end{adjustbox}
\end{table}

To take a closer look at the differences in time consumption, \autoref{table:interpKind_detail} shows the CPU time and wall time separately for the correct and equal analyzed verification tasks that either
failed or  succeeded to use invariants. It is surprising that the increase in CPU time is higher for the tasks where invariant generation was successful. Compared to the baseline
about \SI{40}{\percent} more time are needed for these tasks but only \SI{13}{\percent} more time is needed for the tasks where invariant generation was not successful.
When using the wall time instead of the CPU time for
comparison, the numbers are changing, for unsuccessful invariant generation the time decreases by \SIrange{2}{4}{\percent}
and for the successful invariant generation even from \SIrange{2}{10}{\percent}.
Both comparisons lead to the conclusion that other threads are influencing our measurement, and in fact when executing the benchmark set limited to one virtual core, wall time and CPU time are equal and
the time for generating invariants is still smaller then the difference in the measured times. Our research did not come to any conclusion where the additional time --- accounting about \SI{3}{\hour} in our
experiments --- could be spent. Invariant generation, as well as all other parts of \CPAchecker{}, are run single-threaded, the \ac{SMT} solver \MathSAT{} is also running single-threaded and while 
profiling the application we did not find any additional threads being used. Also when looking at the ratio of wall and CPU time, it stays approximately the same for the tasks with failed invariant 
generation (about \SI{50}{\percent}) and successful invariant generation (about \SI{90}{\percent}), which means that there is no evidence for additional
time in configurations with invariant generation in particular, but a part
of the used CPU time is always spent differently, for example, for garbage collection or resource measurement.


When we look back at \autoref{table:interpKind} we can see that some of the configurations have one unsafe verification task,
\texttt{ldv-linux-3.0/usb\_urb-drivers-input-misc-keyspan\_re\\mote.ko\_false-unreach-call.cil.out.i.pp.i}, where the analyses concluded that this program is safe.
This is a side-effect of using invariants. Without invariants the analysis of this task does not terminate, with invariants being added to either the path or the abstraction formula, the analysis
terminates and reports a wrong result. The baseline is not able to analyze this program, even in \SI{900}{\second}. Therefore we do not know if the wrong result is caused by invalid invariants,
a wrong usage of invariants or if the program cannot be analyzed correctly with the given \textbf{base300} configuration. In the SV-COMP~2016 there were two tools that terminated in time, one
of them (Blast) reported a specification violation, so we can be sure that the problem is not that this verification task has a wrong label.

Another remarkable point is that some of the tasks are running into a timeout because of the sliced prefix generation. This issue can be observed better with increasing amount of possible
slices that need to be tested. The initial idea was to increase the number of abstraction states by changing the block operator $\blk$ (cf. \autoref{title:predicatecpa})
such that abstractions should be computed more often. The default configuration
is that an abstraction is only computed at each occurrence of a loop head. We add that additionally, an abstraction is computed when control-flow meets. With this modification, the infeasible
counterexamples consist of more abstraction states than before, which means that there are also more states that can be removed for creating different infeasible prefixes. But with this increased
number of possibilities, the number of timeouts rises, for example, for \textbf{int-check-abs} from \num{1563} to \num{2394}. At the same time \textbf{base300} has \num{1414} timeouts. For all other
configurations for generating invariants with
this approach the numbers are comparable. The reason for the longer prefix generation times is, in our opinion, that removing certain formulas from the satisfiability check has an impact on the \ac{SMT}
solver, which is in turn not able to prove unsatisfiability. Formulas leading to such a behavior when removed could, \eg, be related to pointer-aliasing handling, because for that, many relations are
introduced. By removing some relations, the possible state space grows and makes the unsatisfiability-check harder.

Overall checking interpolants on invariance with $k$-induction seems to be working for only a small set of verification tasks. For the other tasks, either the invariant generation takes too long, or the
found invariants do not influence the analysis in the expected way. The idea to increase the amount of interpolants being checked by changing the behavior of the $\blk$ operator made the performance even worse.
While more infeasible sliced prefixes do also mean more interpolants, and potentially a higher success rate in finding invariants, the additional time necessary for the prefix generation is just too high.

\subsubsection*{Path Invariants}
%%% predicate_bitprecise_pathinvariants.2016-09-10_1338.results.pathInvariants-invCPA %%%
 %% correct %%
\providecommand{\predicateBitprecisePathinvariantsResultsPathInvariantsInvCPACorrectPlain}{}
  \renewcommand{\predicateBitprecisePathinvariantsResultsPathInvariantsInvCPACorrectPlain}{1846\xspace}

  % cpu-time-sum
\providecommand{\predicateBitprecisePathinvariantsResultsPathInvariantsInvCPACorrectCpuTimeSumPlain}{}
  \renewcommand{\predicateBitprecisePathinvariantsResultsPathInvariantsInvCPACorrectCpuTimeSumPlain}{111598.89775255592\xspace}
\providecommand{\predicateBitprecisePathinvariantsResultsPathInvariantsInvCPACorrectCpuTimeSumPlainHours}{}
  \renewcommand{\predicateBitprecisePathinvariantsResultsPathInvariantsInvCPACorrectCpuTimeSumPlainHours}{30.99969382015442\xspace}

  % wall-time-sum
\providecommand{\predicateBitprecisePathinvariantsResultsPathInvariantsInvCPACorrectWallTimeSumPlain}{}
  \renewcommand{\predicateBitprecisePathinvariantsResultsPathInvariantsInvCPACorrectWallTimeSumPlain}{62256.46204423421\xspace}
\providecommand{\predicateBitprecisePathinvariantsResultsPathInvariantsInvCPACorrectWallTimeSumPlainHours}{}
  \renewcommand{\predicateBitprecisePathinvariantsResultsPathInvariantsInvCPACorrectWallTimeSumPlainHours}{17.29346167895395\xspace}

  % cpu-time-avg
\providecommand{\predicateBitprecisePathinvariantsResultsPathInvariantsInvCPACorrectCpuTimeAvgPlain}{}
  \renewcommand{\predicateBitprecisePathinvariantsResultsPathInvariantsInvCPACorrectCpuTimeAvgPlain}{60.4544408193694\xspace}
\providecommand{\predicateBitprecisePathinvariantsResultsPathInvariantsInvCPACorrectCpuTimeAvgPlainHours}{}
  \renewcommand{\predicateBitprecisePathinvariantsResultsPathInvariantsInvCPACorrectCpuTimeAvgPlainHours}{0.016792900227602613\xspace}

  % wall-time-avg
\providecommand{\predicateBitprecisePathinvariantsResultsPathInvariantsInvCPACorrectWallTimeAvgPlain}{}
  \renewcommand{\predicateBitprecisePathinvariantsResultsPathInvariantsInvCPACorrectWallTimeAvgPlain}{33.72506069568484\xspace}
\providecommand{\predicateBitprecisePathinvariantsResultsPathInvariantsInvCPACorrectWallTimeAvgPlainHours}{}
  \renewcommand{\predicateBitprecisePathinvariantsResultsPathInvariantsInvCPACorrectWallTimeAvgPlainHours}{0.00936807241546801\xspace}

  % inv-succ
\providecommand{\predicateBitprecisePathinvariantsResultsPathInvariantsInvCPACorrectInvSuccPlain}{}
  \renewcommand{\predicateBitprecisePathinvariantsResultsPathInvariantsInvCPACorrectInvSuccPlain}{1428\xspace}

  % inv-tries
\providecommand{\predicateBitprecisePathinvariantsResultsPathInvariantsInvCPACorrectInvTriesPlain}{}
  \renewcommand{\predicateBitprecisePathinvariantsResultsPathInvariantsInvCPACorrectInvTriesPlain}{4719\xspace}

  % inv-time-sum
\providecommand{\predicateBitprecisePathinvariantsResultsPathInvariantsInvCPACorrectInvTimeSumPlain}{}
  \renewcommand{\predicateBitprecisePathinvariantsResultsPathInvariantsInvCPACorrectInvTimeSumPlain}{8484.282000000007\xspace}
\providecommand{\predicateBitprecisePathinvariantsResultsPathInvariantsInvCPACorrectInvTimeSumPlainHours}{}
  \renewcommand{\predicateBitprecisePathinvariantsResultsPathInvariantsInvCPACorrectInvTimeSumPlainHours}{2.356745000000002\xspace}

 %% incorrect %%
\providecommand{\predicateBitprecisePathinvariantsResultsPathInvariantsInvCPAIncorrectPlain}{}
  \renewcommand{\predicateBitprecisePathinvariantsResultsPathInvariantsInvCPAIncorrectPlain}{27\xspace}

  % cpu-time-sum
\providecommand{\predicateBitprecisePathinvariantsResultsPathInvariantsInvCPAIncorrectCpuTimeSumPlain}{}
  \renewcommand{\predicateBitprecisePathinvariantsResultsPathInvariantsInvCPAIncorrectCpuTimeSumPlain}{1183.27833542\xspace}
\providecommand{\predicateBitprecisePathinvariantsResultsPathInvariantsInvCPAIncorrectCpuTimeSumPlainHours}{}
  \renewcommand{\predicateBitprecisePathinvariantsResultsPathInvariantsInvCPAIncorrectCpuTimeSumPlainHours}{0.3286884265055556\xspace}

  % wall-time-sum
\providecommand{\predicateBitprecisePathinvariantsResultsPathInvariantsInvCPAIncorrectWallTimeSumPlain}{}
  \renewcommand{\predicateBitprecisePathinvariantsResultsPathInvariantsInvCPAIncorrectWallTimeSumPlain}{566.5751395224701\xspace}
\providecommand{\predicateBitprecisePathinvariantsResultsPathInvariantsInvCPAIncorrectWallTimeSumPlainHours}{}
  \renewcommand{\predicateBitprecisePathinvariantsResultsPathInvariantsInvCPAIncorrectWallTimeSumPlainHours}{0.15738198320068614\xspace}

  % cpu-time-avg
\providecommand{\predicateBitprecisePathinvariantsResultsPathInvariantsInvCPAIncorrectCpuTimeAvgPlain}{}
  \renewcommand{\predicateBitprecisePathinvariantsResultsPathInvariantsInvCPAIncorrectCpuTimeAvgPlain}{43.82512353407407\xspace}
\providecommand{\predicateBitprecisePathinvariantsResultsPathInvariantsInvCPAIncorrectCpuTimeAvgPlainHours}{}
  \renewcommand{\predicateBitprecisePathinvariantsResultsPathInvariantsInvCPAIncorrectCpuTimeAvgPlainHours}{0.012173645426131688\xspace}

  % wall-time-avg
\providecommand{\predicateBitprecisePathinvariantsResultsPathInvariantsInvCPAIncorrectWallTimeAvgPlain}{}
  \renewcommand{\predicateBitprecisePathinvariantsResultsPathInvariantsInvCPAIncorrectWallTimeAvgPlain}{20.984264426758152\xspace}
\providecommand{\predicateBitprecisePathinvariantsResultsPathInvariantsInvCPAIncorrectWallTimeAvgPlainHours}{}
  \renewcommand{\predicateBitprecisePathinvariantsResultsPathInvariantsInvCPAIncorrectWallTimeAvgPlainHours}{0.005828962340766153\xspace}

  % inv-succ
\providecommand{\predicateBitprecisePathinvariantsResultsPathInvariantsInvCPAIncorrectInvSuccPlain}{}
  \renewcommand{\predicateBitprecisePathinvariantsResultsPathInvariantsInvCPAIncorrectInvSuccPlain}{3\xspace}

  % inv-tries
\providecommand{\predicateBitprecisePathinvariantsResultsPathInvariantsInvCPAIncorrectInvTriesPlain}{}
  \renewcommand{\predicateBitprecisePathinvariantsResultsPathInvariantsInvCPAIncorrectInvTriesPlain}{60\xspace}

  % inv-time-sum
\providecommand{\predicateBitprecisePathinvariantsResultsPathInvariantsInvCPAIncorrectInvTimeSumPlain}{}
  \renewcommand{\predicateBitprecisePathinvariantsResultsPathInvariantsInvCPAIncorrectInvTimeSumPlain}{119.849\xspace}
\providecommand{\predicateBitprecisePathinvariantsResultsPathInvariantsInvCPAIncorrectInvTimeSumPlainHours}{}
  \renewcommand{\predicateBitprecisePathinvariantsResultsPathInvariantsInvCPAIncorrectInvTimeSumPlainHours}{0.03329138888888889\xspace}

 %% timeout %%
\providecommand{\predicateBitprecisePathinvariantsResultsPathInvariantsInvCPATimeoutPlain}{}
  \renewcommand{\predicateBitprecisePathinvariantsResultsPathInvariantsInvCPATimeoutPlain}{1499\xspace}

  % cpu-time-sum
\providecommand{\predicateBitprecisePathinvariantsResultsPathInvariantsInvCPATimeoutCpuTimeSumPlain}{}
  \renewcommand{\predicateBitprecisePathinvariantsResultsPathInvariantsInvCPATimeoutCpuTimeSumPlain}{458267.74822316313\xspace}
\providecommand{\predicateBitprecisePathinvariantsResultsPathInvariantsInvCPATimeoutCpuTimeSumPlainHours}{}
  \renewcommand{\predicateBitprecisePathinvariantsResultsPathInvariantsInvCPATimeoutCpuTimeSumPlainHours}{127.29659672865643\xspace}

  % wall-time-sum
\providecommand{\predicateBitprecisePathinvariantsResultsPathInvariantsInvCPATimeoutWallTimeSumPlain}{}
  \renewcommand{\predicateBitprecisePathinvariantsResultsPathInvariantsInvCPATimeoutWallTimeSumPlain}{385266.7221231276\xspace}
\providecommand{\predicateBitprecisePathinvariantsResultsPathInvariantsInvCPATimeoutWallTimeSumPlainHours}{}
  \renewcommand{\predicateBitprecisePathinvariantsResultsPathInvariantsInvCPATimeoutWallTimeSumPlainHours}{107.018533923091\xspace}

  % cpu-time-avg
\providecommand{\predicateBitprecisePathinvariantsResultsPathInvariantsInvCPATimeoutCpuTimeAvgPlain}{}
  \renewcommand{\predicateBitprecisePathinvariantsResultsPathInvariantsInvCPATimeoutCpuTimeAvgPlain}{305.71564257716017\xspace}
\providecommand{\predicateBitprecisePathinvariantsResultsPathInvariantsInvCPATimeoutCpuTimeAvgPlainHours}{}
  \renewcommand{\predicateBitprecisePathinvariantsResultsPathInvariantsInvCPATimeoutCpuTimeAvgPlainHours}{0.08492101182698894\xspace}

  % wall-time-avg
\providecommand{\predicateBitprecisePathinvariantsResultsPathInvariantsInvCPATimeoutWallTimeAvgPlain}{}
  \renewcommand{\predicateBitprecisePathinvariantsResultsPathInvariantsInvCPATimeoutWallTimeAvgPlain}{257.01582529895103\xspace}
\providecommand{\predicateBitprecisePathinvariantsResultsPathInvariantsInvCPATimeoutWallTimeAvgPlainHours}{}
  \renewcommand{\predicateBitprecisePathinvariantsResultsPathInvariantsInvCPATimeoutWallTimeAvgPlainHours}{0.07139328480526418\xspace}

  % inv-succ
\providecommand{\predicateBitprecisePathinvariantsResultsPathInvariantsInvCPATimeoutInvSuccPlain}{}
  \renewcommand{\predicateBitprecisePathinvariantsResultsPathInvariantsInvCPATimeoutInvSuccPlain}{53883\xspace}

  % inv-tries
\providecommand{\predicateBitprecisePathinvariantsResultsPathInvariantsInvCPATimeoutInvTriesPlain}{}
  \renewcommand{\predicateBitprecisePathinvariantsResultsPathInvariantsInvCPATimeoutInvTriesPlain}{61094\xspace}

  % inv-time-sum
\providecommand{\predicateBitprecisePathinvariantsResultsPathInvariantsInvCPATimeoutInvTimeSumPlain}{}
  \renewcommand{\predicateBitprecisePathinvariantsResultsPathInvariantsInvCPATimeoutInvTimeSumPlain}{14506.15700000001\xspace}
\providecommand{\predicateBitprecisePathinvariantsResultsPathInvariantsInvCPATimeoutInvTimeSumPlainHours}{}
  \renewcommand{\predicateBitprecisePathinvariantsResultsPathInvariantsInvCPATimeoutInvTimeSumPlainHours}{4.029488055555558\xspace}

 %% unknown-or-category-error %%
\providecommand{\predicateBitprecisePathinvariantsResultsPathInvariantsInvCPAUnknownOrCategoryErrorPlain}{}
  \renewcommand{\predicateBitprecisePathinvariantsResultsPathInvariantsInvCPAUnknownOrCategoryErrorPlain}{1615\xspace}

  % cpu-time-sum
\providecommand{\predicateBitprecisePathinvariantsResultsPathInvariantsInvCPAUnknownOrCategoryErrorCpuTimeSumPlain}{}
  \renewcommand{\predicateBitprecisePathinvariantsResultsPathInvariantsInvCPAUnknownOrCategoryErrorCpuTimeSumPlain}{471990.4606770793\xspace}
\providecommand{\predicateBitprecisePathinvariantsResultsPathInvariantsInvCPAUnknownOrCategoryErrorCpuTimeSumPlainHours}{}
  \renewcommand{\predicateBitprecisePathinvariantsResultsPathInvariantsInvCPAUnknownOrCategoryErrorCpuTimeSumPlainHours}{131.1084612991887\xspace}

  % wall-time-sum
\providecommand{\predicateBitprecisePathinvariantsResultsPathInvariantsInvCPAUnknownOrCategoryErrorWallTimeSumPlain}{}
  \renewcommand{\predicateBitprecisePathinvariantsResultsPathInvariantsInvCPAUnknownOrCategoryErrorWallTimeSumPlain}{394047.456606367\xspace}
\providecommand{\predicateBitprecisePathinvariantsResultsPathInvariantsInvCPAUnknownOrCategoryErrorWallTimeSumPlainHours}{}
  \renewcommand{\predicateBitprecisePathinvariantsResultsPathInvariantsInvCPAUnknownOrCategoryErrorWallTimeSumPlainHours}{109.45762683510195\xspace}

  % cpu-time-avg
\providecommand{\predicateBitprecisePathinvariantsResultsPathInvariantsInvCPAUnknownOrCategoryErrorCpuTimeAvgPlain}{}
  \renewcommand{\predicateBitprecisePathinvariantsResultsPathInvariantsInvCPAUnknownOrCategoryErrorCpuTimeAvgPlain}{292.25415521800574\xspace}
\providecommand{\predicateBitprecisePathinvariantsResultsPathInvariantsInvCPAUnknownOrCategoryErrorCpuTimeAvgPlainHours}{}
  \renewcommand{\predicateBitprecisePathinvariantsResultsPathInvariantsInvCPAUnknownOrCategoryErrorCpuTimeAvgPlainHours}{0.08118170978277937\xspace}

  % wall-time-avg
\providecommand{\predicateBitprecisePathinvariantsResultsPathInvariantsInvCPAUnknownOrCategoryErrorWallTimeAvgPlain}{}
  \renewcommand{\predicateBitprecisePathinvariantsResultsPathInvariantsInvCPAUnknownOrCategoryErrorWallTimeAvgPlain}{243.99223319279693\xspace}
\providecommand{\predicateBitprecisePathinvariantsResultsPathInvariantsInvCPAUnknownOrCategoryErrorWallTimeAvgPlainHours}{}
  \renewcommand{\predicateBitprecisePathinvariantsResultsPathInvariantsInvCPAUnknownOrCategoryErrorWallTimeAvgPlainHours}{0.06777562033133248\xspace}

  % inv-succ
\providecommand{\predicateBitprecisePathinvariantsResultsPathInvariantsInvCPAUnknownOrCategoryErrorInvSuccPlain}{}
  \renewcommand{\predicateBitprecisePathinvariantsResultsPathInvariantsInvCPAUnknownOrCategoryErrorInvSuccPlain}{53883\xspace}

  % inv-tries
\providecommand{\predicateBitprecisePathinvariantsResultsPathInvariantsInvCPAUnknownOrCategoryErrorInvTriesPlain}{}
  \renewcommand{\predicateBitprecisePathinvariantsResultsPathInvariantsInvCPAUnknownOrCategoryErrorInvTriesPlain}{61437\xspace}

  % inv-time-sum
\providecommand{\predicateBitprecisePathinvariantsResultsPathInvariantsInvCPAUnknownOrCategoryErrorInvTimeSumPlain}{}
  \renewcommand{\predicateBitprecisePathinvariantsResultsPathInvariantsInvCPAUnknownOrCategoryErrorInvTimeSumPlain}{15480.154000000013\xspace}
\providecommand{\predicateBitprecisePathinvariantsResultsPathInvariantsInvCPAUnknownOrCategoryErrorInvTimeSumPlainHours}{}
  \renewcommand{\predicateBitprecisePathinvariantsResultsPathInvariantsInvCPAUnknownOrCategoryErrorInvTimeSumPlainHours}{4.300042777777781\xspace}

 %% correct-false %%
\providecommand{\predicateBitprecisePathinvariantsResultsPathInvariantsInvCPACorrectFalsePlain}{}
  \renewcommand{\predicateBitprecisePathinvariantsResultsPathInvariantsInvCPACorrectFalsePlain}{519\xspace}

  % cpu-time-sum
\providecommand{\predicateBitprecisePathinvariantsResultsPathInvariantsInvCPACorrectFalseCpuTimeSumPlain}{}
  \renewcommand{\predicateBitprecisePathinvariantsResultsPathInvariantsInvCPACorrectFalseCpuTimeSumPlain}{37347.92062543605\xspace}
\providecommand{\predicateBitprecisePathinvariantsResultsPathInvariantsInvCPACorrectFalseCpuTimeSumPlainHours}{}
  \renewcommand{\predicateBitprecisePathinvariantsResultsPathInvariantsInvCPACorrectFalseCpuTimeSumPlainHours}{10.374422395954458\xspace}

  % wall-time-sum
\providecommand{\predicateBitprecisePathinvariantsResultsPathInvariantsInvCPACorrectFalseWallTimeSumPlain}{}
  \renewcommand{\predicateBitprecisePathinvariantsResultsPathInvariantsInvCPACorrectFalseWallTimeSumPlain}{23382.190789456225\xspace}
\providecommand{\predicateBitprecisePathinvariantsResultsPathInvariantsInvCPACorrectFalseWallTimeSumPlainHours}{}
  \renewcommand{\predicateBitprecisePathinvariantsResultsPathInvariantsInvCPACorrectFalseWallTimeSumPlainHours}{6.4950529970711735\xspace}

  % cpu-time-avg
\providecommand{\predicateBitprecisePathinvariantsResultsPathInvariantsInvCPACorrectFalseCpuTimeAvgPlain}{}
  \renewcommand{\predicateBitprecisePathinvariantsResultsPathInvariantsInvCPACorrectFalseCpuTimeAvgPlain}{71.96131141702514\xspace}
\providecommand{\predicateBitprecisePathinvariantsResultsPathInvariantsInvCPACorrectFalseCpuTimeAvgPlainHours}{}
  \renewcommand{\predicateBitprecisePathinvariantsResultsPathInvariantsInvCPACorrectFalseCpuTimeAvgPlainHours}{0.01998925317139587\xspace}

  % wall-time-avg
\providecommand{\predicateBitprecisePathinvariantsResultsPathInvariantsInvCPACorrectFalseWallTimeAvgPlain}{}
  \renewcommand{\predicateBitprecisePathinvariantsResultsPathInvariantsInvCPACorrectFalseWallTimeAvgPlain}{45.05239073112953\xspace}
\providecommand{\predicateBitprecisePathinvariantsResultsPathInvariantsInvCPACorrectFalseWallTimeAvgPlainHours}{}
  \renewcommand{\predicateBitprecisePathinvariantsResultsPathInvariantsInvCPACorrectFalseWallTimeAvgPlainHours}{0.012514552980869313\xspace}

  % inv-succ
\providecommand{\predicateBitprecisePathinvariantsResultsPathInvariantsInvCPACorrectFalseInvSuccPlain}{}
  \renewcommand{\predicateBitprecisePathinvariantsResultsPathInvariantsInvCPACorrectFalseInvSuccPlain}{455\xspace}

  % inv-tries
\providecommand{\predicateBitprecisePathinvariantsResultsPathInvariantsInvCPACorrectFalseInvTriesPlain}{}
  \renewcommand{\predicateBitprecisePathinvariantsResultsPathInvariantsInvCPACorrectFalseInvTriesPlain}{1489\xspace}

  % inv-time-sum
\providecommand{\predicateBitprecisePathinvariantsResultsPathInvariantsInvCPACorrectFalseInvTimeSumPlain}{}
  \renewcommand{\predicateBitprecisePathinvariantsResultsPathInvariantsInvCPACorrectFalseInvTimeSumPlain}{2068.579000000002\xspace}
\providecommand{\predicateBitprecisePathinvariantsResultsPathInvariantsInvCPACorrectFalseInvTimeSumPlainHours}{}
  \renewcommand{\predicateBitprecisePathinvariantsResultsPathInvariantsInvCPACorrectFalseInvTimeSumPlainHours}{0.5746052777777784\xspace}

 %% correct-true %%
\providecommand{\predicateBitprecisePathinvariantsResultsPathInvariantsInvCPACorrectTruePlain}{}
  \renewcommand{\predicateBitprecisePathinvariantsResultsPathInvariantsInvCPACorrectTruePlain}{1327\xspace}

  % cpu-time-sum
\providecommand{\predicateBitprecisePathinvariantsResultsPathInvariantsInvCPACorrectTrueCpuTimeSumPlain}{}
  \renewcommand{\predicateBitprecisePathinvariantsResultsPathInvariantsInvCPACorrectTrueCpuTimeSumPlain}{74250.97712712\xspace}
\providecommand{\predicateBitprecisePathinvariantsResultsPathInvariantsInvCPACorrectTrueCpuTimeSumPlainHours}{}
  \renewcommand{\predicateBitprecisePathinvariantsResultsPathInvariantsInvCPACorrectTrueCpuTimeSumPlainHours}{20.6252714242\xspace}

  % wall-time-sum
\providecommand{\predicateBitprecisePathinvariantsResultsPathInvariantsInvCPACorrectTrueWallTimeSumPlain}{}
  \renewcommand{\predicateBitprecisePathinvariantsResultsPathInvariantsInvCPACorrectTrueWallTimeSumPlain}{38874.27125477804\xspace}
\providecommand{\predicateBitprecisePathinvariantsResultsPathInvariantsInvCPACorrectTrueWallTimeSumPlainHours}{}
  \renewcommand{\predicateBitprecisePathinvariantsResultsPathInvariantsInvCPACorrectTrueWallTimeSumPlainHours}{10.798408681882789\xspace}

  % cpu-time-avg
\providecommand{\predicateBitprecisePathinvariantsResultsPathInvariantsInvCPACorrectTrueCpuTimeAvgPlain}{}
  \renewcommand{\predicateBitprecisePathinvariantsResultsPathInvariantsInvCPACorrectTrueCpuTimeAvgPlain}{55.954014413805574\xspace}
\providecommand{\predicateBitprecisePathinvariantsResultsPathInvariantsInvCPACorrectTrueCpuTimeAvgPlainHours}{}
  \renewcommand{\predicateBitprecisePathinvariantsResultsPathInvariantsInvCPACorrectTrueCpuTimeAvgPlainHours}{0.015542781781612659\xspace}

  % wall-time-avg
\providecommand{\predicateBitprecisePathinvariantsResultsPathInvariantsInvCPACorrectTrueWallTimeAvgPlain}{}
  \renewcommand{\predicateBitprecisePathinvariantsResultsPathInvariantsInvCPACorrectTrueWallTimeAvgPlain}{29.294853997571998\xspace}
\providecommand{\predicateBitprecisePathinvariantsResultsPathInvariantsInvCPACorrectTrueWallTimeAvgPlainHours}{}
  \renewcommand{\predicateBitprecisePathinvariantsResultsPathInvariantsInvCPACorrectTrueWallTimeAvgPlainHours}{0.00813745944377\xspace}

  % inv-succ
\providecommand{\predicateBitprecisePathinvariantsResultsPathInvariantsInvCPACorrectTrueInvSuccPlain}{}
  \renewcommand{\predicateBitprecisePathinvariantsResultsPathInvariantsInvCPACorrectTrueInvSuccPlain}{973\xspace}

  % inv-tries
\providecommand{\predicateBitprecisePathinvariantsResultsPathInvariantsInvCPACorrectTrueInvTriesPlain}{}
  \renewcommand{\predicateBitprecisePathinvariantsResultsPathInvariantsInvCPACorrectTrueInvTriesPlain}{3230\xspace}

  % inv-time-sum
\providecommand{\predicateBitprecisePathinvariantsResultsPathInvariantsInvCPACorrectTrueInvTimeSumPlain}{}
  \renewcommand{\predicateBitprecisePathinvariantsResultsPathInvariantsInvCPACorrectTrueInvTimeSumPlain}{6415.702999999995\xspace}
\providecommand{\predicateBitprecisePathinvariantsResultsPathInvariantsInvCPACorrectTrueInvTimeSumPlainHours}{}
  \renewcommand{\predicateBitprecisePathinvariantsResultsPathInvariantsInvCPACorrectTrueInvTimeSumPlainHours}{1.7821397222222208\xspace}

 %% incorrect-false %%
\providecommand{\predicateBitprecisePathinvariantsResultsPathInvariantsInvCPAIncorrectFalsePlain}{}
  \renewcommand{\predicateBitprecisePathinvariantsResultsPathInvariantsInvCPAIncorrectFalsePlain}{27\xspace}

  % cpu-time-sum
\providecommand{\predicateBitprecisePathinvariantsResultsPathInvariantsInvCPAIncorrectFalseCpuTimeSumPlain}{}
  \renewcommand{\predicateBitprecisePathinvariantsResultsPathInvariantsInvCPAIncorrectFalseCpuTimeSumPlain}{1183.27833542\xspace}
\providecommand{\predicateBitprecisePathinvariantsResultsPathInvariantsInvCPAIncorrectFalseCpuTimeSumPlainHours}{}
  \renewcommand{\predicateBitprecisePathinvariantsResultsPathInvariantsInvCPAIncorrectFalseCpuTimeSumPlainHours}{0.3286884265055556\xspace}

  % wall-time-sum
\providecommand{\predicateBitprecisePathinvariantsResultsPathInvariantsInvCPAIncorrectFalseWallTimeSumPlain}{}
  \renewcommand{\predicateBitprecisePathinvariantsResultsPathInvariantsInvCPAIncorrectFalseWallTimeSumPlain}{566.5751395224701\xspace}
\providecommand{\predicateBitprecisePathinvariantsResultsPathInvariantsInvCPAIncorrectFalseWallTimeSumPlainHours}{}
  \renewcommand{\predicateBitprecisePathinvariantsResultsPathInvariantsInvCPAIncorrectFalseWallTimeSumPlainHours}{0.15738198320068614\xspace}

  % cpu-time-avg
\providecommand{\predicateBitprecisePathinvariantsResultsPathInvariantsInvCPAIncorrectFalseCpuTimeAvgPlain}{}
  \renewcommand{\predicateBitprecisePathinvariantsResultsPathInvariantsInvCPAIncorrectFalseCpuTimeAvgPlain}{43.82512353407407\xspace}
\providecommand{\predicateBitprecisePathinvariantsResultsPathInvariantsInvCPAIncorrectFalseCpuTimeAvgPlainHours}{}
  \renewcommand{\predicateBitprecisePathinvariantsResultsPathInvariantsInvCPAIncorrectFalseCpuTimeAvgPlainHours}{0.012173645426131688\xspace}

  % wall-time-avg
\providecommand{\predicateBitprecisePathinvariantsResultsPathInvariantsInvCPAIncorrectFalseWallTimeAvgPlain}{}
  \renewcommand{\predicateBitprecisePathinvariantsResultsPathInvariantsInvCPAIncorrectFalseWallTimeAvgPlain}{20.984264426758152\xspace}
\providecommand{\predicateBitprecisePathinvariantsResultsPathInvariantsInvCPAIncorrectFalseWallTimeAvgPlainHours}{}
  \renewcommand{\predicateBitprecisePathinvariantsResultsPathInvariantsInvCPAIncorrectFalseWallTimeAvgPlainHours}{0.005828962340766153\xspace}

  % inv-succ
\providecommand{\predicateBitprecisePathinvariantsResultsPathInvariantsInvCPAIncorrectFalseInvSuccPlain}{}
  \renewcommand{\predicateBitprecisePathinvariantsResultsPathInvariantsInvCPAIncorrectFalseInvSuccPlain}{3\xspace}

  % inv-tries
\providecommand{\predicateBitprecisePathinvariantsResultsPathInvariantsInvCPAIncorrectFalseInvTriesPlain}{}
  \renewcommand{\predicateBitprecisePathinvariantsResultsPathInvariantsInvCPAIncorrectFalseInvTriesPlain}{60\xspace}

  % inv-time-sum
\providecommand{\predicateBitprecisePathinvariantsResultsPathInvariantsInvCPAIncorrectFalseInvTimeSumPlain}{}
  \renewcommand{\predicateBitprecisePathinvariantsResultsPathInvariantsInvCPAIncorrectFalseInvTimeSumPlain}{119.849\xspace}
\providecommand{\predicateBitprecisePathinvariantsResultsPathInvariantsInvCPAIncorrectFalseInvTimeSumPlainHours}{}
  \renewcommand{\predicateBitprecisePathinvariantsResultsPathInvariantsInvCPAIncorrectFalseInvTimeSumPlainHours}{0.03329138888888889\xspace}

 %% incorrect-true %%
\providecommand{\predicateBitprecisePathinvariantsResultsPathInvariantsInvCPAIncorrectTruePlain}{}
  \renewcommand{\predicateBitprecisePathinvariantsResultsPathInvariantsInvCPAIncorrectTruePlain}{0\xspace}

  % cpu-time-sum
\providecommand{\predicateBitprecisePathinvariantsResultsPathInvariantsInvCPAIncorrectTrueCpuTimeSumPlain}{}
  \renewcommand{\predicateBitprecisePathinvariantsResultsPathInvariantsInvCPAIncorrectTrueCpuTimeSumPlain}{0.0\xspace}
\providecommand{\predicateBitprecisePathinvariantsResultsPathInvariantsInvCPAIncorrectTrueCpuTimeSumPlainHours}{}
  \renewcommand{\predicateBitprecisePathinvariantsResultsPathInvariantsInvCPAIncorrectTrueCpuTimeSumPlainHours}{0.0\xspace}

  % wall-time-sum
\providecommand{\predicateBitprecisePathinvariantsResultsPathInvariantsInvCPAIncorrectTrueWallTimeSumPlain}{}
  \renewcommand{\predicateBitprecisePathinvariantsResultsPathInvariantsInvCPAIncorrectTrueWallTimeSumPlain}{0.0\xspace}
\providecommand{\predicateBitprecisePathinvariantsResultsPathInvariantsInvCPAIncorrectTrueWallTimeSumPlainHours}{}
  \renewcommand{\predicateBitprecisePathinvariantsResultsPathInvariantsInvCPAIncorrectTrueWallTimeSumPlainHours}{0.0\xspace}

  % cpu-time-avg
\providecommand{\predicateBitprecisePathinvariantsResultsPathInvariantsInvCPAIncorrectTrueCpuTimeAvgPlain}{}
  \renewcommand{\predicateBitprecisePathinvariantsResultsPathInvariantsInvCPAIncorrectTrueCpuTimeAvgPlain}{NaN\xspace}
\providecommand{\predicateBitprecisePathinvariantsResultsPathInvariantsInvCPAIncorrectTrueCpuTimeAvgPlainHours}{}
  \renewcommand{\predicateBitprecisePathinvariantsResultsPathInvariantsInvCPAIncorrectTrueCpuTimeAvgPlainHours}{NaN\xspace}

  % wall-time-avg
\providecommand{\predicateBitprecisePathinvariantsResultsPathInvariantsInvCPAIncorrectTrueWallTimeAvgPlain}{}
  \renewcommand{\predicateBitprecisePathinvariantsResultsPathInvariantsInvCPAIncorrectTrueWallTimeAvgPlain}{NaN\xspace}
\providecommand{\predicateBitprecisePathinvariantsResultsPathInvariantsInvCPAIncorrectTrueWallTimeAvgPlainHours}{}
  \renewcommand{\predicateBitprecisePathinvariantsResultsPathInvariantsInvCPAIncorrectTrueWallTimeAvgPlainHours}{NaN\xspace}

  % inv-succ
\providecommand{\predicateBitprecisePathinvariantsResultsPathInvariantsInvCPAIncorrectTrueInvSuccPlain}{}
  \renewcommand{\predicateBitprecisePathinvariantsResultsPathInvariantsInvCPAIncorrectTrueInvSuccPlain}{0\xspace}

  % inv-tries
\providecommand{\predicateBitprecisePathinvariantsResultsPathInvariantsInvCPAIncorrectTrueInvTriesPlain}{}
  \renewcommand{\predicateBitprecisePathinvariantsResultsPathInvariantsInvCPAIncorrectTrueInvTriesPlain}{0\xspace}

  % inv-time-sum
\providecommand{\predicateBitprecisePathinvariantsResultsPathInvariantsInvCPAIncorrectTrueInvTimeSumPlain}{}
  \renewcommand{\predicateBitprecisePathinvariantsResultsPathInvariantsInvCPAIncorrectTrueInvTimeSumPlain}{0.0\xspace}
\providecommand{\predicateBitprecisePathinvariantsResultsPathInvariantsInvCPAIncorrectTrueInvTimeSumPlainHours}{}
  \renewcommand{\predicateBitprecisePathinvariantsResultsPathInvariantsInvCPAIncorrectTrueInvTimeSumPlainHours}{0.0\xspace}

 %% all %%
\providecommand{\predicateBitprecisePathinvariantsResultsPathInvariantsInvCPAAllPlain}{}
  \renewcommand{\predicateBitprecisePathinvariantsResultsPathInvariantsInvCPAAllPlain}{3488\xspace}

  % cpu-time-sum
\providecommand{\predicateBitprecisePathinvariantsResultsPathInvariantsInvCPAAllCpuTimeSumPlain}{}
  \renewcommand{\predicateBitprecisePathinvariantsResultsPathInvariantsInvCPAAllCpuTimeSumPlain}{584772.6367650569\xspace}
\providecommand{\predicateBitprecisePathinvariantsResultsPathInvariantsInvCPAAllCpuTimeSumPlainHours}{}
  \renewcommand{\predicateBitprecisePathinvariantsResultsPathInvariantsInvCPAAllCpuTimeSumPlainHours}{162.43684354584914\xspace}

  % wall-time-sum
\providecommand{\predicateBitprecisePathinvariantsResultsPathInvariantsInvCPAAllWallTimeSumPlain}{}
  \renewcommand{\predicateBitprecisePathinvariantsResultsPathInvariantsInvCPAAllWallTimeSumPlain}{456870.49379012437\xspace}
\providecommand{\predicateBitprecisePathinvariantsResultsPathInvariantsInvCPAAllWallTimeSumPlainHours}{}
  \renewcommand{\predicateBitprecisePathinvariantsResultsPathInvariantsInvCPAAllWallTimeSumPlainHours}{126.90847049725677\xspace}

  % cpu-time-avg
\providecommand{\predicateBitprecisePathinvariantsResultsPathInvariantsInvCPAAllCpuTimeAvgPlain}{}
  \renewcommand{\predicateBitprecisePathinvariantsResultsPathInvariantsInvCPAAllCpuTimeAvgPlain}{167.65270549456906\xspace}
\providecommand{\predicateBitprecisePathinvariantsResultsPathInvariantsInvCPAAllCpuTimeAvgPlainHours}{}
  \renewcommand{\predicateBitprecisePathinvariantsResultsPathInvariantsInvCPAAllCpuTimeAvgPlainHours}{0.04657019597071363\xspace}

  % wall-time-avg
\providecommand{\predicateBitprecisePathinvariantsResultsPathInvariantsInvCPAAllWallTimeAvgPlain}{}
  \renewcommand{\predicateBitprecisePathinvariantsResultsPathInvariantsInvCPAAllWallTimeAvgPlain}{130.98351312790263\xspace}
\providecommand{\predicateBitprecisePathinvariantsResultsPathInvariantsInvCPAAllWallTimeAvgPlainHours}{}
  \renewcommand{\predicateBitprecisePathinvariantsResultsPathInvariantsInvCPAAllWallTimeAvgPlainHours}{0.036384309202195174\xspace}

  % inv-succ
\providecommand{\predicateBitprecisePathinvariantsResultsPathInvariantsInvCPAAllInvSuccPlain}{}
  \renewcommand{\predicateBitprecisePathinvariantsResultsPathInvariantsInvCPAAllInvSuccPlain}{55314\xspace}

  % inv-tries
\providecommand{\predicateBitprecisePathinvariantsResultsPathInvariantsInvCPAAllInvTriesPlain}{}
  \renewcommand{\predicateBitprecisePathinvariantsResultsPathInvariantsInvCPAAllInvTriesPlain}{66216\xspace}

  % inv-time-sum
\providecommand{\predicateBitprecisePathinvariantsResultsPathInvariantsInvCPAAllInvTimeSumPlain}{}
  \renewcommand{\predicateBitprecisePathinvariantsResultsPathInvariantsInvCPAAllInvTimeSumPlain}{24084.28500000001\xspace}
\providecommand{\predicateBitprecisePathinvariantsResultsPathInvariantsInvCPAAllInvTimeSumPlainHours}{}
  \renewcommand{\predicateBitprecisePathinvariantsResultsPathInvariantsInvCPAAllInvTimeSumPlainHours}{6.69007916666667\xspace}

 %% equal-only %%
\providecommand{\predicateBitprecisePathinvariantsResultsPathInvariantsInvCPAEqualOnlyPlain}{}
  \renewcommand{\predicateBitprecisePathinvariantsResultsPathInvariantsInvCPAEqualOnlyPlain}{1832\xspace}

  % cpu-time-sum
\providecommand{\predicateBitprecisePathinvariantsResultsPathInvariantsInvCPAEqualOnlyCpuTimeSumPlain}{}
  \renewcommand{\predicateBitprecisePathinvariantsResultsPathInvariantsInvCPAEqualOnlyCpuTimeSumPlain}{109915.92727186791\xspace}
\providecommand{\predicateBitprecisePathinvariantsResultsPathInvariantsInvCPAEqualOnlyCpuTimeSumPlainHours}{}
  \renewcommand{\predicateBitprecisePathinvariantsResultsPathInvariantsInvCPAEqualOnlyCpuTimeSumPlainHours}{30.53220201996331\xspace}

  % wall-time-sum
\providecommand{\predicateBitprecisePathinvariantsResultsPathInvariantsInvCPAEqualOnlyWallTimeSumPlain}{}
  \renewcommand{\predicateBitprecisePathinvariantsResultsPathInvariantsInvCPAEqualOnlyWallTimeSumPlain}{60932.65291714148\xspace}
\providecommand{\predicateBitprecisePathinvariantsResultsPathInvariantsInvCPAEqualOnlyWallTimeSumPlainHours}{}
  \renewcommand{\predicateBitprecisePathinvariantsResultsPathInvariantsInvCPAEqualOnlyWallTimeSumPlainHours}{16.92573692142819\xspace}

  % cpu-time-avg
\providecommand{\predicateBitprecisePathinvariantsResultsPathInvariantsInvCPAEqualOnlyCpuTimeAvgPlain}{}
  \renewcommand{\predicateBitprecisePathinvariantsResultsPathInvariantsInvCPAEqualOnlyCpuTimeAvgPlain}{59.99777689512441\xspace}
\providecommand{\predicateBitprecisePathinvariantsResultsPathInvariantsInvCPAEqualOnlyCpuTimeAvgPlainHours}{}
  \renewcommand{\predicateBitprecisePathinvariantsResultsPathInvariantsInvCPAEqualOnlyCpuTimeAvgPlainHours}{0.01666604913753456\xspace}

  % wall-time-avg
\providecommand{\predicateBitprecisePathinvariantsResultsPathInvariantsInvCPAEqualOnlyWallTimeAvgPlain}{}
  \renewcommand{\predicateBitprecisePathinvariantsResultsPathInvariantsInvCPAEqualOnlyWallTimeAvgPlain}{33.260181723330504\xspace}
\providecommand{\predicateBitprecisePathinvariantsResultsPathInvariantsInvCPAEqualOnlyWallTimeAvgPlainHours}{}
  \renewcommand{\predicateBitprecisePathinvariantsResultsPathInvariantsInvCPAEqualOnlyWallTimeAvgPlainHours}{0.009238939367591808\xspace}

  % inv-succ
\providecommand{\predicateBitprecisePathinvariantsResultsPathInvariantsInvCPAEqualOnlyInvSuccPlain}{}
  \renewcommand{\predicateBitprecisePathinvariantsResultsPathInvariantsInvCPAEqualOnlyInvSuccPlain}{1411\xspace}

  % inv-tries
\providecommand{\predicateBitprecisePathinvariantsResultsPathInvariantsInvCPAEqualOnlyInvTriesPlain}{}
  \renewcommand{\predicateBitprecisePathinvariantsResultsPathInvariantsInvCPAEqualOnlyInvTriesPlain}{4606\xspace}

  % inv-time-sum
\providecommand{\predicateBitprecisePathinvariantsResultsPathInvariantsInvCPAEqualOnlyInvTimeSumPlain}{}
  \renewcommand{\predicateBitprecisePathinvariantsResultsPathInvariantsInvCPAEqualOnlyInvTimeSumPlain}{8417.126000000004\xspace}
\providecommand{\predicateBitprecisePathinvariantsResultsPathInvariantsInvCPAEqualOnlyInvTimeSumPlainHours}{}
  \renewcommand{\predicateBitprecisePathinvariantsResultsPathInvariantsInvCPAEqualOnlyInvTimeSumPlainHours}{2.3380905555555564\xspace}

%%% predicate_bitprecise_pathinvariants.2016-09-10_1338.results.pathInvariants-policyCPA %%%
 %% correct %%
\providecommand{\predicateBitprecisePathinvariantsResultsPathInvariantsPolicyCPACorrectPlain}{}
  \renewcommand{\predicateBitprecisePathinvariantsResultsPathInvariantsPolicyCPACorrectPlain}{1866\xspace}

  % cpu-time-sum
\providecommand{\predicateBitprecisePathinvariantsResultsPathInvariantsPolicyCPACorrectCpuTimeSumPlain}{}
  \renewcommand{\predicateBitprecisePathinvariantsResultsPathInvariantsPolicyCPACorrectCpuTimeSumPlain}{113204.03672177006\xspace}
\providecommand{\predicateBitprecisePathinvariantsResultsPathInvariantsPolicyCPACorrectCpuTimeSumPlainHours}{}
  \renewcommand{\predicateBitprecisePathinvariantsResultsPathInvariantsPolicyCPACorrectCpuTimeSumPlainHours}{31.445565756047237\xspace}

  % wall-time-sum
\providecommand{\predicateBitprecisePathinvariantsResultsPathInvariantsPolicyCPACorrectWallTimeSumPlain}{}
  \renewcommand{\predicateBitprecisePathinvariantsResultsPathInvariantsPolicyCPACorrectWallTimeSumPlain}{70722.2949705107\xspace}
\providecommand{\predicateBitprecisePathinvariantsResultsPathInvariantsPolicyCPACorrectWallTimeSumPlainHours}{}
  \renewcommand{\predicateBitprecisePathinvariantsResultsPathInvariantsPolicyCPACorrectWallTimeSumPlainHours}{19.64508193625297\xspace}

  % cpu-time-avg
\providecommand{\predicateBitprecisePathinvariantsResultsPathInvariantsPolicyCPACorrectCpuTimeAvgPlain}{}
  \renewcommand{\predicateBitprecisePathinvariantsResultsPathInvariantsPolicyCPACorrectCpuTimeAvgPlain}{60.66668634607184\xspace}
\providecommand{\predicateBitprecisePathinvariantsResultsPathInvariantsPolicyCPACorrectCpuTimeAvgPlainHours}{}
  \renewcommand{\predicateBitprecisePathinvariantsResultsPathInvariantsPolicyCPACorrectCpuTimeAvgPlainHours}{0.01685185731835329\xspace}

  % wall-time-avg
\providecommand{\predicateBitprecisePathinvariantsResultsPathInvariantsPolicyCPACorrectWallTimeAvgPlain}{}
  \renewcommand{\predicateBitprecisePathinvariantsResultsPathInvariantsPolicyCPACorrectWallTimeAvgPlain}{37.900479619780654\xspace}
\providecommand{\predicateBitprecisePathinvariantsResultsPathInvariantsPolicyCPACorrectWallTimeAvgPlainHours}{}
  \renewcommand{\predicateBitprecisePathinvariantsResultsPathInvariantsPolicyCPACorrectWallTimeAvgPlainHours}{0.010527911005494626\xspace}

  % inv-succ
\providecommand{\predicateBitprecisePathinvariantsResultsPathInvariantsPolicyCPACorrectInvSuccPlain}{}
  \renewcommand{\predicateBitprecisePathinvariantsResultsPathInvariantsPolicyCPACorrectInvSuccPlain}{1611\xspace}

  % inv-tries
\providecommand{\predicateBitprecisePathinvariantsResultsPathInvariantsPolicyCPACorrectInvTriesPlain}{}
  \renewcommand{\predicateBitprecisePathinvariantsResultsPathInvariantsPolicyCPACorrectInvTriesPlain}{4600\xspace}

  % inv-time-sum
\providecommand{\predicateBitprecisePathinvariantsResultsPathInvariantsPolicyCPACorrectInvTimeSumPlain}{}
  \renewcommand{\predicateBitprecisePathinvariantsResultsPathInvariantsPolicyCPACorrectInvTimeSumPlain}{13829.819000000007\xspace}
\providecommand{\predicateBitprecisePathinvariantsResultsPathInvariantsPolicyCPACorrectInvTimeSumPlainHours}{}
  \renewcommand{\predicateBitprecisePathinvariantsResultsPathInvariantsPolicyCPACorrectInvTimeSumPlainHours}{3.8416163888888906\xspace}

 %% incorrect %%
\providecommand{\predicateBitprecisePathinvariantsResultsPathInvariantsPolicyCPAIncorrectPlain}{}
  \renewcommand{\predicateBitprecisePathinvariantsResultsPathInvariantsPolicyCPAIncorrectPlain}{27\xspace}

  % cpu-time-sum
\providecommand{\predicateBitprecisePathinvariantsResultsPathInvariantsPolicyCPAIncorrectCpuTimeSumPlain}{}
  \renewcommand{\predicateBitprecisePathinvariantsResultsPathInvariantsPolicyCPAIncorrectCpuTimeSumPlain}{1057.8719591509998\xspace}
\providecommand{\predicateBitprecisePathinvariantsResultsPathInvariantsPolicyCPAIncorrectCpuTimeSumPlainHours}{}
  \renewcommand{\predicateBitprecisePathinvariantsResultsPathInvariantsPolicyCPAIncorrectCpuTimeSumPlainHours}{0.29385332198638886\xspace}

  % wall-time-sum
\providecommand{\predicateBitprecisePathinvariantsResultsPathInvariantsPolicyCPAIncorrectWallTimeSumPlain}{}
  \renewcommand{\predicateBitprecisePathinvariantsResultsPathInvariantsPolicyCPAIncorrectWallTimeSumPlain}{621.06020808258\xspace}
\providecommand{\predicateBitprecisePathinvariantsResultsPathInvariantsPolicyCPAIncorrectWallTimeSumPlainHours}{}
  \renewcommand{\predicateBitprecisePathinvariantsResultsPathInvariantsPolicyCPAIncorrectWallTimeSumPlainHours}{0.1725167244673833\xspace}

  % cpu-time-avg
\providecommand{\predicateBitprecisePathinvariantsResultsPathInvariantsPolicyCPAIncorrectCpuTimeAvgPlain}{}
  \renewcommand{\predicateBitprecisePathinvariantsResultsPathInvariantsPolicyCPAIncorrectCpuTimeAvgPlain}{39.18044293151851\xspace}
\providecommand{\predicateBitprecisePathinvariantsResultsPathInvariantsPolicyCPAIncorrectCpuTimeAvgPlainHours}{}
  \renewcommand{\predicateBitprecisePathinvariantsResultsPathInvariantsPolicyCPAIncorrectCpuTimeAvgPlainHours}{0.010883456369866254\xspace}

  % wall-time-avg
\providecommand{\predicateBitprecisePathinvariantsResultsPathInvariantsPolicyCPAIncorrectWallTimeAvgPlain}{}
  \renewcommand{\predicateBitprecisePathinvariantsResultsPathInvariantsPolicyCPAIncorrectWallTimeAvgPlain}{23.00222992898444\xspace}
\providecommand{\predicateBitprecisePathinvariantsResultsPathInvariantsPolicyCPAIncorrectWallTimeAvgPlainHours}{}
  \renewcommand{\predicateBitprecisePathinvariantsResultsPathInvariantsPolicyCPAIncorrectWallTimeAvgPlainHours}{0.006389508313606789\xspace}

  % inv-succ
\providecommand{\predicateBitprecisePathinvariantsResultsPathInvariantsPolicyCPAIncorrectInvSuccPlain}{}
  \renewcommand{\predicateBitprecisePathinvariantsResultsPathInvariantsPolicyCPAIncorrectInvSuccPlain}{3\xspace}

  % inv-tries
\providecommand{\predicateBitprecisePathinvariantsResultsPathInvariantsPolicyCPAIncorrectInvTriesPlain}{}
  \renewcommand{\predicateBitprecisePathinvariantsResultsPathInvariantsPolicyCPAIncorrectInvTriesPlain}{60\xspace}

  % inv-time-sum
\providecommand{\predicateBitprecisePathinvariantsResultsPathInvariantsPolicyCPAIncorrectInvTimeSumPlain}{}
  \renewcommand{\predicateBitprecisePathinvariantsResultsPathInvariantsPolicyCPAIncorrectInvTimeSumPlain}{168.88299999999998\xspace}
\providecommand{\predicateBitprecisePathinvariantsResultsPathInvariantsPolicyCPAIncorrectInvTimeSumPlainHours}{}
  \renewcommand{\predicateBitprecisePathinvariantsResultsPathInvariantsPolicyCPAIncorrectInvTimeSumPlainHours}{0.04691194444444444\xspace}

 %% timeout %%
\providecommand{\predicateBitprecisePathinvariantsResultsPathInvariantsPolicyCPATimeoutPlain}{}
  \renewcommand{\predicateBitprecisePathinvariantsResultsPathInvariantsPolicyCPATimeoutPlain}{1485\xspace}

  % cpu-time-sum
\providecommand{\predicateBitprecisePathinvariantsResultsPathInvariantsPolicyCPATimeoutCpuTimeSumPlain}{}
  \renewcommand{\predicateBitprecisePathinvariantsResultsPathInvariantsPolicyCPATimeoutCpuTimeSumPlain}{453937.33718315023\xspace}
\providecommand{\predicateBitprecisePathinvariantsResultsPathInvariantsPolicyCPATimeoutCpuTimeSumPlainHours}{}
  \renewcommand{\predicateBitprecisePathinvariantsResultsPathInvariantsPolicyCPATimeoutCpuTimeSumPlainHours}{126.09370477309729\xspace}

  % wall-time-sum
\providecommand{\predicateBitprecisePathinvariantsResultsPathInvariantsPolicyCPATimeoutWallTimeSumPlain}{}
  \renewcommand{\predicateBitprecisePathinvariantsResultsPathInvariantsPolicyCPATimeoutWallTimeSumPlain}{391961.627122883\xspace}
\providecommand{\predicateBitprecisePathinvariantsResultsPathInvariantsPolicyCPATimeoutWallTimeSumPlainHours}{}
  \renewcommand{\predicateBitprecisePathinvariantsResultsPathInvariantsPolicyCPATimeoutWallTimeSumPlainHours}{108.87822975635639\xspace}

  % cpu-time-avg
\providecommand{\predicateBitprecisePathinvariantsResultsPathInvariantsPolicyCPATimeoutCpuTimeAvgPlain}{}
  \renewcommand{\predicateBitprecisePathinvariantsResultsPathInvariantsPolicyCPATimeoutCpuTimeAvgPlain}{305.6817085408419\xspace}
\providecommand{\predicateBitprecisePathinvariantsResultsPathInvariantsPolicyCPATimeoutCpuTimeAvgPlainHours}{}
  \renewcommand{\predicateBitprecisePathinvariantsResultsPathInvariantsPolicyCPATimeoutCpuTimeAvgPlainHours}{0.08491158570578941\xspace}

  % wall-time-avg
\providecommand{\predicateBitprecisePathinvariantsResultsPathInvariantsPolicyCPATimeoutWallTimeAvgPlain}{}
  \renewcommand{\predicateBitprecisePathinvariantsResultsPathInvariantsPolicyCPATimeoutWallTimeAvgPlain}{263.9472236517731\xspace}
\providecommand{\predicateBitprecisePathinvariantsResultsPathInvariantsPolicyCPATimeoutWallTimeAvgPlainHours}{}
  \renewcommand{\predicateBitprecisePathinvariantsResultsPathInvariantsPolicyCPATimeoutWallTimeAvgPlainHours}{0.07331867323660363\xspace}

  % inv-succ
\providecommand{\predicateBitprecisePathinvariantsResultsPathInvariantsPolicyCPATimeoutInvSuccPlain}{}
  \renewcommand{\predicateBitprecisePathinvariantsResultsPathInvariantsPolicyCPATimeoutInvSuccPlain}{51884\xspace}

  % inv-tries
\providecommand{\predicateBitprecisePathinvariantsResultsPathInvariantsPolicyCPATimeoutInvTriesPlain}{}
  \renewcommand{\predicateBitprecisePathinvariantsResultsPathInvariantsPolicyCPATimeoutInvTriesPlain}{57927\xspace}

  % inv-time-sum
\providecommand{\predicateBitprecisePathinvariantsResultsPathInvariantsPolicyCPATimeoutInvTimeSumPlain}{}
  \renewcommand{\predicateBitprecisePathinvariantsResultsPathInvariantsPolicyCPATimeoutInvTimeSumPlain}{17444.481\xspace}
\providecommand{\predicateBitprecisePathinvariantsResultsPathInvariantsPolicyCPATimeoutInvTimeSumPlainHours}{}
  \renewcommand{\predicateBitprecisePathinvariantsResultsPathInvariantsPolicyCPATimeoutInvTimeSumPlainHours}{4.845689166666666\xspace}

 %% unknown-or-category-error %%
\providecommand{\predicateBitprecisePathinvariantsResultsPathInvariantsPolicyCPAUnknownOrCategoryErrorPlain}{}
  \renewcommand{\predicateBitprecisePathinvariantsResultsPathInvariantsPolicyCPAUnknownOrCategoryErrorPlain}{1595\xspace}

  % cpu-time-sum
\providecommand{\predicateBitprecisePathinvariantsResultsPathInvariantsPolicyCPAUnknownOrCategoryErrorCpuTimeSumPlain}{}
  \renewcommand{\predicateBitprecisePathinvariantsResultsPathInvariantsPolicyCPAUnknownOrCategoryErrorCpuTimeSumPlain}{465644.634696504\xspace}
\providecommand{\predicateBitprecisePathinvariantsResultsPathInvariantsPolicyCPAUnknownOrCategoryErrorCpuTimeSumPlainHours}{}
  \renewcommand{\predicateBitprecisePathinvariantsResultsPathInvariantsPolicyCPAUnknownOrCategoryErrorCpuTimeSumPlainHours}{129.34573186014\xspace}

  % wall-time-sum
\providecommand{\predicateBitprecisePathinvariantsResultsPathInvariantsPolicyCPAUnknownOrCategoryErrorWallTimeSumPlain}{}
  \renewcommand{\predicateBitprecisePathinvariantsResultsPathInvariantsPolicyCPAUnknownOrCategoryErrorWallTimeSumPlain}{398978.6903715182\xspace}
\providecommand{\predicateBitprecisePathinvariantsResultsPathInvariantsPolicyCPAUnknownOrCategoryErrorWallTimeSumPlainHours}{}
  \renewcommand{\predicateBitprecisePathinvariantsResultsPathInvariantsPolicyCPAUnknownOrCategoryErrorWallTimeSumPlainHours}{110.8274139920884\xspace}

  % cpu-time-avg
\providecommand{\predicateBitprecisePathinvariantsResultsPathInvariantsPolicyCPAUnknownOrCategoryErrorCpuTimeAvgPlain}{}
  \renewcommand{\predicateBitprecisePathinvariantsResultsPathInvariantsPolicyCPAUnknownOrCategoryErrorCpuTimeAvgPlain}{291.94020984106834\xspace}
\providecommand{\predicateBitprecisePathinvariantsResultsPathInvariantsPolicyCPAUnknownOrCategoryErrorCpuTimeAvgPlainHours}{}
  \renewcommand{\predicateBitprecisePathinvariantsResultsPathInvariantsPolicyCPAUnknownOrCategoryErrorCpuTimeAvgPlainHours}{0.0810945027336301\xspace}

  % wall-time-avg
\providecommand{\predicateBitprecisePathinvariantsResultsPathInvariantsPolicyCPAUnknownOrCategoryErrorWallTimeAvgPlain}{}
  \renewcommand{\predicateBitprecisePathinvariantsResultsPathInvariantsPolicyCPAUnknownOrCategoryErrorWallTimeAvgPlain}{250.1433795432716\xspace}
\providecommand{\predicateBitprecisePathinvariantsResultsPathInvariantsPolicyCPAUnknownOrCategoryErrorWallTimeAvgPlainHours}{}
  \renewcommand{\predicateBitprecisePathinvariantsResultsPathInvariantsPolicyCPAUnknownOrCategoryErrorWallTimeAvgPlainHours}{0.06948427209535323\xspace}

  % inv-succ
\providecommand{\predicateBitprecisePathinvariantsResultsPathInvariantsPolicyCPAUnknownOrCategoryErrorInvSuccPlain}{}
  \renewcommand{\predicateBitprecisePathinvariantsResultsPathInvariantsPolicyCPAUnknownOrCategoryErrorInvSuccPlain}{51890\xspace}

  % inv-tries
\providecommand{\predicateBitprecisePathinvariantsResultsPathInvariantsPolicyCPAUnknownOrCategoryErrorInvTriesPlain}{}
  \renewcommand{\predicateBitprecisePathinvariantsResultsPathInvariantsPolicyCPAUnknownOrCategoryErrorInvTriesPlain}{58259\xspace}

  % inv-time-sum
\providecommand{\predicateBitprecisePathinvariantsResultsPathInvariantsPolicyCPAUnknownOrCategoryErrorInvTimeSumPlain}{}
  \renewcommand{\predicateBitprecisePathinvariantsResultsPathInvariantsPolicyCPAUnknownOrCategoryErrorInvTimeSumPlain}{18697.801000000014\xspace}
\providecommand{\predicateBitprecisePathinvariantsResultsPathInvariantsPolicyCPAUnknownOrCategoryErrorInvTimeSumPlainHours}{}
  \renewcommand{\predicateBitprecisePathinvariantsResultsPathInvariantsPolicyCPAUnknownOrCategoryErrorInvTimeSumPlainHours}{5.193833611111115\xspace}

 %% correct-false %%
\providecommand{\predicateBitprecisePathinvariantsResultsPathInvariantsPolicyCPACorrectFalsePlain}{}
  \renewcommand{\predicateBitprecisePathinvariantsResultsPathInvariantsPolicyCPACorrectFalsePlain}{529\xspace}

  % cpu-time-sum
\providecommand{\predicateBitprecisePathinvariantsResultsPathInvariantsPolicyCPACorrectFalseCpuTimeSumPlain}{}
  \renewcommand{\predicateBitprecisePathinvariantsResultsPathInvariantsPolicyCPACorrectFalseCpuTimeSumPlain}{39744.338430942014\xspace}
\providecommand{\predicateBitprecisePathinvariantsResultsPathInvariantsPolicyCPACorrectFalseCpuTimeSumPlainHours}{}
  \renewcommand{\predicateBitprecisePathinvariantsResultsPathInvariantsPolicyCPACorrectFalseCpuTimeSumPlainHours}{11.040094008595004\xspace}

  % wall-time-sum
\providecommand{\predicateBitprecisePathinvariantsResultsPathInvariantsPolicyCPACorrectFalseWallTimeSumPlain}{}
  \renewcommand{\predicateBitprecisePathinvariantsResultsPathInvariantsPolicyCPACorrectFalseWallTimeSumPlain}{26965.408998726285\xspace}
\providecommand{\predicateBitprecisePathinvariantsResultsPathInvariantsPolicyCPACorrectFalseWallTimeSumPlainHours}{}
  \renewcommand{\predicateBitprecisePathinvariantsResultsPathInvariantsPolicyCPACorrectFalseWallTimeSumPlainHours}{7.4903913885350795\xspace}

  % cpu-time-avg
\providecommand{\predicateBitprecisePathinvariantsResultsPathInvariantsPolicyCPACorrectFalseCpuTimeAvgPlain}{}
  \renewcommand{\predicateBitprecisePathinvariantsResultsPathInvariantsPolicyCPACorrectFalseCpuTimeAvgPlain}{75.13107453864275\xspace}
\providecommand{\predicateBitprecisePathinvariantsResultsPathInvariantsPolicyCPACorrectFalseCpuTimeAvgPlainHours}{}
  \renewcommand{\predicateBitprecisePathinvariantsResultsPathInvariantsPolicyCPACorrectFalseCpuTimeAvgPlainHours}{0.020869742927400764\xspace}

  % wall-time-avg
\providecommand{\predicateBitprecisePathinvariantsResultsPathInvariantsPolicyCPACorrectFalseWallTimeAvgPlain}{}
  \renewcommand{\predicateBitprecisePathinvariantsResultsPathInvariantsPolicyCPACorrectFalseWallTimeAvgPlain}{50.97430812613664\xspace}
\providecommand{\predicateBitprecisePathinvariantsResultsPathInvariantsPolicyCPACorrectFalseWallTimeAvgPlainHours}{}
  \renewcommand{\predicateBitprecisePathinvariantsResultsPathInvariantsPolicyCPACorrectFalseWallTimeAvgPlainHours}{0.014159530035037955\xspace}

  % inv-succ
\providecommand{\predicateBitprecisePathinvariantsResultsPathInvariantsPolicyCPACorrectFalseInvSuccPlain}{}
  \renewcommand{\predicateBitprecisePathinvariantsResultsPathInvariantsPolicyCPACorrectFalseInvSuccPlain}{440\xspace}

  % inv-tries
\providecommand{\predicateBitprecisePathinvariantsResultsPathInvariantsPolicyCPACorrectFalseInvTriesPlain}{}
  \renewcommand{\predicateBitprecisePathinvariantsResultsPathInvariantsPolicyCPACorrectFalseInvTriesPlain}{1527\xspace}

  % inv-time-sum
\providecommand{\predicateBitprecisePathinvariantsResultsPathInvariantsPolicyCPACorrectFalseInvTimeSumPlain}{}
  \renewcommand{\predicateBitprecisePathinvariantsResultsPathInvariantsPolicyCPACorrectFalseInvTimeSumPlain}{3416.9859999999962\xspace}
\providecommand{\predicateBitprecisePathinvariantsResultsPathInvariantsPolicyCPACorrectFalseInvTimeSumPlainHours}{}
  \renewcommand{\predicateBitprecisePathinvariantsResultsPathInvariantsPolicyCPACorrectFalseInvTimeSumPlainHours}{0.9491627777777767\xspace}

 %% correct-true %%
\providecommand{\predicateBitprecisePathinvariantsResultsPathInvariantsPolicyCPACorrectTruePlain}{}
  \renewcommand{\predicateBitprecisePathinvariantsResultsPathInvariantsPolicyCPACorrectTruePlain}{1337\xspace}

  % cpu-time-sum
\providecommand{\predicateBitprecisePathinvariantsResultsPathInvariantsPolicyCPACorrectTrueCpuTimeSumPlain}{}
  \renewcommand{\predicateBitprecisePathinvariantsResultsPathInvariantsPolicyCPACorrectTrueCpuTimeSumPlain}{73459.6982908281\xspace}
\providecommand{\predicateBitprecisePathinvariantsResultsPathInvariantsPolicyCPACorrectTrueCpuTimeSumPlainHours}{}
  \renewcommand{\predicateBitprecisePathinvariantsResultsPathInvariantsPolicyCPACorrectTrueCpuTimeSumPlainHours}{20.405471747452253\xspace}

  % wall-time-sum
\providecommand{\predicateBitprecisePathinvariantsResultsPathInvariantsPolicyCPACorrectTrueWallTimeSumPlain}{}
  \renewcommand{\predicateBitprecisePathinvariantsResultsPathInvariantsPolicyCPACorrectTrueWallTimeSumPlain}{43756.88597178452\xspace}
\providecommand{\predicateBitprecisePathinvariantsResultsPathInvariantsPolicyCPACorrectTrueWallTimeSumPlainHours}{}
  \renewcommand{\predicateBitprecisePathinvariantsResultsPathInvariantsPolicyCPACorrectTrueWallTimeSumPlainHours}{12.154690547717921\xspace}

  % cpu-time-avg
\providecommand{\predicateBitprecisePathinvariantsResultsPathInvariantsPolicyCPACorrectTrueCpuTimeAvgPlain}{}
  \renewcommand{\predicateBitprecisePathinvariantsResultsPathInvariantsPolicyCPACorrectTrueCpuTimeAvgPlain}{54.94367860196567\xspace}
\providecommand{\predicateBitprecisePathinvariantsResultsPathInvariantsPolicyCPACorrectTrueCpuTimeAvgPlainHours}{}
  \renewcommand{\predicateBitprecisePathinvariantsResultsPathInvariantsPolicyCPACorrectTrueCpuTimeAvgPlainHours}{0.015262132944990464\xspace}

  % wall-time-avg
\providecommand{\predicateBitprecisePathinvariantsResultsPathInvariantsPolicyCPACorrectTrueWallTimeAvgPlain}{}
  \renewcommand{\predicateBitprecisePathinvariantsResultsPathInvariantsPolicyCPACorrectTrueWallTimeAvgPlain}{32.72766340447608\xspace}
\providecommand{\predicateBitprecisePathinvariantsResultsPathInvariantsPolicyCPACorrectTrueWallTimeAvgPlainHours}{}
  \renewcommand{\predicateBitprecisePathinvariantsResultsPathInvariantsPolicyCPACorrectTrueWallTimeAvgPlainHours}{0.009091017612354467\xspace}

  % inv-succ
\providecommand{\predicateBitprecisePathinvariantsResultsPathInvariantsPolicyCPACorrectTrueInvSuccPlain}{}
  \renewcommand{\predicateBitprecisePathinvariantsResultsPathInvariantsPolicyCPACorrectTrueInvSuccPlain}{1171\xspace}

  % inv-tries
\providecommand{\predicateBitprecisePathinvariantsResultsPathInvariantsPolicyCPACorrectTrueInvTriesPlain}{}
  \renewcommand{\predicateBitprecisePathinvariantsResultsPathInvariantsPolicyCPACorrectTrueInvTriesPlain}{3073\xspace}

  % inv-time-sum
\providecommand{\predicateBitprecisePathinvariantsResultsPathInvariantsPolicyCPACorrectTrueInvTimeSumPlain}{}
  \renewcommand{\predicateBitprecisePathinvariantsResultsPathInvariantsPolicyCPACorrectTrueInvTimeSumPlain}{10412.833000000002\xspace}
\providecommand{\predicateBitprecisePathinvariantsResultsPathInvariantsPolicyCPACorrectTrueInvTimeSumPlainHours}{}
  \renewcommand{\predicateBitprecisePathinvariantsResultsPathInvariantsPolicyCPACorrectTrueInvTimeSumPlainHours}{2.8924536111111117\xspace}

 %% incorrect-false %%
\providecommand{\predicateBitprecisePathinvariantsResultsPathInvariantsPolicyCPAIncorrectFalsePlain}{}
  \renewcommand{\predicateBitprecisePathinvariantsResultsPathInvariantsPolicyCPAIncorrectFalsePlain}{27\xspace}

  % cpu-time-sum
\providecommand{\predicateBitprecisePathinvariantsResultsPathInvariantsPolicyCPAIncorrectFalseCpuTimeSumPlain}{}
  \renewcommand{\predicateBitprecisePathinvariantsResultsPathInvariantsPolicyCPAIncorrectFalseCpuTimeSumPlain}{1057.8719591509998\xspace}
\providecommand{\predicateBitprecisePathinvariantsResultsPathInvariantsPolicyCPAIncorrectFalseCpuTimeSumPlainHours}{}
  \renewcommand{\predicateBitprecisePathinvariantsResultsPathInvariantsPolicyCPAIncorrectFalseCpuTimeSumPlainHours}{0.29385332198638886\xspace}

  % wall-time-sum
\providecommand{\predicateBitprecisePathinvariantsResultsPathInvariantsPolicyCPAIncorrectFalseWallTimeSumPlain}{}
  \renewcommand{\predicateBitprecisePathinvariantsResultsPathInvariantsPolicyCPAIncorrectFalseWallTimeSumPlain}{621.06020808258\xspace}
\providecommand{\predicateBitprecisePathinvariantsResultsPathInvariantsPolicyCPAIncorrectFalseWallTimeSumPlainHours}{}
  \renewcommand{\predicateBitprecisePathinvariantsResultsPathInvariantsPolicyCPAIncorrectFalseWallTimeSumPlainHours}{0.1725167244673833\xspace}

  % cpu-time-avg
\providecommand{\predicateBitprecisePathinvariantsResultsPathInvariantsPolicyCPAIncorrectFalseCpuTimeAvgPlain}{}
  \renewcommand{\predicateBitprecisePathinvariantsResultsPathInvariantsPolicyCPAIncorrectFalseCpuTimeAvgPlain}{39.18044293151851\xspace}
\providecommand{\predicateBitprecisePathinvariantsResultsPathInvariantsPolicyCPAIncorrectFalseCpuTimeAvgPlainHours}{}
  \renewcommand{\predicateBitprecisePathinvariantsResultsPathInvariantsPolicyCPAIncorrectFalseCpuTimeAvgPlainHours}{0.010883456369866254\xspace}

  % wall-time-avg
\providecommand{\predicateBitprecisePathinvariantsResultsPathInvariantsPolicyCPAIncorrectFalseWallTimeAvgPlain}{}
  \renewcommand{\predicateBitprecisePathinvariantsResultsPathInvariantsPolicyCPAIncorrectFalseWallTimeAvgPlain}{23.00222992898444\xspace}
\providecommand{\predicateBitprecisePathinvariantsResultsPathInvariantsPolicyCPAIncorrectFalseWallTimeAvgPlainHours}{}
  \renewcommand{\predicateBitprecisePathinvariantsResultsPathInvariantsPolicyCPAIncorrectFalseWallTimeAvgPlainHours}{0.006389508313606789\xspace}

  % inv-succ
\providecommand{\predicateBitprecisePathinvariantsResultsPathInvariantsPolicyCPAIncorrectFalseInvSuccPlain}{}
  \renewcommand{\predicateBitprecisePathinvariantsResultsPathInvariantsPolicyCPAIncorrectFalseInvSuccPlain}{3\xspace}

  % inv-tries
\providecommand{\predicateBitprecisePathinvariantsResultsPathInvariantsPolicyCPAIncorrectFalseInvTriesPlain}{}
  \renewcommand{\predicateBitprecisePathinvariantsResultsPathInvariantsPolicyCPAIncorrectFalseInvTriesPlain}{60\xspace}

  % inv-time-sum
\providecommand{\predicateBitprecisePathinvariantsResultsPathInvariantsPolicyCPAIncorrectFalseInvTimeSumPlain}{}
  \renewcommand{\predicateBitprecisePathinvariantsResultsPathInvariantsPolicyCPAIncorrectFalseInvTimeSumPlain}{168.88299999999998\xspace}
\providecommand{\predicateBitprecisePathinvariantsResultsPathInvariantsPolicyCPAIncorrectFalseInvTimeSumPlainHours}{}
  \renewcommand{\predicateBitprecisePathinvariantsResultsPathInvariantsPolicyCPAIncorrectFalseInvTimeSumPlainHours}{0.04691194444444444\xspace}

 %% incorrect-true %%
\providecommand{\predicateBitprecisePathinvariantsResultsPathInvariantsPolicyCPAIncorrectTruePlain}{}
  \renewcommand{\predicateBitprecisePathinvariantsResultsPathInvariantsPolicyCPAIncorrectTruePlain}{0\xspace}

  % cpu-time-sum
\providecommand{\predicateBitprecisePathinvariantsResultsPathInvariantsPolicyCPAIncorrectTrueCpuTimeSumPlain}{}
  \renewcommand{\predicateBitprecisePathinvariantsResultsPathInvariantsPolicyCPAIncorrectTrueCpuTimeSumPlain}{0.0\xspace}
\providecommand{\predicateBitprecisePathinvariantsResultsPathInvariantsPolicyCPAIncorrectTrueCpuTimeSumPlainHours}{}
  \renewcommand{\predicateBitprecisePathinvariantsResultsPathInvariantsPolicyCPAIncorrectTrueCpuTimeSumPlainHours}{0.0\xspace}

  % wall-time-sum
\providecommand{\predicateBitprecisePathinvariantsResultsPathInvariantsPolicyCPAIncorrectTrueWallTimeSumPlain}{}
  \renewcommand{\predicateBitprecisePathinvariantsResultsPathInvariantsPolicyCPAIncorrectTrueWallTimeSumPlain}{0.0\xspace}
\providecommand{\predicateBitprecisePathinvariantsResultsPathInvariantsPolicyCPAIncorrectTrueWallTimeSumPlainHours}{}
  \renewcommand{\predicateBitprecisePathinvariantsResultsPathInvariantsPolicyCPAIncorrectTrueWallTimeSumPlainHours}{0.0\xspace}

  % cpu-time-avg
\providecommand{\predicateBitprecisePathinvariantsResultsPathInvariantsPolicyCPAIncorrectTrueCpuTimeAvgPlain}{}
  \renewcommand{\predicateBitprecisePathinvariantsResultsPathInvariantsPolicyCPAIncorrectTrueCpuTimeAvgPlain}{NaN\xspace}
\providecommand{\predicateBitprecisePathinvariantsResultsPathInvariantsPolicyCPAIncorrectTrueCpuTimeAvgPlainHours}{}
  \renewcommand{\predicateBitprecisePathinvariantsResultsPathInvariantsPolicyCPAIncorrectTrueCpuTimeAvgPlainHours}{NaN\xspace}

  % wall-time-avg
\providecommand{\predicateBitprecisePathinvariantsResultsPathInvariantsPolicyCPAIncorrectTrueWallTimeAvgPlain}{}
  \renewcommand{\predicateBitprecisePathinvariantsResultsPathInvariantsPolicyCPAIncorrectTrueWallTimeAvgPlain}{NaN\xspace}
\providecommand{\predicateBitprecisePathinvariantsResultsPathInvariantsPolicyCPAIncorrectTrueWallTimeAvgPlainHours}{}
  \renewcommand{\predicateBitprecisePathinvariantsResultsPathInvariantsPolicyCPAIncorrectTrueWallTimeAvgPlainHours}{NaN\xspace}

  % inv-succ
\providecommand{\predicateBitprecisePathinvariantsResultsPathInvariantsPolicyCPAIncorrectTrueInvSuccPlain}{}
  \renewcommand{\predicateBitprecisePathinvariantsResultsPathInvariantsPolicyCPAIncorrectTrueInvSuccPlain}{0\xspace}

  % inv-tries
\providecommand{\predicateBitprecisePathinvariantsResultsPathInvariantsPolicyCPAIncorrectTrueInvTriesPlain}{}
  \renewcommand{\predicateBitprecisePathinvariantsResultsPathInvariantsPolicyCPAIncorrectTrueInvTriesPlain}{0\xspace}

  % inv-time-sum
\providecommand{\predicateBitprecisePathinvariantsResultsPathInvariantsPolicyCPAIncorrectTrueInvTimeSumPlain}{}
  \renewcommand{\predicateBitprecisePathinvariantsResultsPathInvariantsPolicyCPAIncorrectTrueInvTimeSumPlain}{0.0\xspace}
\providecommand{\predicateBitprecisePathinvariantsResultsPathInvariantsPolicyCPAIncorrectTrueInvTimeSumPlainHours}{}
  \renewcommand{\predicateBitprecisePathinvariantsResultsPathInvariantsPolicyCPAIncorrectTrueInvTimeSumPlainHours}{0.0\xspace}

 %% all %%
\providecommand{\predicateBitprecisePathinvariantsResultsPathInvariantsPolicyCPAAllPlain}{}
  \renewcommand{\predicateBitprecisePathinvariantsResultsPathInvariantsPolicyCPAAllPlain}{3488\xspace}

  % cpu-time-sum
\providecommand{\predicateBitprecisePathinvariantsResultsPathInvariantsPolicyCPAAllCpuTimeSumPlain}{}
  \renewcommand{\predicateBitprecisePathinvariantsResultsPathInvariantsPolicyCPAAllCpuTimeSumPlain}{579906.5433774246\xspace}
\providecommand{\predicateBitprecisePathinvariantsResultsPathInvariantsPolicyCPAAllCpuTimeSumPlainHours}{}
  \renewcommand{\predicateBitprecisePathinvariantsResultsPathInvariantsPolicyCPAAllCpuTimeSumPlainHours}{161.08515093817348\xspace}

  % wall-time-sum
\providecommand{\predicateBitprecisePathinvariantsResultsPathInvariantsPolicyCPAAllWallTimeSumPlain}{}
  \renewcommand{\predicateBitprecisePathinvariantsResultsPathInvariantsPolicyCPAAllWallTimeSumPlain}{470322.0455501116\xspace}
\providecommand{\predicateBitprecisePathinvariantsResultsPathInvariantsPolicyCPAAllWallTimeSumPlainHours}{}
  \renewcommand{\predicateBitprecisePathinvariantsResultsPathInvariantsPolicyCPAAllWallTimeSumPlainHours}{130.6450126528088\xspace}

  % cpu-time-avg
\providecommand{\predicateBitprecisePathinvariantsResultsPathInvariantsPolicyCPAAllCpuTimeAvgPlain}{}
  \renewcommand{\predicateBitprecisePathinvariantsResultsPathInvariantsPolicyCPAAllCpuTimeAvgPlain}{166.25760991325245\xspace}
\providecommand{\predicateBitprecisePathinvariantsResultsPathInvariantsPolicyCPAAllCpuTimeAvgPlainHours}{}
  \renewcommand{\predicateBitprecisePathinvariantsResultsPathInvariantsPolicyCPAAllCpuTimeAvgPlainHours}{0.046182669420347905\xspace}

  % wall-time-avg
\providecommand{\predicateBitprecisePathinvariantsResultsPathInvariantsPolicyCPAAllWallTimeAvgPlain}{}
  \renewcommand{\predicateBitprecisePathinvariantsResultsPathInvariantsPolicyCPAAllWallTimeAvgPlain}{134.84003599487144\xspace}
\providecommand{\predicateBitprecisePathinvariantsResultsPathInvariantsPolicyCPAAllWallTimeAvgPlainHours}{}
  \renewcommand{\predicateBitprecisePathinvariantsResultsPathInvariantsPolicyCPAAllWallTimeAvgPlainHours}{0.03745556555413096\xspace}

  % inv-succ
\providecommand{\predicateBitprecisePathinvariantsResultsPathInvariantsPolicyCPAAllInvSuccPlain}{}
  \renewcommand{\predicateBitprecisePathinvariantsResultsPathInvariantsPolicyCPAAllInvSuccPlain}{53504\xspace}

  % inv-tries
\providecommand{\predicateBitprecisePathinvariantsResultsPathInvariantsPolicyCPAAllInvTriesPlain}{}
  \renewcommand{\predicateBitprecisePathinvariantsResultsPathInvariantsPolicyCPAAllInvTriesPlain}{62919\xspace}

  % inv-time-sum
\providecommand{\predicateBitprecisePathinvariantsResultsPathInvariantsPolicyCPAAllInvTimeSumPlain}{}
  \renewcommand{\predicateBitprecisePathinvariantsResultsPathInvariantsPolicyCPAAllInvTimeSumPlain}{32696.50300000008\xspace}
\providecommand{\predicateBitprecisePathinvariantsResultsPathInvariantsPolicyCPAAllInvTimeSumPlainHours}{}
  \renewcommand{\predicateBitprecisePathinvariantsResultsPathInvariantsPolicyCPAAllInvTimeSumPlainHours}{9.082361944444466\xspace}

 %% equal-only %%
\providecommand{\predicateBitprecisePathinvariantsResultsPathInvariantsPolicyCPAEqualOnlyPlain}{}
  \renewcommand{\predicateBitprecisePathinvariantsResultsPathInvariantsPolicyCPAEqualOnlyPlain}{1832\xspace}

  % cpu-time-sum
\providecommand{\predicateBitprecisePathinvariantsResultsPathInvariantsPolicyCPAEqualOnlyCpuTimeSumPlain}{}
  \renewcommand{\predicateBitprecisePathinvariantsResultsPathInvariantsPolicyCPAEqualOnlyCpuTimeSumPlain}{107339.06055763812\xspace}
\providecommand{\predicateBitprecisePathinvariantsResultsPathInvariantsPolicyCPAEqualOnlyCpuTimeSumPlainHours}{}
  \renewcommand{\predicateBitprecisePathinvariantsResultsPathInvariantsPolicyCPAEqualOnlyCpuTimeSumPlainHours}{29.816405710455033\xspace}

  % wall-time-sum
\providecommand{\predicateBitprecisePathinvariantsResultsPathInvariantsPolicyCPAEqualOnlyWallTimeSumPlain}{}
  \renewcommand{\predicateBitprecisePathinvariantsResultsPathInvariantsPolicyCPAEqualOnlyWallTimeSumPlain}{66312.94423985337\xspace}
\providecommand{\predicateBitprecisePathinvariantsResultsPathInvariantsPolicyCPAEqualOnlyWallTimeSumPlainHours}{}
  \renewcommand{\predicateBitprecisePathinvariantsResultsPathInvariantsPolicyCPAEqualOnlyWallTimeSumPlainHours}{18.42026228884816\xspace}

  % cpu-time-avg
\providecommand{\predicateBitprecisePathinvariantsResultsPathInvariantsPolicyCPAEqualOnlyCpuTimeAvgPlain}{}
  \renewcommand{\predicateBitprecisePathinvariantsResultsPathInvariantsPolicyCPAEqualOnlyCpuTimeAvgPlain}{58.5911902607195\xspace}
\providecommand{\predicateBitprecisePathinvariantsResultsPathInvariantsPolicyCPAEqualOnlyCpuTimeAvgPlainHours}{}
  \renewcommand{\predicateBitprecisePathinvariantsResultsPathInvariantsPolicyCPAEqualOnlyCpuTimeAvgPlainHours}{0.01627533062797764\xspace}

  % wall-time-avg
\providecommand{\predicateBitprecisePathinvariantsResultsPathInvariantsPolicyCPAEqualOnlyWallTimeAvgPlain}{}
  \renewcommand{\predicateBitprecisePathinvariantsResultsPathInvariantsPolicyCPAEqualOnlyWallTimeAvgPlain}{36.19702196498547\xspace}
\providecommand{\predicateBitprecisePathinvariantsResultsPathInvariantsPolicyCPAEqualOnlyWallTimeAvgPlainHours}{}
  \renewcommand{\predicateBitprecisePathinvariantsResultsPathInvariantsPolicyCPAEqualOnlyWallTimeAvgPlainHours}{0.010054728323607074\xspace}

  % inv-succ
\providecommand{\predicateBitprecisePathinvariantsResultsPathInvariantsPolicyCPAEqualOnlyInvSuccPlain}{}
  \renewcommand{\predicateBitprecisePathinvariantsResultsPathInvariantsPolicyCPAEqualOnlyInvSuccPlain}{1566\xspace}

  % inv-tries
\providecommand{\predicateBitprecisePathinvariantsResultsPathInvariantsPolicyCPAEqualOnlyInvTriesPlain}{}
  \renewcommand{\predicateBitprecisePathinvariantsResultsPathInvariantsPolicyCPAEqualOnlyInvTriesPlain}{4455\xspace}

  % inv-time-sum
\providecommand{\predicateBitprecisePathinvariantsResultsPathInvariantsPolicyCPAEqualOnlyInvTimeSumPlain}{}
  \renewcommand{\predicateBitprecisePathinvariantsResultsPathInvariantsPolicyCPAEqualOnlyInvTimeSumPlain}{13490.005000000012\xspace}
\providecommand{\predicateBitprecisePathinvariantsResultsPathInvariantsPolicyCPAEqualOnlyInvTimeSumPlainHours}{}
  \renewcommand{\predicateBitprecisePathinvariantsResultsPathInvariantsPolicyCPAEqualOnlyInvTimeSumPlainHours}{3.7472236111111146\xspace}

%%% predicate_base.2016-09-03_1927.results.pred-bitvectors %%%
 %% correct %%
\providecommand{\predicateBaseResultsPredBitvectorsCorrectPlain}{}
  \renewcommand{\predicateBaseResultsPredBitvectorsCorrectPlain}{1944\xspace}

  % cpu-time-sum
\providecommand{\predicateBaseResultsPredBitvectorsCorrectCpuTimeSumPlain}{}
  \renewcommand{\predicateBaseResultsPredBitvectorsCorrectCpuTimeSumPlain}{93675.79971751993\xspace}
\providecommand{\predicateBaseResultsPredBitvectorsCorrectCpuTimeSumPlainHours}{}
  \renewcommand{\predicateBaseResultsPredBitvectorsCorrectCpuTimeSumPlainHours}{26.02105547708887\xspace}

  % wall-time-sum
\providecommand{\predicateBaseResultsPredBitvectorsCorrectWallTimeSumPlain}{}
  \renewcommand{\predicateBaseResultsPredBitvectorsCorrectWallTimeSumPlain}{64278.471849201764\xspace}
\providecommand{\predicateBaseResultsPredBitvectorsCorrectWallTimeSumPlainHours}{}
  \renewcommand{\predicateBaseResultsPredBitvectorsCorrectWallTimeSumPlainHours}{17.85513106922271\xspace}

  % cpu-time-avg
\providecommand{\predicateBaseResultsPredBitvectorsCorrectCpuTimeAvgPlain}{}
  \renewcommand{\predicateBaseResultsPredBitvectorsCorrectCpuTimeAvgPlain}{48.18713977238679\xspace}
\providecommand{\predicateBaseResultsPredBitvectorsCorrectCpuTimeAvgPlainHours}{}
  \renewcommand{\predicateBaseResultsPredBitvectorsCorrectCpuTimeAvgPlainHours}{0.013385316603440776\xspace}

  % wall-time-avg
\providecommand{\predicateBaseResultsPredBitvectorsCorrectWallTimeAvgPlain}{}
  \renewcommand{\predicateBaseResultsPredBitvectorsCorrectWallTimeAvgPlain}{33.06505753559762\xspace}
\providecommand{\predicateBaseResultsPredBitvectorsCorrectWallTimeAvgPlainHours}{}
  \renewcommand{\predicateBaseResultsPredBitvectorsCorrectWallTimeAvgPlainHours}{0.009184738204332672\xspace}

  % inv-succ
\providecommand{\predicateBaseResultsPredBitvectorsCorrectInvSuccPlain}{}
  \renewcommand{\predicateBaseResultsPredBitvectorsCorrectInvSuccPlain}{0\xspace}

  % inv-tries
\providecommand{\predicateBaseResultsPredBitvectorsCorrectInvTriesPlain}{}
  \renewcommand{\predicateBaseResultsPredBitvectorsCorrectInvTriesPlain}{0\xspace}

  % inv-time-sum
\providecommand{\predicateBaseResultsPredBitvectorsCorrectInvTimeSumPlain}{}
  \renewcommand{\predicateBaseResultsPredBitvectorsCorrectInvTimeSumPlain}{0.0\xspace}
\providecommand{\predicateBaseResultsPredBitvectorsCorrectInvTimeSumPlainHours}{}
  \renewcommand{\predicateBaseResultsPredBitvectorsCorrectInvTimeSumPlainHours}{0.0\xspace}

 %% incorrect %%
\providecommand{\predicateBaseResultsPredBitvectorsIncorrectPlain}{}
  \renewcommand{\predicateBaseResultsPredBitvectorsIncorrectPlain}{27\xspace}

  % cpu-time-sum
\providecommand{\predicateBaseResultsPredBitvectorsIncorrectCpuTimeSumPlain}{}
  \renewcommand{\predicateBaseResultsPredBitvectorsIncorrectCpuTimeSumPlain}{712.5640386970001\xspace}
\providecommand{\predicateBaseResultsPredBitvectorsIncorrectCpuTimeSumPlainHours}{}
  \renewcommand{\predicateBaseResultsPredBitvectorsIncorrectCpuTimeSumPlainHours}{0.19793445519361114\xspace}

  % wall-time-sum
\providecommand{\predicateBaseResultsPredBitvectorsIncorrectWallTimeSumPlain}{}
  \renewcommand{\predicateBaseResultsPredBitvectorsIncorrectWallTimeSumPlain}{432.08190250372\xspace}
\providecommand{\predicateBaseResultsPredBitvectorsIncorrectWallTimeSumPlainHours}{}
  \renewcommand{\predicateBaseResultsPredBitvectorsIncorrectWallTimeSumPlainHours}{0.12002275069547778\xspace}

  % cpu-time-avg
\providecommand{\predicateBaseResultsPredBitvectorsIncorrectCpuTimeAvgPlain}{}
  \renewcommand{\predicateBaseResultsPredBitvectorsIncorrectCpuTimeAvgPlain}{26.391260692481485\xspace}
\providecommand{\predicateBaseResultsPredBitvectorsIncorrectCpuTimeAvgPlainHours}{}
  \renewcommand{\predicateBaseResultsPredBitvectorsIncorrectCpuTimeAvgPlainHours}{0.007330905747911523\xspace}

  % wall-time-avg
\providecommand{\predicateBaseResultsPredBitvectorsIncorrectWallTimeAvgPlain}{}
  \renewcommand{\predicateBaseResultsPredBitvectorsIncorrectWallTimeAvgPlain}{16.003033426063705\xspace}
\providecommand{\predicateBaseResultsPredBitvectorsIncorrectWallTimeAvgPlainHours}{}
  \renewcommand{\predicateBaseResultsPredBitvectorsIncorrectWallTimeAvgPlainHours}{0.004445287062795474\xspace}

  % inv-succ
\providecommand{\predicateBaseResultsPredBitvectorsIncorrectInvSuccPlain}{}
  \renewcommand{\predicateBaseResultsPredBitvectorsIncorrectInvSuccPlain}{0\xspace}

  % inv-tries
\providecommand{\predicateBaseResultsPredBitvectorsIncorrectInvTriesPlain}{}
  \renewcommand{\predicateBaseResultsPredBitvectorsIncorrectInvTriesPlain}{0\xspace}

  % inv-time-sum
\providecommand{\predicateBaseResultsPredBitvectorsIncorrectInvTimeSumPlain}{}
  \renewcommand{\predicateBaseResultsPredBitvectorsIncorrectInvTimeSumPlain}{0.0\xspace}
\providecommand{\predicateBaseResultsPredBitvectorsIncorrectInvTimeSumPlainHours}{}
  \renewcommand{\predicateBaseResultsPredBitvectorsIncorrectInvTimeSumPlainHours}{0.0\xspace}

 %% timeout %%
\providecommand{\predicateBaseResultsPredBitvectorsTimeoutPlain}{}
  \renewcommand{\predicateBaseResultsPredBitvectorsTimeoutPlain}{1414\xspace}

  % cpu-time-sum
\providecommand{\predicateBaseResultsPredBitvectorsTimeoutCpuTimeSumPlain}{}
  \renewcommand{\predicateBaseResultsPredBitvectorsTimeoutCpuTimeSumPlain}{432542.40992774273\xspace}
\providecommand{\predicateBaseResultsPredBitvectorsTimeoutCpuTimeSumPlainHours}{}
  \renewcommand{\predicateBaseResultsPredBitvectorsTimeoutCpuTimeSumPlainHours}{120.15066942437298\xspace}

  % wall-time-sum
\providecommand{\predicateBaseResultsPredBitvectorsTimeoutWallTimeSumPlain}{}
  \renewcommand{\predicateBaseResultsPredBitvectorsTimeoutWallTimeSumPlain}{389237.6199243135\xspace}
\providecommand{\predicateBaseResultsPredBitvectorsTimeoutWallTimeSumPlainHours}{}
  \renewcommand{\predicateBaseResultsPredBitvectorsTimeoutWallTimeSumPlainHours}{108.12156109008708\xspace}

  % cpu-time-avg
\providecommand{\predicateBaseResultsPredBitvectorsTimeoutCpuTimeAvgPlain}{}
  \renewcommand{\predicateBaseResultsPredBitvectorsTimeoutCpuTimeAvgPlain}{305.8998655783188\xspace}
\providecommand{\predicateBaseResultsPredBitvectorsTimeoutCpuTimeAvgPlainHours}{}
  \renewcommand{\predicateBaseResultsPredBitvectorsTimeoutCpuTimeAvgPlainHours}{0.08497218488286633\xspace}

  % wall-time-avg
\providecommand{\predicateBaseResultsPredBitvectorsTimeoutWallTimeAvgPlain}{}
  \renewcommand{\predicateBaseResultsPredBitvectorsTimeoutWallTimeAvgPlain}{275.2741300737719\xspace}
\providecommand{\predicateBaseResultsPredBitvectorsTimeoutWallTimeAvgPlainHours}{}
  \renewcommand{\predicateBaseResultsPredBitvectorsTimeoutWallTimeAvgPlainHours}{0.07646503613160331\xspace}

  % inv-succ
\providecommand{\predicateBaseResultsPredBitvectorsTimeoutInvSuccPlain}{}
  \renewcommand{\predicateBaseResultsPredBitvectorsTimeoutInvSuccPlain}{0\xspace}

  % inv-tries
\providecommand{\predicateBaseResultsPredBitvectorsTimeoutInvTriesPlain}{}
  \renewcommand{\predicateBaseResultsPredBitvectorsTimeoutInvTriesPlain}{0\xspace}

  % inv-time-sum
\providecommand{\predicateBaseResultsPredBitvectorsTimeoutInvTimeSumPlain}{}
  \renewcommand{\predicateBaseResultsPredBitvectorsTimeoutInvTimeSumPlain}{0.0\xspace}
\providecommand{\predicateBaseResultsPredBitvectorsTimeoutInvTimeSumPlainHours}{}
  \renewcommand{\predicateBaseResultsPredBitvectorsTimeoutInvTimeSumPlainHours}{0.0\xspace}

 %% unknown-or-category-error %%
\providecommand{\predicateBaseResultsPredBitvectorsUnknownOrCategoryErrorPlain}{}
  \renewcommand{\predicateBaseResultsPredBitvectorsUnknownOrCategoryErrorPlain}{1517\xspace}

  % cpu-time-sum
\providecommand{\predicateBaseResultsPredBitvectorsUnknownOrCategoryErrorCpuTimeSumPlain}{}
  \renewcommand{\predicateBaseResultsPredBitvectorsUnknownOrCategoryErrorCpuTimeSumPlain}{440619.1379288969\xspace}
\providecommand{\predicateBaseResultsPredBitvectorsUnknownOrCategoryErrorCpuTimeSumPlainHours}{}
  \renewcommand{\predicateBaseResultsPredBitvectorsUnknownOrCategoryErrorCpuTimeSumPlainHours}{122.39420498024913\xspace}

  % wall-time-sum
\providecommand{\predicateBaseResultsPredBitvectorsUnknownOrCategoryErrorWallTimeSumPlain}{}
  \renewcommand{\predicateBaseResultsPredBitvectorsUnknownOrCategoryErrorWallTimeSumPlain}{394431.6659314679\xspace}
\providecommand{\predicateBaseResultsPredBitvectorsUnknownOrCategoryErrorWallTimeSumPlainHours}{}
  \renewcommand{\predicateBaseResultsPredBitvectorsUnknownOrCategoryErrorWallTimeSumPlainHours}{109.56435164762998\xspace}

  % cpu-time-avg
\providecommand{\predicateBaseResultsPredBitvectorsUnknownOrCategoryErrorCpuTimeAvgPlain}{}
  \renewcommand{\predicateBaseResultsPredBitvectorsUnknownOrCategoryErrorCpuTimeAvgPlain}{290.4542768153572\xspace}
\providecommand{\predicateBaseResultsPredBitvectorsUnknownOrCategoryErrorCpuTimeAvgPlainHours}{}
  \renewcommand{\predicateBaseResultsPredBitvectorsUnknownOrCategoryErrorCpuTimeAvgPlainHours}{0.08068174355982145\xspace}

  % wall-time-avg
\providecommand{\predicateBaseResultsPredBitvectorsUnknownOrCategoryErrorWallTimeAvgPlain}{}
  \renewcommand{\predicateBaseResultsPredBitvectorsUnknownOrCategoryErrorWallTimeAvgPlain}{260.0076901328068\xspace}
\providecommand{\predicateBaseResultsPredBitvectorsUnknownOrCategoryErrorWallTimeAvgPlainHours}{}
  \renewcommand{\predicateBaseResultsPredBitvectorsUnknownOrCategoryErrorWallTimeAvgPlainHours}{0.0722243583702241\xspace}

  % inv-succ
\providecommand{\predicateBaseResultsPredBitvectorsUnknownOrCategoryErrorInvSuccPlain}{}
  \renewcommand{\predicateBaseResultsPredBitvectorsUnknownOrCategoryErrorInvSuccPlain}{0\xspace}

  % inv-tries
\providecommand{\predicateBaseResultsPredBitvectorsUnknownOrCategoryErrorInvTriesPlain}{}
  \renewcommand{\predicateBaseResultsPredBitvectorsUnknownOrCategoryErrorInvTriesPlain}{0\xspace}

  % inv-time-sum
\providecommand{\predicateBaseResultsPredBitvectorsUnknownOrCategoryErrorInvTimeSumPlain}{}
  \renewcommand{\predicateBaseResultsPredBitvectorsUnknownOrCategoryErrorInvTimeSumPlain}{0.0\xspace}
\providecommand{\predicateBaseResultsPredBitvectorsUnknownOrCategoryErrorInvTimeSumPlainHours}{}
  \renewcommand{\predicateBaseResultsPredBitvectorsUnknownOrCategoryErrorInvTimeSumPlainHours}{0.0\xspace}

 %% correct-false %%
\providecommand{\predicateBaseResultsPredBitvectorsCorrectFalsePlain}{}
  \renewcommand{\predicateBaseResultsPredBitvectorsCorrectFalsePlain}{553\xspace}

  % cpu-time-sum
\providecommand{\predicateBaseResultsPredBitvectorsCorrectFalseCpuTimeSumPlain}{}
  \renewcommand{\predicateBaseResultsPredBitvectorsCorrectFalseCpuTimeSumPlain}{38293.01043458401\xspace}
\providecommand{\predicateBaseResultsPredBitvectorsCorrectFalseCpuTimeSumPlainHours}{}
  \renewcommand{\predicateBaseResultsPredBitvectorsCorrectFalseCpuTimeSumPlainHours}{10.636947342940003\xspace}

  % wall-time-sum
\providecommand{\predicateBaseResultsPredBitvectorsCorrectFalseWallTimeSumPlain}{}
  \renewcommand{\predicateBaseResultsPredBitvectorsCorrectFalseWallTimeSumPlain}{28014.069984672435\xspace}
\providecommand{\predicateBaseResultsPredBitvectorsCorrectFalseWallTimeSumPlainHours}{}
  \renewcommand{\predicateBaseResultsPredBitvectorsCorrectFalseWallTimeSumPlainHours}{7.781686106853454\xspace}

  % cpu-time-avg
\providecommand{\predicateBaseResultsPredBitvectorsCorrectFalseCpuTimeAvgPlain}{}
  \renewcommand{\predicateBaseResultsPredBitvectorsCorrectFalseCpuTimeAvgPlain}{69.2459501529548\xspace}
\providecommand{\predicateBaseResultsPredBitvectorsCorrectFalseCpuTimeAvgPlainHours}{}
  \renewcommand{\predicateBaseResultsPredBitvectorsCorrectFalseCpuTimeAvgPlainHours}{0.019234986153598557\xspace}

  % wall-time-avg
\providecommand{\predicateBaseResultsPredBitvectorsCorrectFalseWallTimeAvgPlain}{}
  \renewcommand{\predicateBaseResultsPredBitvectorsCorrectFalseWallTimeAvgPlain}{50.65835440266263\xspace}
\providecommand{\predicateBaseResultsPredBitvectorsCorrectFalseWallTimeAvgPlainHours}{}
  \renewcommand{\predicateBaseResultsPredBitvectorsCorrectFalseWallTimeAvgPlainHours}{0.014071765111850732\xspace}

  % inv-succ
\providecommand{\predicateBaseResultsPredBitvectorsCorrectFalseInvSuccPlain}{}
  \renewcommand{\predicateBaseResultsPredBitvectorsCorrectFalseInvSuccPlain}{0\xspace}

  % inv-tries
\providecommand{\predicateBaseResultsPredBitvectorsCorrectFalseInvTriesPlain}{}
  \renewcommand{\predicateBaseResultsPredBitvectorsCorrectFalseInvTriesPlain}{0\xspace}

  % inv-time-sum
\providecommand{\predicateBaseResultsPredBitvectorsCorrectFalseInvTimeSumPlain}{}
  \renewcommand{\predicateBaseResultsPredBitvectorsCorrectFalseInvTimeSumPlain}{0.0\xspace}
\providecommand{\predicateBaseResultsPredBitvectorsCorrectFalseInvTimeSumPlainHours}{}
  \renewcommand{\predicateBaseResultsPredBitvectorsCorrectFalseInvTimeSumPlainHours}{0.0\xspace}

 %% correct-true %%
\providecommand{\predicateBaseResultsPredBitvectorsCorrectTruePlain}{}
  \renewcommand{\predicateBaseResultsPredBitvectorsCorrectTruePlain}{1391\xspace}

  % cpu-time-sum
\providecommand{\predicateBaseResultsPredBitvectorsCorrectTrueCpuTimeSumPlain}{}
  \renewcommand{\predicateBaseResultsPredBitvectorsCorrectTrueCpuTimeSumPlain}{55382.789282936064\xspace}
\providecommand{\predicateBaseResultsPredBitvectorsCorrectTrueCpuTimeSumPlainHours}{}
  \renewcommand{\predicateBaseResultsPredBitvectorsCorrectTrueCpuTimeSumPlainHours}{15.384108134148907\xspace}

  % wall-time-sum
\providecommand{\predicateBaseResultsPredBitvectorsCorrectTrueWallTimeSumPlain}{}
  \renewcommand{\predicateBaseResultsPredBitvectorsCorrectTrueWallTimeSumPlain}{36264.40186452916\xspace}
\providecommand{\predicateBaseResultsPredBitvectorsCorrectTrueWallTimeSumPlainHours}{}
  \renewcommand{\predicateBaseResultsPredBitvectorsCorrectTrueWallTimeSumPlainHours}{10.07344496236921\xspace}

  % cpu-time-avg
\providecommand{\predicateBaseResultsPredBitvectorsCorrectTrueCpuTimeAvgPlain}{}
  \renewcommand{\predicateBaseResultsPredBitvectorsCorrectTrueCpuTimeAvgPlain}{39.81508934790515\xspace}
\providecommand{\predicateBaseResultsPredBitvectorsCorrectTrueCpuTimeAvgPlainHours}{}
  \renewcommand{\predicateBaseResultsPredBitvectorsCorrectTrueCpuTimeAvgPlainHours}{0.011059747041084764\xspace}

  % wall-time-avg
\providecommand{\predicateBaseResultsPredBitvectorsCorrectTrueWallTimeAvgPlain}{}
  \renewcommand{\predicateBaseResultsPredBitvectorsCorrectTrueWallTimeAvgPlain}{26.07074181490234\xspace}
\providecommand{\predicateBaseResultsPredBitvectorsCorrectTrueWallTimeAvgPlainHours}{}
  \renewcommand{\predicateBaseResultsPredBitvectorsCorrectTrueWallTimeAvgPlainHours}{0.007241872726361761\xspace}

  % inv-succ
\providecommand{\predicateBaseResultsPredBitvectorsCorrectTrueInvSuccPlain}{}
  \renewcommand{\predicateBaseResultsPredBitvectorsCorrectTrueInvSuccPlain}{0\xspace}

  % inv-tries
\providecommand{\predicateBaseResultsPredBitvectorsCorrectTrueInvTriesPlain}{}
  \renewcommand{\predicateBaseResultsPredBitvectorsCorrectTrueInvTriesPlain}{0\xspace}

  % inv-time-sum
\providecommand{\predicateBaseResultsPredBitvectorsCorrectTrueInvTimeSumPlain}{}
  \renewcommand{\predicateBaseResultsPredBitvectorsCorrectTrueInvTimeSumPlain}{0.0\xspace}
\providecommand{\predicateBaseResultsPredBitvectorsCorrectTrueInvTimeSumPlainHours}{}
  \renewcommand{\predicateBaseResultsPredBitvectorsCorrectTrueInvTimeSumPlainHours}{0.0\xspace}

 %% incorrect-false %%
\providecommand{\predicateBaseResultsPredBitvectorsIncorrectFalsePlain}{}
  \renewcommand{\predicateBaseResultsPredBitvectorsIncorrectFalsePlain}{27\xspace}

  % cpu-time-sum
\providecommand{\predicateBaseResultsPredBitvectorsIncorrectFalseCpuTimeSumPlain}{}
  \renewcommand{\predicateBaseResultsPredBitvectorsIncorrectFalseCpuTimeSumPlain}{712.5640386970001\xspace}
\providecommand{\predicateBaseResultsPredBitvectorsIncorrectFalseCpuTimeSumPlainHours}{}
  \renewcommand{\predicateBaseResultsPredBitvectorsIncorrectFalseCpuTimeSumPlainHours}{0.19793445519361114\xspace}

  % wall-time-sum
\providecommand{\predicateBaseResultsPredBitvectorsIncorrectFalseWallTimeSumPlain}{}
  \renewcommand{\predicateBaseResultsPredBitvectorsIncorrectFalseWallTimeSumPlain}{432.08190250372\xspace}
\providecommand{\predicateBaseResultsPredBitvectorsIncorrectFalseWallTimeSumPlainHours}{}
  \renewcommand{\predicateBaseResultsPredBitvectorsIncorrectFalseWallTimeSumPlainHours}{0.12002275069547778\xspace}

  % cpu-time-avg
\providecommand{\predicateBaseResultsPredBitvectorsIncorrectFalseCpuTimeAvgPlain}{}
  \renewcommand{\predicateBaseResultsPredBitvectorsIncorrectFalseCpuTimeAvgPlain}{26.391260692481485\xspace}
\providecommand{\predicateBaseResultsPredBitvectorsIncorrectFalseCpuTimeAvgPlainHours}{}
  \renewcommand{\predicateBaseResultsPredBitvectorsIncorrectFalseCpuTimeAvgPlainHours}{0.007330905747911523\xspace}

  % wall-time-avg
\providecommand{\predicateBaseResultsPredBitvectorsIncorrectFalseWallTimeAvgPlain}{}
  \renewcommand{\predicateBaseResultsPredBitvectorsIncorrectFalseWallTimeAvgPlain}{16.003033426063705\xspace}
\providecommand{\predicateBaseResultsPredBitvectorsIncorrectFalseWallTimeAvgPlainHours}{}
  \renewcommand{\predicateBaseResultsPredBitvectorsIncorrectFalseWallTimeAvgPlainHours}{0.004445287062795474\xspace}

  % inv-succ
\providecommand{\predicateBaseResultsPredBitvectorsIncorrectFalseInvSuccPlain}{}
  \renewcommand{\predicateBaseResultsPredBitvectorsIncorrectFalseInvSuccPlain}{0\xspace}

  % inv-tries
\providecommand{\predicateBaseResultsPredBitvectorsIncorrectFalseInvTriesPlain}{}
  \renewcommand{\predicateBaseResultsPredBitvectorsIncorrectFalseInvTriesPlain}{0\xspace}

  % inv-time-sum
\providecommand{\predicateBaseResultsPredBitvectorsIncorrectFalseInvTimeSumPlain}{}
  \renewcommand{\predicateBaseResultsPredBitvectorsIncorrectFalseInvTimeSumPlain}{0.0\xspace}
\providecommand{\predicateBaseResultsPredBitvectorsIncorrectFalseInvTimeSumPlainHours}{}
  \renewcommand{\predicateBaseResultsPredBitvectorsIncorrectFalseInvTimeSumPlainHours}{0.0\xspace}

 %% incorrect-true %%
\providecommand{\predicateBaseResultsPredBitvectorsIncorrectTruePlain}{}
  \renewcommand{\predicateBaseResultsPredBitvectorsIncorrectTruePlain}{0\xspace}

  % cpu-time-sum
\providecommand{\predicateBaseResultsPredBitvectorsIncorrectTrueCpuTimeSumPlain}{}
  \renewcommand{\predicateBaseResultsPredBitvectorsIncorrectTrueCpuTimeSumPlain}{0.0\xspace}
\providecommand{\predicateBaseResultsPredBitvectorsIncorrectTrueCpuTimeSumPlainHours}{}
  \renewcommand{\predicateBaseResultsPredBitvectorsIncorrectTrueCpuTimeSumPlainHours}{0.0\xspace}

  % wall-time-sum
\providecommand{\predicateBaseResultsPredBitvectorsIncorrectTrueWallTimeSumPlain}{}
  \renewcommand{\predicateBaseResultsPredBitvectorsIncorrectTrueWallTimeSumPlain}{0.0\xspace}
\providecommand{\predicateBaseResultsPredBitvectorsIncorrectTrueWallTimeSumPlainHours}{}
  \renewcommand{\predicateBaseResultsPredBitvectorsIncorrectTrueWallTimeSumPlainHours}{0.0\xspace}

  % cpu-time-avg
\providecommand{\predicateBaseResultsPredBitvectorsIncorrectTrueCpuTimeAvgPlain}{}
  \renewcommand{\predicateBaseResultsPredBitvectorsIncorrectTrueCpuTimeAvgPlain}{NaN\xspace}
\providecommand{\predicateBaseResultsPredBitvectorsIncorrectTrueCpuTimeAvgPlainHours}{}
  \renewcommand{\predicateBaseResultsPredBitvectorsIncorrectTrueCpuTimeAvgPlainHours}{NaN\xspace}

  % wall-time-avg
\providecommand{\predicateBaseResultsPredBitvectorsIncorrectTrueWallTimeAvgPlain}{}
  \renewcommand{\predicateBaseResultsPredBitvectorsIncorrectTrueWallTimeAvgPlain}{NaN\xspace}
\providecommand{\predicateBaseResultsPredBitvectorsIncorrectTrueWallTimeAvgPlainHours}{}
  \renewcommand{\predicateBaseResultsPredBitvectorsIncorrectTrueWallTimeAvgPlainHours}{NaN\xspace}

  % inv-succ
\providecommand{\predicateBaseResultsPredBitvectorsIncorrectTrueInvSuccPlain}{}
  \renewcommand{\predicateBaseResultsPredBitvectorsIncorrectTrueInvSuccPlain}{0\xspace}

  % inv-tries
\providecommand{\predicateBaseResultsPredBitvectorsIncorrectTrueInvTriesPlain}{}
  \renewcommand{\predicateBaseResultsPredBitvectorsIncorrectTrueInvTriesPlain}{0\xspace}

  % inv-time-sum
\providecommand{\predicateBaseResultsPredBitvectorsIncorrectTrueInvTimeSumPlain}{}
  \renewcommand{\predicateBaseResultsPredBitvectorsIncorrectTrueInvTimeSumPlain}{0.0\xspace}
\providecommand{\predicateBaseResultsPredBitvectorsIncorrectTrueInvTimeSumPlainHours}{}
  \renewcommand{\predicateBaseResultsPredBitvectorsIncorrectTrueInvTimeSumPlainHours}{0.0\xspace}

 %% all %%
\providecommand{\predicateBaseResultsPredBitvectorsAllPlain}{}
  \renewcommand{\predicateBaseResultsPredBitvectorsAllPlain}{3488\xspace}

  % cpu-time-sum
\providecommand{\predicateBaseResultsPredBitvectorsAllCpuTimeSumPlain}{}
  \renewcommand{\predicateBaseResultsPredBitvectorsAllCpuTimeSumPlain}{535007.5016851136\xspace}
\providecommand{\predicateBaseResultsPredBitvectorsAllCpuTimeSumPlainHours}{}
  \renewcommand{\predicateBaseResultsPredBitvectorsAllCpuTimeSumPlainHours}{148.61319491253155\xspace}

  % wall-time-sum
\providecommand{\predicateBaseResultsPredBitvectorsAllWallTimeSumPlain}{}
  \renewcommand{\predicateBaseResultsPredBitvectorsAllWallTimeSumPlain}{459142.2196831733\xspace}
\providecommand{\predicateBaseResultsPredBitvectorsAllWallTimeSumPlainHours}{}
  \renewcommand{\predicateBaseResultsPredBitvectorsAllWallTimeSumPlainHours}{127.53950546754814\xspace}

  % cpu-time-avg
\providecommand{\predicateBaseResultsPredBitvectorsAllCpuTimeAvgPlain}{}
  \renewcommand{\predicateBaseResultsPredBitvectorsAllCpuTimeAvgPlain}{153.38517823541102\xspace}
\providecommand{\predicateBaseResultsPredBitvectorsAllCpuTimeAvgPlainHours}{}
  \renewcommand{\predicateBaseResultsPredBitvectorsAllCpuTimeAvgPlainHours}{0.04260699395428084\xspace}

  % wall-time-avg
\providecommand{\predicateBaseResultsPredBitvectorsAllWallTimeAvgPlain}{}
  \renewcommand{\predicateBaseResultsPredBitvectorsAllWallTimeAvgPlain}{131.63481068898318\xspace}
\providecommand{\predicateBaseResultsPredBitvectorsAllWallTimeAvgPlainHours}{}
  \renewcommand{\predicateBaseResultsPredBitvectorsAllWallTimeAvgPlainHours}{0.036565225191384214\xspace}

  % inv-succ
\providecommand{\predicateBaseResultsPredBitvectorsAllInvSuccPlain}{}
  \renewcommand{\predicateBaseResultsPredBitvectorsAllInvSuccPlain}{0\xspace}

  % inv-tries
\providecommand{\predicateBaseResultsPredBitvectorsAllInvTriesPlain}{}
  \renewcommand{\predicateBaseResultsPredBitvectorsAllInvTriesPlain}{0\xspace}

  % inv-time-sum
\providecommand{\predicateBaseResultsPredBitvectorsAllInvTimeSumPlain}{}
  \renewcommand{\predicateBaseResultsPredBitvectorsAllInvTimeSumPlain}{0.0\xspace}
\providecommand{\predicateBaseResultsPredBitvectorsAllInvTimeSumPlainHours}{}
  \renewcommand{\predicateBaseResultsPredBitvectorsAllInvTimeSumPlainHours}{0.0\xspace}

 %% equal-only %%
\providecommand{\predicateBaseResultsPredBitvectorsEqualOnlyPlain}{}
  \renewcommand{\predicateBaseResultsPredBitvectorsEqualOnlyPlain}{1832\xspace}

  % cpu-time-sum
\providecommand{\predicateBaseResultsPredBitvectorsEqualOnlyCpuTimeSumPlain}{}
  \renewcommand{\predicateBaseResultsPredBitvectorsEqualOnlyCpuTimeSumPlain}{76538.70975279305\xspace}
\providecommand{\predicateBaseResultsPredBitvectorsEqualOnlyCpuTimeSumPlainHours}{}
  \renewcommand{\predicateBaseResultsPredBitvectorsEqualOnlyCpuTimeSumPlainHours}{21.26075270910918\xspace}

  % wall-time-sum
\providecommand{\predicateBaseResultsPredBitvectorsEqualOnlyWallTimeSumPlain}{}
  \renewcommand{\predicateBaseResultsPredBitvectorsEqualOnlyWallTimeSumPlain}{50601.00435876745\xspace}
\providecommand{\predicateBaseResultsPredBitvectorsEqualOnlyWallTimeSumPlainHours}{}
  \renewcommand{\predicateBaseResultsPredBitvectorsEqualOnlyWallTimeSumPlainHours}{14.05583454410207\xspace}

  % cpu-time-avg
\providecommand{\predicateBaseResultsPredBitvectorsEqualOnlyCpuTimeAvgPlain}{}
  \renewcommand{\predicateBaseResultsPredBitvectorsEqualOnlyCpuTimeAvgPlain}{41.77877169912284\xspace}
\providecommand{\predicateBaseResultsPredBitvectorsEqualOnlyCpuTimeAvgPlainHours}{}
  \renewcommand{\predicateBaseResultsPredBitvectorsEqualOnlyCpuTimeAvgPlainHours}{0.011605214360867457\xspace}

  % wall-time-avg
\providecommand{\predicateBaseResultsPredBitvectorsEqualOnlyWallTimeAvgPlain}{}
  \renewcommand{\predicateBaseResultsPredBitvectorsEqualOnlyWallTimeAvgPlain}{27.6206355670128\xspace}
\providecommand{\predicateBaseResultsPredBitvectorsEqualOnlyWallTimeAvgPlainHours}{}
  \renewcommand{\predicateBaseResultsPredBitvectorsEqualOnlyWallTimeAvgPlainHours}{0.007672398768614667\xspace}

  % inv-succ
\providecommand{\predicateBaseResultsPredBitvectorsEqualOnlyInvSuccPlain}{}
  \renewcommand{\predicateBaseResultsPredBitvectorsEqualOnlyInvSuccPlain}{0\xspace}

  % inv-tries
\providecommand{\predicateBaseResultsPredBitvectorsEqualOnlyInvTriesPlain}{}
  \renewcommand{\predicateBaseResultsPredBitvectorsEqualOnlyInvTriesPlain}{0\xspace}

  % inv-time-sum
\providecommand{\predicateBaseResultsPredBitvectorsEqualOnlyInvTimeSumPlain}{}
  \renewcommand{\predicateBaseResultsPredBitvectorsEqualOnlyInvTimeSumPlain}{0.0\xspace}
\providecommand{\predicateBaseResultsPredBitvectorsEqualOnlyInvTimeSumPlainHours}{}
  \renewcommand{\predicateBaseResultsPredBitvectorsEqualOnlyInvTimeSumPlainHours}{0.0\xspace}


  
  \begin{table}
   \caption{Details on analyses using path invariants for generating auxiliary invariants and their baseline}
 \label{table:pathinv_overall}
\begin{adjustbox}{max width=\textwidth}
  \begin{tabular}{l
                  S[table-format=4.0, round-mode=off, round-precision=3]
                  S[table-format=3.0, round-mode=off, round-precision=3]
                  S[table-format=2.0, round-mode=off, round-precision=3]
                  S[table-format=1.3, round-mode=figures, round-precision=3]
                  S[table-format=4.0, round-mode=off, round-precision=3]
                  S[table-format=4.0, round-mode=off, round-precision=3]
                  S[table-format=3.0, round-mode=figures, round-precision=3]
                  S[table-format=2.1, round-mode=figures, round-precision=3]
                  S[table-format=2.1, round-mode=figures, round-precision=3]}
\toprule
 & \multicolumn{2}{c}{\textbf{correct}} & \multicolumn{1}{c}{\textbf{wrong}} & \multicolumn{3}{c}{\textbf{Invariants (equal)}} & \multicolumn{3}{c}{\textbf{CPU time (h)}}\\
 & \multicolumn{1}{c}{proof} & \multicolumn{1}{c}{alarm} & \multicolumn{1}{c}{alarm} & \multicolumn{1}{c}{time (h)} & \multicolumn{1}{c}{tries} & \multicolumn{1}{c}{succ} & \multicolumn{1}{c}{all} & \multicolumn{1}{c}{correct} & \multicolumn{1}{c}{equal} \\
\cmidrule(lr){2-3}\cmidrule(lr){4-4}\cmidrule(lr){5-7}\cmidrule(lr){8-10}

  \textbf{base300} & \predicateBaseResultsPredBitvectorsCorrectTruePlain & \predicateBaseResultsPredBitvectorsCorrectFalsePlain & \predicateBaseResultsPredBitvectorsIncorrectFalsePlain &  &  &  & \predicateBaseResultsPredBitvectorsAllCpuTimeSumPlainHours & \predicateBaseResultsPredBitvectorsCorrectCpuTimeSumPlainHours & \predicateBaseResultsPredBitvectorsEqualOnlyCpuTimeSumPlainHours \\
\textbf{path-inv} & \predicateBitprecisePathinvariantsResultsPathInvariantsInvCPACorrectTruePlain & \predicateBitprecisePathinvariantsResultsPathInvariantsInvCPACorrectFalsePlain & \predicateBitprecisePathinvariantsResultsPathInvariantsInvCPAIncorrectFalsePlain & \predicateBitprecisePathinvariantsResultsPathInvariantsInvCPACorrectInvTimeSumPlainHours & \predicateBitprecisePathinvariantsResultsPathInvariantsInvCPACorrectInvTriesPlain & \predicateBitprecisePathinvariantsResultsPathInvariantsInvCPACorrectInvSuccPlain & \predicateBitprecisePathinvariantsResultsPathInvariantsInvCPAAllCpuTimeSumPlainHours & \predicateBitprecisePathinvariantsResultsPathInvariantsInvCPACorrectCpuTimeSumPlainHours & \predicateBitprecisePathinvariantsResultsPathInvariantsInvCPAEqualOnlyCpuTimeSumPlainHours \\
\textbf{path-policy} & \predicateBitprecisePathinvariantsResultsPathInvariantsPolicyCPACorrectTruePlain & \predicateBitprecisePathinvariantsResultsPathInvariantsPolicyCPACorrectFalsePlain & \predicateBitprecisePathinvariantsResultsPathInvariantsPolicyCPAIncorrectFalsePlain & \predicateBitprecisePathinvariantsResultsPathInvariantsPolicyCPACorrectInvTimeSumPlainHours & \predicateBitprecisePathinvariantsResultsPathInvariantsPolicyCPACorrectInvTriesPlain & \predicateBitprecisePathinvariantsResultsPathInvariantsPolicyCPACorrectInvSuccPlain & \predicateBitprecisePathinvariantsResultsPathInvariantsPolicyCPAAllCpuTimeSumPlainHours & \predicateBitprecisePathinvariantsResultsPathInvariantsPolicyCPACorrectCpuTimeSumPlainHours & \predicateBitprecisePathinvariantsResultsPathInvariantsPolicyCPAEqualOnlyCpuTimeSumPlainHours \\[0.5ex]
\textbf{400s-inv} & 1364 & 575 & 27 & & & & 196.1 & 35.55 &\\
\textbf{400s-policy} & 1371 & 576 & 27 & & & & 196.1 & 34.72 & \\
\bottomrule
 \end{tabular}
 \end{adjustbox}


\end{table}
As stated in \autoref{related:pathInv} and \autoref{background:pathinvariants}, path invariants were already a part of the \CPAchecker{} framework before this master's thesis, but during this work we
found an issue with the old implementation. The conversion of formulas did not consider different pointer encodings and thus lead to wrong formulas used as precision increment in the \PredicateCPA{}.
This was not recognized, because adding additional formulas to the precision --- correct and incorrect ones --- does not lead to wrong behavior, only the runtime increases with increasing size of the
precision. In our earlier work we found analyses using path invariants generated by the \InvariantsCPA{} or the \PolicyCPA{} perform better than the baseline in terms of the number of correctly
analyzed tasks. Path invariants improved the results by about \SI{1.5}{\percent}\,\sidenote{This is not comparable to our results as we use bit-precise analyses and the earlier evaluation used unbounded integers and 
rationals.}. When it comes to the time measured, analyses with path invariants took significantly longer (more than \SI{10}{\percent}) than the baseline.
This comes on the one hand from the time needed for the invariant generation, and on the other hand from the additional time needed for repeated refinements, if the generated invariants were not
strong enough to refute the counterexample\,\sidenote{This can be the case due to over-approximation while converting the formulas or because of the fact that the formula encoding was not correct,
and therefore many formulas added to the precision could not be used afterwards.}.

Our evaluation shows completely different results. The number of successfully analyzed tasks while using path invariants is about \SI{4}{\percent} lower than \textbf{base300}.
This is mainly caused by the additional time needed for invariant generation. When increasing the time limit to \SI{400}{\second} (\textbf{400s-inv}, \textbf{400s-policy}),
only verification tasks where the invariant generation is successful have a changing outcome, such that the overall results become approximately equal to \textbf{base300}. In
\autoref{table:pathinv_overall} one can see that while \textbf{path-policy} takes about \SI{1.5}{\hour} more time for generating invariants compared to \textbf{path-inv}, it
can successfully analyze \num{20} tasks more, and additionally the CPU time for equally and successfully analyzed tasks is approximately \SI{0.8}{\hour} less than for
\textbf{path-inv}. This behavior is less noticeable for the \textbf{400s} configurations but still the analysis using the \PolicyCPA{} performs better.

What is also interesting is that there are many tasks that can be correctly analyzed by \textbf{base300} in \SI{15}{\second} to \SI{30}{\second}, but are running into a timeout with \textbf{path-inv} and 
\textbf{path-policy}. The invariant generation in these cases is not the problem; instead the usage of the generated invariants leads to loop unrollings, and in turn, to more refinements than 
\textbf{base300}. The aim of path invariants was initially to prevent loop unrollings (and therefore refinements) but it seems that this is not working as expected with this invariant generation 
strategy. From \autoref{table:pathinv_overall} we can see furthermore that the number of correctly analyzed safe programs decreases in a higher ratio than the number of correct alarms. For the \textbf{400s} 
analyses, the number of correct alarms is even higher than \textbf{base300}, while the number of correct proofs is still smaller than \textbf{base300}.

\begin{table}
 \caption{A selection of tasks and their results with path invariants}
 \label{table:pathinv_detail_succ}
\begin{adjustbox}{max width=\textwidth}
  \begin{tabular}{lll}
  \toprule
  \textbf{file name} & \textbf{path-inv} & \textbf{path-policy} \\
  \cmidrule(lr){1-1}\cmidrule(lr){2-2}\cmidrule(lr){3-3}
loop-acceleration/array\_true-unreach-call3.i & ✓ & ✗ \\
loop-acceleration/functions\_true-unreach-call1.i & ✗ & ✓ \\
loop-acceleration/nested\_true-unreach-call1.i & ✓ & ✗  \\
loop-acceleration/simple\_true-unreach-call1.i & ✗ & ✓ \\
loop-new/count\_by\_1\_true-unreach-call.i & ✓ & ✗ \\
loop-new/count\_by\_1\_variant\_true-unreach-call.i & ✓ & ✗ \\
loop-new/count\_by\_nondet\_true-unreach-call.i & ✗ & ✓ \\
\bottomrule
 \end{tabular}
 \end{adjustbox}

\end{table}

Both path invariant generation approaches yield better results than \textbf{base300} for the tasks in the loops category. The tasks in the loops category do not consist of many lines of code. Instead
they have loops and some conditions that are complicated to track without having relations between variables. While \textbf{base300} times out on some of the tasks after \SI{300}{\second},
either \textbf{path-inv} or \textbf{path-policy} are able to successfully prove the safety of these programs (cf. \autoref{table:pathinv_detail_succ}). Furthermore, the safety of
these tasks can be proved within \SI{10}{\second}. So the 
speedup due to the invariants is enormous. For all these tasks \textbf{base300} times out after many refinements, with invariants only a small amount of refinements (between one and five) are necessary. 
Most of the computed invariants are very simple, for example, for the task \emph{loop-new/count\_by\_1\_true-unreach-call.i} (cf. \autoref{listing:easyinv}) the computed invariant
is that at the location of their \texttt{\_\_VERIFIER\_assert} call \texttt{i = 10000}. This is very helpful in contrast to the interpolants found by the \ac{SMT} solvers, which
force a loop unrolling in this case, because in each loop iteration the interpolant found is just that \texttt{i} equals the next higher number.


\begin{lstlisting}[language=C, label=listing:easyinv, caption=The source code of \mbox{loop-new/count\_by\_1\_true-unreach-call.i}, float, captionpos=b, frame = single]
 int main() {
    int i;
    for (i = 0; i < 1000000; i++) ;
    __VERIFIER_assert(i == 1000000);
    return 0;
}
\end{lstlisting}

To sum up, we have three cases, first there are tasks where invariant generation is not successful and thus the additional time spent effectively slows down the analysis. Second, there are 
tasks where invariant generation is successful, but the found invariants slow down the analysis, for example by forcing a loop to be unrolled. Unfortunately this case is appearing to be more common than the 
last case where the auxiliary invariants speed up the analysis. The last case leads, in many cases, to results within seconds where otherwise five minutes are not enough to analyze the verification task.
For a better performance it will therefore be necessary to classify the error traces and only compute path invariants for certain cases. Such conditions could, \eg, be that the loops in the infeasible 
counterexample path do not consist of more than \texttt{X} statements, or that the loop conditions and the loop iteration statements may only be simple increment or decrement operations. In general, taking 
the complexity of the loop body into account might help, but finding appropriate heuristics for that is future work.