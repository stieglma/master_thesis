\subsection{Parallel Analyses}\label{title:evalParallel}
\begin{figure}
  \centering
  %\tikzsetnextfilename{parallel_all.pdf}
\begin{small}
    \begin{tikzpicture}
      \begin{semilogyaxis}[
	  % Which column to be taken from each file
	  /pgfplots/table/y index=3,
	  /pgfplots/table/header=false,
	  % axis labels
	  xlabel=n-th fastest correct result,
	  ylabel=CPU time (\second),
	  % axis ranges
	  xmin= 0,
	  xmax=2150,
	  ymin=2,
	  ymax=650,
	  xtick distance=500,
	  width=\textwidth,
	  height=.7\textwidth,
	  mark repeat=500,
	  cycle list name=exotic,
	  % legend
	  legend entries={base600, base300, basePar, async-abs},%, async-path, async-prec, async-prec-abs, async-abs-path, async-prec-path, async-prec-abs-path},
	  every axis legend/.append style={at={(0,1)}, anchor=north west, outer xsep=5pt, outer ysep=5pt,},
	  ]
	  \addplot+[mark phase = 100] table {../resources/quantile_base_long.csv};
	  \addplot+[mark phase = 200] table {../resources/quantile_base_short.csv};
	  \addplot+[mark phase = 300] table {../resources/quantile_base_parallel.csv};
	  \addplot+[mark phase = 400] table {../resources/quantile_parallel_abs.csv};
	  %\addplot+[mark phase = 200] table {../resources/quantile_parallel_path.csv};
	  %\addplot+[mark phase = 250] table {../resources/quantile_parallel_prec.csv};
	  %\addplot+[mark phase = 300] table {../resources/quantile_parallel_prec_abs.csv};
	  %\addplot+[mark phase = 350] table {../resources/quantile_parallel_abs_path.csv};
	  %\addplot+[mark phase = 400] table {../resources/quantile_parallel_prec_path.csv};
	  %\addplot+[mark phase = 450] table {../resources/quantile_parallel_prec_abs_path.csv};
      \end{semilogyaxis}
    \end{tikzpicture}
    \end{small}

  \caption{A quantile plot showing the best concurrent analysis and the three baselines}
  \label{fig:quantile_parallel}
\end{figure}

In this section we move away from trying to generate invariants but keeping the impact on the overall time of the analysis low, to using approximately half of the overall analysis time for
invariant generation. By limiting the overall analysis time to \SI{600}{\second} and the CPU time to \SI{300}{\second} for both of the analysis threads we can achieve this.
Tracking the CPU time a thread needs is, however, not very precise in our case. On the one hand, threads started from within a thread are not counted towards this time,
and on the other hand, additionally running threads, for example for handling the resource limits or also the Java garbage collector can also not be counted towards these
limits. So the thread-wise CPU time limits are a best-effort approach to achieve a certain distribution.

As described in \autoref{title:configs} we only have parallel configurations combining an analysis using the \PredicateCPA{} and an analysis using the \InvariantsCPA{}.
The \InvariantsCPA{} is configured to be continuously refined, and provides the intermediate finished reached sets to the \PredicateCPA{}, which computes invariants from them.
As baselines, we use all three configurations \textbf{base300}, \textbf{base600} and \textbf{basePar} to be able to draw more precise conclusions out of the results. \autoref{fig:quantile_parallel} gives 
an overview on the performance of all baselines and the best parallel analysis using invariants\,\sidenote{All other parallel configurations using invariants are quite similar to \textbf{async-abs} and 
are excluded from \autoref{fig:quantile_parallel} for better visibility.}. All parallel-analysis configurations will be discussed in more detail in the following paragraphs. Exact numbers for all 
configurations can be found in \autoref{table:parallel}.
As can be seen in \autoref{fig:quantile_parallel}, \textbf{base300} has the same curve as \textbf{base600}, it just stops earlier. This is exactly what is expected as both configurations are equal,
only \textbf{base600} has twice the amount of time. More interesting is that the number of successfully analyzed tasks increases by \num{78} from \textbf{base300} to \textbf{base600} but by instead
using \textbf{basePar} we can further increase the number of successfully analyzed tasks by \num{28}.
This means that there are many programs quite hard to analyze with the \textbf{base300} or even \textbf{base600} which are easier for the analysis using the \InvariantsCPA{}.
By generating invariants
and using them at any possible location we can once again increase the number of successfully analyzed tasks. While all configurations using invariants perform strictly better than the baseline,
\textbf{async-abs} is the best configuration we evaluated in this setting, with an increase of \num{54} more correctly analyzed tasks, compared to \textbf{basePar}.




%%% predicate_base.2016-09-03_1927.results.pred-bitvectors %%%
 %% correct %%
\providecommand{\predicateBaseResultsPredBitvectorsCorrectPlain}{}
  \renewcommand{\predicateBaseResultsPredBitvectorsCorrectPlain}{1944\xspace}

  % cpu-time-sum
\providecommand{\predicateBaseResultsPredBitvectorsCorrectCpuTimeSumPlain}{}
  \renewcommand{\predicateBaseResultsPredBitvectorsCorrectCpuTimeSumPlain}{93675.79971751993\xspace}
\providecommand{\predicateBaseResultsPredBitvectorsCorrectCpuTimeSumPlainHours}{}
  \renewcommand{\predicateBaseResultsPredBitvectorsCorrectCpuTimeSumPlainHours}{26.02105547708887\xspace}

  % wall-time-sum
\providecommand{\predicateBaseResultsPredBitvectorsCorrectWallTimeSumPlain}{}
  \renewcommand{\predicateBaseResultsPredBitvectorsCorrectWallTimeSumPlain}{64278.471849201764\xspace}
\providecommand{\predicateBaseResultsPredBitvectorsCorrectWallTimeSumPlainHours}{}
  \renewcommand{\predicateBaseResultsPredBitvectorsCorrectWallTimeSumPlainHours}{17.85513106922271\xspace}

  % cpu-time-avg
\providecommand{\predicateBaseResultsPredBitvectorsCorrectCpuTimeAvgPlain}{}
  \renewcommand{\predicateBaseResultsPredBitvectorsCorrectCpuTimeAvgPlain}{48.18713977238679\xspace}
\providecommand{\predicateBaseResultsPredBitvectorsCorrectCpuTimeAvgPlainHours}{}
  \renewcommand{\predicateBaseResultsPredBitvectorsCorrectCpuTimeAvgPlainHours}{0.013385316603440776\xspace}

  % wall-time-avg
\providecommand{\predicateBaseResultsPredBitvectorsCorrectWallTimeAvgPlain}{}
  \renewcommand{\predicateBaseResultsPredBitvectorsCorrectWallTimeAvgPlain}{33.06505753559762\xspace}
\providecommand{\predicateBaseResultsPredBitvectorsCorrectWallTimeAvgPlainHours}{}
  \renewcommand{\predicateBaseResultsPredBitvectorsCorrectWallTimeAvgPlainHours}{0.009184738204332672\xspace}

  % inv-succ
\providecommand{\predicateBaseResultsPredBitvectorsCorrectInvSuccPlain}{}
  \renewcommand{\predicateBaseResultsPredBitvectorsCorrectInvSuccPlain}{0\xspace}

  % inv-tries
\providecommand{\predicateBaseResultsPredBitvectorsCorrectInvTriesPlain}{}
  \renewcommand{\predicateBaseResultsPredBitvectorsCorrectInvTriesPlain}{0\xspace}

  % inv-time-sum
\providecommand{\predicateBaseResultsPredBitvectorsCorrectInvTimeSumPlain}{}
  \renewcommand{\predicateBaseResultsPredBitvectorsCorrectInvTimeSumPlain}{0.0\xspace}
\providecommand{\predicateBaseResultsPredBitvectorsCorrectInvTimeSumPlainHours}{}
  \renewcommand{\predicateBaseResultsPredBitvectorsCorrectInvTimeSumPlainHours}{0.0\xspace}

  % finished-main
\providecommand{\predicateBaseResultsPredBitvectorsCorrectFinishedMainPlain}{}
  \renewcommand{\predicateBaseResultsPredBitvectorsCorrectFinishedMainPlain}{1944\xspace}

 %% incorrect %%
\providecommand{\predicateBaseResultsPredBitvectorsIncorrectPlain}{}
  \renewcommand{\predicateBaseResultsPredBitvectorsIncorrectPlain}{27\xspace}

  % cpu-time-sum
\providecommand{\predicateBaseResultsPredBitvectorsIncorrectCpuTimeSumPlain}{}
  \renewcommand{\predicateBaseResultsPredBitvectorsIncorrectCpuTimeSumPlain}{712.5640386970001\xspace}
\providecommand{\predicateBaseResultsPredBitvectorsIncorrectCpuTimeSumPlainHours}{}
  \renewcommand{\predicateBaseResultsPredBitvectorsIncorrectCpuTimeSumPlainHours}{0.19793445519361114\xspace}

  % wall-time-sum
\providecommand{\predicateBaseResultsPredBitvectorsIncorrectWallTimeSumPlain}{}
  \renewcommand{\predicateBaseResultsPredBitvectorsIncorrectWallTimeSumPlain}{432.08190250372\xspace}
\providecommand{\predicateBaseResultsPredBitvectorsIncorrectWallTimeSumPlainHours}{}
  \renewcommand{\predicateBaseResultsPredBitvectorsIncorrectWallTimeSumPlainHours}{0.12002275069547778\xspace}

  % cpu-time-avg
\providecommand{\predicateBaseResultsPredBitvectorsIncorrectCpuTimeAvgPlain}{}
  \renewcommand{\predicateBaseResultsPredBitvectorsIncorrectCpuTimeAvgPlain}{26.391260692481485\xspace}
\providecommand{\predicateBaseResultsPredBitvectorsIncorrectCpuTimeAvgPlainHours}{}
  \renewcommand{\predicateBaseResultsPredBitvectorsIncorrectCpuTimeAvgPlainHours}{0.007330905747911523\xspace}

  % wall-time-avg
\providecommand{\predicateBaseResultsPredBitvectorsIncorrectWallTimeAvgPlain}{}
  \renewcommand{\predicateBaseResultsPredBitvectorsIncorrectWallTimeAvgPlain}{16.003033426063705\xspace}
\providecommand{\predicateBaseResultsPredBitvectorsIncorrectWallTimeAvgPlainHours}{}
  \renewcommand{\predicateBaseResultsPredBitvectorsIncorrectWallTimeAvgPlainHours}{0.004445287062795474\xspace}

  % inv-succ
\providecommand{\predicateBaseResultsPredBitvectorsIncorrectInvSuccPlain}{}
  \renewcommand{\predicateBaseResultsPredBitvectorsIncorrectInvSuccPlain}{0\xspace}

  % inv-tries
\providecommand{\predicateBaseResultsPredBitvectorsIncorrectInvTriesPlain}{}
  \renewcommand{\predicateBaseResultsPredBitvectorsIncorrectInvTriesPlain}{0\xspace}

  % inv-time-sum
\providecommand{\predicateBaseResultsPredBitvectorsIncorrectInvTimeSumPlain}{}
  \renewcommand{\predicateBaseResultsPredBitvectorsIncorrectInvTimeSumPlain}{0.0\xspace}
\providecommand{\predicateBaseResultsPredBitvectorsIncorrectInvTimeSumPlainHours}{}
  \renewcommand{\predicateBaseResultsPredBitvectorsIncorrectInvTimeSumPlainHours}{0.0\xspace}

  % finished-main
\providecommand{\predicateBaseResultsPredBitvectorsIncorrectFinishedMainPlain}{}
  \renewcommand{\predicateBaseResultsPredBitvectorsIncorrectFinishedMainPlain}{27\xspace}

 %% timeout %%
\providecommand{\predicateBaseResultsPredBitvectorsTimeoutPlain}{}
  \renewcommand{\predicateBaseResultsPredBitvectorsTimeoutPlain}{1414\xspace}

  % cpu-time-sum
\providecommand{\predicateBaseResultsPredBitvectorsTimeoutCpuTimeSumPlain}{}
  \renewcommand{\predicateBaseResultsPredBitvectorsTimeoutCpuTimeSumPlain}{432542.40992774273\xspace}
\providecommand{\predicateBaseResultsPredBitvectorsTimeoutCpuTimeSumPlainHours}{}
  \renewcommand{\predicateBaseResultsPredBitvectorsTimeoutCpuTimeSumPlainHours}{120.15066942437298\xspace}

  % wall-time-sum
\providecommand{\predicateBaseResultsPredBitvectorsTimeoutWallTimeSumPlain}{}
  \renewcommand{\predicateBaseResultsPredBitvectorsTimeoutWallTimeSumPlain}{389237.6199243135\xspace}
\providecommand{\predicateBaseResultsPredBitvectorsTimeoutWallTimeSumPlainHours}{}
  \renewcommand{\predicateBaseResultsPredBitvectorsTimeoutWallTimeSumPlainHours}{108.12156109008708\xspace}

  % cpu-time-avg
\providecommand{\predicateBaseResultsPredBitvectorsTimeoutCpuTimeAvgPlain}{}
  \renewcommand{\predicateBaseResultsPredBitvectorsTimeoutCpuTimeAvgPlain}{305.8998655783188\xspace}
\providecommand{\predicateBaseResultsPredBitvectorsTimeoutCpuTimeAvgPlainHours}{}
  \renewcommand{\predicateBaseResultsPredBitvectorsTimeoutCpuTimeAvgPlainHours}{0.08497218488286633\xspace}

  % wall-time-avg
\providecommand{\predicateBaseResultsPredBitvectorsTimeoutWallTimeAvgPlain}{}
  \renewcommand{\predicateBaseResultsPredBitvectorsTimeoutWallTimeAvgPlain}{275.2741300737719\xspace}
\providecommand{\predicateBaseResultsPredBitvectorsTimeoutWallTimeAvgPlainHours}{}
  \renewcommand{\predicateBaseResultsPredBitvectorsTimeoutWallTimeAvgPlainHours}{0.07646503613160331\xspace}

  % inv-succ
\providecommand{\predicateBaseResultsPredBitvectorsTimeoutInvSuccPlain}{}
  \renewcommand{\predicateBaseResultsPredBitvectorsTimeoutInvSuccPlain}{0\xspace}

  % inv-tries
\providecommand{\predicateBaseResultsPredBitvectorsTimeoutInvTriesPlain}{}
  \renewcommand{\predicateBaseResultsPredBitvectorsTimeoutInvTriesPlain}{0\xspace}

  % inv-time-sum
\providecommand{\predicateBaseResultsPredBitvectorsTimeoutInvTimeSumPlain}{}
  \renewcommand{\predicateBaseResultsPredBitvectorsTimeoutInvTimeSumPlain}{0.0\xspace}
\providecommand{\predicateBaseResultsPredBitvectorsTimeoutInvTimeSumPlainHours}{}
  \renewcommand{\predicateBaseResultsPredBitvectorsTimeoutInvTimeSumPlainHours}{0.0\xspace}

  % finished-main
\providecommand{\predicateBaseResultsPredBitvectorsTimeoutFinishedMainPlain}{}
  \renewcommand{\predicateBaseResultsPredBitvectorsTimeoutFinishedMainPlain}{1414\xspace}

 %% unknown-or-category-error %%
\providecommand{\predicateBaseResultsPredBitvectorsUnknownOrCategoryErrorPlain}{}
  \renewcommand{\predicateBaseResultsPredBitvectorsUnknownOrCategoryErrorPlain}{1517\xspace}

  % cpu-time-sum
\providecommand{\predicateBaseResultsPredBitvectorsUnknownOrCategoryErrorCpuTimeSumPlain}{}
  \renewcommand{\predicateBaseResultsPredBitvectorsUnknownOrCategoryErrorCpuTimeSumPlain}{440619.1379288969\xspace}
\providecommand{\predicateBaseResultsPredBitvectorsUnknownOrCategoryErrorCpuTimeSumPlainHours}{}
  \renewcommand{\predicateBaseResultsPredBitvectorsUnknownOrCategoryErrorCpuTimeSumPlainHours}{122.39420498024913\xspace}

  % wall-time-sum
\providecommand{\predicateBaseResultsPredBitvectorsUnknownOrCategoryErrorWallTimeSumPlain}{}
  \renewcommand{\predicateBaseResultsPredBitvectorsUnknownOrCategoryErrorWallTimeSumPlain}{394431.6659314679\xspace}
\providecommand{\predicateBaseResultsPredBitvectorsUnknownOrCategoryErrorWallTimeSumPlainHours}{}
  \renewcommand{\predicateBaseResultsPredBitvectorsUnknownOrCategoryErrorWallTimeSumPlainHours}{109.56435164762998\xspace}

  % cpu-time-avg
\providecommand{\predicateBaseResultsPredBitvectorsUnknownOrCategoryErrorCpuTimeAvgPlain}{}
  \renewcommand{\predicateBaseResultsPredBitvectorsUnknownOrCategoryErrorCpuTimeAvgPlain}{290.4542768153572\xspace}
\providecommand{\predicateBaseResultsPredBitvectorsUnknownOrCategoryErrorCpuTimeAvgPlainHours}{}
  \renewcommand{\predicateBaseResultsPredBitvectorsUnknownOrCategoryErrorCpuTimeAvgPlainHours}{0.08068174355982145\xspace}

  % wall-time-avg
\providecommand{\predicateBaseResultsPredBitvectorsUnknownOrCategoryErrorWallTimeAvgPlain}{}
  \renewcommand{\predicateBaseResultsPredBitvectorsUnknownOrCategoryErrorWallTimeAvgPlain}{260.0076901328068\xspace}
\providecommand{\predicateBaseResultsPredBitvectorsUnknownOrCategoryErrorWallTimeAvgPlainHours}{}
  \renewcommand{\predicateBaseResultsPredBitvectorsUnknownOrCategoryErrorWallTimeAvgPlainHours}{0.0722243583702241\xspace}

  % inv-succ
\providecommand{\predicateBaseResultsPredBitvectorsUnknownOrCategoryErrorInvSuccPlain}{}
  \renewcommand{\predicateBaseResultsPredBitvectorsUnknownOrCategoryErrorInvSuccPlain}{0\xspace}

  % inv-tries
\providecommand{\predicateBaseResultsPredBitvectorsUnknownOrCategoryErrorInvTriesPlain}{}
  \renewcommand{\predicateBaseResultsPredBitvectorsUnknownOrCategoryErrorInvTriesPlain}{0\xspace}

  % inv-time-sum
\providecommand{\predicateBaseResultsPredBitvectorsUnknownOrCategoryErrorInvTimeSumPlain}{}
  \renewcommand{\predicateBaseResultsPredBitvectorsUnknownOrCategoryErrorInvTimeSumPlain}{0.0\xspace}
\providecommand{\predicateBaseResultsPredBitvectorsUnknownOrCategoryErrorInvTimeSumPlainHours}{}
  \renewcommand{\predicateBaseResultsPredBitvectorsUnknownOrCategoryErrorInvTimeSumPlainHours}{0.0\xspace}

  % finished-main
\providecommand{\predicateBaseResultsPredBitvectorsUnknownOrCategoryErrorFinishedMainPlain}{}
  \renewcommand{\predicateBaseResultsPredBitvectorsUnknownOrCategoryErrorFinishedMainPlain}{1517\xspace}

 %% correct-false %%
\providecommand{\predicateBaseResultsPredBitvectorsCorrectFalsePlain}{}
  \renewcommand{\predicateBaseResultsPredBitvectorsCorrectFalsePlain}{553\xspace}

  % cpu-time-sum
\providecommand{\predicateBaseResultsPredBitvectorsCorrectFalseCpuTimeSumPlain}{}
  \renewcommand{\predicateBaseResultsPredBitvectorsCorrectFalseCpuTimeSumPlain}{38293.01043458401\xspace}
\providecommand{\predicateBaseResultsPredBitvectorsCorrectFalseCpuTimeSumPlainHours}{}
  \renewcommand{\predicateBaseResultsPredBitvectorsCorrectFalseCpuTimeSumPlainHours}{10.636947342940003\xspace}

  % wall-time-sum
\providecommand{\predicateBaseResultsPredBitvectorsCorrectFalseWallTimeSumPlain}{}
  \renewcommand{\predicateBaseResultsPredBitvectorsCorrectFalseWallTimeSumPlain}{28014.069984672435\xspace}
\providecommand{\predicateBaseResultsPredBitvectorsCorrectFalseWallTimeSumPlainHours}{}
  \renewcommand{\predicateBaseResultsPredBitvectorsCorrectFalseWallTimeSumPlainHours}{7.781686106853454\xspace}

  % cpu-time-avg
\providecommand{\predicateBaseResultsPredBitvectorsCorrectFalseCpuTimeAvgPlain}{}
  \renewcommand{\predicateBaseResultsPredBitvectorsCorrectFalseCpuTimeAvgPlain}{69.2459501529548\xspace}
\providecommand{\predicateBaseResultsPredBitvectorsCorrectFalseCpuTimeAvgPlainHours}{}
  \renewcommand{\predicateBaseResultsPredBitvectorsCorrectFalseCpuTimeAvgPlainHours}{0.019234986153598557\xspace}

  % wall-time-avg
\providecommand{\predicateBaseResultsPredBitvectorsCorrectFalseWallTimeAvgPlain}{}
  \renewcommand{\predicateBaseResultsPredBitvectorsCorrectFalseWallTimeAvgPlain}{50.65835440266263\xspace}
\providecommand{\predicateBaseResultsPredBitvectorsCorrectFalseWallTimeAvgPlainHours}{}
  \renewcommand{\predicateBaseResultsPredBitvectorsCorrectFalseWallTimeAvgPlainHours}{0.014071765111850732\xspace}

  % inv-succ
\providecommand{\predicateBaseResultsPredBitvectorsCorrectFalseInvSuccPlain}{}
  \renewcommand{\predicateBaseResultsPredBitvectorsCorrectFalseInvSuccPlain}{0\xspace}

  % inv-tries
\providecommand{\predicateBaseResultsPredBitvectorsCorrectFalseInvTriesPlain}{}
  \renewcommand{\predicateBaseResultsPredBitvectorsCorrectFalseInvTriesPlain}{0\xspace}

  % inv-time-sum
\providecommand{\predicateBaseResultsPredBitvectorsCorrectFalseInvTimeSumPlain}{}
  \renewcommand{\predicateBaseResultsPredBitvectorsCorrectFalseInvTimeSumPlain}{0.0\xspace}
\providecommand{\predicateBaseResultsPredBitvectorsCorrectFalseInvTimeSumPlainHours}{}
  \renewcommand{\predicateBaseResultsPredBitvectorsCorrectFalseInvTimeSumPlainHours}{0.0\xspace}

  % finished-main
\providecommand{\predicateBaseResultsPredBitvectorsCorrectFalseFinishedMainPlain}{}
  \renewcommand{\predicateBaseResultsPredBitvectorsCorrectFalseFinishedMainPlain}{553\xspace}

 %% correct-true %%
\providecommand{\predicateBaseResultsPredBitvectorsCorrectTruePlain}{}
  \renewcommand{\predicateBaseResultsPredBitvectorsCorrectTruePlain}{1391\xspace}

  % cpu-time-sum
\providecommand{\predicateBaseResultsPredBitvectorsCorrectTrueCpuTimeSumPlain}{}
  \renewcommand{\predicateBaseResultsPredBitvectorsCorrectTrueCpuTimeSumPlain}{55382.789282936064\xspace}
\providecommand{\predicateBaseResultsPredBitvectorsCorrectTrueCpuTimeSumPlainHours}{}
  \renewcommand{\predicateBaseResultsPredBitvectorsCorrectTrueCpuTimeSumPlainHours}{15.384108134148907\xspace}

  % wall-time-sum
\providecommand{\predicateBaseResultsPredBitvectorsCorrectTrueWallTimeSumPlain}{}
  \renewcommand{\predicateBaseResultsPredBitvectorsCorrectTrueWallTimeSumPlain}{36264.40186452916\xspace}
\providecommand{\predicateBaseResultsPredBitvectorsCorrectTrueWallTimeSumPlainHours}{}
  \renewcommand{\predicateBaseResultsPredBitvectorsCorrectTrueWallTimeSumPlainHours}{10.07344496236921\xspace}

  % cpu-time-avg
\providecommand{\predicateBaseResultsPredBitvectorsCorrectTrueCpuTimeAvgPlain}{}
  \renewcommand{\predicateBaseResultsPredBitvectorsCorrectTrueCpuTimeAvgPlain}{39.81508934790515\xspace}
\providecommand{\predicateBaseResultsPredBitvectorsCorrectTrueCpuTimeAvgPlainHours}{}
  \renewcommand{\predicateBaseResultsPredBitvectorsCorrectTrueCpuTimeAvgPlainHours}{0.011059747041084764\xspace}

  % wall-time-avg
\providecommand{\predicateBaseResultsPredBitvectorsCorrectTrueWallTimeAvgPlain}{}
  \renewcommand{\predicateBaseResultsPredBitvectorsCorrectTrueWallTimeAvgPlain}{26.07074181490234\xspace}
\providecommand{\predicateBaseResultsPredBitvectorsCorrectTrueWallTimeAvgPlainHours}{}
  \renewcommand{\predicateBaseResultsPredBitvectorsCorrectTrueWallTimeAvgPlainHours}{0.007241872726361761\xspace}

  % inv-succ
\providecommand{\predicateBaseResultsPredBitvectorsCorrectTrueInvSuccPlain}{}
  \renewcommand{\predicateBaseResultsPredBitvectorsCorrectTrueInvSuccPlain}{0\xspace}

  % inv-tries
\providecommand{\predicateBaseResultsPredBitvectorsCorrectTrueInvTriesPlain}{}
  \renewcommand{\predicateBaseResultsPredBitvectorsCorrectTrueInvTriesPlain}{0\xspace}

  % inv-time-sum
\providecommand{\predicateBaseResultsPredBitvectorsCorrectTrueInvTimeSumPlain}{}
  \renewcommand{\predicateBaseResultsPredBitvectorsCorrectTrueInvTimeSumPlain}{0.0\xspace}
\providecommand{\predicateBaseResultsPredBitvectorsCorrectTrueInvTimeSumPlainHours}{}
  \renewcommand{\predicateBaseResultsPredBitvectorsCorrectTrueInvTimeSumPlainHours}{0.0\xspace}

  % finished-main
\providecommand{\predicateBaseResultsPredBitvectorsCorrectTrueFinishedMainPlain}{}
  \renewcommand{\predicateBaseResultsPredBitvectorsCorrectTrueFinishedMainPlain}{1391\xspace}

 %% incorrect-false %%
\providecommand{\predicateBaseResultsPredBitvectorsIncorrectFalsePlain}{}
  \renewcommand{\predicateBaseResultsPredBitvectorsIncorrectFalsePlain}{27\xspace}

  % cpu-time-sum
\providecommand{\predicateBaseResultsPredBitvectorsIncorrectFalseCpuTimeSumPlain}{}
  \renewcommand{\predicateBaseResultsPredBitvectorsIncorrectFalseCpuTimeSumPlain}{712.5640386970001\xspace}
\providecommand{\predicateBaseResultsPredBitvectorsIncorrectFalseCpuTimeSumPlainHours}{}
  \renewcommand{\predicateBaseResultsPredBitvectorsIncorrectFalseCpuTimeSumPlainHours}{0.19793445519361114\xspace}

  % wall-time-sum
\providecommand{\predicateBaseResultsPredBitvectorsIncorrectFalseWallTimeSumPlain}{}
  \renewcommand{\predicateBaseResultsPredBitvectorsIncorrectFalseWallTimeSumPlain}{432.08190250372\xspace}
\providecommand{\predicateBaseResultsPredBitvectorsIncorrectFalseWallTimeSumPlainHours}{}
  \renewcommand{\predicateBaseResultsPredBitvectorsIncorrectFalseWallTimeSumPlainHours}{0.12002275069547778\xspace}

  % cpu-time-avg
\providecommand{\predicateBaseResultsPredBitvectorsIncorrectFalseCpuTimeAvgPlain}{}
  \renewcommand{\predicateBaseResultsPredBitvectorsIncorrectFalseCpuTimeAvgPlain}{26.391260692481485\xspace}
\providecommand{\predicateBaseResultsPredBitvectorsIncorrectFalseCpuTimeAvgPlainHours}{}
  \renewcommand{\predicateBaseResultsPredBitvectorsIncorrectFalseCpuTimeAvgPlainHours}{0.007330905747911523\xspace}

  % wall-time-avg
\providecommand{\predicateBaseResultsPredBitvectorsIncorrectFalseWallTimeAvgPlain}{}
  \renewcommand{\predicateBaseResultsPredBitvectorsIncorrectFalseWallTimeAvgPlain}{16.003033426063705\xspace}
\providecommand{\predicateBaseResultsPredBitvectorsIncorrectFalseWallTimeAvgPlainHours}{}
  \renewcommand{\predicateBaseResultsPredBitvectorsIncorrectFalseWallTimeAvgPlainHours}{0.004445287062795474\xspace}

  % inv-succ
\providecommand{\predicateBaseResultsPredBitvectorsIncorrectFalseInvSuccPlain}{}
  \renewcommand{\predicateBaseResultsPredBitvectorsIncorrectFalseInvSuccPlain}{0\xspace}

  % inv-tries
\providecommand{\predicateBaseResultsPredBitvectorsIncorrectFalseInvTriesPlain}{}
  \renewcommand{\predicateBaseResultsPredBitvectorsIncorrectFalseInvTriesPlain}{0\xspace}

  % inv-time-sum
\providecommand{\predicateBaseResultsPredBitvectorsIncorrectFalseInvTimeSumPlain}{}
  \renewcommand{\predicateBaseResultsPredBitvectorsIncorrectFalseInvTimeSumPlain}{0.0\xspace}
\providecommand{\predicateBaseResultsPredBitvectorsIncorrectFalseInvTimeSumPlainHours}{}
  \renewcommand{\predicateBaseResultsPredBitvectorsIncorrectFalseInvTimeSumPlainHours}{0.0\xspace}

  % finished-main
\providecommand{\predicateBaseResultsPredBitvectorsIncorrectFalseFinishedMainPlain}{}
  \renewcommand{\predicateBaseResultsPredBitvectorsIncorrectFalseFinishedMainPlain}{27\xspace}

 %% incorrect-true %%
\providecommand{\predicateBaseResultsPredBitvectorsIncorrectTruePlain}{}
  \renewcommand{\predicateBaseResultsPredBitvectorsIncorrectTruePlain}{0\xspace}

  % cpu-time-sum
\providecommand{\predicateBaseResultsPredBitvectorsIncorrectTrueCpuTimeSumPlain}{}
  \renewcommand{\predicateBaseResultsPredBitvectorsIncorrectTrueCpuTimeSumPlain}{0.0\xspace}
\providecommand{\predicateBaseResultsPredBitvectorsIncorrectTrueCpuTimeSumPlainHours}{}
  \renewcommand{\predicateBaseResultsPredBitvectorsIncorrectTrueCpuTimeSumPlainHours}{0.0\xspace}

  % wall-time-sum
\providecommand{\predicateBaseResultsPredBitvectorsIncorrectTrueWallTimeSumPlain}{}
  \renewcommand{\predicateBaseResultsPredBitvectorsIncorrectTrueWallTimeSumPlain}{0.0\xspace}
\providecommand{\predicateBaseResultsPredBitvectorsIncorrectTrueWallTimeSumPlainHours}{}
  \renewcommand{\predicateBaseResultsPredBitvectorsIncorrectTrueWallTimeSumPlainHours}{0.0\xspace}

  % cpu-time-avg
\providecommand{\predicateBaseResultsPredBitvectorsIncorrectTrueCpuTimeAvgPlain}{}
  \renewcommand{\predicateBaseResultsPredBitvectorsIncorrectTrueCpuTimeAvgPlain}{NaN\xspace}
\providecommand{\predicateBaseResultsPredBitvectorsIncorrectTrueCpuTimeAvgPlainHours}{}
  \renewcommand{\predicateBaseResultsPredBitvectorsIncorrectTrueCpuTimeAvgPlainHours}{NaN\xspace}

  % wall-time-avg
\providecommand{\predicateBaseResultsPredBitvectorsIncorrectTrueWallTimeAvgPlain}{}
  \renewcommand{\predicateBaseResultsPredBitvectorsIncorrectTrueWallTimeAvgPlain}{NaN\xspace}
\providecommand{\predicateBaseResultsPredBitvectorsIncorrectTrueWallTimeAvgPlainHours}{}
  \renewcommand{\predicateBaseResultsPredBitvectorsIncorrectTrueWallTimeAvgPlainHours}{NaN\xspace}

  % inv-succ
\providecommand{\predicateBaseResultsPredBitvectorsIncorrectTrueInvSuccPlain}{}
  \renewcommand{\predicateBaseResultsPredBitvectorsIncorrectTrueInvSuccPlain}{0\xspace}

  % inv-tries
\providecommand{\predicateBaseResultsPredBitvectorsIncorrectTrueInvTriesPlain}{}
  \renewcommand{\predicateBaseResultsPredBitvectorsIncorrectTrueInvTriesPlain}{0\xspace}

  % inv-time-sum
\providecommand{\predicateBaseResultsPredBitvectorsIncorrectTrueInvTimeSumPlain}{}
  \renewcommand{\predicateBaseResultsPredBitvectorsIncorrectTrueInvTimeSumPlain}{0.0\xspace}
\providecommand{\predicateBaseResultsPredBitvectorsIncorrectTrueInvTimeSumPlainHours}{}
  \renewcommand{\predicateBaseResultsPredBitvectorsIncorrectTrueInvTimeSumPlainHours}{0.0\xspace}

  % finished-main
\providecommand{\predicateBaseResultsPredBitvectorsIncorrectTrueFinishedMainPlain}{}
  \renewcommand{\predicateBaseResultsPredBitvectorsIncorrectTrueFinishedMainPlain}{0\xspace}

 %% all %%
\providecommand{\predicateBaseResultsPredBitvectorsAllPlain}{}
  \renewcommand{\predicateBaseResultsPredBitvectorsAllPlain}{3488\xspace}

  % cpu-time-sum
\providecommand{\predicateBaseResultsPredBitvectorsAllCpuTimeSumPlain}{}
  \renewcommand{\predicateBaseResultsPredBitvectorsAllCpuTimeSumPlain}{535007.5016851136\xspace}
\providecommand{\predicateBaseResultsPredBitvectorsAllCpuTimeSumPlainHours}{}
  \renewcommand{\predicateBaseResultsPredBitvectorsAllCpuTimeSumPlainHours}{148.61319491253155\xspace}

  % wall-time-sum
\providecommand{\predicateBaseResultsPredBitvectorsAllWallTimeSumPlain}{}
  \renewcommand{\predicateBaseResultsPredBitvectorsAllWallTimeSumPlain}{459142.2196831733\xspace}
\providecommand{\predicateBaseResultsPredBitvectorsAllWallTimeSumPlainHours}{}
  \renewcommand{\predicateBaseResultsPredBitvectorsAllWallTimeSumPlainHours}{127.53950546754814\xspace}

  % cpu-time-avg
\providecommand{\predicateBaseResultsPredBitvectorsAllCpuTimeAvgPlain}{}
  \renewcommand{\predicateBaseResultsPredBitvectorsAllCpuTimeAvgPlain}{153.38517823541102\xspace}
\providecommand{\predicateBaseResultsPredBitvectorsAllCpuTimeAvgPlainHours}{}
  \renewcommand{\predicateBaseResultsPredBitvectorsAllCpuTimeAvgPlainHours}{0.04260699395428084\xspace}

  % wall-time-avg
\providecommand{\predicateBaseResultsPredBitvectorsAllWallTimeAvgPlain}{}
  \renewcommand{\predicateBaseResultsPredBitvectorsAllWallTimeAvgPlain}{131.63481068898318\xspace}
\providecommand{\predicateBaseResultsPredBitvectorsAllWallTimeAvgPlainHours}{}
  \renewcommand{\predicateBaseResultsPredBitvectorsAllWallTimeAvgPlainHours}{0.036565225191384214\xspace}

  % inv-succ
\providecommand{\predicateBaseResultsPredBitvectorsAllInvSuccPlain}{}
  \renewcommand{\predicateBaseResultsPredBitvectorsAllInvSuccPlain}{0\xspace}

  % inv-tries
\providecommand{\predicateBaseResultsPredBitvectorsAllInvTriesPlain}{}
  \renewcommand{\predicateBaseResultsPredBitvectorsAllInvTriesPlain}{0\xspace}

  % inv-time-sum
\providecommand{\predicateBaseResultsPredBitvectorsAllInvTimeSumPlain}{}
  \renewcommand{\predicateBaseResultsPredBitvectorsAllInvTimeSumPlain}{0.0\xspace}
\providecommand{\predicateBaseResultsPredBitvectorsAllInvTimeSumPlainHours}{}
  \renewcommand{\predicateBaseResultsPredBitvectorsAllInvTimeSumPlainHours}{0.0\xspace}

  % finished-main
\providecommand{\predicateBaseResultsPredBitvectorsAllFinishedMainPlain}{}
  \renewcommand{\predicateBaseResultsPredBitvectorsAllFinishedMainPlain}{3488\xspace}

 %% equal-only %%
\providecommand{\predicateBaseResultsPredBitvectorsEqualOnlyPlain}{}
  \renewcommand{\predicateBaseResultsPredBitvectorsEqualOnlyPlain}{1865\xspace}

  % cpu-time-sum
\providecommand{\predicateBaseResultsPredBitvectorsEqualOnlyCpuTimeSumPlain}{}
  \renewcommand{\predicateBaseResultsPredBitvectorsEqualOnlyCpuTimeSumPlain}{75203.42298178401\xspace}
\providecommand{\predicateBaseResultsPredBitvectorsEqualOnlyCpuTimeSumPlainHours}{}
  \renewcommand{\predicateBaseResultsPredBitvectorsEqualOnlyCpuTimeSumPlainHours}{20.889839717162225\xspace}

  % wall-time-sum
\providecommand{\predicateBaseResultsPredBitvectorsEqualOnlyWallTimeSumPlain}{}
  \renewcommand{\predicateBaseResultsPredBitvectorsEqualOnlyWallTimeSumPlain}{49506.36484003198\xspace}
\providecommand{\predicateBaseResultsPredBitvectorsEqualOnlyWallTimeSumPlainHours}{}
  \renewcommand{\predicateBaseResultsPredBitvectorsEqualOnlyWallTimeSumPlainHours}{13.751768011119994\xspace}

  % cpu-time-avg
\providecommand{\predicateBaseResultsPredBitvectorsEqualOnlyCpuTimeAvgPlain}{}
  \renewcommand{\predicateBaseResultsPredBitvectorsEqualOnlyCpuTimeAvgPlain}{40.323551196667026\xspace}
\providecommand{\predicateBaseResultsPredBitvectorsEqualOnlyCpuTimeAvgPlainHours}{}
  \renewcommand{\predicateBaseResultsPredBitvectorsEqualOnlyCpuTimeAvgPlainHours}{0.011200986443518619\xspace}

  % wall-time-avg
\providecommand{\predicateBaseResultsPredBitvectorsEqualOnlyWallTimeAvgPlain}{}
  \renewcommand{\predicateBaseResultsPredBitvectorsEqualOnlyWallTimeAvgPlain}{26.54496774264449\xspace}
\providecommand{\predicateBaseResultsPredBitvectorsEqualOnlyWallTimeAvgPlainHours}{}
  \renewcommand{\predicateBaseResultsPredBitvectorsEqualOnlyWallTimeAvgPlainHours}{0.007373602150734581\xspace}

  % inv-succ
\providecommand{\predicateBaseResultsPredBitvectorsEqualOnlyInvSuccPlain}{}
  \renewcommand{\predicateBaseResultsPredBitvectorsEqualOnlyInvSuccPlain}{0\xspace}

  % inv-tries
\providecommand{\predicateBaseResultsPredBitvectorsEqualOnlyInvTriesPlain}{}
  \renewcommand{\predicateBaseResultsPredBitvectorsEqualOnlyInvTriesPlain}{0\xspace}

  % inv-time-sum
\providecommand{\predicateBaseResultsPredBitvectorsEqualOnlyInvTimeSumPlain}{}
  \renewcommand{\predicateBaseResultsPredBitvectorsEqualOnlyInvTimeSumPlain}{0.0\xspace}
\providecommand{\predicateBaseResultsPredBitvectorsEqualOnlyInvTimeSumPlainHours}{}
  \renewcommand{\predicateBaseResultsPredBitvectorsEqualOnlyInvTimeSumPlainHours}{0.0\xspace}

  % finished-main
\providecommand{\predicateBaseResultsPredBitvectorsEqualOnlyFinishedMainPlain}{}
  \renewcommand{\predicateBaseResultsPredBitvectorsEqualOnlyFinishedMainPlain}{1865\xspace}

%%% predicate_base_longtimeout.2016-09-03_1353.results.pred-bitvectors %%%
 %% correct %%
\providecommand{\predicateBaseLongtimeoutResultsPredBitvectorsCorrectPlain}{}
  \renewcommand{\predicateBaseLongtimeoutResultsPredBitvectorsCorrectPlain}{2022\xspace}

  % cpu-time-sum
\providecommand{\predicateBaseLongtimeoutResultsPredBitvectorsCorrectCpuTimeSumPlain}{}
  \renewcommand{\predicateBaseLongtimeoutResultsPredBitvectorsCorrectCpuTimeSumPlain}{127824.350385768\xspace}
\providecommand{\predicateBaseLongtimeoutResultsPredBitvectorsCorrectCpuTimeSumPlainHours}{}
  \renewcommand{\predicateBaseLongtimeoutResultsPredBitvectorsCorrectCpuTimeSumPlainHours}{35.506763996046665\xspace}

  % wall-time-sum
\providecommand{\predicateBaseLongtimeoutResultsPredBitvectorsCorrectWallTimeSumPlain}{}
  \renewcommand{\predicateBaseLongtimeoutResultsPredBitvectorsCorrectWallTimeSumPlain}{96207.8136818381\xspace}
\providecommand{\predicateBaseLongtimeoutResultsPredBitvectorsCorrectWallTimeSumPlainHours}{}
  \renewcommand{\predicateBaseLongtimeoutResultsPredBitvectorsCorrectWallTimeSumPlainHours}{26.724392689399473\xspace}

  % cpu-time-avg
\providecommand{\predicateBaseLongtimeoutResultsPredBitvectorsCorrectCpuTimeAvgPlain}{}
  \renewcommand{\predicateBaseLongtimeoutResultsPredBitvectorsCorrectCpuTimeAvgPlain}{63.21679049741247\xspace}
\providecommand{\predicateBaseLongtimeoutResultsPredBitvectorsCorrectCpuTimeAvgPlainHours}{}
  \renewcommand{\predicateBaseLongtimeoutResultsPredBitvectorsCorrectCpuTimeAvgPlainHours}{0.017560219582614573\xspace}

  % wall-time-avg
\providecommand{\predicateBaseLongtimeoutResultsPredBitvectorsCorrectWallTimeAvgPlain}{}
  \renewcommand{\predicateBaseLongtimeoutResultsPredBitvectorsCorrectWallTimeAvgPlain}{47.5805211087231\xspace}
\providecommand{\predicateBaseLongtimeoutResultsPredBitvectorsCorrectWallTimeAvgPlainHours}{}
  \renewcommand{\predicateBaseLongtimeoutResultsPredBitvectorsCorrectWallTimeAvgPlainHours}{0.01321681141908975\xspace}

  % inv-succ
\providecommand{\predicateBaseLongtimeoutResultsPredBitvectorsCorrectInvSuccPlain}{}
  \renewcommand{\predicateBaseLongtimeoutResultsPredBitvectorsCorrectInvSuccPlain}{0\xspace}

  % inv-tries
\providecommand{\predicateBaseLongtimeoutResultsPredBitvectorsCorrectInvTriesPlain}{}
  \renewcommand{\predicateBaseLongtimeoutResultsPredBitvectorsCorrectInvTriesPlain}{0\xspace}

  % inv-time-sum
\providecommand{\predicateBaseLongtimeoutResultsPredBitvectorsCorrectInvTimeSumPlain}{}
  \renewcommand{\predicateBaseLongtimeoutResultsPredBitvectorsCorrectInvTimeSumPlain}{0.0\xspace}
\providecommand{\predicateBaseLongtimeoutResultsPredBitvectorsCorrectInvTimeSumPlainHours}{}
  \renewcommand{\predicateBaseLongtimeoutResultsPredBitvectorsCorrectInvTimeSumPlainHours}{0.0\xspace}

  % finished-main
\providecommand{\predicateBaseLongtimeoutResultsPredBitvectorsCorrectFinishedMainPlain}{}
  \renewcommand{\predicateBaseLongtimeoutResultsPredBitvectorsCorrectFinishedMainPlain}{2022\xspace}

 %% incorrect %%
\providecommand{\predicateBaseLongtimeoutResultsPredBitvectorsIncorrectPlain}{}
  \renewcommand{\predicateBaseLongtimeoutResultsPredBitvectorsIncorrectPlain}{27\xspace}

  % cpu-time-sum
\providecommand{\predicateBaseLongtimeoutResultsPredBitvectorsIncorrectCpuTimeSumPlain}{}
  \renewcommand{\predicateBaseLongtimeoutResultsPredBitvectorsIncorrectCpuTimeSumPlain}{707.1392141769999\xspace}
\providecommand{\predicateBaseLongtimeoutResultsPredBitvectorsIncorrectCpuTimeSumPlainHours}{}
  \renewcommand{\predicateBaseLongtimeoutResultsPredBitvectorsIncorrectCpuTimeSumPlainHours}{0.19642755949361107\xspace}

  % wall-time-sum
\providecommand{\predicateBaseLongtimeoutResultsPredBitvectorsIncorrectWallTimeSumPlain}{}
  \renewcommand{\predicateBaseLongtimeoutResultsPredBitvectorsIncorrectWallTimeSumPlain}{427.00321269012005\xspace}
\providecommand{\predicateBaseLongtimeoutResultsPredBitvectorsIncorrectWallTimeSumPlainHours}{}
  \renewcommand{\predicateBaseLongtimeoutResultsPredBitvectorsIncorrectWallTimeSumPlainHours}{0.11861200352503334\xspace}

  % cpu-time-avg
\providecommand{\predicateBaseLongtimeoutResultsPredBitvectorsIncorrectCpuTimeAvgPlain}{}
  \renewcommand{\predicateBaseLongtimeoutResultsPredBitvectorsIncorrectCpuTimeAvgPlain}{26.19034126581481\xspace}
\providecommand{\predicateBaseLongtimeoutResultsPredBitvectorsIncorrectCpuTimeAvgPlainHours}{}
  \renewcommand{\predicateBaseLongtimeoutResultsPredBitvectorsIncorrectCpuTimeAvgPlainHours}{0.007275094796059669\xspace}

  % wall-time-avg
\providecommand{\predicateBaseLongtimeoutResultsPredBitvectorsIncorrectWallTimeAvgPlain}{}
  \renewcommand{\predicateBaseLongtimeoutResultsPredBitvectorsIncorrectWallTimeAvgPlain}{15.81493380333778\xspace}
\providecommand{\predicateBaseLongtimeoutResultsPredBitvectorsIncorrectWallTimeAvgPlainHours}{}
  \renewcommand{\predicateBaseLongtimeoutResultsPredBitvectorsIncorrectWallTimeAvgPlainHours}{0.0043930371675938275\xspace}

  % inv-succ
\providecommand{\predicateBaseLongtimeoutResultsPredBitvectorsIncorrectInvSuccPlain}{}
  \renewcommand{\predicateBaseLongtimeoutResultsPredBitvectorsIncorrectInvSuccPlain}{0\xspace}

  % inv-tries
\providecommand{\predicateBaseLongtimeoutResultsPredBitvectorsIncorrectInvTriesPlain}{}
  \renewcommand{\predicateBaseLongtimeoutResultsPredBitvectorsIncorrectInvTriesPlain}{0\xspace}

  % inv-time-sum
\providecommand{\predicateBaseLongtimeoutResultsPredBitvectorsIncorrectInvTimeSumPlain}{}
  \renewcommand{\predicateBaseLongtimeoutResultsPredBitvectorsIncorrectInvTimeSumPlain}{0.0\xspace}
\providecommand{\predicateBaseLongtimeoutResultsPredBitvectorsIncorrectInvTimeSumPlainHours}{}
  \renewcommand{\predicateBaseLongtimeoutResultsPredBitvectorsIncorrectInvTimeSumPlainHours}{0.0\xspace}

  % finished-main
\providecommand{\predicateBaseLongtimeoutResultsPredBitvectorsIncorrectFinishedMainPlain}{}
  \renewcommand{\predicateBaseLongtimeoutResultsPredBitvectorsIncorrectFinishedMainPlain}{27\xspace}

 %% timeout %%
\providecommand{\predicateBaseLongtimeoutResultsPredBitvectorsTimeoutPlain}{}
  \renewcommand{\predicateBaseLongtimeoutResultsPredBitvectorsTimeoutPlain}{1314\xspace}

  % cpu-time-sum
\providecommand{\predicateBaseLongtimeoutResultsPredBitvectorsTimeoutCpuTimeSumPlain}{}
  \renewcommand{\predicateBaseLongtimeoutResultsPredBitvectorsTimeoutCpuTimeSumPlain}{796607.5371394024\xspace}
\providecommand{\predicateBaseLongtimeoutResultsPredBitvectorsTimeoutCpuTimeSumPlainHours}{}
  \renewcommand{\predicateBaseLongtimeoutResultsPredBitvectorsTimeoutCpuTimeSumPlainHours}{221.27987142761177\xspace}

  % wall-time-sum
\providecommand{\predicateBaseLongtimeoutResultsPredBitvectorsTimeoutWallTimeSumPlain}{}
  \renewcommand{\predicateBaseLongtimeoutResultsPredBitvectorsTimeoutWallTimeSumPlain}{752287.8574705243\xspace}
\providecommand{\predicateBaseLongtimeoutResultsPredBitvectorsTimeoutWallTimeSumPlainHours}{}
  \renewcommand{\predicateBaseLongtimeoutResultsPredBitvectorsTimeoutWallTimeSumPlainHours}{208.96884929736785\xspace}

  % cpu-time-avg
\providecommand{\predicateBaseLongtimeoutResultsPredBitvectorsTimeoutCpuTimeAvgPlain}{}
  \renewcommand{\predicateBaseLongtimeoutResultsPredBitvectorsTimeoutCpuTimeAvgPlain}{606.2462230893473\xspace}
\providecommand{\predicateBaseLongtimeoutResultsPredBitvectorsTimeoutCpuTimeAvgPlainHours}{}
  \renewcommand{\predicateBaseLongtimeoutResultsPredBitvectorsTimeoutCpuTimeAvgPlainHours}{0.1684017286359298\xspace}

  % wall-time-avg
\providecommand{\predicateBaseLongtimeoutResultsPredBitvectorsTimeoutWallTimeAvgPlain}{}
  \renewcommand{\predicateBaseLongtimeoutResultsPredBitvectorsTimeoutWallTimeAvgPlain}{572.5173953352544\xspace}
\providecommand{\predicateBaseLongtimeoutResultsPredBitvectorsTimeoutWallTimeAvgPlainHours}{}
  \renewcommand{\predicateBaseLongtimeoutResultsPredBitvectorsTimeoutWallTimeAvgPlainHours}{0.15903260981534845\xspace}

  % inv-succ
\providecommand{\predicateBaseLongtimeoutResultsPredBitvectorsTimeoutInvSuccPlain}{}
  \renewcommand{\predicateBaseLongtimeoutResultsPredBitvectorsTimeoutInvSuccPlain}{0\xspace}

  % inv-tries
\providecommand{\predicateBaseLongtimeoutResultsPredBitvectorsTimeoutInvTriesPlain}{}
  \renewcommand{\predicateBaseLongtimeoutResultsPredBitvectorsTimeoutInvTriesPlain}{0\xspace}

  % inv-time-sum
\providecommand{\predicateBaseLongtimeoutResultsPredBitvectorsTimeoutInvTimeSumPlain}{}
  \renewcommand{\predicateBaseLongtimeoutResultsPredBitvectorsTimeoutInvTimeSumPlain}{0.0\xspace}
\providecommand{\predicateBaseLongtimeoutResultsPredBitvectorsTimeoutInvTimeSumPlainHours}{}
  \renewcommand{\predicateBaseLongtimeoutResultsPredBitvectorsTimeoutInvTimeSumPlainHours}{0.0\xspace}

  % finished-main
\providecommand{\predicateBaseLongtimeoutResultsPredBitvectorsTimeoutFinishedMainPlain}{}
  \renewcommand{\predicateBaseLongtimeoutResultsPredBitvectorsTimeoutFinishedMainPlain}{1314\xspace}

 %% unknown-or-category-error %%
\providecommand{\predicateBaseLongtimeoutResultsPredBitvectorsUnknownOrCategoryErrorPlain}{}
  \renewcommand{\predicateBaseLongtimeoutResultsPredBitvectorsUnknownOrCategoryErrorPlain}{1439\xspace}

  % cpu-time-sum
\providecommand{\predicateBaseLongtimeoutResultsPredBitvectorsUnknownOrCategoryErrorCpuTimeSumPlain}{}
  \renewcommand{\predicateBaseLongtimeoutResultsPredBitvectorsUnknownOrCategoryErrorCpuTimeSumPlain}{815470.6349393608\xspace}
\providecommand{\predicateBaseLongtimeoutResultsPredBitvectorsUnknownOrCategoryErrorCpuTimeSumPlainHours}{}
  \renewcommand{\predicateBaseLongtimeoutResultsPredBitvectorsUnknownOrCategoryErrorCpuTimeSumPlainHours}{226.5196208164891\xspace}

  % wall-time-sum
\providecommand{\predicateBaseLongtimeoutResultsPredBitvectorsUnknownOrCategoryErrorWallTimeSumPlain}{}
  \renewcommand{\predicateBaseLongtimeoutResultsPredBitvectorsUnknownOrCategoryErrorWallTimeSumPlain}{767569.8588805315\xspace}
\providecommand{\predicateBaseLongtimeoutResultsPredBitvectorsUnknownOrCategoryErrorWallTimeSumPlainHours}{}
  \renewcommand{\predicateBaseLongtimeoutResultsPredBitvectorsUnknownOrCategoryErrorWallTimeSumPlainHours}{213.21384968903652\xspace}

  % cpu-time-avg
\providecommand{\predicateBaseLongtimeoutResultsPredBitvectorsUnknownOrCategoryErrorCpuTimeAvgPlain}{}
  \renewcommand{\predicateBaseLongtimeoutResultsPredBitvectorsUnknownOrCategoryErrorCpuTimeAvgPlain}{566.6925885610568\xspace}
\providecommand{\predicateBaseLongtimeoutResultsPredBitvectorsUnknownOrCategoryErrorCpuTimeAvgPlainHours}{}
  \renewcommand{\predicateBaseLongtimeoutResultsPredBitvectorsUnknownOrCategoryErrorCpuTimeAvgPlainHours}{0.15741460793362688\xspace}

  % wall-time-avg
\providecommand{\predicateBaseLongtimeoutResultsPredBitvectorsUnknownOrCategoryErrorWallTimeAvgPlain}{}
  \renewcommand{\predicateBaseLongtimeoutResultsPredBitvectorsUnknownOrCategoryErrorWallTimeAvgPlain}{533.4050443923082\xspace}
\providecommand{\predicateBaseLongtimeoutResultsPredBitvectorsUnknownOrCategoryErrorWallTimeAvgPlainHours}{}
  \renewcommand{\predicateBaseLongtimeoutResultsPredBitvectorsUnknownOrCategoryErrorWallTimeAvgPlainHours}{0.1481680678867523\xspace}

  % inv-succ
\providecommand{\predicateBaseLongtimeoutResultsPredBitvectorsUnknownOrCategoryErrorInvSuccPlain}{}
  \renewcommand{\predicateBaseLongtimeoutResultsPredBitvectorsUnknownOrCategoryErrorInvSuccPlain}{0\xspace}

  % inv-tries
\providecommand{\predicateBaseLongtimeoutResultsPredBitvectorsUnknownOrCategoryErrorInvTriesPlain}{}
  \renewcommand{\predicateBaseLongtimeoutResultsPredBitvectorsUnknownOrCategoryErrorInvTriesPlain}{0\xspace}

  % inv-time-sum
\providecommand{\predicateBaseLongtimeoutResultsPredBitvectorsUnknownOrCategoryErrorInvTimeSumPlain}{}
  \renewcommand{\predicateBaseLongtimeoutResultsPredBitvectorsUnknownOrCategoryErrorInvTimeSumPlain}{0.0\xspace}
\providecommand{\predicateBaseLongtimeoutResultsPredBitvectorsUnknownOrCategoryErrorInvTimeSumPlainHours}{}
  \renewcommand{\predicateBaseLongtimeoutResultsPredBitvectorsUnknownOrCategoryErrorInvTimeSumPlainHours}{0.0\xspace}

  % finished-main
\providecommand{\predicateBaseLongtimeoutResultsPredBitvectorsUnknownOrCategoryErrorFinishedMainPlain}{}
  \renewcommand{\predicateBaseLongtimeoutResultsPredBitvectorsUnknownOrCategoryErrorFinishedMainPlain}{1439\xspace}

 %% correct-false %%
\providecommand{\predicateBaseLongtimeoutResultsPredBitvectorsCorrectFalsePlain}{}
  \renewcommand{\predicateBaseLongtimeoutResultsPredBitvectorsCorrectFalsePlain}{588\xspace}

  % cpu-time-sum
\providecommand{\predicateBaseLongtimeoutResultsPredBitvectorsCorrectFalseCpuTimeSumPlain}{}
  \renewcommand{\predicateBaseLongtimeoutResultsPredBitvectorsCorrectFalseCpuTimeSumPlain}{53552.234097685985\xspace}
\providecommand{\predicateBaseLongtimeoutResultsPredBitvectorsCorrectFalseCpuTimeSumPlainHours}{}
  \renewcommand{\predicateBaseLongtimeoutResultsPredBitvectorsCorrectFalseCpuTimeSumPlainHours}{14.875620582690551\xspace}

  % wall-time-sum
\providecommand{\predicateBaseLongtimeoutResultsPredBitvectorsCorrectFalseWallTimeSumPlain}{}
  \renewcommand{\predicateBaseLongtimeoutResultsPredBitvectorsCorrectFalseWallTimeSumPlain}{42274.32463979353\xspace}
\providecommand{\predicateBaseLongtimeoutResultsPredBitvectorsCorrectFalseWallTimeSumPlainHours}{}
  \renewcommand{\predicateBaseLongtimeoutResultsPredBitvectorsCorrectFalseWallTimeSumPlainHours}{11.742867955498202\xspace}

  % cpu-time-avg
\providecommand{\predicateBaseLongtimeoutResultsPredBitvectorsCorrectFalseCpuTimeAvgPlain}{}
  \renewcommand{\predicateBaseLongtimeoutResultsPredBitvectorsCorrectFalseCpuTimeAvgPlain}{91.0752280572891\xspace}
\providecommand{\predicateBaseLongtimeoutResultsPredBitvectorsCorrectFalseCpuTimeAvgPlainHours}{}
  \renewcommand{\predicateBaseLongtimeoutResultsPredBitvectorsCorrectFalseCpuTimeAvgPlainHours}{0.02529867446035808\xspace}

  % wall-time-avg
\providecommand{\predicateBaseLongtimeoutResultsPredBitvectorsCorrectFalseWallTimeAvgPlain}{}
  \renewcommand{\predicateBaseLongtimeoutResultsPredBitvectorsCorrectFalseWallTimeAvgPlain}{71.89510993162165\xspace}
\providecommand{\predicateBaseLongtimeoutResultsPredBitvectorsCorrectFalseWallTimeAvgPlainHours}{}
  \renewcommand{\predicateBaseLongtimeoutResultsPredBitvectorsCorrectFalseWallTimeAvgPlainHours}{0.0199708638698949\xspace}

  % inv-succ
\providecommand{\predicateBaseLongtimeoutResultsPredBitvectorsCorrectFalseInvSuccPlain}{}
  \renewcommand{\predicateBaseLongtimeoutResultsPredBitvectorsCorrectFalseInvSuccPlain}{0\xspace}

  % inv-tries
\providecommand{\predicateBaseLongtimeoutResultsPredBitvectorsCorrectFalseInvTriesPlain}{}
  \renewcommand{\predicateBaseLongtimeoutResultsPredBitvectorsCorrectFalseInvTriesPlain}{0\xspace}

  % inv-time-sum
\providecommand{\predicateBaseLongtimeoutResultsPredBitvectorsCorrectFalseInvTimeSumPlain}{}
  \renewcommand{\predicateBaseLongtimeoutResultsPredBitvectorsCorrectFalseInvTimeSumPlain}{0.0\xspace}
\providecommand{\predicateBaseLongtimeoutResultsPredBitvectorsCorrectFalseInvTimeSumPlainHours}{}
  \renewcommand{\predicateBaseLongtimeoutResultsPredBitvectorsCorrectFalseInvTimeSumPlainHours}{0.0\xspace}

  % finished-main
\providecommand{\predicateBaseLongtimeoutResultsPredBitvectorsCorrectFalseFinishedMainPlain}{}
  \renewcommand{\predicateBaseLongtimeoutResultsPredBitvectorsCorrectFalseFinishedMainPlain}{588\xspace}

 %% correct-true %%
\providecommand{\predicateBaseLongtimeoutResultsPredBitvectorsCorrectTruePlain}{}
  \renewcommand{\predicateBaseLongtimeoutResultsPredBitvectorsCorrectTruePlain}{1434\xspace}

  % cpu-time-sum
\providecommand{\predicateBaseLongtimeoutResultsPredBitvectorsCorrectTrueCpuTimeSumPlain}{}
  \renewcommand{\predicateBaseLongtimeoutResultsPredBitvectorsCorrectTrueCpuTimeSumPlain}{74272.11628808215\xspace}
\providecommand{\predicateBaseLongtimeoutResultsPredBitvectorsCorrectTrueCpuTimeSumPlainHours}{}
  \renewcommand{\predicateBaseLongtimeoutResultsPredBitvectorsCorrectTrueCpuTimeSumPlainHours}{20.631143413356153\xspace}

  % wall-time-sum
\providecommand{\predicateBaseLongtimeoutResultsPredBitvectorsCorrectTrueWallTimeSumPlain}{}
  \renewcommand{\predicateBaseLongtimeoutResultsPredBitvectorsCorrectTrueWallTimeSumPlain}{53933.489042044544\xspace}
\providecommand{\predicateBaseLongtimeoutResultsPredBitvectorsCorrectTrueWallTimeSumPlainHours}{}
  \renewcommand{\predicateBaseLongtimeoutResultsPredBitvectorsCorrectTrueWallTimeSumPlainHours}{14.981524733901262\xspace}

  % cpu-time-avg
\providecommand{\predicateBaseLongtimeoutResultsPredBitvectorsCorrectTrueCpuTimeAvgPlain}{}
  \renewcommand{\predicateBaseLongtimeoutResultsPredBitvectorsCorrectTrueCpuTimeAvgPlain}{51.79366547286063\xspace}
\providecommand{\predicateBaseLongtimeoutResultsPredBitvectorsCorrectTrueCpuTimeAvgPlainHours}{}
  \renewcommand{\predicateBaseLongtimeoutResultsPredBitvectorsCorrectTrueCpuTimeAvgPlainHours}{0.014387129298016842\xspace}

  % wall-time-avg
\providecommand{\predicateBaseLongtimeoutResultsPredBitvectorsCorrectTrueWallTimeAvgPlain}{}
  \renewcommand{\predicateBaseLongtimeoutResultsPredBitvectorsCorrectTrueWallTimeAvgPlain}{37.610522344521996\xspace}
\providecommand{\predicateBaseLongtimeoutResultsPredBitvectorsCorrectTrueWallTimeAvgPlainHours}{}
  \renewcommand{\predicateBaseLongtimeoutResultsPredBitvectorsCorrectTrueWallTimeAvgPlainHours}{0.010447367317922777\xspace}

  % inv-succ
\providecommand{\predicateBaseLongtimeoutResultsPredBitvectorsCorrectTrueInvSuccPlain}{}
  \renewcommand{\predicateBaseLongtimeoutResultsPredBitvectorsCorrectTrueInvSuccPlain}{0\xspace}

  % inv-tries
\providecommand{\predicateBaseLongtimeoutResultsPredBitvectorsCorrectTrueInvTriesPlain}{}
  \renewcommand{\predicateBaseLongtimeoutResultsPredBitvectorsCorrectTrueInvTriesPlain}{0\xspace}

  % inv-time-sum
\providecommand{\predicateBaseLongtimeoutResultsPredBitvectorsCorrectTrueInvTimeSumPlain}{}
  \renewcommand{\predicateBaseLongtimeoutResultsPredBitvectorsCorrectTrueInvTimeSumPlain}{0.0\xspace}
\providecommand{\predicateBaseLongtimeoutResultsPredBitvectorsCorrectTrueInvTimeSumPlainHours}{}
  \renewcommand{\predicateBaseLongtimeoutResultsPredBitvectorsCorrectTrueInvTimeSumPlainHours}{0.0\xspace}

  % finished-main
\providecommand{\predicateBaseLongtimeoutResultsPredBitvectorsCorrectTrueFinishedMainPlain}{}
  \renewcommand{\predicateBaseLongtimeoutResultsPredBitvectorsCorrectTrueFinishedMainPlain}{1434\xspace}

 %% incorrect-false %%
\providecommand{\predicateBaseLongtimeoutResultsPredBitvectorsIncorrectFalsePlain}{}
  \renewcommand{\predicateBaseLongtimeoutResultsPredBitvectorsIncorrectFalsePlain}{27\xspace}

  % cpu-time-sum
\providecommand{\predicateBaseLongtimeoutResultsPredBitvectorsIncorrectFalseCpuTimeSumPlain}{}
  \renewcommand{\predicateBaseLongtimeoutResultsPredBitvectorsIncorrectFalseCpuTimeSumPlain}{707.1392141769999\xspace}
\providecommand{\predicateBaseLongtimeoutResultsPredBitvectorsIncorrectFalseCpuTimeSumPlainHours}{}
  \renewcommand{\predicateBaseLongtimeoutResultsPredBitvectorsIncorrectFalseCpuTimeSumPlainHours}{0.19642755949361107\xspace}

  % wall-time-sum
\providecommand{\predicateBaseLongtimeoutResultsPredBitvectorsIncorrectFalseWallTimeSumPlain}{}
  \renewcommand{\predicateBaseLongtimeoutResultsPredBitvectorsIncorrectFalseWallTimeSumPlain}{427.00321269012005\xspace}
\providecommand{\predicateBaseLongtimeoutResultsPredBitvectorsIncorrectFalseWallTimeSumPlainHours}{}
  \renewcommand{\predicateBaseLongtimeoutResultsPredBitvectorsIncorrectFalseWallTimeSumPlainHours}{0.11861200352503334\xspace}

  % cpu-time-avg
\providecommand{\predicateBaseLongtimeoutResultsPredBitvectorsIncorrectFalseCpuTimeAvgPlain}{}
  \renewcommand{\predicateBaseLongtimeoutResultsPredBitvectorsIncorrectFalseCpuTimeAvgPlain}{26.19034126581481\xspace}
\providecommand{\predicateBaseLongtimeoutResultsPredBitvectorsIncorrectFalseCpuTimeAvgPlainHours}{}
  \renewcommand{\predicateBaseLongtimeoutResultsPredBitvectorsIncorrectFalseCpuTimeAvgPlainHours}{0.007275094796059669\xspace}

  % wall-time-avg
\providecommand{\predicateBaseLongtimeoutResultsPredBitvectorsIncorrectFalseWallTimeAvgPlain}{}
  \renewcommand{\predicateBaseLongtimeoutResultsPredBitvectorsIncorrectFalseWallTimeAvgPlain}{15.81493380333778\xspace}
\providecommand{\predicateBaseLongtimeoutResultsPredBitvectorsIncorrectFalseWallTimeAvgPlainHours}{}
  \renewcommand{\predicateBaseLongtimeoutResultsPredBitvectorsIncorrectFalseWallTimeAvgPlainHours}{0.0043930371675938275\xspace}

  % inv-succ
\providecommand{\predicateBaseLongtimeoutResultsPredBitvectorsIncorrectFalseInvSuccPlain}{}
  \renewcommand{\predicateBaseLongtimeoutResultsPredBitvectorsIncorrectFalseInvSuccPlain}{0\xspace}

  % inv-tries
\providecommand{\predicateBaseLongtimeoutResultsPredBitvectorsIncorrectFalseInvTriesPlain}{}
  \renewcommand{\predicateBaseLongtimeoutResultsPredBitvectorsIncorrectFalseInvTriesPlain}{0\xspace}

  % inv-time-sum
\providecommand{\predicateBaseLongtimeoutResultsPredBitvectorsIncorrectFalseInvTimeSumPlain}{}
  \renewcommand{\predicateBaseLongtimeoutResultsPredBitvectorsIncorrectFalseInvTimeSumPlain}{0.0\xspace}
\providecommand{\predicateBaseLongtimeoutResultsPredBitvectorsIncorrectFalseInvTimeSumPlainHours}{}
  \renewcommand{\predicateBaseLongtimeoutResultsPredBitvectorsIncorrectFalseInvTimeSumPlainHours}{0.0\xspace}

  % finished-main
\providecommand{\predicateBaseLongtimeoutResultsPredBitvectorsIncorrectFalseFinishedMainPlain}{}
  \renewcommand{\predicateBaseLongtimeoutResultsPredBitvectorsIncorrectFalseFinishedMainPlain}{27\xspace}

 %% incorrect-true %%
\providecommand{\predicateBaseLongtimeoutResultsPredBitvectorsIncorrectTruePlain}{}
  \renewcommand{\predicateBaseLongtimeoutResultsPredBitvectorsIncorrectTruePlain}{0\xspace}

  % cpu-time-sum
\providecommand{\predicateBaseLongtimeoutResultsPredBitvectorsIncorrectTrueCpuTimeSumPlain}{}
  \renewcommand{\predicateBaseLongtimeoutResultsPredBitvectorsIncorrectTrueCpuTimeSumPlain}{0.0\xspace}
\providecommand{\predicateBaseLongtimeoutResultsPredBitvectorsIncorrectTrueCpuTimeSumPlainHours}{}
  \renewcommand{\predicateBaseLongtimeoutResultsPredBitvectorsIncorrectTrueCpuTimeSumPlainHours}{0.0\xspace}

  % wall-time-sum
\providecommand{\predicateBaseLongtimeoutResultsPredBitvectorsIncorrectTrueWallTimeSumPlain}{}
  \renewcommand{\predicateBaseLongtimeoutResultsPredBitvectorsIncorrectTrueWallTimeSumPlain}{0.0\xspace}
\providecommand{\predicateBaseLongtimeoutResultsPredBitvectorsIncorrectTrueWallTimeSumPlainHours}{}
  \renewcommand{\predicateBaseLongtimeoutResultsPredBitvectorsIncorrectTrueWallTimeSumPlainHours}{0.0\xspace}

  % cpu-time-avg
\providecommand{\predicateBaseLongtimeoutResultsPredBitvectorsIncorrectTrueCpuTimeAvgPlain}{}
  \renewcommand{\predicateBaseLongtimeoutResultsPredBitvectorsIncorrectTrueCpuTimeAvgPlain}{NaN\xspace}
\providecommand{\predicateBaseLongtimeoutResultsPredBitvectorsIncorrectTrueCpuTimeAvgPlainHours}{}
  \renewcommand{\predicateBaseLongtimeoutResultsPredBitvectorsIncorrectTrueCpuTimeAvgPlainHours}{NaN\xspace}

  % wall-time-avg
\providecommand{\predicateBaseLongtimeoutResultsPredBitvectorsIncorrectTrueWallTimeAvgPlain}{}
  \renewcommand{\predicateBaseLongtimeoutResultsPredBitvectorsIncorrectTrueWallTimeAvgPlain}{NaN\xspace}
\providecommand{\predicateBaseLongtimeoutResultsPredBitvectorsIncorrectTrueWallTimeAvgPlainHours}{}
  \renewcommand{\predicateBaseLongtimeoutResultsPredBitvectorsIncorrectTrueWallTimeAvgPlainHours}{NaN\xspace}

  % inv-succ
\providecommand{\predicateBaseLongtimeoutResultsPredBitvectorsIncorrectTrueInvSuccPlain}{}
  \renewcommand{\predicateBaseLongtimeoutResultsPredBitvectorsIncorrectTrueInvSuccPlain}{0\xspace}

  % inv-tries
\providecommand{\predicateBaseLongtimeoutResultsPredBitvectorsIncorrectTrueInvTriesPlain}{}
  \renewcommand{\predicateBaseLongtimeoutResultsPredBitvectorsIncorrectTrueInvTriesPlain}{0\xspace}

  % inv-time-sum
\providecommand{\predicateBaseLongtimeoutResultsPredBitvectorsIncorrectTrueInvTimeSumPlain}{}
  \renewcommand{\predicateBaseLongtimeoutResultsPredBitvectorsIncorrectTrueInvTimeSumPlain}{0.0\xspace}
\providecommand{\predicateBaseLongtimeoutResultsPredBitvectorsIncorrectTrueInvTimeSumPlainHours}{}
  \renewcommand{\predicateBaseLongtimeoutResultsPredBitvectorsIncorrectTrueInvTimeSumPlainHours}{0.0\xspace}

  % finished-main
\providecommand{\predicateBaseLongtimeoutResultsPredBitvectorsIncorrectTrueFinishedMainPlain}{}
  \renewcommand{\predicateBaseLongtimeoutResultsPredBitvectorsIncorrectTrueFinishedMainPlain}{0\xspace}

 %% all %%
\providecommand{\predicateBaseLongtimeoutResultsPredBitvectorsAllPlain}{}
  \renewcommand{\predicateBaseLongtimeoutResultsPredBitvectorsAllPlain}{3488\xspace}

  % cpu-time-sum
\providecommand{\predicateBaseLongtimeoutResultsPredBitvectorsAllCpuTimeSumPlain}{}
  \renewcommand{\predicateBaseLongtimeoutResultsPredBitvectorsAllCpuTimeSumPlain}{944002.1245393046\xspace}
\providecommand{\predicateBaseLongtimeoutResultsPredBitvectorsAllCpuTimeSumPlainHours}{}
  \renewcommand{\predicateBaseLongtimeoutResultsPredBitvectorsAllCpuTimeSumPlainHours}{262.22281237202907\xspace}

  % wall-time-sum
\providecommand{\predicateBaseLongtimeoutResultsPredBitvectorsAllWallTimeSumPlain}{}
  \renewcommand{\predicateBaseLongtimeoutResultsPredBitvectorsAllWallTimeSumPlain}{864204.6757750589\xspace}
\providecommand{\predicateBaseLongtimeoutResultsPredBitvectorsAllWallTimeSumPlainHours}{}
  \renewcommand{\predicateBaseLongtimeoutResultsPredBitvectorsAllWallTimeSumPlainHours}{240.05685438196082\xspace}

  % cpu-time-avg
\providecommand{\predicateBaseLongtimeoutResultsPredBitvectorsAllCpuTimeAvgPlain}{}
  \renewcommand{\predicateBaseLongtimeoutResultsPredBitvectorsAllCpuTimeAvgPlain}{270.64281093443367\xspace}
\providecommand{\predicateBaseLongtimeoutResultsPredBitvectorsAllCpuTimeAvgPlainHours}{}
  \renewcommand{\predicateBaseLongtimeoutResultsPredBitvectorsAllCpuTimeAvgPlainHours}{0.07517855859289824\xspace}

  % wall-time-avg
\providecommand{\predicateBaseLongtimeoutResultsPredBitvectorsAllWallTimeAvgPlain}{}
  \renewcommand{\predicateBaseLongtimeoutResultsPredBitvectorsAllWallTimeAvgPlain}{247.7651019997302\xspace}
\providecommand{\predicateBaseLongtimeoutResultsPredBitvectorsAllWallTimeAvgPlainHours}{}
  \renewcommand{\predicateBaseLongtimeoutResultsPredBitvectorsAllWallTimeAvgPlainHours}{0.0688236394443695\xspace}

  % inv-succ
\providecommand{\predicateBaseLongtimeoutResultsPredBitvectorsAllInvSuccPlain}{}
  \renewcommand{\predicateBaseLongtimeoutResultsPredBitvectorsAllInvSuccPlain}{0\xspace}

  % inv-tries
\providecommand{\predicateBaseLongtimeoutResultsPredBitvectorsAllInvTriesPlain}{}
  \renewcommand{\predicateBaseLongtimeoutResultsPredBitvectorsAllInvTriesPlain}{0\xspace}

  % inv-time-sum
\providecommand{\predicateBaseLongtimeoutResultsPredBitvectorsAllInvTimeSumPlain}{}
  \renewcommand{\predicateBaseLongtimeoutResultsPredBitvectorsAllInvTimeSumPlain}{0.0\xspace}
\providecommand{\predicateBaseLongtimeoutResultsPredBitvectorsAllInvTimeSumPlainHours}{}
  \renewcommand{\predicateBaseLongtimeoutResultsPredBitvectorsAllInvTimeSumPlainHours}{0.0\xspace}

  % finished-main
\providecommand{\predicateBaseLongtimeoutResultsPredBitvectorsAllFinishedMainPlain}{}
  \renewcommand{\predicateBaseLongtimeoutResultsPredBitvectorsAllFinishedMainPlain}{3488\xspace}

 %% equal-only %%
\providecommand{\predicateBaseLongtimeoutResultsPredBitvectorsEqualOnlyPlain}{}
  \renewcommand{\predicateBaseLongtimeoutResultsPredBitvectorsEqualOnlyPlain}{1865\xspace}

  % cpu-time-sum
\providecommand{\predicateBaseLongtimeoutResultsPredBitvectorsEqualOnlyCpuTimeSumPlain}{}
  \renewcommand{\predicateBaseLongtimeoutResultsPredBitvectorsEqualOnlyCpuTimeSumPlain}{75860.71799421417\xspace}
\providecommand{\predicateBaseLongtimeoutResultsPredBitvectorsEqualOnlyCpuTimeSumPlainHours}{}
  \renewcommand{\predicateBaseLongtimeoutResultsPredBitvectorsEqualOnlyCpuTimeSumPlainHours}{21.07242166505949\xspace}

  % wall-time-sum
\providecommand{\predicateBaseLongtimeoutResultsPredBitvectorsEqualOnlyWallTimeSumPlain}{}
  \renewcommand{\predicateBaseLongtimeoutResultsPredBitvectorsEqualOnlyWallTimeSumPlain}{50162.51113866954\xspace}
\providecommand{\predicateBaseLongtimeoutResultsPredBitvectorsEqualOnlyWallTimeSumPlainHours}{}
  \renewcommand{\predicateBaseLongtimeoutResultsPredBitvectorsEqualOnlyWallTimeSumPlainHours}{13.93403087185265\xspace}

  % cpu-time-avg
\providecommand{\predicateBaseLongtimeoutResultsPredBitvectorsEqualOnlyCpuTimeAvgPlain}{}
  \renewcommand{\predicateBaseLongtimeoutResultsPredBitvectorsEqualOnlyCpuTimeAvgPlain}{40.67598820065103\xspace}
\providecommand{\predicateBaseLongtimeoutResultsPredBitvectorsEqualOnlyCpuTimeAvgPlainHours}{}
  \renewcommand{\predicateBaseLongtimeoutResultsPredBitvectorsEqualOnlyCpuTimeAvgPlainHours}{0.011298885611291953\xspace}

  % wall-time-avg
\providecommand{\predicateBaseLongtimeoutResultsPredBitvectorsEqualOnlyWallTimeAvgPlain}{}
  \renewcommand{\predicateBaseLongtimeoutResultsPredBitvectorsEqualOnlyWallTimeAvgPlain}{26.896788814300024\xspace}
\providecommand{\predicateBaseLongtimeoutResultsPredBitvectorsEqualOnlyWallTimeAvgPlainHours}{}
  \renewcommand{\predicateBaseLongtimeoutResultsPredBitvectorsEqualOnlyWallTimeAvgPlainHours}{0.007471330226194451\xspace}

  % inv-succ
\providecommand{\predicateBaseLongtimeoutResultsPredBitvectorsEqualOnlyInvSuccPlain}{}
  \renewcommand{\predicateBaseLongtimeoutResultsPredBitvectorsEqualOnlyInvSuccPlain}{0\xspace}

  % inv-tries
\providecommand{\predicateBaseLongtimeoutResultsPredBitvectorsEqualOnlyInvTriesPlain}{}
  \renewcommand{\predicateBaseLongtimeoutResultsPredBitvectorsEqualOnlyInvTriesPlain}{0\xspace}

  % inv-time-sum
\providecommand{\predicateBaseLongtimeoutResultsPredBitvectorsEqualOnlyInvTimeSumPlain}{}
  \renewcommand{\predicateBaseLongtimeoutResultsPredBitvectorsEqualOnlyInvTimeSumPlain}{0.0\xspace}
\providecommand{\predicateBaseLongtimeoutResultsPredBitvectorsEqualOnlyInvTimeSumPlainHours}{}
  \renewcommand{\predicateBaseLongtimeoutResultsPredBitvectorsEqualOnlyInvTimeSumPlainHours}{0.0\xspace}

  % finished-main
\providecommand{\predicateBaseLongtimeoutResultsPredBitvectorsEqualOnlyFinishedMainPlain}{}
  \renewcommand{\predicateBaseLongtimeoutResultsPredBitvectorsEqualOnlyFinishedMainPlain}{1865\xspace}

%%% predicate_bitprecise_parallel_invariants.2016-09-05_0219.results.async-invariants-prec %%%
 %% correct %%
\providecommand{\predicateBitpreciseParallelInvariantsResultsAsyncInvariantsPrecCorrectPlain}{}
  \renewcommand{\predicateBitpreciseParallelInvariantsResultsAsyncInvariantsPrecCorrectPlain}{2075\xspace}

  % cpu-time-sum
\providecommand{\predicateBitpreciseParallelInvariantsResultsAsyncInvariantsPrecCorrectCpuTimeSumPlain}{}
  \renewcommand{\predicateBitpreciseParallelInvariantsResultsAsyncInvariantsPrecCorrectCpuTimeSumPlain}{190739.40533795083\xspace}
\providecommand{\predicateBitpreciseParallelInvariantsResultsAsyncInvariantsPrecCorrectCpuTimeSumPlainHours}{}
  \renewcommand{\predicateBitpreciseParallelInvariantsResultsAsyncInvariantsPrecCorrectCpuTimeSumPlainHours}{52.98316814943078\xspace}

  % wall-time-sum
\providecommand{\predicateBitpreciseParallelInvariantsResultsAsyncInvariantsPrecCorrectWallTimeSumPlain}{}
  \renewcommand{\predicateBitpreciseParallelInvariantsResultsAsyncInvariantsPrecCorrectWallTimeSumPlain}{79168.96180891515\xspace}
\providecommand{\predicateBitpreciseParallelInvariantsResultsAsyncInvariantsPrecCorrectWallTimeSumPlainHours}{}
  \renewcommand{\predicateBitpreciseParallelInvariantsResultsAsyncInvariantsPrecCorrectWallTimeSumPlainHours}{21.991378280254207\xspace}

  % cpu-time-avg
\providecommand{\predicateBitpreciseParallelInvariantsResultsAsyncInvariantsPrecCorrectCpuTimeAvgPlain}{}
  \renewcommand{\predicateBitpreciseParallelInvariantsResultsAsyncInvariantsPrecCorrectCpuTimeAvgPlain}{91.92260498214497\xspace}
\providecommand{\predicateBitpreciseParallelInvariantsResultsAsyncInvariantsPrecCorrectCpuTimeAvgPlainHours}{}
  \renewcommand{\predicateBitpreciseParallelInvariantsResultsAsyncInvariantsPrecCorrectCpuTimeAvgPlainHours}{0.025534056939484715\xspace}

  % wall-time-avg
\providecommand{\predicateBitpreciseParallelInvariantsResultsAsyncInvariantsPrecCorrectWallTimeAvgPlain}{}
  \renewcommand{\predicateBitpreciseParallelInvariantsResultsAsyncInvariantsPrecCorrectWallTimeAvgPlain}{38.15371653441694\xspace}
\providecommand{\predicateBitpreciseParallelInvariantsResultsAsyncInvariantsPrecCorrectWallTimeAvgPlainHours}{}
  \renewcommand{\predicateBitpreciseParallelInvariantsResultsAsyncInvariantsPrecCorrectWallTimeAvgPlainHours}{0.010598254592893593\xspace}

  % inv-succ
\providecommand{\predicateBitpreciseParallelInvariantsResultsAsyncInvariantsPrecCorrectInvSuccPlain}{}
  \renewcommand{\predicateBitpreciseParallelInvariantsResultsAsyncInvariantsPrecCorrectInvSuccPlain}{0\xspace}

  % inv-tries
\providecommand{\predicateBitpreciseParallelInvariantsResultsAsyncInvariantsPrecCorrectInvTriesPlain}{}
  \renewcommand{\predicateBitpreciseParallelInvariantsResultsAsyncInvariantsPrecCorrectInvTriesPlain}{0\xspace}

  % inv-time-sum
\providecommand{\predicateBitpreciseParallelInvariantsResultsAsyncInvariantsPrecCorrectInvTimeSumPlain}{}
  \renewcommand{\predicateBitpreciseParallelInvariantsResultsAsyncInvariantsPrecCorrectInvTimeSumPlain}{0.0\xspace}
\providecommand{\predicateBitpreciseParallelInvariantsResultsAsyncInvariantsPrecCorrectInvTimeSumPlainHours}{}
  \renewcommand{\predicateBitpreciseParallelInvariantsResultsAsyncInvariantsPrecCorrectInvTimeSumPlainHours}{0.0\xspace}

  % finished-main
\providecommand{\predicateBitpreciseParallelInvariantsResultsAsyncInvariantsPrecCorrectFinishedMainPlain}{}
  \renewcommand{\predicateBitpreciseParallelInvariantsResultsAsyncInvariantsPrecCorrectFinishedMainPlain}{1108\xspace}

 %% incorrect %%
\providecommand{\predicateBitpreciseParallelInvariantsResultsAsyncInvariantsPrecIncorrectPlain}{}
  \renewcommand{\predicateBitpreciseParallelInvariantsResultsAsyncInvariantsPrecIncorrectPlain}{18\xspace}

  % cpu-time-sum
\providecommand{\predicateBitpreciseParallelInvariantsResultsAsyncInvariantsPrecIncorrectCpuTimeSumPlain}{}
  \renewcommand{\predicateBitpreciseParallelInvariantsResultsAsyncInvariantsPrecIncorrectCpuTimeSumPlain}{1303.0098760540002\xspace}
\providecommand{\predicateBitpreciseParallelInvariantsResultsAsyncInvariantsPrecIncorrectCpuTimeSumPlainHours}{}
  \renewcommand{\predicateBitpreciseParallelInvariantsResultsAsyncInvariantsPrecIncorrectCpuTimeSumPlainHours}{0.36194718779277785\xspace}

  % wall-time-sum
\providecommand{\predicateBitpreciseParallelInvariantsResultsAsyncInvariantsPrecIncorrectWallTimeSumPlain}{}
  \renewcommand{\predicateBitpreciseParallelInvariantsResultsAsyncInvariantsPrecIncorrectWallTimeSumPlain}{680.3062491414299\xspace}
\providecommand{\predicateBitpreciseParallelInvariantsResultsAsyncInvariantsPrecIncorrectWallTimeSumPlainHours}{}
  \renewcommand{\predicateBitpreciseParallelInvariantsResultsAsyncInvariantsPrecIncorrectWallTimeSumPlainHours}{0.18897395809484166\xspace}

  % cpu-time-avg
\providecommand{\predicateBitpreciseParallelInvariantsResultsAsyncInvariantsPrecIncorrectCpuTimeAvgPlain}{}
  \renewcommand{\predicateBitpreciseParallelInvariantsResultsAsyncInvariantsPrecIncorrectCpuTimeAvgPlain}{72.38943755855557\xspace}
\providecommand{\predicateBitpreciseParallelInvariantsResultsAsyncInvariantsPrecIncorrectCpuTimeAvgPlainHours}{}
  \renewcommand{\predicateBitpreciseParallelInvariantsResultsAsyncInvariantsPrecIncorrectCpuTimeAvgPlainHours}{0.020108177099598768\xspace}

  % wall-time-avg
\providecommand{\predicateBitpreciseParallelInvariantsResultsAsyncInvariantsPrecIncorrectWallTimeAvgPlain}{}
  \renewcommand{\predicateBitpreciseParallelInvariantsResultsAsyncInvariantsPrecIncorrectWallTimeAvgPlain}{37.79479161896833\xspace}
\providecommand{\predicateBitpreciseParallelInvariantsResultsAsyncInvariantsPrecIncorrectWallTimeAvgPlainHours}{}
  \renewcommand{\predicateBitpreciseParallelInvariantsResultsAsyncInvariantsPrecIncorrectWallTimeAvgPlainHours}{0.010498553227491204\xspace}

  % inv-succ
\providecommand{\predicateBitpreciseParallelInvariantsResultsAsyncInvariantsPrecIncorrectInvSuccPlain}{}
  \renewcommand{\predicateBitpreciseParallelInvariantsResultsAsyncInvariantsPrecIncorrectInvSuccPlain}{0\xspace}

  % inv-tries
\providecommand{\predicateBitpreciseParallelInvariantsResultsAsyncInvariantsPrecIncorrectInvTriesPlain}{}
  \renewcommand{\predicateBitpreciseParallelInvariantsResultsAsyncInvariantsPrecIncorrectInvTriesPlain}{0\xspace}

  % inv-time-sum
\providecommand{\predicateBitpreciseParallelInvariantsResultsAsyncInvariantsPrecIncorrectInvTimeSumPlain}{}
  \renewcommand{\predicateBitpreciseParallelInvariantsResultsAsyncInvariantsPrecIncorrectInvTimeSumPlain}{0.0\xspace}
\providecommand{\predicateBitpreciseParallelInvariantsResultsAsyncInvariantsPrecIncorrectInvTimeSumPlainHours}{}
  \renewcommand{\predicateBitpreciseParallelInvariantsResultsAsyncInvariantsPrecIncorrectInvTimeSumPlainHours}{0.0\xspace}

  % finished-main
\providecommand{\predicateBitpreciseParallelInvariantsResultsAsyncInvariantsPrecIncorrectFinishedMainPlain}{}
  \renewcommand{\predicateBitpreciseParallelInvariantsResultsAsyncInvariantsPrecIncorrectFinishedMainPlain}{18\xspace}

 %% timeout %%
\providecommand{\predicateBitpreciseParallelInvariantsResultsAsyncInvariantsPrecTimeoutPlain}{}
  \renewcommand{\predicateBitpreciseParallelInvariantsResultsAsyncInvariantsPrecTimeoutPlain}{1228\xspace}

  % cpu-time-sum
\providecommand{\predicateBitpreciseParallelInvariantsResultsAsyncInvariantsPrecTimeoutCpuTimeSumPlain}{}
  \renewcommand{\predicateBitpreciseParallelInvariantsResultsAsyncInvariantsPrecTimeoutCpuTimeSumPlain}{754947.799360049\xspace}
\providecommand{\predicateBitpreciseParallelInvariantsResultsAsyncInvariantsPrecTimeoutCpuTimeSumPlainHours}{}
  \renewcommand{\predicateBitpreciseParallelInvariantsResultsAsyncInvariantsPrecTimeoutCpuTimeSumPlainHours}{209.70772204445805\xspace}

  % wall-time-sum
\providecommand{\predicateBitpreciseParallelInvariantsResultsAsyncInvariantsPrecTimeoutWallTimeSumPlain}{}
  \renewcommand{\predicateBitpreciseParallelInvariantsResultsAsyncInvariantsPrecTimeoutWallTimeSumPlain}{421390.34141539957\xspace}
\providecommand{\predicateBitpreciseParallelInvariantsResultsAsyncInvariantsPrecTimeoutWallTimeSumPlainHours}{}
  \renewcommand{\predicateBitpreciseParallelInvariantsResultsAsyncInvariantsPrecTimeoutWallTimeSumPlainHours}{117.05287261538876\xspace}

  % cpu-time-avg
\providecommand{\predicateBitpreciseParallelInvariantsResultsAsyncInvariantsPrecTimeoutCpuTimeAvgPlain}{}
  \renewcommand{\predicateBitpreciseParallelInvariantsResultsAsyncInvariantsPrecTimeoutCpuTimeAvgPlain}{614.7783382410822\xspace}
\providecommand{\predicateBitpreciseParallelInvariantsResultsAsyncInvariantsPrecTimeoutCpuTimeAvgPlainHours}{}
  \renewcommand{\predicateBitpreciseParallelInvariantsResultsAsyncInvariantsPrecTimeoutCpuTimeAvgPlainHours}{0.17077176062252283\xspace}

  % wall-time-avg
\providecommand{\predicateBitpreciseParallelInvariantsResultsAsyncInvariantsPrecTimeoutWallTimeAvgPlain}{}
  \renewcommand{\predicateBitpreciseParallelInvariantsResultsAsyncInvariantsPrecTimeoutWallTimeAvgPlain}{343.1517438236153\xspace}
\providecommand{\predicateBitpreciseParallelInvariantsResultsAsyncInvariantsPrecTimeoutWallTimeAvgPlainHours}{}
  \renewcommand{\predicateBitpreciseParallelInvariantsResultsAsyncInvariantsPrecTimeoutWallTimeAvgPlainHours}{0.09531992883989314\xspace}

  % inv-succ
\providecommand{\predicateBitpreciseParallelInvariantsResultsAsyncInvariantsPrecTimeoutInvSuccPlain}{}
  \renewcommand{\predicateBitpreciseParallelInvariantsResultsAsyncInvariantsPrecTimeoutInvSuccPlain}{0\xspace}

  % inv-tries
\providecommand{\predicateBitpreciseParallelInvariantsResultsAsyncInvariantsPrecTimeoutInvTriesPlain}{}
  \renewcommand{\predicateBitpreciseParallelInvariantsResultsAsyncInvariantsPrecTimeoutInvTriesPlain}{0\xspace}

  % inv-time-sum
\providecommand{\predicateBitpreciseParallelInvariantsResultsAsyncInvariantsPrecTimeoutInvTimeSumPlain}{}
  \renewcommand{\predicateBitpreciseParallelInvariantsResultsAsyncInvariantsPrecTimeoutInvTimeSumPlain}{0.0\xspace}
\providecommand{\predicateBitpreciseParallelInvariantsResultsAsyncInvariantsPrecTimeoutInvTimeSumPlainHours}{}
  \renewcommand{\predicateBitpreciseParallelInvariantsResultsAsyncInvariantsPrecTimeoutInvTimeSumPlainHours}{0.0\xspace}

  % finished-main
\providecommand{\predicateBitpreciseParallelInvariantsResultsAsyncInvariantsPrecTimeoutFinishedMainPlain}{}
  \renewcommand{\predicateBitpreciseParallelInvariantsResultsAsyncInvariantsPrecTimeoutFinishedMainPlain}{0\xspace}

 %% unknown-or-category-error %%
\providecommand{\predicateBitpreciseParallelInvariantsResultsAsyncInvariantsPrecUnknownOrCategoryErrorPlain}{}
  \renewcommand{\predicateBitpreciseParallelInvariantsResultsAsyncInvariantsPrecUnknownOrCategoryErrorPlain}{1395\xspace}

  % cpu-time-sum
\providecommand{\predicateBitpreciseParallelInvariantsResultsAsyncInvariantsPrecUnknownOrCategoryErrorCpuTimeSumPlain}{}
  \renewcommand{\predicateBitpreciseParallelInvariantsResultsAsyncInvariantsPrecUnknownOrCategoryErrorCpuTimeSumPlain}{814098.337082185\xspace}
\providecommand{\predicateBitpreciseParallelInvariantsResultsAsyncInvariantsPrecUnknownOrCategoryErrorCpuTimeSumPlainHours}{}
  \renewcommand{\predicateBitpreciseParallelInvariantsResultsAsyncInvariantsPrecUnknownOrCategoryErrorCpuTimeSumPlainHours}{226.1384269672736\xspace}

  % wall-time-sum
\providecommand{\predicateBitpreciseParallelInvariantsResultsAsyncInvariantsPrecUnknownOrCategoryErrorWallTimeSumPlain}{}
  \renewcommand{\predicateBitpreciseParallelInvariantsResultsAsyncInvariantsPrecUnknownOrCategoryErrorWallTimeSumPlain}{457235.12576245755\xspace}
\providecommand{\predicateBitpreciseParallelInvariantsResultsAsyncInvariantsPrecUnknownOrCategoryErrorWallTimeSumPlainHours}{}
  \renewcommand{\predicateBitpreciseParallelInvariantsResultsAsyncInvariantsPrecUnknownOrCategoryErrorWallTimeSumPlainHours}{127.00975715623821\xspace}

  % cpu-time-avg
\providecommand{\predicateBitpreciseParallelInvariantsResultsAsyncInvariantsPrecUnknownOrCategoryErrorCpuTimeAvgPlain}{}
  \renewcommand{\predicateBitpreciseParallelInvariantsResultsAsyncInvariantsPrecUnknownOrCategoryErrorCpuTimeAvgPlain}{583.5830373348996\xspace}
\providecommand{\predicateBitpreciseParallelInvariantsResultsAsyncInvariantsPrecUnknownOrCategoryErrorCpuTimeAvgPlainHours}{}
  \renewcommand{\predicateBitpreciseParallelInvariantsResultsAsyncInvariantsPrecUnknownOrCategoryErrorCpuTimeAvgPlainHours}{0.16210639925969433\xspace}

  % wall-time-avg
\providecommand{\predicateBitpreciseParallelInvariantsResultsAsyncInvariantsPrecUnknownOrCategoryErrorWallTimeAvgPlain}{}
  \renewcommand{\predicateBitpreciseParallelInvariantsResultsAsyncInvariantsPrecUnknownOrCategoryErrorWallTimeAvgPlain}{327.76711524190506\xspace}
\providecommand{\predicateBitpreciseParallelInvariantsResultsAsyncInvariantsPrecUnknownOrCategoryErrorWallTimeAvgPlainHours}{}
  \renewcommand{\predicateBitpreciseParallelInvariantsResultsAsyncInvariantsPrecUnknownOrCategoryErrorWallTimeAvgPlainHours}{0.09104642090052918\xspace}

  % inv-succ
\providecommand{\predicateBitpreciseParallelInvariantsResultsAsyncInvariantsPrecUnknownOrCategoryErrorInvSuccPlain}{}
  \renewcommand{\predicateBitpreciseParallelInvariantsResultsAsyncInvariantsPrecUnknownOrCategoryErrorInvSuccPlain}{0\xspace}

  % inv-tries
\providecommand{\predicateBitpreciseParallelInvariantsResultsAsyncInvariantsPrecUnknownOrCategoryErrorInvTriesPlain}{}
  \renewcommand{\predicateBitpreciseParallelInvariantsResultsAsyncInvariantsPrecUnknownOrCategoryErrorInvTriesPlain}{0\xspace}

  % inv-time-sum
\providecommand{\predicateBitpreciseParallelInvariantsResultsAsyncInvariantsPrecUnknownOrCategoryErrorInvTimeSumPlain}{}
  \renewcommand{\predicateBitpreciseParallelInvariantsResultsAsyncInvariantsPrecUnknownOrCategoryErrorInvTimeSumPlain}{0.0\xspace}
\providecommand{\predicateBitpreciseParallelInvariantsResultsAsyncInvariantsPrecUnknownOrCategoryErrorInvTimeSumPlainHours}{}
  \renewcommand{\predicateBitpreciseParallelInvariantsResultsAsyncInvariantsPrecUnknownOrCategoryErrorInvTimeSumPlainHours}{0.0\xspace}

  % finished-main
\providecommand{\predicateBitpreciseParallelInvariantsResultsAsyncInvariantsPrecUnknownOrCategoryErrorFinishedMainPlain}{}
  \renewcommand{\predicateBitpreciseParallelInvariantsResultsAsyncInvariantsPrecUnknownOrCategoryErrorFinishedMainPlain}{0\xspace}

 %% correct-false %%
\providecommand{\predicateBitpreciseParallelInvariantsResultsAsyncInvariantsPrecCorrectFalsePlain}{}
  \renewcommand{\predicateBitpreciseParallelInvariantsResultsAsyncInvariantsPrecCorrectFalsePlain}{549\xspace}

  % cpu-time-sum
\providecommand{\predicateBitpreciseParallelInvariantsResultsAsyncInvariantsPrecCorrectFalseCpuTimeSumPlain}{}
  \renewcommand{\predicateBitpreciseParallelInvariantsResultsAsyncInvariantsPrecCorrectFalseCpuTimeSumPlain}{75324.7972408311\xspace}
\providecommand{\predicateBitpreciseParallelInvariantsResultsAsyncInvariantsPrecCorrectFalseCpuTimeSumPlainHours}{}
  \renewcommand{\predicateBitpreciseParallelInvariantsResultsAsyncInvariantsPrecCorrectFalseCpuTimeSumPlainHours}{20.923554789119752\xspace}

  % wall-time-sum
\providecommand{\predicateBitpreciseParallelInvariantsResultsAsyncInvariantsPrecCorrectFalseWallTimeSumPlain}{}
  \renewcommand{\predicateBitpreciseParallelInvariantsResultsAsyncInvariantsPrecCorrectFalseWallTimeSumPlain}{33738.42693519465\xspace}
\providecommand{\predicateBitpreciseParallelInvariantsResultsAsyncInvariantsPrecCorrectFalseWallTimeSumPlainHours}{}
  \renewcommand{\predicateBitpreciseParallelInvariantsResultsAsyncInvariantsPrecCorrectFalseWallTimeSumPlainHours}{9.371785259776292\xspace}

  % cpu-time-avg
\providecommand{\predicateBitpreciseParallelInvariantsResultsAsyncInvariantsPrecCorrectFalseCpuTimeAvgPlain}{}
  \renewcommand{\predicateBitpreciseParallelInvariantsResultsAsyncInvariantsPrecCorrectFalseCpuTimeAvgPlain}{137.203637961441\xspace}
\providecommand{\predicateBitpreciseParallelInvariantsResultsAsyncInvariantsPrecCorrectFalseCpuTimeAvgPlainHours}{}
  \renewcommand{\predicateBitpreciseParallelInvariantsResultsAsyncInvariantsPrecCorrectFalseCpuTimeAvgPlainHours}{0.03811212165595583\xspace}

  % wall-time-avg
\providecommand{\predicateBitpreciseParallelInvariantsResultsAsyncInvariantsPrecCorrectFalseWallTimeAvgPlain}{}
  \renewcommand{\predicateBitpreciseParallelInvariantsResultsAsyncInvariantsPrecCorrectFalseWallTimeAvgPlain}{61.45432957230355\xspace}
\providecommand{\predicateBitpreciseParallelInvariantsResultsAsyncInvariantsPrecCorrectFalseWallTimeAvgPlainHours}{}
  \renewcommand{\predicateBitpreciseParallelInvariantsResultsAsyncInvariantsPrecCorrectFalseWallTimeAvgPlainHours}{0.017070647103417654\xspace}

  % inv-succ
\providecommand{\predicateBitpreciseParallelInvariantsResultsAsyncInvariantsPrecCorrectFalseInvSuccPlain}{}
  \renewcommand{\predicateBitpreciseParallelInvariantsResultsAsyncInvariantsPrecCorrectFalseInvSuccPlain}{0\xspace}

  % inv-tries
\providecommand{\predicateBitpreciseParallelInvariantsResultsAsyncInvariantsPrecCorrectFalseInvTriesPlain}{}
  \renewcommand{\predicateBitpreciseParallelInvariantsResultsAsyncInvariantsPrecCorrectFalseInvTriesPlain}{0\xspace}

  % inv-time-sum
\providecommand{\predicateBitpreciseParallelInvariantsResultsAsyncInvariantsPrecCorrectFalseInvTimeSumPlain}{}
  \renewcommand{\predicateBitpreciseParallelInvariantsResultsAsyncInvariantsPrecCorrectFalseInvTimeSumPlain}{0.0\xspace}
\providecommand{\predicateBitpreciseParallelInvariantsResultsAsyncInvariantsPrecCorrectFalseInvTimeSumPlainHours}{}
  \renewcommand{\predicateBitpreciseParallelInvariantsResultsAsyncInvariantsPrecCorrectFalseInvTimeSumPlainHours}{0.0\xspace}

  % finished-main
\providecommand{\predicateBitpreciseParallelInvariantsResultsAsyncInvariantsPrecCorrectFalseFinishedMainPlain}{}
  \renewcommand{\predicateBitpreciseParallelInvariantsResultsAsyncInvariantsPrecCorrectFalseFinishedMainPlain}{549\xspace}

 %% correct-true %%
\providecommand{\predicateBitpreciseParallelInvariantsResultsAsyncInvariantsPrecCorrectTruePlain}{}
  \renewcommand{\predicateBitpreciseParallelInvariantsResultsAsyncInvariantsPrecCorrectTruePlain}{1526\xspace}

  % cpu-time-sum
\providecommand{\predicateBitpreciseParallelInvariantsResultsAsyncInvariantsPrecCorrectTrueCpuTimeSumPlain}{}
  \renewcommand{\predicateBitpreciseParallelInvariantsResultsAsyncInvariantsPrecCorrectTrueCpuTimeSumPlain}{115414.60809711998\xspace}
\providecommand{\predicateBitpreciseParallelInvariantsResultsAsyncInvariantsPrecCorrectTrueCpuTimeSumPlainHours}{}
  \renewcommand{\predicateBitpreciseParallelInvariantsResultsAsyncInvariantsPrecCorrectTrueCpuTimeSumPlainHours}{32.059613360311104\xspace}

  % wall-time-sum
\providecommand{\predicateBitpreciseParallelInvariantsResultsAsyncInvariantsPrecCorrectTrueWallTimeSumPlain}{}
  \renewcommand{\predicateBitpreciseParallelInvariantsResultsAsyncInvariantsPrecCorrectTrueWallTimeSumPlain}{45430.53487372059\xspace}
\providecommand{\predicateBitpreciseParallelInvariantsResultsAsyncInvariantsPrecCorrectTrueWallTimeSumPlainHours}{}
  \renewcommand{\predicateBitpreciseParallelInvariantsResultsAsyncInvariantsPrecCorrectTrueWallTimeSumPlainHours}{12.619593020477943\xspace}

  % cpu-time-avg
\providecommand{\predicateBitpreciseParallelInvariantsResultsAsyncInvariantsPrecCorrectTrueCpuTimeAvgPlain}{}
  \renewcommand{\predicateBitpreciseParallelInvariantsResultsAsyncInvariantsPrecCorrectTrueCpuTimeAvgPlain}{75.63211539785058\xspace}
\providecommand{\predicateBitpreciseParallelInvariantsResultsAsyncInvariantsPrecCorrectTrueCpuTimeAvgPlainHours}{}
  \renewcommand{\predicateBitpreciseParallelInvariantsResultsAsyncInvariantsPrecCorrectTrueCpuTimeAvgPlainHours}{0.021008920943847383\xspace}

  % wall-time-avg
\providecommand{\predicateBitpreciseParallelInvariantsResultsAsyncInvariantsPrecCorrectTrueWallTimeAvgPlain}{}
  \renewcommand{\predicateBitpreciseParallelInvariantsResultsAsyncInvariantsPrecCorrectTrueWallTimeAvgPlain}{29.770992708860152\xspace}
\providecommand{\predicateBitpreciseParallelInvariantsResultsAsyncInvariantsPrecCorrectTrueWallTimeAvgPlainHours}{}
  \renewcommand{\predicateBitpreciseParallelInvariantsResultsAsyncInvariantsPrecCorrectTrueWallTimeAvgPlainHours}{0.008269720196905597\xspace}

  % inv-succ
\providecommand{\predicateBitpreciseParallelInvariantsResultsAsyncInvariantsPrecCorrectTrueInvSuccPlain}{}
  \renewcommand{\predicateBitpreciseParallelInvariantsResultsAsyncInvariantsPrecCorrectTrueInvSuccPlain}{0\xspace}

  % inv-tries
\providecommand{\predicateBitpreciseParallelInvariantsResultsAsyncInvariantsPrecCorrectTrueInvTriesPlain}{}
  \renewcommand{\predicateBitpreciseParallelInvariantsResultsAsyncInvariantsPrecCorrectTrueInvTriesPlain}{0\xspace}

  % inv-time-sum
\providecommand{\predicateBitpreciseParallelInvariantsResultsAsyncInvariantsPrecCorrectTrueInvTimeSumPlain}{}
  \renewcommand{\predicateBitpreciseParallelInvariantsResultsAsyncInvariantsPrecCorrectTrueInvTimeSumPlain}{0.0\xspace}
\providecommand{\predicateBitpreciseParallelInvariantsResultsAsyncInvariantsPrecCorrectTrueInvTimeSumPlainHours}{}
  \renewcommand{\predicateBitpreciseParallelInvariantsResultsAsyncInvariantsPrecCorrectTrueInvTimeSumPlainHours}{0.0\xspace}

  % finished-main
\providecommand{\predicateBitpreciseParallelInvariantsResultsAsyncInvariantsPrecCorrectTrueFinishedMainPlain}{}
  \renewcommand{\predicateBitpreciseParallelInvariantsResultsAsyncInvariantsPrecCorrectTrueFinishedMainPlain}{559\xspace}

 %% incorrect-false %%
\providecommand{\predicateBitpreciseParallelInvariantsResultsAsyncInvariantsPrecIncorrectFalsePlain}{}
  \renewcommand{\predicateBitpreciseParallelInvariantsResultsAsyncInvariantsPrecIncorrectFalsePlain}{18\xspace}

  % cpu-time-sum
\providecommand{\predicateBitpreciseParallelInvariantsResultsAsyncInvariantsPrecIncorrectFalseCpuTimeSumPlain}{}
  \renewcommand{\predicateBitpreciseParallelInvariantsResultsAsyncInvariantsPrecIncorrectFalseCpuTimeSumPlain}{1303.0098760540002\xspace}
\providecommand{\predicateBitpreciseParallelInvariantsResultsAsyncInvariantsPrecIncorrectFalseCpuTimeSumPlainHours}{}
  \renewcommand{\predicateBitpreciseParallelInvariantsResultsAsyncInvariantsPrecIncorrectFalseCpuTimeSumPlainHours}{0.36194718779277785\xspace}

  % wall-time-sum
\providecommand{\predicateBitpreciseParallelInvariantsResultsAsyncInvariantsPrecIncorrectFalseWallTimeSumPlain}{}
  \renewcommand{\predicateBitpreciseParallelInvariantsResultsAsyncInvariantsPrecIncorrectFalseWallTimeSumPlain}{680.3062491414299\xspace}
\providecommand{\predicateBitpreciseParallelInvariantsResultsAsyncInvariantsPrecIncorrectFalseWallTimeSumPlainHours}{}
  \renewcommand{\predicateBitpreciseParallelInvariantsResultsAsyncInvariantsPrecIncorrectFalseWallTimeSumPlainHours}{0.18897395809484166\xspace}

  % cpu-time-avg
\providecommand{\predicateBitpreciseParallelInvariantsResultsAsyncInvariantsPrecIncorrectFalseCpuTimeAvgPlain}{}
  \renewcommand{\predicateBitpreciseParallelInvariantsResultsAsyncInvariantsPrecIncorrectFalseCpuTimeAvgPlain}{72.38943755855557\xspace}
\providecommand{\predicateBitpreciseParallelInvariantsResultsAsyncInvariantsPrecIncorrectFalseCpuTimeAvgPlainHours}{}
  \renewcommand{\predicateBitpreciseParallelInvariantsResultsAsyncInvariantsPrecIncorrectFalseCpuTimeAvgPlainHours}{0.020108177099598768\xspace}

  % wall-time-avg
\providecommand{\predicateBitpreciseParallelInvariantsResultsAsyncInvariantsPrecIncorrectFalseWallTimeAvgPlain}{}
  \renewcommand{\predicateBitpreciseParallelInvariantsResultsAsyncInvariantsPrecIncorrectFalseWallTimeAvgPlain}{37.79479161896833\xspace}
\providecommand{\predicateBitpreciseParallelInvariantsResultsAsyncInvariantsPrecIncorrectFalseWallTimeAvgPlainHours}{}
  \renewcommand{\predicateBitpreciseParallelInvariantsResultsAsyncInvariantsPrecIncorrectFalseWallTimeAvgPlainHours}{0.010498553227491204\xspace}

  % inv-succ
\providecommand{\predicateBitpreciseParallelInvariantsResultsAsyncInvariantsPrecIncorrectFalseInvSuccPlain}{}
  \renewcommand{\predicateBitpreciseParallelInvariantsResultsAsyncInvariantsPrecIncorrectFalseInvSuccPlain}{0\xspace}

  % inv-tries
\providecommand{\predicateBitpreciseParallelInvariantsResultsAsyncInvariantsPrecIncorrectFalseInvTriesPlain}{}
  \renewcommand{\predicateBitpreciseParallelInvariantsResultsAsyncInvariantsPrecIncorrectFalseInvTriesPlain}{0\xspace}

  % inv-time-sum
\providecommand{\predicateBitpreciseParallelInvariantsResultsAsyncInvariantsPrecIncorrectFalseInvTimeSumPlain}{}
  \renewcommand{\predicateBitpreciseParallelInvariantsResultsAsyncInvariantsPrecIncorrectFalseInvTimeSumPlain}{0.0\xspace}
\providecommand{\predicateBitpreciseParallelInvariantsResultsAsyncInvariantsPrecIncorrectFalseInvTimeSumPlainHours}{}
  \renewcommand{\predicateBitpreciseParallelInvariantsResultsAsyncInvariantsPrecIncorrectFalseInvTimeSumPlainHours}{0.0\xspace}

  % finished-main
\providecommand{\predicateBitpreciseParallelInvariantsResultsAsyncInvariantsPrecIncorrectFalseFinishedMainPlain}{}
  \renewcommand{\predicateBitpreciseParallelInvariantsResultsAsyncInvariantsPrecIncorrectFalseFinishedMainPlain}{18\xspace}

 %% incorrect-true %%
\providecommand{\predicateBitpreciseParallelInvariantsResultsAsyncInvariantsPrecIncorrectTruePlain}{}
  \renewcommand{\predicateBitpreciseParallelInvariantsResultsAsyncInvariantsPrecIncorrectTruePlain}{0\xspace}

  % cpu-time-sum
\providecommand{\predicateBitpreciseParallelInvariantsResultsAsyncInvariantsPrecIncorrectTrueCpuTimeSumPlain}{}
  \renewcommand{\predicateBitpreciseParallelInvariantsResultsAsyncInvariantsPrecIncorrectTrueCpuTimeSumPlain}{0.0\xspace}
\providecommand{\predicateBitpreciseParallelInvariantsResultsAsyncInvariantsPrecIncorrectTrueCpuTimeSumPlainHours}{}
  \renewcommand{\predicateBitpreciseParallelInvariantsResultsAsyncInvariantsPrecIncorrectTrueCpuTimeSumPlainHours}{0.0\xspace}

  % wall-time-sum
\providecommand{\predicateBitpreciseParallelInvariantsResultsAsyncInvariantsPrecIncorrectTrueWallTimeSumPlain}{}
  \renewcommand{\predicateBitpreciseParallelInvariantsResultsAsyncInvariantsPrecIncorrectTrueWallTimeSumPlain}{0.0\xspace}
\providecommand{\predicateBitpreciseParallelInvariantsResultsAsyncInvariantsPrecIncorrectTrueWallTimeSumPlainHours}{}
  \renewcommand{\predicateBitpreciseParallelInvariantsResultsAsyncInvariantsPrecIncorrectTrueWallTimeSumPlainHours}{0.0\xspace}

  % cpu-time-avg
\providecommand{\predicateBitpreciseParallelInvariantsResultsAsyncInvariantsPrecIncorrectTrueCpuTimeAvgPlain}{}
  \renewcommand{\predicateBitpreciseParallelInvariantsResultsAsyncInvariantsPrecIncorrectTrueCpuTimeAvgPlain}{NaN\xspace}
\providecommand{\predicateBitpreciseParallelInvariantsResultsAsyncInvariantsPrecIncorrectTrueCpuTimeAvgPlainHours}{}
  \renewcommand{\predicateBitpreciseParallelInvariantsResultsAsyncInvariantsPrecIncorrectTrueCpuTimeAvgPlainHours}{NaN\xspace}

  % wall-time-avg
\providecommand{\predicateBitpreciseParallelInvariantsResultsAsyncInvariantsPrecIncorrectTrueWallTimeAvgPlain}{}
  \renewcommand{\predicateBitpreciseParallelInvariantsResultsAsyncInvariantsPrecIncorrectTrueWallTimeAvgPlain}{NaN\xspace}
\providecommand{\predicateBitpreciseParallelInvariantsResultsAsyncInvariantsPrecIncorrectTrueWallTimeAvgPlainHours}{}
  \renewcommand{\predicateBitpreciseParallelInvariantsResultsAsyncInvariantsPrecIncorrectTrueWallTimeAvgPlainHours}{NaN\xspace}

  % inv-succ
\providecommand{\predicateBitpreciseParallelInvariantsResultsAsyncInvariantsPrecIncorrectTrueInvSuccPlain}{}
  \renewcommand{\predicateBitpreciseParallelInvariantsResultsAsyncInvariantsPrecIncorrectTrueInvSuccPlain}{0\xspace}

  % inv-tries
\providecommand{\predicateBitpreciseParallelInvariantsResultsAsyncInvariantsPrecIncorrectTrueInvTriesPlain}{}
  \renewcommand{\predicateBitpreciseParallelInvariantsResultsAsyncInvariantsPrecIncorrectTrueInvTriesPlain}{0\xspace}

  % inv-time-sum
\providecommand{\predicateBitpreciseParallelInvariantsResultsAsyncInvariantsPrecIncorrectTrueInvTimeSumPlain}{}
  \renewcommand{\predicateBitpreciseParallelInvariantsResultsAsyncInvariantsPrecIncorrectTrueInvTimeSumPlain}{0.0\xspace}
\providecommand{\predicateBitpreciseParallelInvariantsResultsAsyncInvariantsPrecIncorrectTrueInvTimeSumPlainHours}{}
  \renewcommand{\predicateBitpreciseParallelInvariantsResultsAsyncInvariantsPrecIncorrectTrueInvTimeSumPlainHours}{0.0\xspace}

  % finished-main
\providecommand{\predicateBitpreciseParallelInvariantsResultsAsyncInvariantsPrecIncorrectTrueFinishedMainPlain}{}
  \renewcommand{\predicateBitpreciseParallelInvariantsResultsAsyncInvariantsPrecIncorrectTrueFinishedMainPlain}{0\xspace}

 %% all %%
\providecommand{\predicateBitpreciseParallelInvariantsResultsAsyncInvariantsPrecAllPlain}{}
  \renewcommand{\predicateBitpreciseParallelInvariantsResultsAsyncInvariantsPrecAllPlain}{3488\xspace}

  % cpu-time-sum
\providecommand{\predicateBitpreciseParallelInvariantsResultsAsyncInvariantsPrecAllCpuTimeSumPlain}{}
  \renewcommand{\predicateBitpreciseParallelInvariantsResultsAsyncInvariantsPrecAllCpuTimeSumPlain}{1006140.7522961877\xspace}
\providecommand{\predicateBitpreciseParallelInvariantsResultsAsyncInvariantsPrecAllCpuTimeSumPlainHours}{}
  \renewcommand{\predicateBitpreciseParallelInvariantsResultsAsyncInvariantsPrecAllCpuTimeSumPlainHours}{279.48354230449655\xspace}

  % wall-time-sum
\providecommand{\predicateBitpreciseParallelInvariantsResultsAsyncInvariantsPrecAllWallTimeSumPlain}{}
  \renewcommand{\predicateBitpreciseParallelInvariantsResultsAsyncInvariantsPrecAllWallTimeSumPlain}{537084.3938205144\xspace}
\providecommand{\predicateBitpreciseParallelInvariantsResultsAsyncInvariantsPrecAllWallTimeSumPlainHours}{}
  \renewcommand{\predicateBitpreciseParallelInvariantsResultsAsyncInvariantsPrecAllWallTimeSumPlainHours}{149.19010939458735\xspace}

  % cpu-time-avg
\providecommand{\predicateBitpreciseParallelInvariantsResultsAsyncInvariantsPrecAllCpuTimeAvgPlain}{}
  \renewcommand{\predicateBitpreciseParallelInvariantsResultsAsyncInvariantsPrecAllCpuTimeAvgPlain}{288.4577844885859\xspace}
\providecommand{\predicateBitpreciseParallelInvariantsResultsAsyncInvariantsPrecAllCpuTimeAvgPlainHours}{}
  \renewcommand{\predicateBitpreciseParallelInvariantsResultsAsyncInvariantsPrecAllCpuTimeAvgPlainHours}{0.08012716235794053\xspace}

  % wall-time-avg
\providecommand{\predicateBitpreciseParallelInvariantsResultsAsyncInvariantsPrecAllWallTimeAvgPlain}{}
  \renewcommand{\predicateBitpreciseParallelInvariantsResultsAsyncInvariantsPrecAllWallTimeAvgPlain}{153.98061749441354\xspace}
\providecommand{\predicateBitpreciseParallelInvariantsResultsAsyncInvariantsPrecAllWallTimeAvgPlainHours}{}
  \renewcommand{\predicateBitpreciseParallelInvariantsResultsAsyncInvariantsPrecAllWallTimeAvgPlainHours}{0.042772393748448205\xspace}

  % inv-succ
\providecommand{\predicateBitpreciseParallelInvariantsResultsAsyncInvariantsPrecAllInvSuccPlain}{}
  \renewcommand{\predicateBitpreciseParallelInvariantsResultsAsyncInvariantsPrecAllInvSuccPlain}{0\xspace}

  % inv-tries
\providecommand{\predicateBitpreciseParallelInvariantsResultsAsyncInvariantsPrecAllInvTriesPlain}{}
  \renewcommand{\predicateBitpreciseParallelInvariantsResultsAsyncInvariantsPrecAllInvTriesPlain}{0\xspace}

  % inv-time-sum
\providecommand{\predicateBitpreciseParallelInvariantsResultsAsyncInvariantsPrecAllInvTimeSumPlain}{}
  \renewcommand{\predicateBitpreciseParallelInvariantsResultsAsyncInvariantsPrecAllInvTimeSumPlain}{0.0\xspace}
\providecommand{\predicateBitpreciseParallelInvariantsResultsAsyncInvariantsPrecAllInvTimeSumPlainHours}{}
  \renewcommand{\predicateBitpreciseParallelInvariantsResultsAsyncInvariantsPrecAllInvTimeSumPlainHours}{0.0\xspace}

  % finished-main
\providecommand{\predicateBitpreciseParallelInvariantsResultsAsyncInvariantsPrecAllFinishedMainPlain}{}
  \renewcommand{\predicateBitpreciseParallelInvariantsResultsAsyncInvariantsPrecAllFinishedMainPlain}{1126\xspace}

 %% equal-only %%
\providecommand{\predicateBitpreciseParallelInvariantsResultsAsyncInvariantsPrecEqualOnlyPlain}{}
  \renewcommand{\predicateBitpreciseParallelInvariantsResultsAsyncInvariantsPrecEqualOnlyPlain}{1865\xspace}

  % cpu-time-sum
\providecommand{\predicateBitpreciseParallelInvariantsResultsAsyncInvariantsPrecEqualOnlyCpuTimeSumPlain}{}
  \renewcommand{\predicateBitpreciseParallelInvariantsResultsAsyncInvariantsPrecEqualOnlyCpuTimeSumPlain}{142576.431441922\xspace}
\providecommand{\predicateBitpreciseParallelInvariantsResultsAsyncInvariantsPrecEqualOnlyCpuTimeSumPlainHours}{}
  \renewcommand{\predicateBitpreciseParallelInvariantsResultsAsyncInvariantsPrecEqualOnlyCpuTimeSumPlainHours}{39.604564289422775\xspace}

  % wall-time-sum
\providecommand{\predicateBitpreciseParallelInvariantsResultsAsyncInvariantsPrecEqualOnlyWallTimeSumPlain}{}
  \renewcommand{\predicateBitpreciseParallelInvariantsResultsAsyncInvariantsPrecEqualOnlyWallTimeSumPlain}{54996.45782351187\xspace}
\providecommand{\predicateBitpreciseParallelInvariantsResultsAsyncInvariantsPrecEqualOnlyWallTimeSumPlainHours}{}
  \renewcommand{\predicateBitpreciseParallelInvariantsResultsAsyncInvariantsPrecEqualOnlyWallTimeSumPlainHours}{15.276793839864407\xspace}

  % cpu-time-avg
\providecommand{\predicateBitpreciseParallelInvariantsResultsAsyncInvariantsPrecEqualOnlyCpuTimeAvgPlain}{}
  \renewcommand{\predicateBitpreciseParallelInvariantsResultsAsyncInvariantsPrecEqualOnlyCpuTimeAvgPlain}{76.44848870880536\xspace}
\providecommand{\predicateBitpreciseParallelInvariantsResultsAsyncInvariantsPrecEqualOnlyCpuTimeAvgPlainHours}{}
  \renewcommand{\predicateBitpreciseParallelInvariantsResultsAsyncInvariantsPrecEqualOnlyCpuTimeAvgPlainHours}{0.021235691308001486\xspace}

  % wall-time-avg
\providecommand{\predicateBitpreciseParallelInvariantsResultsAsyncInvariantsPrecEqualOnlyWallTimeAvgPlain}{}
  \renewcommand{\predicateBitpreciseParallelInvariantsResultsAsyncInvariantsPrecEqualOnlyWallTimeAvgPlain}{29.48871733164175\xspace}
\providecommand{\predicateBitpreciseParallelInvariantsResultsAsyncInvariantsPrecEqualOnlyWallTimeAvgPlainHours}{}
  \renewcommand{\predicateBitpreciseParallelInvariantsResultsAsyncInvariantsPrecEqualOnlyWallTimeAvgPlainHours}{0.008191310369900487\xspace}

  % inv-succ
\providecommand{\predicateBitpreciseParallelInvariantsResultsAsyncInvariantsPrecEqualOnlyInvSuccPlain}{}
  \renewcommand{\predicateBitpreciseParallelInvariantsResultsAsyncInvariantsPrecEqualOnlyInvSuccPlain}{0\xspace}

  % inv-tries
\providecommand{\predicateBitpreciseParallelInvariantsResultsAsyncInvariantsPrecEqualOnlyInvTriesPlain}{}
  \renewcommand{\predicateBitpreciseParallelInvariantsResultsAsyncInvariantsPrecEqualOnlyInvTriesPlain}{0\xspace}

  % inv-time-sum
\providecommand{\predicateBitpreciseParallelInvariantsResultsAsyncInvariantsPrecEqualOnlyInvTimeSumPlain}{}
  \renewcommand{\predicateBitpreciseParallelInvariantsResultsAsyncInvariantsPrecEqualOnlyInvTimeSumPlain}{0.0\xspace}
\providecommand{\predicateBitpreciseParallelInvariantsResultsAsyncInvariantsPrecEqualOnlyInvTimeSumPlainHours}{}
  \renewcommand{\predicateBitpreciseParallelInvariantsResultsAsyncInvariantsPrecEqualOnlyInvTimeSumPlainHours}{0.0\xspace}

  % finished-main
\providecommand{\predicateBitpreciseParallelInvariantsResultsAsyncInvariantsPrecEqualOnlyFinishedMainPlain}{}
  \renewcommand{\predicateBitpreciseParallelInvariantsResultsAsyncInvariantsPrecEqualOnlyFinishedMainPlain}{1028\xspace}

%%% predicate_bitprecise_parallel_invariants.2016-09-05_0219.results.async-invariants-prec-path %%%
 %% correct %%
\providecommand{\predicateBitpreciseParallelInvariantsResultsAsyncInvariantsPrecPathCorrectPlain}{}
  \renewcommand{\predicateBitpreciseParallelInvariantsResultsAsyncInvariantsPrecPathCorrectPlain}{2086\xspace}

  % cpu-time-sum
\providecommand{\predicateBitpreciseParallelInvariantsResultsAsyncInvariantsPrecPathCorrectCpuTimeSumPlain}{}
  \renewcommand{\predicateBitpreciseParallelInvariantsResultsAsyncInvariantsPrecPathCorrectCpuTimeSumPlain}{192825.18462640536\xspace}
\providecommand{\predicateBitpreciseParallelInvariantsResultsAsyncInvariantsPrecPathCorrectCpuTimeSumPlainHours}{}
  \renewcommand{\predicateBitpreciseParallelInvariantsResultsAsyncInvariantsPrecPathCorrectCpuTimeSumPlainHours}{53.5625512851126\xspace}

  % wall-time-sum
\providecommand{\predicateBitpreciseParallelInvariantsResultsAsyncInvariantsPrecPathCorrectWallTimeSumPlain}{}
  \renewcommand{\predicateBitpreciseParallelInvariantsResultsAsyncInvariantsPrecPathCorrectWallTimeSumPlain}{78541.1241371655\xspace}
\providecommand{\predicateBitpreciseParallelInvariantsResultsAsyncInvariantsPrecPathCorrectWallTimeSumPlainHours}{}
  \renewcommand{\predicateBitpreciseParallelInvariantsResultsAsyncInvariantsPrecPathCorrectWallTimeSumPlainHours}{21.81697892699042\xspace}

  % cpu-time-avg
\providecommand{\predicateBitpreciseParallelInvariantsResultsAsyncInvariantsPrecPathCorrectCpuTimeAvgPlain}{}
  \renewcommand{\predicateBitpreciseParallelInvariantsResultsAsyncInvariantsPrecPathCorrectCpuTimeAvgPlain}{92.43776827727966\xspace}
\providecommand{\predicateBitpreciseParallelInvariantsResultsAsyncInvariantsPrecPathCorrectCpuTimeAvgPlainHours}{}
  \renewcommand{\predicateBitpreciseParallelInvariantsResultsAsyncInvariantsPrecPathCorrectCpuTimeAvgPlainHours}{0.025677157854799904\xspace}

  % wall-time-avg
\providecommand{\predicateBitpreciseParallelInvariantsResultsAsyncInvariantsPrecPathCorrectWallTimeAvgPlain}{}
  \renewcommand{\predicateBitpreciseParallelInvariantsResultsAsyncInvariantsPrecPathCorrectWallTimeAvgPlain}{37.65154560746189\xspace}
\providecommand{\predicateBitpreciseParallelInvariantsResultsAsyncInvariantsPrecPathCorrectWallTimeAvgPlainHours}{}
  \renewcommand{\predicateBitpreciseParallelInvariantsResultsAsyncInvariantsPrecPathCorrectWallTimeAvgPlainHours}{0.010458762668739414\xspace}

  % inv-succ
\providecommand{\predicateBitpreciseParallelInvariantsResultsAsyncInvariantsPrecPathCorrectInvSuccPlain}{}
  \renewcommand{\predicateBitpreciseParallelInvariantsResultsAsyncInvariantsPrecPathCorrectInvSuccPlain}{0\xspace}

  % inv-tries
\providecommand{\predicateBitpreciseParallelInvariantsResultsAsyncInvariantsPrecPathCorrectInvTriesPlain}{}
  \renewcommand{\predicateBitpreciseParallelInvariantsResultsAsyncInvariantsPrecPathCorrectInvTriesPlain}{0\xspace}

  % inv-time-sum
\providecommand{\predicateBitpreciseParallelInvariantsResultsAsyncInvariantsPrecPathCorrectInvTimeSumPlain}{}
  \renewcommand{\predicateBitpreciseParallelInvariantsResultsAsyncInvariantsPrecPathCorrectInvTimeSumPlain}{0.0\xspace}
\providecommand{\predicateBitpreciseParallelInvariantsResultsAsyncInvariantsPrecPathCorrectInvTimeSumPlainHours}{}
  \renewcommand{\predicateBitpreciseParallelInvariantsResultsAsyncInvariantsPrecPathCorrectInvTimeSumPlainHours}{0.0\xspace}

  % finished-main
\providecommand{\predicateBitpreciseParallelInvariantsResultsAsyncInvariantsPrecPathCorrectFinishedMainPlain}{}
  \renewcommand{\predicateBitpreciseParallelInvariantsResultsAsyncInvariantsPrecPathCorrectFinishedMainPlain}{1111\xspace}

 %% incorrect %%
\providecommand{\predicateBitpreciseParallelInvariantsResultsAsyncInvariantsPrecPathIncorrectPlain}{}
  \renewcommand{\predicateBitpreciseParallelInvariantsResultsAsyncInvariantsPrecPathIncorrectPlain}{18\xspace}

  % cpu-time-sum
\providecommand{\predicateBitpreciseParallelInvariantsResultsAsyncInvariantsPrecPathIncorrectCpuTimeSumPlain}{}
  \renewcommand{\predicateBitpreciseParallelInvariantsResultsAsyncInvariantsPrecPathIncorrectCpuTimeSumPlain}{1108.9655518420002\xspace}
\providecommand{\predicateBitpreciseParallelInvariantsResultsAsyncInvariantsPrecPathIncorrectCpuTimeSumPlainHours}{}
  \renewcommand{\predicateBitpreciseParallelInvariantsResultsAsyncInvariantsPrecPathIncorrectCpuTimeSumPlainHours}{0.3080459866227778\xspace}

  % wall-time-sum
\providecommand{\predicateBitpreciseParallelInvariantsResultsAsyncInvariantsPrecPathIncorrectWallTimeSumPlain}{}
  \renewcommand{\predicateBitpreciseParallelInvariantsResultsAsyncInvariantsPrecPathIncorrectWallTimeSumPlain}{563.58012366253\xspace}
\providecommand{\predicateBitpreciseParallelInvariantsResultsAsyncInvariantsPrecPathIncorrectWallTimeSumPlainHours}{}
  \renewcommand{\predicateBitpreciseParallelInvariantsResultsAsyncInvariantsPrecPathIncorrectWallTimeSumPlainHours}{0.15655003435070278\xspace}

  % cpu-time-avg
\providecommand{\predicateBitpreciseParallelInvariantsResultsAsyncInvariantsPrecPathIncorrectCpuTimeAvgPlain}{}
  \renewcommand{\predicateBitpreciseParallelInvariantsResultsAsyncInvariantsPrecPathIncorrectCpuTimeAvgPlain}{61.609197324555566\xspace}
\providecommand{\predicateBitpreciseParallelInvariantsResultsAsyncInvariantsPrecPathIncorrectCpuTimeAvgPlainHours}{}
  \renewcommand{\predicateBitpreciseParallelInvariantsResultsAsyncInvariantsPrecPathIncorrectCpuTimeAvgPlainHours}{0.017113665923487658\xspace}

  % wall-time-avg
\providecommand{\predicateBitpreciseParallelInvariantsResultsAsyncInvariantsPrecPathIncorrectWallTimeAvgPlain}{}
  \renewcommand{\predicateBitpreciseParallelInvariantsResultsAsyncInvariantsPrecPathIncorrectWallTimeAvgPlain}{31.310006870140555\xspace}
\providecommand{\predicateBitpreciseParallelInvariantsResultsAsyncInvariantsPrecPathIncorrectWallTimeAvgPlainHours}{}
  \renewcommand{\predicateBitpreciseParallelInvariantsResultsAsyncInvariantsPrecPathIncorrectWallTimeAvgPlainHours}{0.0086972241305946\xspace}

  % inv-succ
\providecommand{\predicateBitpreciseParallelInvariantsResultsAsyncInvariantsPrecPathIncorrectInvSuccPlain}{}
  \renewcommand{\predicateBitpreciseParallelInvariantsResultsAsyncInvariantsPrecPathIncorrectInvSuccPlain}{0\xspace}

  % inv-tries
\providecommand{\predicateBitpreciseParallelInvariantsResultsAsyncInvariantsPrecPathIncorrectInvTriesPlain}{}
  \renewcommand{\predicateBitpreciseParallelInvariantsResultsAsyncInvariantsPrecPathIncorrectInvTriesPlain}{0\xspace}

  % inv-time-sum
\providecommand{\predicateBitpreciseParallelInvariantsResultsAsyncInvariantsPrecPathIncorrectInvTimeSumPlain}{}
  \renewcommand{\predicateBitpreciseParallelInvariantsResultsAsyncInvariantsPrecPathIncorrectInvTimeSumPlain}{0.0\xspace}
\providecommand{\predicateBitpreciseParallelInvariantsResultsAsyncInvariantsPrecPathIncorrectInvTimeSumPlainHours}{}
  \renewcommand{\predicateBitpreciseParallelInvariantsResultsAsyncInvariantsPrecPathIncorrectInvTimeSumPlainHours}{0.0\xspace}

  % finished-main
\providecommand{\predicateBitpreciseParallelInvariantsResultsAsyncInvariantsPrecPathIncorrectFinishedMainPlain}{}
  \renewcommand{\predicateBitpreciseParallelInvariantsResultsAsyncInvariantsPrecPathIncorrectFinishedMainPlain}{18\xspace}

 %% timeout %%
\providecommand{\predicateBitpreciseParallelInvariantsResultsAsyncInvariantsPrecPathTimeoutPlain}{}
  \renewcommand{\predicateBitpreciseParallelInvariantsResultsAsyncInvariantsPrecPathTimeoutPlain}{1215\xspace}

  % cpu-time-sum
\providecommand{\predicateBitpreciseParallelInvariantsResultsAsyncInvariantsPrecPathTimeoutCpuTimeSumPlain}{}
  \renewcommand{\predicateBitpreciseParallelInvariantsResultsAsyncInvariantsPrecPathTimeoutCpuTimeSumPlain}{747039.554269208\xspace}
\providecommand{\predicateBitpreciseParallelInvariantsResultsAsyncInvariantsPrecPathTimeoutCpuTimeSumPlainHours}{}
  \renewcommand{\predicateBitpreciseParallelInvariantsResultsAsyncInvariantsPrecPathTimeoutCpuTimeSumPlainHours}{207.51098729700223\xspace}

  % wall-time-sum
\providecommand{\predicateBitpreciseParallelInvariantsResultsAsyncInvariantsPrecPathTimeoutWallTimeSumPlain}{}
  \renewcommand{\predicateBitpreciseParallelInvariantsResultsAsyncInvariantsPrecPathTimeoutWallTimeSumPlain}{417558.2006771677\xspace}
\providecommand{\predicateBitpreciseParallelInvariantsResultsAsyncInvariantsPrecPathTimeoutWallTimeSumPlainHours}{}
  \renewcommand{\predicateBitpreciseParallelInvariantsResultsAsyncInvariantsPrecPathTimeoutWallTimeSumPlainHours}{115.98838907699103\xspace}

  % cpu-time-avg
\providecommand{\predicateBitpreciseParallelInvariantsResultsAsyncInvariantsPrecPathTimeoutCpuTimeAvgPlain}{}
  \renewcommand{\predicateBitpreciseParallelInvariantsResultsAsyncInvariantsPrecPathTimeoutCpuTimeAvgPlain}{614.8473697688954\xspace}
\providecommand{\predicateBitpreciseParallelInvariantsResultsAsyncInvariantsPrecPathTimeoutCpuTimeAvgPlainHours}{}
  \renewcommand{\predicateBitpreciseParallelInvariantsResultsAsyncInvariantsPrecPathTimeoutCpuTimeAvgPlainHours}{0.1707909360469154\xspace}

  % wall-time-avg
\providecommand{\predicateBitpreciseParallelInvariantsResultsAsyncInvariantsPrecPathTimeoutWallTimeAvgPlain}{}
  \renewcommand{\predicateBitpreciseParallelInvariantsResultsAsyncInvariantsPrecPathTimeoutWallTimeAvgPlain}{343.6693009688623\xspace}
\providecommand{\predicateBitpreciseParallelInvariantsResultsAsyncInvariantsPrecPathTimeoutWallTimeAvgPlainHours}{}
  \renewcommand{\predicateBitpreciseParallelInvariantsResultsAsyncInvariantsPrecPathTimeoutWallTimeAvgPlainHours}{0.09546369471357286\xspace}

  % inv-succ
\providecommand{\predicateBitpreciseParallelInvariantsResultsAsyncInvariantsPrecPathTimeoutInvSuccPlain}{}
  \renewcommand{\predicateBitpreciseParallelInvariantsResultsAsyncInvariantsPrecPathTimeoutInvSuccPlain}{0\xspace}

  % inv-tries
\providecommand{\predicateBitpreciseParallelInvariantsResultsAsyncInvariantsPrecPathTimeoutInvTriesPlain}{}
  \renewcommand{\predicateBitpreciseParallelInvariantsResultsAsyncInvariantsPrecPathTimeoutInvTriesPlain}{0\xspace}

  % inv-time-sum
\providecommand{\predicateBitpreciseParallelInvariantsResultsAsyncInvariantsPrecPathTimeoutInvTimeSumPlain}{}
  \renewcommand{\predicateBitpreciseParallelInvariantsResultsAsyncInvariantsPrecPathTimeoutInvTimeSumPlain}{0.0\xspace}
\providecommand{\predicateBitpreciseParallelInvariantsResultsAsyncInvariantsPrecPathTimeoutInvTimeSumPlainHours}{}
  \renewcommand{\predicateBitpreciseParallelInvariantsResultsAsyncInvariantsPrecPathTimeoutInvTimeSumPlainHours}{0.0\xspace}

  % finished-main
\providecommand{\predicateBitpreciseParallelInvariantsResultsAsyncInvariantsPrecPathTimeoutFinishedMainPlain}{}
  \renewcommand{\predicateBitpreciseParallelInvariantsResultsAsyncInvariantsPrecPathTimeoutFinishedMainPlain}{0\xspace}

 %% unknown-or-category-error %%
\providecommand{\predicateBitpreciseParallelInvariantsResultsAsyncInvariantsPrecPathUnknownOrCategoryErrorPlain}{}
  \renewcommand{\predicateBitpreciseParallelInvariantsResultsAsyncInvariantsPrecPathUnknownOrCategoryErrorPlain}{1384\xspace}

  % cpu-time-sum
\providecommand{\predicateBitpreciseParallelInvariantsResultsAsyncInvariantsPrecPathUnknownOrCategoryErrorCpuTimeSumPlain}{}
  \renewcommand{\predicateBitpreciseParallelInvariantsResultsAsyncInvariantsPrecPathUnknownOrCategoryErrorCpuTimeSumPlain}{807513.6796776715\xspace}
\providecommand{\predicateBitpreciseParallelInvariantsResultsAsyncInvariantsPrecPathUnknownOrCategoryErrorCpuTimeSumPlainHours}{}
  \renewcommand{\predicateBitpreciseParallelInvariantsResultsAsyncInvariantsPrecPathUnknownOrCategoryErrorCpuTimeSumPlainHours}{224.30935546601987\xspace}

  % wall-time-sum
\providecommand{\predicateBitpreciseParallelInvariantsResultsAsyncInvariantsPrecPathUnknownOrCategoryErrorWallTimeSumPlain}{}
  \renewcommand{\predicateBitpreciseParallelInvariantsResultsAsyncInvariantsPrecPathUnknownOrCategoryErrorWallTimeSumPlain}{454683.6627366637\xspace}
\providecommand{\predicateBitpreciseParallelInvariantsResultsAsyncInvariantsPrecPathUnknownOrCategoryErrorWallTimeSumPlainHours}{}
  \renewcommand{\predicateBitpreciseParallelInvariantsResultsAsyncInvariantsPrecPathUnknownOrCategoryErrorWallTimeSumPlainHours}{126.30101742685103\xspace}

  % cpu-time-avg
\providecommand{\predicateBitpreciseParallelInvariantsResultsAsyncInvariantsPrecPathUnknownOrCategoryErrorCpuTimeAvgPlain}{}
  \renewcommand{\predicateBitpreciseParallelInvariantsResultsAsyncInvariantsPrecPathUnknownOrCategoryErrorCpuTimeAvgPlain}{583.4636413856008\xspace}
\providecommand{\predicateBitpreciseParallelInvariantsResultsAsyncInvariantsPrecPathUnknownOrCategoryErrorCpuTimeAvgPlainHours}{}
  \renewcommand{\predicateBitpreciseParallelInvariantsResultsAsyncInvariantsPrecPathUnknownOrCategoryErrorCpuTimeAvgPlainHours}{0.16207323371822244\xspace}

  % wall-time-avg
\providecommand{\predicateBitpreciseParallelInvariantsResultsAsyncInvariantsPrecPathUnknownOrCategoryErrorWallTimeAvgPlain}{}
  \renewcommand{\predicateBitpreciseParallelInvariantsResultsAsyncInvariantsPrecPathUnknownOrCategoryErrorWallTimeAvgPlain}{328.52865804672234\xspace}
\providecommand{\predicateBitpreciseParallelInvariantsResultsAsyncInvariantsPrecPathUnknownOrCategoryErrorWallTimeAvgPlainHours}{}
  \renewcommand{\predicateBitpreciseParallelInvariantsResultsAsyncInvariantsPrecPathUnknownOrCategoryErrorWallTimeAvgPlainHours}{0.09125796056853398\xspace}

  % inv-succ
\providecommand{\predicateBitpreciseParallelInvariantsResultsAsyncInvariantsPrecPathUnknownOrCategoryErrorInvSuccPlain}{}
  \renewcommand{\predicateBitpreciseParallelInvariantsResultsAsyncInvariantsPrecPathUnknownOrCategoryErrorInvSuccPlain}{0\xspace}

  % inv-tries
\providecommand{\predicateBitpreciseParallelInvariantsResultsAsyncInvariantsPrecPathUnknownOrCategoryErrorInvTriesPlain}{}
  \renewcommand{\predicateBitpreciseParallelInvariantsResultsAsyncInvariantsPrecPathUnknownOrCategoryErrorInvTriesPlain}{0\xspace}

  % inv-time-sum
\providecommand{\predicateBitpreciseParallelInvariantsResultsAsyncInvariantsPrecPathUnknownOrCategoryErrorInvTimeSumPlain}{}
  \renewcommand{\predicateBitpreciseParallelInvariantsResultsAsyncInvariantsPrecPathUnknownOrCategoryErrorInvTimeSumPlain}{0.0\xspace}
\providecommand{\predicateBitpreciseParallelInvariantsResultsAsyncInvariantsPrecPathUnknownOrCategoryErrorInvTimeSumPlainHours}{}
  \renewcommand{\predicateBitpreciseParallelInvariantsResultsAsyncInvariantsPrecPathUnknownOrCategoryErrorInvTimeSumPlainHours}{0.0\xspace}

  % finished-main
\providecommand{\predicateBitpreciseParallelInvariantsResultsAsyncInvariantsPrecPathUnknownOrCategoryErrorFinishedMainPlain}{}
  \renewcommand{\predicateBitpreciseParallelInvariantsResultsAsyncInvariantsPrecPathUnknownOrCategoryErrorFinishedMainPlain}{2\xspace}

 %% correct-false %%
\providecommand{\predicateBitpreciseParallelInvariantsResultsAsyncInvariantsPrecPathCorrectFalsePlain}{}
  \renewcommand{\predicateBitpreciseParallelInvariantsResultsAsyncInvariantsPrecPathCorrectFalsePlain}{561\xspace}

  % cpu-time-sum
\providecommand{\predicateBitpreciseParallelInvariantsResultsAsyncInvariantsPrecPathCorrectFalseCpuTimeSumPlain}{}
  \renewcommand{\predicateBitpreciseParallelInvariantsResultsAsyncInvariantsPrecPathCorrectFalseCpuTimeSumPlain}{78008.03660640196\xspace}
\providecommand{\predicateBitpreciseParallelInvariantsResultsAsyncInvariantsPrecPathCorrectFalseCpuTimeSumPlainHours}{}
  \renewcommand{\predicateBitpreciseParallelInvariantsResultsAsyncInvariantsPrecPathCorrectFalseCpuTimeSumPlainHours}{21.668899057333878\xspace}

  % wall-time-sum
\providecommand{\predicateBitpreciseParallelInvariantsResultsAsyncInvariantsPrecPathCorrectFalseWallTimeSumPlain}{}
  \renewcommand{\predicateBitpreciseParallelInvariantsResultsAsyncInvariantsPrecPathCorrectFalseWallTimeSumPlain}{33791.18114089528\xspace}
\providecommand{\predicateBitpreciseParallelInvariantsResultsAsyncInvariantsPrecPathCorrectFalseWallTimeSumPlainHours}{}
  \renewcommand{\predicateBitpreciseParallelInvariantsResultsAsyncInvariantsPrecPathCorrectFalseWallTimeSumPlainHours}{9.386439205804244\xspace}

  % cpu-time-avg
\providecommand{\predicateBitpreciseParallelInvariantsResultsAsyncInvariantsPrecPathCorrectFalseCpuTimeAvgPlain}{}
  \renewcommand{\predicateBitpreciseParallelInvariantsResultsAsyncInvariantsPrecPathCorrectFalseCpuTimeAvgPlain}{139.05175865668798\xspace}
\providecommand{\predicateBitpreciseParallelInvariantsResultsAsyncInvariantsPrecPathCorrectFalseCpuTimeAvgPlainHours}{}
  \renewcommand{\predicateBitpreciseParallelInvariantsResultsAsyncInvariantsPrecPathCorrectFalseCpuTimeAvgPlainHours}{0.03862548851574666\xspace}

  % wall-time-avg
\providecommand{\predicateBitpreciseParallelInvariantsResultsAsyncInvariantsPrecPathCorrectFalseWallTimeAvgPlain}{}
  \renewcommand{\predicateBitpreciseParallelInvariantsResultsAsyncInvariantsPrecPathCorrectFalseWallTimeAvgPlain}{60.233834475749156\xspace}
\providecommand{\predicateBitpreciseParallelInvariantsResultsAsyncInvariantsPrecPathCorrectFalseWallTimeAvgPlainHours}{}
  \renewcommand{\predicateBitpreciseParallelInvariantsResultsAsyncInvariantsPrecPathCorrectFalseWallTimeAvgPlainHours}{0.016731620687708098\xspace}

  % inv-succ
\providecommand{\predicateBitpreciseParallelInvariantsResultsAsyncInvariantsPrecPathCorrectFalseInvSuccPlain}{}
  \renewcommand{\predicateBitpreciseParallelInvariantsResultsAsyncInvariantsPrecPathCorrectFalseInvSuccPlain}{0\xspace}

  % inv-tries
\providecommand{\predicateBitpreciseParallelInvariantsResultsAsyncInvariantsPrecPathCorrectFalseInvTriesPlain}{}
  \renewcommand{\predicateBitpreciseParallelInvariantsResultsAsyncInvariantsPrecPathCorrectFalseInvTriesPlain}{0\xspace}

  % inv-time-sum
\providecommand{\predicateBitpreciseParallelInvariantsResultsAsyncInvariantsPrecPathCorrectFalseInvTimeSumPlain}{}
  \renewcommand{\predicateBitpreciseParallelInvariantsResultsAsyncInvariantsPrecPathCorrectFalseInvTimeSumPlain}{0.0\xspace}
\providecommand{\predicateBitpreciseParallelInvariantsResultsAsyncInvariantsPrecPathCorrectFalseInvTimeSumPlainHours}{}
  \renewcommand{\predicateBitpreciseParallelInvariantsResultsAsyncInvariantsPrecPathCorrectFalseInvTimeSumPlainHours}{0.0\xspace}

  % finished-main
\providecommand{\predicateBitpreciseParallelInvariantsResultsAsyncInvariantsPrecPathCorrectFalseFinishedMainPlain}{}
  \renewcommand{\predicateBitpreciseParallelInvariantsResultsAsyncInvariantsPrecPathCorrectFalseFinishedMainPlain}{561\xspace}

 %% correct-true %%
\providecommand{\predicateBitpreciseParallelInvariantsResultsAsyncInvariantsPrecPathCorrectTruePlain}{}
  \renewcommand{\predicateBitpreciseParallelInvariantsResultsAsyncInvariantsPrecPathCorrectTruePlain}{1525\xspace}

  % cpu-time-sum
\providecommand{\predicateBitpreciseParallelInvariantsResultsAsyncInvariantsPrecPathCorrectTrueCpuTimeSumPlain}{}
  \renewcommand{\predicateBitpreciseParallelInvariantsResultsAsyncInvariantsPrecPathCorrectTrueCpuTimeSumPlain}{114817.1480200028\xspace}
\providecommand{\predicateBitpreciseParallelInvariantsResultsAsyncInvariantsPrecPathCorrectTrueCpuTimeSumPlainHours}{}
  \renewcommand{\predicateBitpreciseParallelInvariantsResultsAsyncInvariantsPrecPathCorrectTrueCpuTimeSumPlainHours}{31.893652227778556\xspace}

  % wall-time-sum
\providecommand{\predicateBitpreciseParallelInvariantsResultsAsyncInvariantsPrecPathCorrectTrueWallTimeSumPlain}{}
  \renewcommand{\predicateBitpreciseParallelInvariantsResultsAsyncInvariantsPrecPathCorrectTrueWallTimeSumPlain}{44749.942996270285\xspace}
\providecommand{\predicateBitpreciseParallelInvariantsResultsAsyncInvariantsPrecPathCorrectTrueWallTimeSumPlainHours}{}
  \renewcommand{\predicateBitpreciseParallelInvariantsResultsAsyncInvariantsPrecPathCorrectTrueWallTimeSumPlainHours}{12.430539721186191\xspace}

  % cpu-time-avg
\providecommand{\predicateBitpreciseParallelInvariantsResultsAsyncInvariantsPrecPathCorrectTrueCpuTimeAvgPlain}{}
  \renewcommand{\predicateBitpreciseParallelInvariantsResultsAsyncInvariantsPrecPathCorrectTrueCpuTimeAvgPlain}{75.2899331278707\xspace}
\providecommand{\predicateBitpreciseParallelInvariantsResultsAsyncInvariantsPrecPathCorrectTrueCpuTimeAvgPlainHours}{}
  \renewcommand{\predicateBitpreciseParallelInvariantsResultsAsyncInvariantsPrecPathCorrectTrueCpuTimeAvgPlainHours}{0.020913870313297417\xspace}

  % wall-time-avg
\providecommand{\predicateBitpreciseParallelInvariantsResultsAsyncInvariantsPrecPathCorrectTrueWallTimeAvgPlain}{}
  \renewcommand{\predicateBitpreciseParallelInvariantsResultsAsyncInvariantsPrecPathCorrectTrueWallTimeAvgPlain}{29.344224915587073\xspace}
\providecommand{\predicateBitpreciseParallelInvariantsResultsAsyncInvariantsPrecPathCorrectTrueWallTimeAvgPlainHours}{}
  \renewcommand{\predicateBitpreciseParallelInvariantsResultsAsyncInvariantsPrecPathCorrectTrueWallTimeAvgPlainHours}{0.008151173587663076\xspace}

  % inv-succ
\providecommand{\predicateBitpreciseParallelInvariantsResultsAsyncInvariantsPrecPathCorrectTrueInvSuccPlain}{}
  \renewcommand{\predicateBitpreciseParallelInvariantsResultsAsyncInvariantsPrecPathCorrectTrueInvSuccPlain}{0\xspace}

  % inv-tries
\providecommand{\predicateBitpreciseParallelInvariantsResultsAsyncInvariantsPrecPathCorrectTrueInvTriesPlain}{}
  \renewcommand{\predicateBitpreciseParallelInvariantsResultsAsyncInvariantsPrecPathCorrectTrueInvTriesPlain}{0\xspace}

  % inv-time-sum
\providecommand{\predicateBitpreciseParallelInvariantsResultsAsyncInvariantsPrecPathCorrectTrueInvTimeSumPlain}{}
  \renewcommand{\predicateBitpreciseParallelInvariantsResultsAsyncInvariantsPrecPathCorrectTrueInvTimeSumPlain}{0.0\xspace}
\providecommand{\predicateBitpreciseParallelInvariantsResultsAsyncInvariantsPrecPathCorrectTrueInvTimeSumPlainHours}{}
  \renewcommand{\predicateBitpreciseParallelInvariantsResultsAsyncInvariantsPrecPathCorrectTrueInvTimeSumPlainHours}{0.0\xspace}

  % finished-main
\providecommand{\predicateBitpreciseParallelInvariantsResultsAsyncInvariantsPrecPathCorrectTrueFinishedMainPlain}{}
  \renewcommand{\predicateBitpreciseParallelInvariantsResultsAsyncInvariantsPrecPathCorrectTrueFinishedMainPlain}{550\xspace}

 %% incorrect-false %%
\providecommand{\predicateBitpreciseParallelInvariantsResultsAsyncInvariantsPrecPathIncorrectFalsePlain}{}
  \renewcommand{\predicateBitpreciseParallelInvariantsResultsAsyncInvariantsPrecPathIncorrectFalsePlain}{17\xspace}

  % cpu-time-sum
\providecommand{\predicateBitpreciseParallelInvariantsResultsAsyncInvariantsPrecPathIncorrectFalseCpuTimeSumPlain}{}
  \renewcommand{\predicateBitpreciseParallelInvariantsResultsAsyncInvariantsPrecPathIncorrectFalseCpuTimeSumPlain}{1093.240709635\xspace}
\providecommand{\predicateBitpreciseParallelInvariantsResultsAsyncInvariantsPrecPathIncorrectFalseCpuTimeSumPlainHours}{}
  \renewcommand{\predicateBitpreciseParallelInvariantsResultsAsyncInvariantsPrecPathIncorrectFalseCpuTimeSumPlainHours}{0.30367797489861115\xspace}

  % wall-time-sum
\providecommand{\predicateBitpreciseParallelInvariantsResultsAsyncInvariantsPrecPathIncorrectFalseWallTimeSumPlain}{}
  \renewcommand{\predicateBitpreciseParallelInvariantsResultsAsyncInvariantsPrecPathIncorrectFalseWallTimeSumPlain}{558.58445572811\xspace}
\providecommand{\predicateBitpreciseParallelInvariantsResultsAsyncInvariantsPrecPathIncorrectFalseWallTimeSumPlainHours}{}
  \renewcommand{\predicateBitpreciseParallelInvariantsResultsAsyncInvariantsPrecPathIncorrectFalseWallTimeSumPlainHours}{0.15516234881336388\xspace}

  % cpu-time-avg
\providecommand{\predicateBitpreciseParallelInvariantsResultsAsyncInvariantsPrecPathIncorrectFalseCpuTimeAvgPlain}{}
  \renewcommand{\predicateBitpreciseParallelInvariantsResultsAsyncInvariantsPrecPathIncorrectFalseCpuTimeAvgPlain}{64.30827703735294\xspace}
\providecommand{\predicateBitpreciseParallelInvariantsResultsAsyncInvariantsPrecPathIncorrectFalseCpuTimeAvgPlainHours}{}
  \renewcommand{\predicateBitpreciseParallelInvariantsResultsAsyncInvariantsPrecPathIncorrectFalseCpuTimeAvgPlainHours}{0.017863410288153594\xspace}

  % wall-time-avg
\providecommand{\predicateBitpreciseParallelInvariantsResultsAsyncInvariantsPrecPathIncorrectFalseWallTimeAvgPlain}{}
  \renewcommand{\predicateBitpreciseParallelInvariantsResultsAsyncInvariantsPrecPathIncorrectFalseWallTimeAvgPlain}{32.85790916047706\xspace}
\providecommand{\predicateBitpreciseParallelInvariantsResultsAsyncInvariantsPrecPathIncorrectFalseWallTimeAvgPlainHours}{}
  \renewcommand{\predicateBitpreciseParallelInvariantsResultsAsyncInvariantsPrecPathIncorrectFalseWallTimeAvgPlainHours}{0.009127196989021405\xspace}

  % inv-succ
\providecommand{\predicateBitpreciseParallelInvariantsResultsAsyncInvariantsPrecPathIncorrectFalseInvSuccPlain}{}
  \renewcommand{\predicateBitpreciseParallelInvariantsResultsAsyncInvariantsPrecPathIncorrectFalseInvSuccPlain}{0\xspace}

  % inv-tries
\providecommand{\predicateBitpreciseParallelInvariantsResultsAsyncInvariantsPrecPathIncorrectFalseInvTriesPlain}{}
  \renewcommand{\predicateBitpreciseParallelInvariantsResultsAsyncInvariantsPrecPathIncorrectFalseInvTriesPlain}{0\xspace}

  % inv-time-sum
\providecommand{\predicateBitpreciseParallelInvariantsResultsAsyncInvariantsPrecPathIncorrectFalseInvTimeSumPlain}{}
  \renewcommand{\predicateBitpreciseParallelInvariantsResultsAsyncInvariantsPrecPathIncorrectFalseInvTimeSumPlain}{0.0\xspace}
\providecommand{\predicateBitpreciseParallelInvariantsResultsAsyncInvariantsPrecPathIncorrectFalseInvTimeSumPlainHours}{}
  \renewcommand{\predicateBitpreciseParallelInvariantsResultsAsyncInvariantsPrecPathIncorrectFalseInvTimeSumPlainHours}{0.0\xspace}

  % finished-main
\providecommand{\predicateBitpreciseParallelInvariantsResultsAsyncInvariantsPrecPathIncorrectFalseFinishedMainPlain}{}
  \renewcommand{\predicateBitpreciseParallelInvariantsResultsAsyncInvariantsPrecPathIncorrectFalseFinishedMainPlain}{17\xspace}

 %% incorrect-true %%
\providecommand{\predicateBitpreciseParallelInvariantsResultsAsyncInvariantsPrecPathIncorrectTruePlain}{}
  \renewcommand{\predicateBitpreciseParallelInvariantsResultsAsyncInvariantsPrecPathIncorrectTruePlain}{1\xspace}

  % cpu-time-sum
\providecommand{\predicateBitpreciseParallelInvariantsResultsAsyncInvariantsPrecPathIncorrectTrueCpuTimeSumPlain}{}
  \renewcommand{\predicateBitpreciseParallelInvariantsResultsAsyncInvariantsPrecPathIncorrectTrueCpuTimeSumPlain}{15.724842207\xspace}
\providecommand{\predicateBitpreciseParallelInvariantsResultsAsyncInvariantsPrecPathIncorrectTrueCpuTimeSumPlainHours}{}
  \renewcommand{\predicateBitpreciseParallelInvariantsResultsAsyncInvariantsPrecPathIncorrectTrueCpuTimeSumPlainHours}{0.0043680117241666665\xspace}

  % wall-time-sum
\providecommand{\predicateBitpreciseParallelInvariantsResultsAsyncInvariantsPrecPathIncorrectTrueWallTimeSumPlain}{}
  \renewcommand{\predicateBitpreciseParallelInvariantsResultsAsyncInvariantsPrecPathIncorrectTrueWallTimeSumPlain}{4.99566793442\xspace}
\providecommand{\predicateBitpreciseParallelInvariantsResultsAsyncInvariantsPrecPathIncorrectTrueWallTimeSumPlainHours}{}
  \renewcommand{\predicateBitpreciseParallelInvariantsResultsAsyncInvariantsPrecPathIncorrectTrueWallTimeSumPlainHours}{0.001387685537338889\xspace}

  % cpu-time-avg
\providecommand{\predicateBitpreciseParallelInvariantsResultsAsyncInvariantsPrecPathIncorrectTrueCpuTimeAvgPlain}{}
  \renewcommand{\predicateBitpreciseParallelInvariantsResultsAsyncInvariantsPrecPathIncorrectTrueCpuTimeAvgPlain}{15.724842207\xspace}
\providecommand{\predicateBitpreciseParallelInvariantsResultsAsyncInvariantsPrecPathIncorrectTrueCpuTimeAvgPlainHours}{}
  \renewcommand{\predicateBitpreciseParallelInvariantsResultsAsyncInvariantsPrecPathIncorrectTrueCpuTimeAvgPlainHours}{0.0043680117241666665\xspace}

  % wall-time-avg
\providecommand{\predicateBitpreciseParallelInvariantsResultsAsyncInvariantsPrecPathIncorrectTrueWallTimeAvgPlain}{}
  \renewcommand{\predicateBitpreciseParallelInvariantsResultsAsyncInvariantsPrecPathIncorrectTrueWallTimeAvgPlain}{4.99566793442\xspace}
\providecommand{\predicateBitpreciseParallelInvariantsResultsAsyncInvariantsPrecPathIncorrectTrueWallTimeAvgPlainHours}{}
  \renewcommand{\predicateBitpreciseParallelInvariantsResultsAsyncInvariantsPrecPathIncorrectTrueWallTimeAvgPlainHours}{0.001387685537338889\xspace}

  % inv-succ
\providecommand{\predicateBitpreciseParallelInvariantsResultsAsyncInvariantsPrecPathIncorrectTrueInvSuccPlain}{}
  \renewcommand{\predicateBitpreciseParallelInvariantsResultsAsyncInvariantsPrecPathIncorrectTrueInvSuccPlain}{0\xspace}

  % inv-tries
\providecommand{\predicateBitpreciseParallelInvariantsResultsAsyncInvariantsPrecPathIncorrectTrueInvTriesPlain}{}
  \renewcommand{\predicateBitpreciseParallelInvariantsResultsAsyncInvariantsPrecPathIncorrectTrueInvTriesPlain}{0\xspace}

  % inv-time-sum
\providecommand{\predicateBitpreciseParallelInvariantsResultsAsyncInvariantsPrecPathIncorrectTrueInvTimeSumPlain}{}
  \renewcommand{\predicateBitpreciseParallelInvariantsResultsAsyncInvariantsPrecPathIncorrectTrueInvTimeSumPlain}{0.0\xspace}
\providecommand{\predicateBitpreciseParallelInvariantsResultsAsyncInvariantsPrecPathIncorrectTrueInvTimeSumPlainHours}{}
  \renewcommand{\predicateBitpreciseParallelInvariantsResultsAsyncInvariantsPrecPathIncorrectTrueInvTimeSumPlainHours}{0.0\xspace}

  % finished-main
\providecommand{\predicateBitpreciseParallelInvariantsResultsAsyncInvariantsPrecPathIncorrectTrueFinishedMainPlain}{}
  \renewcommand{\predicateBitpreciseParallelInvariantsResultsAsyncInvariantsPrecPathIncorrectTrueFinishedMainPlain}{1\xspace}

 %% all %%
\providecommand{\predicateBitpreciseParallelInvariantsResultsAsyncInvariantsPrecPathAllPlain}{}
  \renewcommand{\predicateBitpreciseParallelInvariantsResultsAsyncInvariantsPrecPathAllPlain}{3488\xspace}

  % cpu-time-sum
\providecommand{\predicateBitpreciseParallelInvariantsResultsAsyncInvariantsPrecPathAllCpuTimeSumPlain}{}
  \renewcommand{\predicateBitpreciseParallelInvariantsResultsAsyncInvariantsPrecPathAllCpuTimeSumPlain}{1001447.8298559207\xspace}
\providecommand{\predicateBitpreciseParallelInvariantsResultsAsyncInvariantsPrecPathAllCpuTimeSumPlainHours}{}
  \renewcommand{\predicateBitpreciseParallelInvariantsResultsAsyncInvariantsPrecPathAllCpuTimeSumPlainHours}{278.17995273775574\xspace}

  % wall-time-sum
\providecommand{\predicateBitpreciseParallelInvariantsResultsAsyncInvariantsPrecPathAllWallTimeSumPlain}{}
  \renewcommand{\predicateBitpreciseParallelInvariantsResultsAsyncInvariantsPrecPathAllWallTimeSumPlain}{533788.3669974915\xspace}
\providecommand{\predicateBitpreciseParallelInvariantsResultsAsyncInvariantsPrecPathAllWallTimeSumPlainHours}{}
  \renewcommand{\predicateBitpreciseParallelInvariantsResultsAsyncInvariantsPrecPathAllWallTimeSumPlainHours}{148.27454638819208\xspace}

  % cpu-time-avg
\providecommand{\predicateBitpreciseParallelInvariantsResultsAsyncInvariantsPrecPathAllCpuTimeAvgPlain}{}
  \renewcommand{\predicateBitpreciseParallelInvariantsResultsAsyncInvariantsPrecPathAllCpuTimeAvgPlain}{287.1123365412617\xspace}
\providecommand{\predicateBitpreciseParallelInvariantsResultsAsyncInvariantsPrecPathAllCpuTimeAvgPlainHours}{}
  \renewcommand{\predicateBitpreciseParallelInvariantsResultsAsyncInvariantsPrecPathAllCpuTimeAvgPlainHours}{0.07975342681701714\xspace}

  % wall-time-avg
\providecommand{\predicateBitpreciseParallelInvariantsResultsAsyncInvariantsPrecPathAllWallTimeAvgPlain}{}
  \renewcommand{\predicateBitpreciseParallelInvariantsResultsAsyncInvariantsPrecPathAllWallTimeAvgPlain}{153.03565567588632\xspace}
\providecommand{\predicateBitpreciseParallelInvariantsResultsAsyncInvariantsPrecPathAllWallTimeAvgPlainHours}{}
  \renewcommand{\predicateBitpreciseParallelInvariantsResultsAsyncInvariantsPrecPathAllWallTimeAvgPlainHours}{0.04250990435441287\xspace}

  % inv-succ
\providecommand{\predicateBitpreciseParallelInvariantsResultsAsyncInvariantsPrecPathAllInvSuccPlain}{}
  \renewcommand{\predicateBitpreciseParallelInvariantsResultsAsyncInvariantsPrecPathAllInvSuccPlain}{0\xspace}

  % inv-tries
\providecommand{\predicateBitpreciseParallelInvariantsResultsAsyncInvariantsPrecPathAllInvTriesPlain}{}
  \renewcommand{\predicateBitpreciseParallelInvariantsResultsAsyncInvariantsPrecPathAllInvTriesPlain}{0\xspace}

  % inv-time-sum
\providecommand{\predicateBitpreciseParallelInvariantsResultsAsyncInvariantsPrecPathAllInvTimeSumPlain}{}
  \renewcommand{\predicateBitpreciseParallelInvariantsResultsAsyncInvariantsPrecPathAllInvTimeSumPlain}{0.0\xspace}
\providecommand{\predicateBitpreciseParallelInvariantsResultsAsyncInvariantsPrecPathAllInvTimeSumPlainHours}{}
  \renewcommand{\predicateBitpreciseParallelInvariantsResultsAsyncInvariantsPrecPathAllInvTimeSumPlainHours}{0.0\xspace}

  % finished-main
\providecommand{\predicateBitpreciseParallelInvariantsResultsAsyncInvariantsPrecPathAllFinishedMainPlain}{}
  \renewcommand{\predicateBitpreciseParallelInvariantsResultsAsyncInvariantsPrecPathAllFinishedMainPlain}{1131\xspace}

 %% equal-only %%
\providecommand{\predicateBitpreciseParallelInvariantsResultsAsyncInvariantsPrecPathEqualOnlyPlain}{}
  \renewcommand{\predicateBitpreciseParallelInvariantsResultsAsyncInvariantsPrecPathEqualOnlyPlain}{1865\xspace}

  % cpu-time-sum
\providecommand{\predicateBitpreciseParallelInvariantsResultsAsyncInvariantsPrecPathEqualOnlyCpuTimeSumPlain}{}
  \renewcommand{\predicateBitpreciseParallelInvariantsResultsAsyncInvariantsPrecPathEqualOnlyCpuTimeSumPlain}{141854.09937561603\xspace}
\providecommand{\predicateBitpreciseParallelInvariantsResultsAsyncInvariantsPrecPathEqualOnlyCpuTimeSumPlainHours}{}
  \renewcommand{\predicateBitpreciseParallelInvariantsResultsAsyncInvariantsPrecPathEqualOnlyCpuTimeSumPlainHours}{39.403916493226674\xspace}

  % wall-time-sum
\providecommand{\predicateBitpreciseParallelInvariantsResultsAsyncInvariantsPrecPathEqualOnlyWallTimeSumPlain}{}
  \renewcommand{\predicateBitpreciseParallelInvariantsResultsAsyncInvariantsPrecPathEqualOnlyWallTimeSumPlain}{54313.74700236485\xspace}
\providecommand{\predicateBitpreciseParallelInvariantsResultsAsyncInvariantsPrecPathEqualOnlyWallTimeSumPlainHours}{}
  \renewcommand{\predicateBitpreciseParallelInvariantsResultsAsyncInvariantsPrecPathEqualOnlyWallTimeSumPlainHours}{15.087151945101347\xspace}

  % cpu-time-avg
\providecommand{\predicateBitpreciseParallelInvariantsResultsAsyncInvariantsPrecPathEqualOnlyCpuTimeAvgPlain}{}
  \renewcommand{\predicateBitpreciseParallelInvariantsResultsAsyncInvariantsPrecPathEqualOnlyCpuTimeAvgPlain}{76.06117928987454\xspace}
\providecommand{\predicateBitpreciseParallelInvariantsResultsAsyncInvariantsPrecPathEqualOnlyCpuTimeAvgPlainHours}{}
  \renewcommand{\predicateBitpreciseParallelInvariantsResultsAsyncInvariantsPrecPathEqualOnlyCpuTimeAvgPlainHours}{0.021128105358298486\xspace}

  % wall-time-avg
\providecommand{\predicateBitpreciseParallelInvariantsResultsAsyncInvariantsPrecPathEqualOnlyWallTimeAvgPlain}{}
  \renewcommand{\predicateBitpreciseParallelInvariantsResultsAsyncInvariantsPrecPathEqualOnlyWallTimeAvgPlain}{29.122652548184906\xspace}
\providecommand{\predicateBitpreciseParallelInvariantsResultsAsyncInvariantsPrecPathEqualOnlyWallTimeAvgPlainHours}{}
  \renewcommand{\predicateBitpreciseParallelInvariantsResultsAsyncInvariantsPrecPathEqualOnlyWallTimeAvgPlainHours}{0.00808962570782914\xspace}

  % inv-succ
\providecommand{\predicateBitpreciseParallelInvariantsResultsAsyncInvariantsPrecPathEqualOnlyInvSuccPlain}{}
  \renewcommand{\predicateBitpreciseParallelInvariantsResultsAsyncInvariantsPrecPathEqualOnlyInvSuccPlain}{0\xspace}

  % inv-tries
\providecommand{\predicateBitpreciseParallelInvariantsResultsAsyncInvariantsPrecPathEqualOnlyInvTriesPlain}{}
  \renewcommand{\predicateBitpreciseParallelInvariantsResultsAsyncInvariantsPrecPathEqualOnlyInvTriesPlain}{0\xspace}

  % inv-time-sum
\providecommand{\predicateBitpreciseParallelInvariantsResultsAsyncInvariantsPrecPathEqualOnlyInvTimeSumPlain}{}
  \renewcommand{\predicateBitpreciseParallelInvariantsResultsAsyncInvariantsPrecPathEqualOnlyInvTimeSumPlain}{0.0\xspace}
\providecommand{\predicateBitpreciseParallelInvariantsResultsAsyncInvariantsPrecPathEqualOnlyInvTimeSumPlainHours}{}
  \renewcommand{\predicateBitpreciseParallelInvariantsResultsAsyncInvariantsPrecPathEqualOnlyInvTimeSumPlainHours}{0.0\xspace}

  % finished-main
\providecommand{\predicateBitpreciseParallelInvariantsResultsAsyncInvariantsPrecPathEqualOnlyFinishedMainPlain}{}
  \renewcommand{\predicateBitpreciseParallelInvariantsResultsAsyncInvariantsPrecPathEqualOnlyFinishedMainPlain}{1021\xspace}

%%% predicate_bitprecise_parallel_invariants.2016-09-05_0219.results.async-invariants-prec-abs %%%
 %% correct %%
\providecommand{\predicateBitpreciseParallelInvariantsResultsAsyncInvariantsPrecAbsCorrectPlain}{}
  \renewcommand{\predicateBitpreciseParallelInvariantsResultsAsyncInvariantsPrecAbsCorrectPlain}{2083\xspace}

  % cpu-time-sum
\providecommand{\predicateBitpreciseParallelInvariantsResultsAsyncInvariantsPrecAbsCorrectCpuTimeSumPlain}{}
  \renewcommand{\predicateBitpreciseParallelInvariantsResultsAsyncInvariantsPrecAbsCorrectCpuTimeSumPlain}{193273.8629155335\xspace}
\providecommand{\predicateBitpreciseParallelInvariantsResultsAsyncInvariantsPrecAbsCorrectCpuTimeSumPlainHours}{}
  \renewcommand{\predicateBitpreciseParallelInvariantsResultsAsyncInvariantsPrecAbsCorrectCpuTimeSumPlainHours}{53.68718414320375\xspace}

  % wall-time-sum
\providecommand{\predicateBitpreciseParallelInvariantsResultsAsyncInvariantsPrecAbsCorrectWallTimeSumPlain}{}
  \renewcommand{\predicateBitpreciseParallelInvariantsResultsAsyncInvariantsPrecAbsCorrectWallTimeSumPlain}{80057.72850609056\xspace}
\providecommand{\predicateBitpreciseParallelInvariantsResultsAsyncInvariantsPrecAbsCorrectWallTimeSumPlainHours}{}
  \renewcommand{\predicateBitpreciseParallelInvariantsResultsAsyncInvariantsPrecAbsCorrectWallTimeSumPlainHours}{22.23825791835849\xspace}

  % cpu-time-avg
\providecommand{\predicateBitpreciseParallelInvariantsResultsAsyncInvariantsPrecAbsCorrectCpuTimeAvgPlain}{}
  \renewcommand{\predicateBitpreciseParallelInvariantsResultsAsyncInvariantsPrecAbsCorrectCpuTimeAvgPlain}{92.78630000745727\xspace}
\providecommand{\predicateBitpreciseParallelInvariantsResultsAsyncInvariantsPrecAbsCorrectCpuTimeAvgPlainHours}{}
  \renewcommand{\predicateBitpreciseParallelInvariantsResultsAsyncInvariantsPrecAbsCorrectCpuTimeAvgPlainHours}{0.025773972224293687\xspace}

  % wall-time-avg
\providecommand{\predicateBitpreciseParallelInvariantsResultsAsyncInvariantsPrecAbsCorrectWallTimeAvgPlain}{}
  \renewcommand{\predicateBitpreciseParallelInvariantsResultsAsyncInvariantsPrecAbsCorrectWallTimeAvgPlain}{38.43385910037953\xspace}
\providecommand{\predicateBitpreciseParallelInvariantsResultsAsyncInvariantsPrecAbsCorrectWallTimeAvgPlainHours}{}
  \renewcommand{\predicateBitpreciseParallelInvariantsResultsAsyncInvariantsPrecAbsCorrectWallTimeAvgPlainHours}{0.010676071972327646\xspace}

  % inv-succ
\providecommand{\predicateBitpreciseParallelInvariantsResultsAsyncInvariantsPrecAbsCorrectInvSuccPlain}{}
  \renewcommand{\predicateBitpreciseParallelInvariantsResultsAsyncInvariantsPrecAbsCorrectInvSuccPlain}{0\xspace}

  % inv-tries
\providecommand{\predicateBitpreciseParallelInvariantsResultsAsyncInvariantsPrecAbsCorrectInvTriesPlain}{}
  \renewcommand{\predicateBitpreciseParallelInvariantsResultsAsyncInvariantsPrecAbsCorrectInvTriesPlain}{0\xspace}

  % inv-time-sum
\providecommand{\predicateBitpreciseParallelInvariantsResultsAsyncInvariantsPrecAbsCorrectInvTimeSumPlain}{}
  \renewcommand{\predicateBitpreciseParallelInvariantsResultsAsyncInvariantsPrecAbsCorrectInvTimeSumPlain}{0.0\xspace}
\providecommand{\predicateBitpreciseParallelInvariantsResultsAsyncInvariantsPrecAbsCorrectInvTimeSumPlainHours}{}
  \renewcommand{\predicateBitpreciseParallelInvariantsResultsAsyncInvariantsPrecAbsCorrectInvTimeSumPlainHours}{0.0\xspace}

  % finished-main
\providecommand{\predicateBitpreciseParallelInvariantsResultsAsyncInvariantsPrecAbsCorrectFinishedMainPlain}{}
  \renewcommand{\predicateBitpreciseParallelInvariantsResultsAsyncInvariantsPrecAbsCorrectFinishedMainPlain}{1110\xspace}

 %% incorrect %%
\providecommand{\predicateBitpreciseParallelInvariantsResultsAsyncInvariantsPrecAbsIncorrectPlain}{}
  \renewcommand{\predicateBitpreciseParallelInvariantsResultsAsyncInvariantsPrecAbsIncorrectPlain}{18\xspace}

  % cpu-time-sum
\providecommand{\predicateBitpreciseParallelInvariantsResultsAsyncInvariantsPrecAbsIncorrectCpuTimeSumPlain}{}
  \renewcommand{\predicateBitpreciseParallelInvariantsResultsAsyncInvariantsPrecAbsIncorrectCpuTimeSumPlain}{1271.545953851\xspace}
\providecommand{\predicateBitpreciseParallelInvariantsResultsAsyncInvariantsPrecAbsIncorrectCpuTimeSumPlainHours}{}
  \renewcommand{\predicateBitpreciseParallelInvariantsResultsAsyncInvariantsPrecAbsIncorrectCpuTimeSumPlainHours}{0.35320720940305556\xspace}

  % wall-time-sum
\providecommand{\predicateBitpreciseParallelInvariantsResultsAsyncInvariantsPrecAbsIncorrectWallTimeSumPlain}{}
  \renewcommand{\predicateBitpreciseParallelInvariantsResultsAsyncInvariantsPrecAbsIncorrectWallTimeSumPlain}{663.9508917329998\xspace}
\providecommand{\predicateBitpreciseParallelInvariantsResultsAsyncInvariantsPrecAbsIncorrectWallTimeSumPlainHours}{}
  \renewcommand{\predicateBitpreciseParallelInvariantsResultsAsyncInvariantsPrecAbsIncorrectWallTimeSumPlainHours}{0.1844308032591666\xspace}

  % cpu-time-avg
\providecommand{\predicateBitpreciseParallelInvariantsResultsAsyncInvariantsPrecAbsIncorrectCpuTimeAvgPlain}{}
  \renewcommand{\predicateBitpreciseParallelInvariantsResultsAsyncInvariantsPrecAbsIncorrectCpuTimeAvgPlain}{70.64144188061111\xspace}
\providecommand{\predicateBitpreciseParallelInvariantsResultsAsyncInvariantsPrecAbsIncorrectCpuTimeAvgPlainHours}{}
  \renewcommand{\predicateBitpreciseParallelInvariantsResultsAsyncInvariantsPrecAbsIncorrectCpuTimeAvgPlainHours}{0.019622622744614196\xspace}

  % wall-time-avg
\providecommand{\predicateBitpreciseParallelInvariantsResultsAsyncInvariantsPrecAbsIncorrectWallTimeAvgPlain}{}
  \renewcommand{\predicateBitpreciseParallelInvariantsResultsAsyncInvariantsPrecAbsIncorrectWallTimeAvgPlain}{36.88616065183332\xspace}
\providecommand{\predicateBitpreciseParallelInvariantsResultsAsyncInvariantsPrecAbsIncorrectWallTimeAvgPlainHours}{}
  \renewcommand{\predicateBitpreciseParallelInvariantsResultsAsyncInvariantsPrecAbsIncorrectWallTimeAvgPlainHours}{0.010246155736620367\xspace}

  % inv-succ
\providecommand{\predicateBitpreciseParallelInvariantsResultsAsyncInvariantsPrecAbsIncorrectInvSuccPlain}{}
  \renewcommand{\predicateBitpreciseParallelInvariantsResultsAsyncInvariantsPrecAbsIncorrectInvSuccPlain}{0\xspace}

  % inv-tries
\providecommand{\predicateBitpreciseParallelInvariantsResultsAsyncInvariantsPrecAbsIncorrectInvTriesPlain}{}
  \renewcommand{\predicateBitpreciseParallelInvariantsResultsAsyncInvariantsPrecAbsIncorrectInvTriesPlain}{0\xspace}

  % inv-time-sum
\providecommand{\predicateBitpreciseParallelInvariantsResultsAsyncInvariantsPrecAbsIncorrectInvTimeSumPlain}{}
  \renewcommand{\predicateBitpreciseParallelInvariantsResultsAsyncInvariantsPrecAbsIncorrectInvTimeSumPlain}{0.0\xspace}
\providecommand{\predicateBitpreciseParallelInvariantsResultsAsyncInvariantsPrecAbsIncorrectInvTimeSumPlainHours}{}
  \renewcommand{\predicateBitpreciseParallelInvariantsResultsAsyncInvariantsPrecAbsIncorrectInvTimeSumPlainHours}{0.0\xspace}

  % finished-main
\providecommand{\predicateBitpreciseParallelInvariantsResultsAsyncInvariantsPrecAbsIncorrectFinishedMainPlain}{}
  \renewcommand{\predicateBitpreciseParallelInvariantsResultsAsyncInvariantsPrecAbsIncorrectFinishedMainPlain}{18\xspace}

 %% timeout %%
\providecommand{\predicateBitpreciseParallelInvariantsResultsAsyncInvariantsPrecAbsTimeoutPlain}{}
  \renewcommand{\predicateBitpreciseParallelInvariantsResultsAsyncInvariantsPrecAbsTimeoutPlain}{1216\xspace}

  % cpu-time-sum
\providecommand{\predicateBitpreciseParallelInvariantsResultsAsyncInvariantsPrecAbsTimeoutCpuTimeSumPlain}{}
  \renewcommand{\predicateBitpreciseParallelInvariantsResultsAsyncInvariantsPrecAbsTimeoutCpuTimeSumPlain}{747833.3609568239\xspace}
\providecommand{\predicateBitpreciseParallelInvariantsResultsAsyncInvariantsPrecAbsTimeoutCpuTimeSumPlainHours}{}
  \renewcommand{\predicateBitpreciseParallelInvariantsResultsAsyncInvariantsPrecAbsTimeoutCpuTimeSumPlainHours}{207.7314891546733\xspace}

  % wall-time-sum
\providecommand{\predicateBitpreciseParallelInvariantsResultsAsyncInvariantsPrecAbsTimeoutWallTimeSumPlain}{}
  \renewcommand{\predicateBitpreciseParallelInvariantsResultsAsyncInvariantsPrecAbsTimeoutWallTimeSumPlain}{416673.4468863032\xspace}
\providecommand{\predicateBitpreciseParallelInvariantsResultsAsyncInvariantsPrecAbsTimeoutWallTimeSumPlainHours}{}
  \renewcommand{\predicateBitpreciseParallelInvariantsResultsAsyncInvariantsPrecAbsTimeoutWallTimeSumPlainHours}{115.74262413508423\xspace}

  % cpu-time-avg
\providecommand{\predicateBitpreciseParallelInvariantsResultsAsyncInvariantsPrecAbsTimeoutCpuTimeAvgPlain}{}
  \renewcommand{\predicateBitpreciseParallelInvariantsResultsAsyncInvariantsPrecAbsTimeoutCpuTimeAvgPlain}{614.9945402605459\xspace}
\providecommand{\predicateBitpreciseParallelInvariantsResultsAsyncInvariantsPrecAbsTimeoutCpuTimeAvgPlainHours}{}
  \renewcommand{\predicateBitpreciseParallelInvariantsResultsAsyncInvariantsPrecAbsTimeoutCpuTimeAvgPlainHours}{0.17083181673904055\xspace}

  % wall-time-avg
\providecommand{\predicateBitpreciseParallelInvariantsResultsAsyncInvariantsPrecAbsTimeoutWallTimeAvgPlain}{}
  \renewcommand{\predicateBitpreciseParallelInvariantsResultsAsyncInvariantsPrecAbsTimeoutWallTimeAvgPlain}{342.6590846104467\xspace}
\providecommand{\predicateBitpreciseParallelInvariantsResultsAsyncInvariantsPrecAbsTimeoutWallTimeAvgPlainHours}{}
  \renewcommand{\predicateBitpreciseParallelInvariantsResultsAsyncInvariantsPrecAbsTimeoutWallTimeAvgPlainHours}{0.09518307905845742\xspace}

  % inv-succ
\providecommand{\predicateBitpreciseParallelInvariantsResultsAsyncInvariantsPrecAbsTimeoutInvSuccPlain}{}
  \renewcommand{\predicateBitpreciseParallelInvariantsResultsAsyncInvariantsPrecAbsTimeoutInvSuccPlain}{0\xspace}

  % inv-tries
\providecommand{\predicateBitpreciseParallelInvariantsResultsAsyncInvariantsPrecAbsTimeoutInvTriesPlain}{}
  \renewcommand{\predicateBitpreciseParallelInvariantsResultsAsyncInvariantsPrecAbsTimeoutInvTriesPlain}{0\xspace}

  % inv-time-sum
\providecommand{\predicateBitpreciseParallelInvariantsResultsAsyncInvariantsPrecAbsTimeoutInvTimeSumPlain}{}
  \renewcommand{\predicateBitpreciseParallelInvariantsResultsAsyncInvariantsPrecAbsTimeoutInvTimeSumPlain}{0.0\xspace}
\providecommand{\predicateBitpreciseParallelInvariantsResultsAsyncInvariantsPrecAbsTimeoutInvTimeSumPlainHours}{}
  \renewcommand{\predicateBitpreciseParallelInvariantsResultsAsyncInvariantsPrecAbsTimeoutInvTimeSumPlainHours}{0.0\xspace}

  % finished-main
\providecommand{\predicateBitpreciseParallelInvariantsResultsAsyncInvariantsPrecAbsTimeoutFinishedMainPlain}{}
  \renewcommand{\predicateBitpreciseParallelInvariantsResultsAsyncInvariantsPrecAbsTimeoutFinishedMainPlain}{0\xspace}

 %% unknown-or-category-error %%
\providecommand{\predicateBitpreciseParallelInvariantsResultsAsyncInvariantsPrecAbsUnknownOrCategoryErrorPlain}{}
  \renewcommand{\predicateBitpreciseParallelInvariantsResultsAsyncInvariantsPrecAbsUnknownOrCategoryErrorPlain}{1387\xspace}

  % cpu-time-sum
\providecommand{\predicateBitpreciseParallelInvariantsResultsAsyncInvariantsPrecAbsUnknownOrCategoryErrorCpuTimeSumPlain}{}
  \renewcommand{\predicateBitpreciseParallelInvariantsResultsAsyncInvariantsPrecAbsUnknownOrCategoryErrorCpuTimeSumPlain}{808865.0782154739\xspace}
\providecommand{\predicateBitpreciseParallelInvariantsResultsAsyncInvariantsPrecAbsUnknownOrCategoryErrorCpuTimeSumPlainHours}{}
  \renewcommand{\predicateBitpreciseParallelInvariantsResultsAsyncInvariantsPrecAbsUnknownOrCategoryErrorCpuTimeSumPlainHours}{224.68474394874275\xspace}

  % wall-time-sum
\providecommand{\predicateBitpreciseParallelInvariantsResultsAsyncInvariantsPrecAbsUnknownOrCategoryErrorWallTimeSumPlain}{}
  \renewcommand{\predicateBitpreciseParallelInvariantsResultsAsyncInvariantsPrecAbsUnknownOrCategoryErrorWallTimeSumPlain}{454399.14287043473\xspace}
\providecommand{\predicateBitpreciseParallelInvariantsResultsAsyncInvariantsPrecAbsUnknownOrCategoryErrorWallTimeSumPlainHours}{}
  \renewcommand{\predicateBitpreciseParallelInvariantsResultsAsyncInvariantsPrecAbsUnknownOrCategoryErrorWallTimeSumPlainHours}{126.22198413067632\xspace}

  % cpu-time-avg
\providecommand{\predicateBitpreciseParallelInvariantsResultsAsyncInvariantsPrecAbsUnknownOrCategoryErrorCpuTimeAvgPlain}{}
  \renewcommand{\predicateBitpreciseParallelInvariantsResultsAsyncInvariantsPrecAbsUnknownOrCategoryErrorCpuTimeAvgPlain}{583.1759756420144\xspace}
\providecommand{\predicateBitpreciseParallelInvariantsResultsAsyncInvariantsPrecAbsUnknownOrCategoryErrorCpuTimeAvgPlainHours}{}
  \renewcommand{\predicateBitpreciseParallelInvariantsResultsAsyncInvariantsPrecAbsUnknownOrCategoryErrorCpuTimeAvgPlainHours}{0.1619933265672262\xspace}

  % wall-time-avg
\providecommand{\predicateBitpreciseParallelInvariantsResultsAsyncInvariantsPrecAbsUnknownOrCategoryErrorWallTimeAvgPlain}{}
  \renewcommand{\predicateBitpreciseParallelInvariantsResultsAsyncInvariantsPrecAbsUnknownOrCategoryErrorWallTimeAvgPlain}{327.612936460299\xspace}
\providecommand{\predicateBitpreciseParallelInvariantsResultsAsyncInvariantsPrecAbsUnknownOrCategoryErrorWallTimeAvgPlainHours}{}
  \renewcommand{\predicateBitpreciseParallelInvariantsResultsAsyncInvariantsPrecAbsUnknownOrCategoryErrorWallTimeAvgPlainHours}{0.09100359346119417\xspace}

  % inv-succ
\providecommand{\predicateBitpreciseParallelInvariantsResultsAsyncInvariantsPrecAbsUnknownOrCategoryErrorInvSuccPlain}{}
  \renewcommand{\predicateBitpreciseParallelInvariantsResultsAsyncInvariantsPrecAbsUnknownOrCategoryErrorInvSuccPlain}{0\xspace}

  % inv-tries
\providecommand{\predicateBitpreciseParallelInvariantsResultsAsyncInvariantsPrecAbsUnknownOrCategoryErrorInvTriesPlain}{}
  \renewcommand{\predicateBitpreciseParallelInvariantsResultsAsyncInvariantsPrecAbsUnknownOrCategoryErrorInvTriesPlain}{0\xspace}

  % inv-time-sum
\providecommand{\predicateBitpreciseParallelInvariantsResultsAsyncInvariantsPrecAbsUnknownOrCategoryErrorInvTimeSumPlain}{}
  \renewcommand{\predicateBitpreciseParallelInvariantsResultsAsyncInvariantsPrecAbsUnknownOrCategoryErrorInvTimeSumPlain}{0.0\xspace}
\providecommand{\predicateBitpreciseParallelInvariantsResultsAsyncInvariantsPrecAbsUnknownOrCategoryErrorInvTimeSumPlainHours}{}
  \renewcommand{\predicateBitpreciseParallelInvariantsResultsAsyncInvariantsPrecAbsUnknownOrCategoryErrorInvTimeSumPlainHours}{0.0\xspace}

  % finished-main
\providecommand{\predicateBitpreciseParallelInvariantsResultsAsyncInvariantsPrecAbsUnknownOrCategoryErrorFinishedMainPlain}{}
  \renewcommand{\predicateBitpreciseParallelInvariantsResultsAsyncInvariantsPrecAbsUnknownOrCategoryErrorFinishedMainPlain}{1\xspace}

 %% correct-false %%
\providecommand{\predicateBitpreciseParallelInvariantsResultsAsyncInvariantsPrecAbsCorrectFalsePlain}{}
  \renewcommand{\predicateBitpreciseParallelInvariantsResultsAsyncInvariantsPrecAbsCorrectFalsePlain}{557\xspace}

  % cpu-time-sum
\providecommand{\predicateBitpreciseParallelInvariantsResultsAsyncInvariantsPrecAbsCorrectFalseCpuTimeSumPlain}{}
  \renewcommand{\predicateBitpreciseParallelInvariantsResultsAsyncInvariantsPrecAbsCorrectFalseCpuTimeSumPlain}{76908.26519517993\xspace}
\providecommand{\predicateBitpreciseParallelInvariantsResultsAsyncInvariantsPrecAbsCorrectFalseCpuTimeSumPlainHours}{}
  \renewcommand{\predicateBitpreciseParallelInvariantsResultsAsyncInvariantsPrecAbsCorrectFalseCpuTimeSumPlainHours}{21.363406998661095\xspace}

  % wall-time-sum
\providecommand{\predicateBitpreciseParallelInvariantsResultsAsyncInvariantsPrecAbsCorrectFalseWallTimeSumPlain}{}
  \renewcommand{\predicateBitpreciseParallelInvariantsResultsAsyncInvariantsPrecAbsCorrectFalseWallTimeSumPlain}{34129.746963505015\xspace}
\providecommand{\predicateBitpreciseParallelInvariantsResultsAsyncInvariantsPrecAbsCorrectFalseWallTimeSumPlainHours}{}
  \renewcommand{\predicateBitpreciseParallelInvariantsResultsAsyncInvariantsPrecAbsCorrectFalseWallTimeSumPlainHours}{9.480485267640281\xspace}

  % cpu-time-avg
\providecommand{\predicateBitpreciseParallelInvariantsResultsAsyncInvariantsPrecAbsCorrectFalseCpuTimeAvgPlain}{}
  \renewcommand{\predicateBitpreciseParallelInvariantsResultsAsyncInvariantsPrecAbsCorrectFalseCpuTimeAvgPlain}{138.07588006315967\xspace}
\providecommand{\predicateBitpreciseParallelInvariantsResultsAsyncInvariantsPrecAbsCorrectFalseCpuTimeAvgPlainHours}{}
  \renewcommand{\predicateBitpreciseParallelInvariantsResultsAsyncInvariantsPrecAbsCorrectFalseCpuTimeAvgPlainHours}{0.038354411128655466\xspace}

  % wall-time-avg
\providecommand{\predicateBitpreciseParallelInvariantsResultsAsyncInvariantsPrecAbsCorrectFalseWallTimeAvgPlain}{}
  \renewcommand{\predicateBitpreciseParallelInvariantsResultsAsyncInvariantsPrecAbsCorrectFalseWallTimeAvgPlain}{61.27423153232498\xspace}
\providecommand{\predicateBitpreciseParallelInvariantsResultsAsyncInvariantsPrecAbsCorrectFalseWallTimeAvgPlainHours}{}
  \renewcommand{\predicateBitpreciseParallelInvariantsResultsAsyncInvariantsPrecAbsCorrectFalseWallTimeAvgPlainHours}{0.01702061987009027\xspace}

  % inv-succ
\providecommand{\predicateBitpreciseParallelInvariantsResultsAsyncInvariantsPrecAbsCorrectFalseInvSuccPlain}{}
  \renewcommand{\predicateBitpreciseParallelInvariantsResultsAsyncInvariantsPrecAbsCorrectFalseInvSuccPlain}{0\xspace}

  % inv-tries
\providecommand{\predicateBitpreciseParallelInvariantsResultsAsyncInvariantsPrecAbsCorrectFalseInvTriesPlain}{}
  \renewcommand{\predicateBitpreciseParallelInvariantsResultsAsyncInvariantsPrecAbsCorrectFalseInvTriesPlain}{0\xspace}

  % inv-time-sum
\providecommand{\predicateBitpreciseParallelInvariantsResultsAsyncInvariantsPrecAbsCorrectFalseInvTimeSumPlain}{}
  \renewcommand{\predicateBitpreciseParallelInvariantsResultsAsyncInvariantsPrecAbsCorrectFalseInvTimeSumPlain}{0.0\xspace}
\providecommand{\predicateBitpreciseParallelInvariantsResultsAsyncInvariantsPrecAbsCorrectFalseInvTimeSumPlainHours}{}
  \renewcommand{\predicateBitpreciseParallelInvariantsResultsAsyncInvariantsPrecAbsCorrectFalseInvTimeSumPlainHours}{0.0\xspace}

  % finished-main
\providecommand{\predicateBitpreciseParallelInvariantsResultsAsyncInvariantsPrecAbsCorrectFalseFinishedMainPlain}{}
  \renewcommand{\predicateBitpreciseParallelInvariantsResultsAsyncInvariantsPrecAbsCorrectFalseFinishedMainPlain}{557\xspace}

 %% correct-true %%
\providecommand{\predicateBitpreciseParallelInvariantsResultsAsyncInvariantsPrecAbsCorrectTruePlain}{}
  \renewcommand{\predicateBitpreciseParallelInvariantsResultsAsyncInvariantsPrecAbsCorrectTruePlain}{1526\xspace}

  % cpu-time-sum
\providecommand{\predicateBitpreciseParallelInvariantsResultsAsyncInvariantsPrecAbsCorrectTrueCpuTimeSumPlain}{}
  \renewcommand{\predicateBitpreciseParallelInvariantsResultsAsyncInvariantsPrecAbsCorrectTrueCpuTimeSumPlain}{116365.59772035287\xspace}
\providecommand{\predicateBitpreciseParallelInvariantsResultsAsyncInvariantsPrecAbsCorrectTrueCpuTimeSumPlainHours}{}
  \renewcommand{\predicateBitpreciseParallelInvariantsResultsAsyncInvariantsPrecAbsCorrectTrueCpuTimeSumPlainHours}{32.32377714454247\xspace}

  % wall-time-sum
\providecommand{\predicateBitpreciseParallelInvariantsResultsAsyncInvariantsPrecAbsCorrectTrueWallTimeSumPlain}{}
  \renewcommand{\predicateBitpreciseParallelInvariantsResultsAsyncInvariantsPrecAbsCorrectTrueWallTimeSumPlain}{45927.98154258556\xspace}
\providecommand{\predicateBitpreciseParallelInvariantsResultsAsyncInvariantsPrecAbsCorrectTrueWallTimeSumPlainHours}{}
  \renewcommand{\predicateBitpreciseParallelInvariantsResultsAsyncInvariantsPrecAbsCorrectTrueWallTimeSumPlainHours}{12.757772650718213\xspace}

  % cpu-time-avg
\providecommand{\predicateBitpreciseParallelInvariantsResultsAsyncInvariantsPrecAbsCorrectTrueCpuTimeAvgPlain}{}
  \renewcommand{\predicateBitpreciseParallelInvariantsResultsAsyncInvariantsPrecAbsCorrectTrueCpuTimeAvgPlain}{76.25530650088655\xspace}
\providecommand{\predicateBitpreciseParallelInvariantsResultsAsyncInvariantsPrecAbsCorrectTrueCpuTimeAvgPlainHours}{}
  \renewcommand{\predicateBitpreciseParallelInvariantsResultsAsyncInvariantsPrecAbsCorrectTrueCpuTimeAvgPlainHours}{0.021182029583579596\xspace}

  % wall-time-avg
\providecommand{\predicateBitpreciseParallelInvariantsResultsAsyncInvariantsPrecAbsCorrectTrueWallTimeAvgPlain}{}
  \renewcommand{\predicateBitpreciseParallelInvariantsResultsAsyncInvariantsPrecAbsCorrectTrueWallTimeAvgPlain}{30.09697348793287\xspace}
\providecommand{\predicateBitpreciseParallelInvariantsResultsAsyncInvariantsPrecAbsCorrectTrueWallTimeAvgPlainHours}{}
  \renewcommand{\predicateBitpreciseParallelInvariantsResultsAsyncInvariantsPrecAbsCorrectTrueWallTimeAvgPlainHours}{0.008360270413314686\xspace}

  % inv-succ
\providecommand{\predicateBitpreciseParallelInvariantsResultsAsyncInvariantsPrecAbsCorrectTrueInvSuccPlain}{}
  \renewcommand{\predicateBitpreciseParallelInvariantsResultsAsyncInvariantsPrecAbsCorrectTrueInvSuccPlain}{0\xspace}

  % inv-tries
\providecommand{\predicateBitpreciseParallelInvariantsResultsAsyncInvariantsPrecAbsCorrectTrueInvTriesPlain}{}
  \renewcommand{\predicateBitpreciseParallelInvariantsResultsAsyncInvariantsPrecAbsCorrectTrueInvTriesPlain}{0\xspace}

  % inv-time-sum
\providecommand{\predicateBitpreciseParallelInvariantsResultsAsyncInvariantsPrecAbsCorrectTrueInvTimeSumPlain}{}
  \renewcommand{\predicateBitpreciseParallelInvariantsResultsAsyncInvariantsPrecAbsCorrectTrueInvTimeSumPlain}{0.0\xspace}
\providecommand{\predicateBitpreciseParallelInvariantsResultsAsyncInvariantsPrecAbsCorrectTrueInvTimeSumPlainHours}{}
  \renewcommand{\predicateBitpreciseParallelInvariantsResultsAsyncInvariantsPrecAbsCorrectTrueInvTimeSumPlainHours}{0.0\xspace}

  % finished-main
\providecommand{\predicateBitpreciseParallelInvariantsResultsAsyncInvariantsPrecAbsCorrectTrueFinishedMainPlain}{}
  \renewcommand{\predicateBitpreciseParallelInvariantsResultsAsyncInvariantsPrecAbsCorrectTrueFinishedMainPlain}{553\xspace}

 %% incorrect-false %%
\providecommand{\predicateBitpreciseParallelInvariantsResultsAsyncInvariantsPrecAbsIncorrectFalsePlain}{}
  \renewcommand{\predicateBitpreciseParallelInvariantsResultsAsyncInvariantsPrecAbsIncorrectFalsePlain}{18\xspace}

  % cpu-time-sum
\providecommand{\predicateBitpreciseParallelInvariantsResultsAsyncInvariantsPrecAbsIncorrectFalseCpuTimeSumPlain}{}
  \renewcommand{\predicateBitpreciseParallelInvariantsResultsAsyncInvariantsPrecAbsIncorrectFalseCpuTimeSumPlain}{1271.545953851\xspace}
\providecommand{\predicateBitpreciseParallelInvariantsResultsAsyncInvariantsPrecAbsIncorrectFalseCpuTimeSumPlainHours}{}
  \renewcommand{\predicateBitpreciseParallelInvariantsResultsAsyncInvariantsPrecAbsIncorrectFalseCpuTimeSumPlainHours}{0.35320720940305556\xspace}

  % wall-time-sum
\providecommand{\predicateBitpreciseParallelInvariantsResultsAsyncInvariantsPrecAbsIncorrectFalseWallTimeSumPlain}{}
  \renewcommand{\predicateBitpreciseParallelInvariantsResultsAsyncInvariantsPrecAbsIncorrectFalseWallTimeSumPlain}{663.9508917329998\xspace}
\providecommand{\predicateBitpreciseParallelInvariantsResultsAsyncInvariantsPrecAbsIncorrectFalseWallTimeSumPlainHours}{}
  \renewcommand{\predicateBitpreciseParallelInvariantsResultsAsyncInvariantsPrecAbsIncorrectFalseWallTimeSumPlainHours}{0.1844308032591666\xspace}

  % cpu-time-avg
\providecommand{\predicateBitpreciseParallelInvariantsResultsAsyncInvariantsPrecAbsIncorrectFalseCpuTimeAvgPlain}{}
  \renewcommand{\predicateBitpreciseParallelInvariantsResultsAsyncInvariantsPrecAbsIncorrectFalseCpuTimeAvgPlain}{70.64144188061111\xspace}
\providecommand{\predicateBitpreciseParallelInvariantsResultsAsyncInvariantsPrecAbsIncorrectFalseCpuTimeAvgPlainHours}{}
  \renewcommand{\predicateBitpreciseParallelInvariantsResultsAsyncInvariantsPrecAbsIncorrectFalseCpuTimeAvgPlainHours}{0.019622622744614196\xspace}

  % wall-time-avg
\providecommand{\predicateBitpreciseParallelInvariantsResultsAsyncInvariantsPrecAbsIncorrectFalseWallTimeAvgPlain}{}
  \renewcommand{\predicateBitpreciseParallelInvariantsResultsAsyncInvariantsPrecAbsIncorrectFalseWallTimeAvgPlain}{36.88616065183332\xspace}
\providecommand{\predicateBitpreciseParallelInvariantsResultsAsyncInvariantsPrecAbsIncorrectFalseWallTimeAvgPlainHours}{}
  \renewcommand{\predicateBitpreciseParallelInvariantsResultsAsyncInvariantsPrecAbsIncorrectFalseWallTimeAvgPlainHours}{0.010246155736620367\xspace}

  % inv-succ
\providecommand{\predicateBitpreciseParallelInvariantsResultsAsyncInvariantsPrecAbsIncorrectFalseInvSuccPlain}{}
  \renewcommand{\predicateBitpreciseParallelInvariantsResultsAsyncInvariantsPrecAbsIncorrectFalseInvSuccPlain}{0\xspace}

  % inv-tries
\providecommand{\predicateBitpreciseParallelInvariantsResultsAsyncInvariantsPrecAbsIncorrectFalseInvTriesPlain}{}
  \renewcommand{\predicateBitpreciseParallelInvariantsResultsAsyncInvariantsPrecAbsIncorrectFalseInvTriesPlain}{0\xspace}

  % inv-time-sum
\providecommand{\predicateBitpreciseParallelInvariantsResultsAsyncInvariantsPrecAbsIncorrectFalseInvTimeSumPlain}{}
  \renewcommand{\predicateBitpreciseParallelInvariantsResultsAsyncInvariantsPrecAbsIncorrectFalseInvTimeSumPlain}{0.0\xspace}
\providecommand{\predicateBitpreciseParallelInvariantsResultsAsyncInvariantsPrecAbsIncorrectFalseInvTimeSumPlainHours}{}
  \renewcommand{\predicateBitpreciseParallelInvariantsResultsAsyncInvariantsPrecAbsIncorrectFalseInvTimeSumPlainHours}{0.0\xspace}

  % finished-main
\providecommand{\predicateBitpreciseParallelInvariantsResultsAsyncInvariantsPrecAbsIncorrectFalseFinishedMainPlain}{}
  \renewcommand{\predicateBitpreciseParallelInvariantsResultsAsyncInvariantsPrecAbsIncorrectFalseFinishedMainPlain}{18\xspace}

 %% incorrect-true %%
\providecommand{\predicateBitpreciseParallelInvariantsResultsAsyncInvariantsPrecAbsIncorrectTruePlain}{}
  \renewcommand{\predicateBitpreciseParallelInvariantsResultsAsyncInvariantsPrecAbsIncorrectTruePlain}{0\xspace}

  % cpu-time-sum
\providecommand{\predicateBitpreciseParallelInvariantsResultsAsyncInvariantsPrecAbsIncorrectTrueCpuTimeSumPlain}{}
  \renewcommand{\predicateBitpreciseParallelInvariantsResultsAsyncInvariantsPrecAbsIncorrectTrueCpuTimeSumPlain}{0.0\xspace}
\providecommand{\predicateBitpreciseParallelInvariantsResultsAsyncInvariantsPrecAbsIncorrectTrueCpuTimeSumPlainHours}{}
  \renewcommand{\predicateBitpreciseParallelInvariantsResultsAsyncInvariantsPrecAbsIncorrectTrueCpuTimeSumPlainHours}{0.0\xspace}

  % wall-time-sum
\providecommand{\predicateBitpreciseParallelInvariantsResultsAsyncInvariantsPrecAbsIncorrectTrueWallTimeSumPlain}{}
  \renewcommand{\predicateBitpreciseParallelInvariantsResultsAsyncInvariantsPrecAbsIncorrectTrueWallTimeSumPlain}{0.0\xspace}
\providecommand{\predicateBitpreciseParallelInvariantsResultsAsyncInvariantsPrecAbsIncorrectTrueWallTimeSumPlainHours}{}
  \renewcommand{\predicateBitpreciseParallelInvariantsResultsAsyncInvariantsPrecAbsIncorrectTrueWallTimeSumPlainHours}{0.0\xspace}

  % cpu-time-avg
\providecommand{\predicateBitpreciseParallelInvariantsResultsAsyncInvariantsPrecAbsIncorrectTrueCpuTimeAvgPlain}{}
  \renewcommand{\predicateBitpreciseParallelInvariantsResultsAsyncInvariantsPrecAbsIncorrectTrueCpuTimeAvgPlain}{NaN\xspace}
\providecommand{\predicateBitpreciseParallelInvariantsResultsAsyncInvariantsPrecAbsIncorrectTrueCpuTimeAvgPlainHours}{}
  \renewcommand{\predicateBitpreciseParallelInvariantsResultsAsyncInvariantsPrecAbsIncorrectTrueCpuTimeAvgPlainHours}{NaN\xspace}

  % wall-time-avg
\providecommand{\predicateBitpreciseParallelInvariantsResultsAsyncInvariantsPrecAbsIncorrectTrueWallTimeAvgPlain}{}
  \renewcommand{\predicateBitpreciseParallelInvariantsResultsAsyncInvariantsPrecAbsIncorrectTrueWallTimeAvgPlain}{NaN\xspace}
\providecommand{\predicateBitpreciseParallelInvariantsResultsAsyncInvariantsPrecAbsIncorrectTrueWallTimeAvgPlainHours}{}
  \renewcommand{\predicateBitpreciseParallelInvariantsResultsAsyncInvariantsPrecAbsIncorrectTrueWallTimeAvgPlainHours}{NaN\xspace}

  % inv-succ
\providecommand{\predicateBitpreciseParallelInvariantsResultsAsyncInvariantsPrecAbsIncorrectTrueInvSuccPlain}{}
  \renewcommand{\predicateBitpreciseParallelInvariantsResultsAsyncInvariantsPrecAbsIncorrectTrueInvSuccPlain}{0\xspace}

  % inv-tries
\providecommand{\predicateBitpreciseParallelInvariantsResultsAsyncInvariantsPrecAbsIncorrectTrueInvTriesPlain}{}
  \renewcommand{\predicateBitpreciseParallelInvariantsResultsAsyncInvariantsPrecAbsIncorrectTrueInvTriesPlain}{0\xspace}

  % inv-time-sum
\providecommand{\predicateBitpreciseParallelInvariantsResultsAsyncInvariantsPrecAbsIncorrectTrueInvTimeSumPlain}{}
  \renewcommand{\predicateBitpreciseParallelInvariantsResultsAsyncInvariantsPrecAbsIncorrectTrueInvTimeSumPlain}{0.0\xspace}
\providecommand{\predicateBitpreciseParallelInvariantsResultsAsyncInvariantsPrecAbsIncorrectTrueInvTimeSumPlainHours}{}
  \renewcommand{\predicateBitpreciseParallelInvariantsResultsAsyncInvariantsPrecAbsIncorrectTrueInvTimeSumPlainHours}{0.0\xspace}

  % finished-main
\providecommand{\predicateBitpreciseParallelInvariantsResultsAsyncInvariantsPrecAbsIncorrectTrueFinishedMainPlain}{}
  \renewcommand{\predicateBitpreciseParallelInvariantsResultsAsyncInvariantsPrecAbsIncorrectTrueFinishedMainPlain}{0\xspace}

 %% all %%
\providecommand{\predicateBitpreciseParallelInvariantsResultsAsyncInvariantsPrecAbsAllPlain}{}
  \renewcommand{\predicateBitpreciseParallelInvariantsResultsAsyncInvariantsPrecAbsAllPlain}{3488\xspace}

  % cpu-time-sum
\providecommand{\predicateBitpreciseParallelInvariantsResultsAsyncInvariantsPrecAbsAllCpuTimeSumPlain}{}
  \renewcommand{\predicateBitpreciseParallelInvariantsResultsAsyncInvariantsPrecAbsAllCpuTimeSumPlain}{1003410.4870848588\xspace}
\providecommand{\predicateBitpreciseParallelInvariantsResultsAsyncInvariantsPrecAbsAllCpuTimeSumPlainHours}{}
  \renewcommand{\predicateBitpreciseParallelInvariantsResultsAsyncInvariantsPrecAbsAllCpuTimeSumPlainHours}{278.72513530134967\xspace}

  % wall-time-sum
\providecommand{\predicateBitpreciseParallelInvariantsResultsAsyncInvariantsPrecAbsAllWallTimeSumPlain}{}
  \renewcommand{\predicateBitpreciseParallelInvariantsResultsAsyncInvariantsPrecAbsAllWallTimeSumPlain}{535120.8222682585\xspace}
\providecommand{\predicateBitpreciseParallelInvariantsResultsAsyncInvariantsPrecAbsAllWallTimeSumPlainHours}{}
  \renewcommand{\predicateBitpreciseParallelInvariantsResultsAsyncInvariantsPrecAbsAllWallTimeSumPlainHours}{148.64467285229404\xspace}

  % cpu-time-avg
\providecommand{\predicateBitpreciseParallelInvariantsResultsAsyncInvariantsPrecAbsAllCpuTimeAvgPlain}{}
  \renewcommand{\predicateBitpreciseParallelInvariantsResultsAsyncInvariantsPrecAbsAllCpuTimeAvgPlain}{287.6750249669893\xspace}
\providecommand{\predicateBitpreciseParallelInvariantsResultsAsyncInvariantsPrecAbsAllCpuTimeAvgPlainHours}{}
  \renewcommand{\predicateBitpreciseParallelInvariantsResultsAsyncInvariantsPrecAbsAllCpuTimeAvgPlainHours}{0.07990972915749704\xspace}

  % wall-time-avg
\providecommand{\predicateBitpreciseParallelInvariantsResultsAsyncInvariantsPrecAbsAllWallTimeAvgPlain}{}
  \renewcommand{\predicateBitpreciseParallelInvariantsResultsAsyncInvariantsPrecAbsAllWallTimeAvgPlain}{153.41766693470714\xspace}
\providecommand{\predicateBitpreciseParallelInvariantsResultsAsyncInvariantsPrecAbsAllWallTimeAvgPlainHours}{}
  \renewcommand{\predicateBitpreciseParallelInvariantsResultsAsyncInvariantsPrecAbsAllWallTimeAvgPlainHours}{0.0426160185929742\xspace}

  % inv-succ
\providecommand{\predicateBitpreciseParallelInvariantsResultsAsyncInvariantsPrecAbsAllInvSuccPlain}{}
  \renewcommand{\predicateBitpreciseParallelInvariantsResultsAsyncInvariantsPrecAbsAllInvSuccPlain}{0\xspace}

  % inv-tries
\providecommand{\predicateBitpreciseParallelInvariantsResultsAsyncInvariantsPrecAbsAllInvTriesPlain}{}
  \renewcommand{\predicateBitpreciseParallelInvariantsResultsAsyncInvariantsPrecAbsAllInvTriesPlain}{0\xspace}

  % inv-time-sum
\providecommand{\predicateBitpreciseParallelInvariantsResultsAsyncInvariantsPrecAbsAllInvTimeSumPlain}{}
  \renewcommand{\predicateBitpreciseParallelInvariantsResultsAsyncInvariantsPrecAbsAllInvTimeSumPlain}{0.0\xspace}
\providecommand{\predicateBitpreciseParallelInvariantsResultsAsyncInvariantsPrecAbsAllInvTimeSumPlainHours}{}
  \renewcommand{\predicateBitpreciseParallelInvariantsResultsAsyncInvariantsPrecAbsAllInvTimeSumPlainHours}{0.0\xspace}

  % finished-main
\providecommand{\predicateBitpreciseParallelInvariantsResultsAsyncInvariantsPrecAbsAllFinishedMainPlain}{}
  \renewcommand{\predicateBitpreciseParallelInvariantsResultsAsyncInvariantsPrecAbsAllFinishedMainPlain}{1129\xspace}

 %% equal-only %%
\providecommand{\predicateBitpreciseParallelInvariantsResultsAsyncInvariantsPrecAbsEqualOnlyPlain}{}
  \renewcommand{\predicateBitpreciseParallelInvariantsResultsAsyncInvariantsPrecAbsEqualOnlyPlain}{1865\xspace}

  % cpu-time-sum
\providecommand{\predicateBitpreciseParallelInvariantsResultsAsyncInvariantsPrecAbsEqualOnlyCpuTimeSumPlain}{}
  \renewcommand{\predicateBitpreciseParallelInvariantsResultsAsyncInvariantsPrecAbsEqualOnlyCpuTimeSumPlain}{141999.98763889662\xspace}
\providecommand{\predicateBitpreciseParallelInvariantsResultsAsyncInvariantsPrecAbsEqualOnlyCpuTimeSumPlainHours}{}
  \renewcommand{\predicateBitpreciseParallelInvariantsResultsAsyncInvariantsPrecAbsEqualOnlyCpuTimeSumPlainHours}{39.44444101080462\xspace}

  % wall-time-sum
\providecommand{\predicateBitpreciseParallelInvariantsResultsAsyncInvariantsPrecAbsEqualOnlyWallTimeSumPlain}{}
  \renewcommand{\predicateBitpreciseParallelInvariantsResultsAsyncInvariantsPrecAbsEqualOnlyWallTimeSumPlain}{54599.87525295709\xspace}
\providecommand{\predicateBitpreciseParallelInvariantsResultsAsyncInvariantsPrecAbsEqualOnlyWallTimeSumPlainHours}{}
  \renewcommand{\predicateBitpreciseParallelInvariantsResultsAsyncInvariantsPrecAbsEqualOnlyWallTimeSumPlainHours}{15.166632014710304\xspace}

  % cpu-time-avg
\providecommand{\predicateBitpreciseParallelInvariantsResultsAsyncInvariantsPrecAbsEqualOnlyCpuTimeAvgPlain}{}
  \renewcommand{\predicateBitpreciseParallelInvariantsResultsAsyncInvariantsPrecAbsEqualOnlyCpuTimeAvgPlain}{76.1394035597301\xspace}
\providecommand{\predicateBitpreciseParallelInvariantsResultsAsyncInvariantsPrecAbsEqualOnlyCpuTimeAvgPlainHours}{}
  \renewcommand{\predicateBitpreciseParallelInvariantsResultsAsyncInvariantsPrecAbsEqualOnlyCpuTimeAvgPlainHours}{0.02114983432214725\xspace}

  % wall-time-avg
\providecommand{\predicateBitpreciseParallelInvariantsResultsAsyncInvariantsPrecAbsEqualOnlyWallTimeAvgPlain}{}
  \renewcommand{\predicateBitpreciseParallelInvariantsResultsAsyncInvariantsPrecAbsEqualOnlyWallTimeAvgPlain}{29.27607252169281\xspace}
\providecommand{\predicateBitpreciseParallelInvariantsResultsAsyncInvariantsPrecAbsEqualOnlyWallTimeAvgPlainHours}{}
  \renewcommand{\predicateBitpreciseParallelInvariantsResultsAsyncInvariantsPrecAbsEqualOnlyWallTimeAvgPlainHours}{0.008132242367136892\xspace}

  % inv-succ
\providecommand{\predicateBitpreciseParallelInvariantsResultsAsyncInvariantsPrecAbsEqualOnlyInvSuccPlain}{}
  \renewcommand{\predicateBitpreciseParallelInvariantsResultsAsyncInvariantsPrecAbsEqualOnlyInvSuccPlain}{0\xspace}

  % inv-tries
\providecommand{\predicateBitpreciseParallelInvariantsResultsAsyncInvariantsPrecAbsEqualOnlyInvTriesPlain}{}
  \renewcommand{\predicateBitpreciseParallelInvariantsResultsAsyncInvariantsPrecAbsEqualOnlyInvTriesPlain}{0\xspace}

  % inv-time-sum
\providecommand{\predicateBitpreciseParallelInvariantsResultsAsyncInvariantsPrecAbsEqualOnlyInvTimeSumPlain}{}
  \renewcommand{\predicateBitpreciseParallelInvariantsResultsAsyncInvariantsPrecAbsEqualOnlyInvTimeSumPlain}{0.0\xspace}
\providecommand{\predicateBitpreciseParallelInvariantsResultsAsyncInvariantsPrecAbsEqualOnlyInvTimeSumPlainHours}{}
  \renewcommand{\predicateBitpreciseParallelInvariantsResultsAsyncInvariantsPrecAbsEqualOnlyInvTimeSumPlainHours}{0.0\xspace}

  % finished-main
\providecommand{\predicateBitpreciseParallelInvariantsResultsAsyncInvariantsPrecAbsEqualOnlyFinishedMainPlain}{}
  \renewcommand{\predicateBitpreciseParallelInvariantsResultsAsyncInvariantsPrecAbsEqualOnlyFinishedMainPlain}{1020\xspace}

%%% predicate_bitprecise_parallel_invariants.2016-09-05_0219.results.async-invariants-prec-abs-path %%%
 %% correct %%
\providecommand{\predicateBitpreciseParallelInvariantsResultsAsyncInvariantsPrecAbsPathCorrectPlain}{}
  \renewcommand{\predicateBitpreciseParallelInvariantsResultsAsyncInvariantsPrecAbsPathCorrectPlain}{2082\xspace}

  % cpu-time-sum
\providecommand{\predicateBitpreciseParallelInvariantsResultsAsyncInvariantsPrecAbsPathCorrectCpuTimeSumPlain}{}
  \renewcommand{\predicateBitpreciseParallelInvariantsResultsAsyncInvariantsPrecAbsPathCorrectCpuTimeSumPlain}{190016.42357471472\xspace}
\providecommand{\predicateBitpreciseParallelInvariantsResultsAsyncInvariantsPrecAbsPathCorrectCpuTimeSumPlainHours}{}
  \renewcommand{\predicateBitpreciseParallelInvariantsResultsAsyncInvariantsPrecAbsPathCorrectCpuTimeSumPlainHours}{52.7823398818652\xspace}

  % wall-time-sum
\providecommand{\predicateBitpreciseParallelInvariantsResultsAsyncInvariantsPrecAbsPathCorrectWallTimeSumPlain}{}
  \renewcommand{\predicateBitpreciseParallelInvariantsResultsAsyncInvariantsPrecAbsPathCorrectWallTimeSumPlain}{76766.65576695475\xspace}
\providecommand{\predicateBitpreciseParallelInvariantsResultsAsyncInvariantsPrecAbsPathCorrectWallTimeSumPlainHours}{}
  \renewcommand{\predicateBitpreciseParallelInvariantsResultsAsyncInvariantsPrecAbsPathCorrectWallTimeSumPlainHours}{21.324071046376318\xspace}

  % cpu-time-avg
\providecommand{\predicateBitpreciseParallelInvariantsResultsAsyncInvariantsPrecAbsPathCorrectCpuTimeAvgPlain}{}
  \renewcommand{\predicateBitpreciseParallelInvariantsResultsAsyncInvariantsPrecAbsPathCorrectCpuTimeAvgPlain}{91.26629374385914\xspace}
\providecommand{\predicateBitpreciseParallelInvariantsResultsAsyncInvariantsPrecAbsPathCorrectCpuTimeAvgPlainHours}{}
  \renewcommand{\predicateBitpreciseParallelInvariantsResultsAsyncInvariantsPrecAbsPathCorrectCpuTimeAvgPlainHours}{0.025351748262183095\xspace}

  % wall-time-avg
\providecommand{\predicateBitpreciseParallelInvariantsResultsAsyncInvariantsPrecAbsPathCorrectWallTimeAvgPlain}{}
  \renewcommand{\predicateBitpreciseParallelInvariantsResultsAsyncInvariantsPrecAbsPathCorrectWallTimeAvgPlain}{36.87159258739421\xspace}
\providecommand{\predicateBitpreciseParallelInvariantsResultsAsyncInvariantsPrecAbsPathCorrectWallTimeAvgPlainHours}{}
  \renewcommand{\predicateBitpreciseParallelInvariantsResultsAsyncInvariantsPrecAbsPathCorrectWallTimeAvgPlainHours}{0.010242109052053947\xspace}

  % inv-succ
\providecommand{\predicateBitpreciseParallelInvariantsResultsAsyncInvariantsPrecAbsPathCorrectInvSuccPlain}{}
  \renewcommand{\predicateBitpreciseParallelInvariantsResultsAsyncInvariantsPrecAbsPathCorrectInvSuccPlain}{0\xspace}

  % inv-tries
\providecommand{\predicateBitpreciseParallelInvariantsResultsAsyncInvariantsPrecAbsPathCorrectInvTriesPlain}{}
  \renewcommand{\predicateBitpreciseParallelInvariantsResultsAsyncInvariantsPrecAbsPathCorrectInvTriesPlain}{0\xspace}

  % inv-time-sum
\providecommand{\predicateBitpreciseParallelInvariantsResultsAsyncInvariantsPrecAbsPathCorrectInvTimeSumPlain}{}
  \renewcommand{\predicateBitpreciseParallelInvariantsResultsAsyncInvariantsPrecAbsPathCorrectInvTimeSumPlain}{0.0\xspace}
\providecommand{\predicateBitpreciseParallelInvariantsResultsAsyncInvariantsPrecAbsPathCorrectInvTimeSumPlainHours}{}
  \renewcommand{\predicateBitpreciseParallelInvariantsResultsAsyncInvariantsPrecAbsPathCorrectInvTimeSumPlainHours}{0.0\xspace}

  % finished-main
\providecommand{\predicateBitpreciseParallelInvariantsResultsAsyncInvariantsPrecAbsPathCorrectFinishedMainPlain}{}
  \renewcommand{\predicateBitpreciseParallelInvariantsResultsAsyncInvariantsPrecAbsPathCorrectFinishedMainPlain}{1106\xspace}

 %% incorrect %%
\providecommand{\predicateBitpreciseParallelInvariantsResultsAsyncInvariantsPrecAbsPathIncorrectPlain}{}
  \renewcommand{\predicateBitpreciseParallelInvariantsResultsAsyncInvariantsPrecAbsPathIncorrectPlain}{19\xspace}

  % cpu-time-sum
\providecommand{\predicateBitpreciseParallelInvariantsResultsAsyncInvariantsPrecAbsPathIncorrectCpuTimeSumPlain}{}
  \renewcommand{\predicateBitpreciseParallelInvariantsResultsAsyncInvariantsPrecAbsPathIncorrectCpuTimeSumPlain}{1579.5558869019999\xspace}
\providecommand{\predicateBitpreciseParallelInvariantsResultsAsyncInvariantsPrecAbsPathIncorrectCpuTimeSumPlainHours}{}
  \renewcommand{\predicateBitpreciseParallelInvariantsResultsAsyncInvariantsPrecAbsPathIncorrectCpuTimeSumPlainHours}{0.4387655241394444\xspace}

  % wall-time-sum
\providecommand{\predicateBitpreciseParallelInvariantsResultsAsyncInvariantsPrecAbsPathIncorrectWallTimeSumPlain}{}
  \renewcommand{\predicateBitpreciseParallelInvariantsResultsAsyncInvariantsPrecAbsPathIncorrectWallTimeSumPlain}{1008.54219961135\xspace}
\providecommand{\predicateBitpreciseParallelInvariantsResultsAsyncInvariantsPrecAbsPathIncorrectWallTimeSumPlainHours}{}
  \renewcommand{\predicateBitpreciseParallelInvariantsResultsAsyncInvariantsPrecAbsPathIncorrectWallTimeSumPlainHours}{0.2801506110031528\xspace}

  % cpu-time-avg
\providecommand{\predicateBitpreciseParallelInvariantsResultsAsyncInvariantsPrecAbsPathIncorrectCpuTimeAvgPlain}{}
  \renewcommand{\predicateBitpreciseParallelInvariantsResultsAsyncInvariantsPrecAbsPathIncorrectCpuTimeAvgPlain}{83.13452036326315\xspace}
\providecommand{\predicateBitpreciseParallelInvariantsResultsAsyncInvariantsPrecAbsPathIncorrectCpuTimeAvgPlainHours}{}
  \renewcommand{\predicateBitpreciseParallelInvariantsResultsAsyncInvariantsPrecAbsPathIncorrectCpuTimeAvgPlainHours}{0.023092922323128654\xspace}

  % wall-time-avg
\providecommand{\predicateBitpreciseParallelInvariantsResultsAsyncInvariantsPrecAbsPathIncorrectWallTimeAvgPlain}{}
  \renewcommand{\predicateBitpreciseParallelInvariantsResultsAsyncInvariantsPrecAbsPathIncorrectWallTimeAvgPlain}{53.08116840059737\xspace}
\providecommand{\predicateBitpreciseParallelInvariantsResultsAsyncInvariantsPrecAbsPathIncorrectWallTimeAvgPlainHours}{}
  \renewcommand{\predicateBitpreciseParallelInvariantsResultsAsyncInvariantsPrecAbsPathIncorrectWallTimeAvgPlainHours}{0.014744769000165936\xspace}

  % inv-succ
\providecommand{\predicateBitpreciseParallelInvariantsResultsAsyncInvariantsPrecAbsPathIncorrectInvSuccPlain}{}
  \renewcommand{\predicateBitpreciseParallelInvariantsResultsAsyncInvariantsPrecAbsPathIncorrectInvSuccPlain}{0\xspace}

  % inv-tries
\providecommand{\predicateBitpreciseParallelInvariantsResultsAsyncInvariantsPrecAbsPathIncorrectInvTriesPlain}{}
  \renewcommand{\predicateBitpreciseParallelInvariantsResultsAsyncInvariantsPrecAbsPathIncorrectInvTriesPlain}{0\xspace}

  % inv-time-sum
\providecommand{\predicateBitpreciseParallelInvariantsResultsAsyncInvariantsPrecAbsPathIncorrectInvTimeSumPlain}{}
  \renewcommand{\predicateBitpreciseParallelInvariantsResultsAsyncInvariantsPrecAbsPathIncorrectInvTimeSumPlain}{0.0\xspace}
\providecommand{\predicateBitpreciseParallelInvariantsResultsAsyncInvariantsPrecAbsPathIncorrectInvTimeSumPlainHours}{}
  \renewcommand{\predicateBitpreciseParallelInvariantsResultsAsyncInvariantsPrecAbsPathIncorrectInvTimeSumPlainHours}{0.0\xspace}

  % finished-main
\providecommand{\predicateBitpreciseParallelInvariantsResultsAsyncInvariantsPrecAbsPathIncorrectFinishedMainPlain}{}
  \renewcommand{\predicateBitpreciseParallelInvariantsResultsAsyncInvariantsPrecAbsPathIncorrectFinishedMainPlain}{19\xspace}

 %% timeout %%
\providecommand{\predicateBitpreciseParallelInvariantsResultsAsyncInvariantsPrecAbsPathTimeoutPlain}{}
  \renewcommand{\predicateBitpreciseParallelInvariantsResultsAsyncInvariantsPrecAbsPathTimeoutPlain}{1227\xspace}

  % cpu-time-sum
\providecommand{\predicateBitpreciseParallelInvariantsResultsAsyncInvariantsPrecAbsPathTimeoutCpuTimeSumPlain}{}
  \renewcommand{\predicateBitpreciseParallelInvariantsResultsAsyncInvariantsPrecAbsPathTimeoutCpuTimeSumPlain}{754413.6698852151\xspace}
\providecommand{\predicateBitpreciseParallelInvariantsResultsAsyncInvariantsPrecAbsPathTimeoutCpuTimeSumPlainHours}{}
  \renewcommand{\predicateBitpreciseParallelInvariantsResultsAsyncInvariantsPrecAbsPathTimeoutCpuTimeSumPlainHours}{209.55935274589308\xspace}

  % wall-time-sum
\providecommand{\predicateBitpreciseParallelInvariantsResultsAsyncInvariantsPrecAbsPathTimeoutWallTimeSumPlain}{}
  \renewcommand{\predicateBitpreciseParallelInvariantsResultsAsyncInvariantsPrecAbsPathTimeoutWallTimeSumPlain}{421233.5861470677\xspace}
\providecommand{\predicateBitpreciseParallelInvariantsResultsAsyncInvariantsPrecAbsPathTimeoutWallTimeSumPlainHours}{}
  \renewcommand{\predicateBitpreciseParallelInvariantsResultsAsyncInvariantsPrecAbsPathTimeoutWallTimeSumPlainHours}{117.00932948529659\xspace}

  % cpu-time-avg
\providecommand{\predicateBitpreciseParallelInvariantsResultsAsyncInvariantsPrecAbsPathTimeoutCpuTimeAvgPlain}{}
  \renewcommand{\predicateBitpreciseParallelInvariantsResultsAsyncInvariantsPrecAbsPathTimeoutCpuTimeAvgPlain}{614.8440667361166\xspace}
\providecommand{\predicateBitpreciseParallelInvariantsResultsAsyncInvariantsPrecAbsPathTimeoutCpuTimeAvgPlainHours}{}
  \renewcommand{\predicateBitpreciseParallelInvariantsResultsAsyncInvariantsPrecAbsPathTimeoutCpuTimeAvgPlainHours}{0.17079001853781017\xspace}

  % wall-time-avg
\providecommand{\predicateBitpreciseParallelInvariantsResultsAsyncInvariantsPrecAbsPathTimeoutWallTimeAvgPlain}{}
  \renewcommand{\predicateBitpreciseParallelInvariantsResultsAsyncInvariantsPrecAbsPathTimeoutWallTimeAvgPlain}{343.30365619157925\xspace}
\providecommand{\predicateBitpreciseParallelInvariantsResultsAsyncInvariantsPrecAbsPathTimeoutWallTimeAvgPlainHours}{}
  \renewcommand{\predicateBitpreciseParallelInvariantsResultsAsyncInvariantsPrecAbsPathTimeoutWallTimeAvgPlainHours}{0.09536212671988313\xspace}

  % inv-succ
\providecommand{\predicateBitpreciseParallelInvariantsResultsAsyncInvariantsPrecAbsPathTimeoutInvSuccPlain}{}
  \renewcommand{\predicateBitpreciseParallelInvariantsResultsAsyncInvariantsPrecAbsPathTimeoutInvSuccPlain}{0\xspace}

  % inv-tries
\providecommand{\predicateBitpreciseParallelInvariantsResultsAsyncInvariantsPrecAbsPathTimeoutInvTriesPlain}{}
  \renewcommand{\predicateBitpreciseParallelInvariantsResultsAsyncInvariantsPrecAbsPathTimeoutInvTriesPlain}{0\xspace}

  % inv-time-sum
\providecommand{\predicateBitpreciseParallelInvariantsResultsAsyncInvariantsPrecAbsPathTimeoutInvTimeSumPlain}{}
  \renewcommand{\predicateBitpreciseParallelInvariantsResultsAsyncInvariantsPrecAbsPathTimeoutInvTimeSumPlain}{0.0\xspace}
\providecommand{\predicateBitpreciseParallelInvariantsResultsAsyncInvariantsPrecAbsPathTimeoutInvTimeSumPlainHours}{}
  \renewcommand{\predicateBitpreciseParallelInvariantsResultsAsyncInvariantsPrecAbsPathTimeoutInvTimeSumPlainHours}{0.0\xspace}

  % finished-main
\providecommand{\predicateBitpreciseParallelInvariantsResultsAsyncInvariantsPrecAbsPathTimeoutFinishedMainPlain}{}
  \renewcommand{\predicateBitpreciseParallelInvariantsResultsAsyncInvariantsPrecAbsPathTimeoutFinishedMainPlain}{1\xspace}

 %% unknown-or-category-error %%
\providecommand{\predicateBitpreciseParallelInvariantsResultsAsyncInvariantsPrecAbsPathUnknownOrCategoryErrorPlain}{}
  \renewcommand{\predicateBitpreciseParallelInvariantsResultsAsyncInvariantsPrecAbsPathUnknownOrCategoryErrorPlain}{1387\xspace}

  % cpu-time-sum
\providecommand{\predicateBitpreciseParallelInvariantsResultsAsyncInvariantsPrecAbsPathUnknownOrCategoryErrorCpuTimeSumPlain}{}
  \renewcommand{\predicateBitpreciseParallelInvariantsResultsAsyncInvariantsPrecAbsPathUnknownOrCategoryErrorCpuTimeSumPlain}{809411.7640123998\xspace}
\providecommand{\predicateBitpreciseParallelInvariantsResultsAsyncInvariantsPrecAbsPathUnknownOrCategoryErrorCpuTimeSumPlainHours}{}
  \renewcommand{\predicateBitpreciseParallelInvariantsResultsAsyncInvariantsPrecAbsPathUnknownOrCategoryErrorCpuTimeSumPlainHours}{224.83660111455552\xspace}

  % wall-time-sum
\providecommand{\predicateBitpreciseParallelInvariantsResultsAsyncInvariantsPrecAbsPathUnknownOrCategoryErrorWallTimeSumPlain}{}
  \renewcommand{\predicateBitpreciseParallelInvariantsResultsAsyncInvariantsPrecAbsPathUnknownOrCategoryErrorWallTimeSumPlain}{454508.41554713267\xspace}
\providecommand{\predicateBitpreciseParallelInvariantsResultsAsyncInvariantsPrecAbsPathUnknownOrCategoryErrorWallTimeSumPlainHours}{}
  \renewcommand{\predicateBitpreciseParallelInvariantsResultsAsyncInvariantsPrecAbsPathUnknownOrCategoryErrorWallTimeSumPlainHours}{126.25233765198129\xspace}

  % cpu-time-avg
\providecommand{\predicateBitpreciseParallelInvariantsResultsAsyncInvariantsPrecAbsPathUnknownOrCategoryErrorCpuTimeAvgPlain}{}
  \renewcommand{\predicateBitpreciseParallelInvariantsResultsAsyncInvariantsPrecAbsPathUnknownOrCategoryErrorCpuTimeAvgPlain}{583.5701254595529\xspace}
\providecommand{\predicateBitpreciseParallelInvariantsResultsAsyncInvariantsPrecAbsPathUnknownOrCategoryErrorCpuTimeAvgPlainHours}{}
  \renewcommand{\predicateBitpreciseParallelInvariantsResultsAsyncInvariantsPrecAbsPathUnknownOrCategoryErrorCpuTimeAvgPlainHours}{0.16210281262765358\xspace}

  % wall-time-avg
\providecommand{\predicateBitpreciseParallelInvariantsResultsAsyncInvariantsPrecAbsPathUnknownOrCategoryErrorWallTimeAvgPlain}{}
  \renewcommand{\predicateBitpreciseParallelInvariantsResultsAsyncInvariantsPrecAbsPathUnknownOrCategoryErrorWallTimeAvgPlain}{327.69171993304445\xspace}
\providecommand{\predicateBitpreciseParallelInvariantsResultsAsyncInvariantsPrecAbsPathUnknownOrCategoryErrorWallTimeAvgPlainHours}{}
  \renewcommand{\predicateBitpreciseParallelInvariantsResultsAsyncInvariantsPrecAbsPathUnknownOrCategoryErrorWallTimeAvgPlainHours}{0.09102547775917902\xspace}

  % inv-succ
\providecommand{\predicateBitpreciseParallelInvariantsResultsAsyncInvariantsPrecAbsPathUnknownOrCategoryErrorInvSuccPlain}{}
  \renewcommand{\predicateBitpreciseParallelInvariantsResultsAsyncInvariantsPrecAbsPathUnknownOrCategoryErrorInvSuccPlain}{0\xspace}

  % inv-tries
\providecommand{\predicateBitpreciseParallelInvariantsResultsAsyncInvariantsPrecAbsPathUnknownOrCategoryErrorInvTriesPlain}{}
  \renewcommand{\predicateBitpreciseParallelInvariantsResultsAsyncInvariantsPrecAbsPathUnknownOrCategoryErrorInvTriesPlain}{0\xspace}

  % inv-time-sum
\providecommand{\predicateBitpreciseParallelInvariantsResultsAsyncInvariantsPrecAbsPathUnknownOrCategoryErrorInvTimeSumPlain}{}
  \renewcommand{\predicateBitpreciseParallelInvariantsResultsAsyncInvariantsPrecAbsPathUnknownOrCategoryErrorInvTimeSumPlain}{0.0\xspace}
\providecommand{\predicateBitpreciseParallelInvariantsResultsAsyncInvariantsPrecAbsPathUnknownOrCategoryErrorInvTimeSumPlainHours}{}
  \renewcommand{\predicateBitpreciseParallelInvariantsResultsAsyncInvariantsPrecAbsPathUnknownOrCategoryErrorInvTimeSumPlainHours}{0.0\xspace}

  % finished-main
\providecommand{\predicateBitpreciseParallelInvariantsResultsAsyncInvariantsPrecAbsPathUnknownOrCategoryErrorFinishedMainPlain}{}
  \renewcommand{\predicateBitpreciseParallelInvariantsResultsAsyncInvariantsPrecAbsPathUnknownOrCategoryErrorFinishedMainPlain}{1\xspace}

 %% correct-false %%
\providecommand{\predicateBitpreciseParallelInvariantsResultsAsyncInvariantsPrecAbsPathCorrectFalsePlain}{}
  \renewcommand{\predicateBitpreciseParallelInvariantsResultsAsyncInvariantsPrecAbsPathCorrectFalsePlain}{551\xspace}

  % cpu-time-sum
\providecommand{\predicateBitpreciseParallelInvariantsResultsAsyncInvariantsPrecAbsPathCorrectFalseCpuTimeSumPlain}{}
  \renewcommand{\predicateBitpreciseParallelInvariantsResultsAsyncInvariantsPrecAbsPathCorrectFalseCpuTimeSumPlain}{73683.09541510507\xspace}
\providecommand{\predicateBitpreciseParallelInvariantsResultsAsyncInvariantsPrecAbsPathCorrectFalseCpuTimeSumPlainHours}{}
  \renewcommand{\predicateBitpreciseParallelInvariantsResultsAsyncInvariantsPrecAbsPathCorrectFalseCpuTimeSumPlainHours}{20.467526504195853\xspace}

  % wall-time-sum
\providecommand{\predicateBitpreciseParallelInvariantsResultsAsyncInvariantsPrecAbsPathCorrectFalseWallTimeSumPlain}{}
  \renewcommand{\predicateBitpreciseParallelInvariantsResultsAsyncInvariantsPrecAbsPathCorrectFalseWallTimeSumPlain}{31299.168867108812\xspace}
\providecommand{\predicateBitpreciseParallelInvariantsResultsAsyncInvariantsPrecAbsPathCorrectFalseWallTimeSumPlainHours}{}
  \renewcommand{\predicateBitpreciseParallelInvariantsResultsAsyncInvariantsPrecAbsPathCorrectFalseWallTimeSumPlainHours}{8.694213574196892\xspace}

  % cpu-time-avg
\providecommand{\predicateBitpreciseParallelInvariantsResultsAsyncInvariantsPrecAbsPathCorrectFalseCpuTimeAvgPlain}{}
  \renewcommand{\predicateBitpreciseParallelInvariantsResultsAsyncInvariantsPrecAbsPathCorrectFalseCpuTimeAvgPlain}{133.72612598022698\xspace}
\providecommand{\predicateBitpreciseParallelInvariantsResultsAsyncInvariantsPrecAbsPathCorrectFalseCpuTimeAvgPlainHours}{}
  \renewcommand{\predicateBitpreciseParallelInvariantsResultsAsyncInvariantsPrecAbsPathCorrectFalseCpuTimeAvgPlainHours}{0.03714614610561861\xspace}

  % wall-time-avg
\providecommand{\predicateBitpreciseParallelInvariantsResultsAsyncInvariantsPrecAbsPathCorrectFalseWallTimeAvgPlain}{}
  \renewcommand{\predicateBitpreciseParallelInvariantsResultsAsyncInvariantsPrecAbsPathCorrectFalseWallTimeAvgPlain}{56.80429921435356\xspace}
\providecommand{\predicateBitpreciseParallelInvariantsResultsAsyncInvariantsPrecAbsPathCorrectFalseWallTimeAvgPlainHours}{}
  \renewcommand{\predicateBitpreciseParallelInvariantsResultsAsyncInvariantsPrecAbsPathCorrectFalseWallTimeAvgPlainHours}{0.0157789720039871\xspace}

  % inv-succ
\providecommand{\predicateBitpreciseParallelInvariantsResultsAsyncInvariantsPrecAbsPathCorrectFalseInvSuccPlain}{}
  \renewcommand{\predicateBitpreciseParallelInvariantsResultsAsyncInvariantsPrecAbsPathCorrectFalseInvSuccPlain}{0\xspace}

  % inv-tries
\providecommand{\predicateBitpreciseParallelInvariantsResultsAsyncInvariantsPrecAbsPathCorrectFalseInvTriesPlain}{}
  \renewcommand{\predicateBitpreciseParallelInvariantsResultsAsyncInvariantsPrecAbsPathCorrectFalseInvTriesPlain}{0\xspace}

  % inv-time-sum
\providecommand{\predicateBitpreciseParallelInvariantsResultsAsyncInvariantsPrecAbsPathCorrectFalseInvTimeSumPlain}{}
  \renewcommand{\predicateBitpreciseParallelInvariantsResultsAsyncInvariantsPrecAbsPathCorrectFalseInvTimeSumPlain}{0.0\xspace}
\providecommand{\predicateBitpreciseParallelInvariantsResultsAsyncInvariantsPrecAbsPathCorrectFalseInvTimeSumPlainHours}{}
  \renewcommand{\predicateBitpreciseParallelInvariantsResultsAsyncInvariantsPrecAbsPathCorrectFalseInvTimeSumPlainHours}{0.0\xspace}

  % finished-main
\providecommand{\predicateBitpreciseParallelInvariantsResultsAsyncInvariantsPrecAbsPathCorrectFalseFinishedMainPlain}{}
  \renewcommand{\predicateBitpreciseParallelInvariantsResultsAsyncInvariantsPrecAbsPathCorrectFalseFinishedMainPlain}{551\xspace}

 %% correct-true %%
\providecommand{\predicateBitpreciseParallelInvariantsResultsAsyncInvariantsPrecAbsPathCorrectTruePlain}{}
  \renewcommand{\predicateBitpreciseParallelInvariantsResultsAsyncInvariantsPrecAbsPathCorrectTruePlain}{1531\xspace}

  % cpu-time-sum
\providecommand{\predicateBitpreciseParallelInvariantsResultsAsyncInvariantsPrecAbsPathCorrectTrueCpuTimeSumPlain}{}
  \renewcommand{\predicateBitpreciseParallelInvariantsResultsAsyncInvariantsPrecAbsPathCorrectTrueCpuTimeSumPlain}{116333.32815961016\xspace}
\providecommand{\predicateBitpreciseParallelInvariantsResultsAsyncInvariantsPrecAbsPathCorrectTrueCpuTimeSumPlainHours}{}
  \renewcommand{\predicateBitpreciseParallelInvariantsResultsAsyncInvariantsPrecAbsPathCorrectTrueCpuTimeSumPlainHours}{32.31481337766949\xspace}

  % wall-time-sum
\providecommand{\predicateBitpreciseParallelInvariantsResultsAsyncInvariantsPrecAbsPathCorrectTrueWallTimeSumPlain}{}
  \renewcommand{\predicateBitpreciseParallelInvariantsResultsAsyncInvariantsPrecAbsPathCorrectTrueWallTimeSumPlain}{45467.48689984571\xspace}
\providecommand{\predicateBitpreciseParallelInvariantsResultsAsyncInvariantsPrecAbsPathCorrectTrueWallTimeSumPlainHours}{}
  \renewcommand{\predicateBitpreciseParallelInvariantsResultsAsyncInvariantsPrecAbsPathCorrectTrueWallTimeSumPlainHours}{12.629857472179363\xspace}

  % cpu-time-avg
\providecommand{\predicateBitpreciseParallelInvariantsResultsAsyncInvariantsPrecAbsPathCorrectTrueCpuTimeAvgPlain}{}
  \renewcommand{\predicateBitpreciseParallelInvariantsResultsAsyncInvariantsPrecAbsPathCorrectTrueCpuTimeAvgPlain}{75.98519148243642\xspace}
\providecommand{\predicateBitpreciseParallelInvariantsResultsAsyncInvariantsPrecAbsPathCorrectTrueCpuTimeAvgPlainHours}{}
  \renewcommand{\predicateBitpreciseParallelInvariantsResultsAsyncInvariantsPrecAbsPathCorrectTrueCpuTimeAvgPlainHours}{0.021106997634010118\xspace}

  % wall-time-avg
\providecommand{\predicateBitpreciseParallelInvariantsResultsAsyncInvariantsPrecAbsPathCorrectTrueWallTimeAvgPlain}{}
  \renewcommand{\predicateBitpreciseParallelInvariantsResultsAsyncInvariantsPrecAbsPathCorrectTrueWallTimeAvgPlain}{29.69790130623495\xspace}
\providecommand{\predicateBitpreciseParallelInvariantsResultsAsyncInvariantsPrecAbsPathCorrectTrueWallTimeAvgPlainHours}{}
  \renewcommand{\predicateBitpreciseParallelInvariantsResultsAsyncInvariantsPrecAbsPathCorrectTrueWallTimeAvgPlainHours}{0.008249417029509709\xspace}

  % inv-succ
\providecommand{\predicateBitpreciseParallelInvariantsResultsAsyncInvariantsPrecAbsPathCorrectTrueInvSuccPlain}{}
  \renewcommand{\predicateBitpreciseParallelInvariantsResultsAsyncInvariantsPrecAbsPathCorrectTrueInvSuccPlain}{0\xspace}

  % inv-tries
\providecommand{\predicateBitpreciseParallelInvariantsResultsAsyncInvariantsPrecAbsPathCorrectTrueInvTriesPlain}{}
  \renewcommand{\predicateBitpreciseParallelInvariantsResultsAsyncInvariantsPrecAbsPathCorrectTrueInvTriesPlain}{0\xspace}

  % inv-time-sum
\providecommand{\predicateBitpreciseParallelInvariantsResultsAsyncInvariantsPrecAbsPathCorrectTrueInvTimeSumPlain}{}
  \renewcommand{\predicateBitpreciseParallelInvariantsResultsAsyncInvariantsPrecAbsPathCorrectTrueInvTimeSumPlain}{0.0\xspace}
\providecommand{\predicateBitpreciseParallelInvariantsResultsAsyncInvariantsPrecAbsPathCorrectTrueInvTimeSumPlainHours}{}
  \renewcommand{\predicateBitpreciseParallelInvariantsResultsAsyncInvariantsPrecAbsPathCorrectTrueInvTimeSumPlainHours}{0.0\xspace}

  % finished-main
\providecommand{\predicateBitpreciseParallelInvariantsResultsAsyncInvariantsPrecAbsPathCorrectTrueFinishedMainPlain}{}
  \renewcommand{\predicateBitpreciseParallelInvariantsResultsAsyncInvariantsPrecAbsPathCorrectTrueFinishedMainPlain}{555\xspace}

 %% incorrect-false %%
\providecommand{\predicateBitpreciseParallelInvariantsResultsAsyncInvariantsPrecAbsPathIncorrectFalsePlain}{}
  \renewcommand{\predicateBitpreciseParallelInvariantsResultsAsyncInvariantsPrecAbsPathIncorrectFalsePlain}{18\xspace}

  % cpu-time-sum
\providecommand{\predicateBitpreciseParallelInvariantsResultsAsyncInvariantsPrecAbsPathIncorrectFalseCpuTimeSumPlain}{}
  \renewcommand{\predicateBitpreciseParallelInvariantsResultsAsyncInvariantsPrecAbsPathIncorrectFalseCpuTimeSumPlain}{1565.262615413\xspace}
\providecommand{\predicateBitpreciseParallelInvariantsResultsAsyncInvariantsPrecAbsPathIncorrectFalseCpuTimeSumPlainHours}{}
  \renewcommand{\predicateBitpreciseParallelInvariantsResultsAsyncInvariantsPrecAbsPathIncorrectFalseCpuTimeSumPlainHours}{0.4347951709480556\xspace}

  % wall-time-sum
\providecommand{\predicateBitpreciseParallelInvariantsResultsAsyncInvariantsPrecAbsPathIncorrectFalseWallTimeSumPlain}{}
  \renewcommand{\predicateBitpreciseParallelInvariantsResultsAsyncInvariantsPrecAbsPathIncorrectFalseWallTimeSumPlain}{1003.70136356322\xspace}
\providecommand{\predicateBitpreciseParallelInvariantsResultsAsyncInvariantsPrecAbsPathIncorrectFalseWallTimeSumPlainHours}{}
  \renewcommand{\predicateBitpreciseParallelInvariantsResultsAsyncInvariantsPrecAbsPathIncorrectFalseWallTimeSumPlainHours}{0.2788059343231167\xspace}

  % cpu-time-avg
\providecommand{\predicateBitpreciseParallelInvariantsResultsAsyncInvariantsPrecAbsPathIncorrectFalseCpuTimeAvgPlain}{}
  \renewcommand{\predicateBitpreciseParallelInvariantsResultsAsyncInvariantsPrecAbsPathIncorrectFalseCpuTimeAvgPlain}{86.95903418961112\xspace}
\providecommand{\predicateBitpreciseParallelInvariantsResultsAsyncInvariantsPrecAbsPathIncorrectFalseCpuTimeAvgPlainHours}{}
  \renewcommand{\predicateBitpreciseParallelInvariantsResultsAsyncInvariantsPrecAbsPathIncorrectFalseCpuTimeAvgPlainHours}{0.024155287274891978\xspace}

  % wall-time-avg
\providecommand{\predicateBitpreciseParallelInvariantsResultsAsyncInvariantsPrecAbsPathIncorrectFalseWallTimeAvgPlain}{}
  \renewcommand{\predicateBitpreciseParallelInvariantsResultsAsyncInvariantsPrecAbsPathIncorrectFalseWallTimeAvgPlain}{55.76118686462333\xspace}
\providecommand{\predicateBitpreciseParallelInvariantsResultsAsyncInvariantsPrecAbsPathIncorrectFalseWallTimeAvgPlainHours}{}
  \renewcommand{\predicateBitpreciseParallelInvariantsResultsAsyncInvariantsPrecAbsPathIncorrectFalseWallTimeAvgPlainHours}{0.015489218573506481\xspace}

  % inv-succ
\providecommand{\predicateBitpreciseParallelInvariantsResultsAsyncInvariantsPrecAbsPathIncorrectFalseInvSuccPlain}{}
  \renewcommand{\predicateBitpreciseParallelInvariantsResultsAsyncInvariantsPrecAbsPathIncorrectFalseInvSuccPlain}{0\xspace}

  % inv-tries
\providecommand{\predicateBitpreciseParallelInvariantsResultsAsyncInvariantsPrecAbsPathIncorrectFalseInvTriesPlain}{}
  \renewcommand{\predicateBitpreciseParallelInvariantsResultsAsyncInvariantsPrecAbsPathIncorrectFalseInvTriesPlain}{0\xspace}

  % inv-time-sum
\providecommand{\predicateBitpreciseParallelInvariantsResultsAsyncInvariantsPrecAbsPathIncorrectFalseInvTimeSumPlain}{}
  \renewcommand{\predicateBitpreciseParallelInvariantsResultsAsyncInvariantsPrecAbsPathIncorrectFalseInvTimeSumPlain}{0.0\xspace}
\providecommand{\predicateBitpreciseParallelInvariantsResultsAsyncInvariantsPrecAbsPathIncorrectFalseInvTimeSumPlainHours}{}
  \renewcommand{\predicateBitpreciseParallelInvariantsResultsAsyncInvariantsPrecAbsPathIncorrectFalseInvTimeSumPlainHours}{0.0\xspace}

  % finished-main
\providecommand{\predicateBitpreciseParallelInvariantsResultsAsyncInvariantsPrecAbsPathIncorrectFalseFinishedMainPlain}{}
  \renewcommand{\predicateBitpreciseParallelInvariantsResultsAsyncInvariantsPrecAbsPathIncorrectFalseFinishedMainPlain}{18\xspace}

 %% incorrect-true %%
\providecommand{\predicateBitpreciseParallelInvariantsResultsAsyncInvariantsPrecAbsPathIncorrectTruePlain}{}
  \renewcommand{\predicateBitpreciseParallelInvariantsResultsAsyncInvariantsPrecAbsPathIncorrectTruePlain}{1\xspace}

  % cpu-time-sum
\providecommand{\predicateBitpreciseParallelInvariantsResultsAsyncInvariantsPrecAbsPathIncorrectTrueCpuTimeSumPlain}{}
  \renewcommand{\predicateBitpreciseParallelInvariantsResultsAsyncInvariantsPrecAbsPathIncorrectTrueCpuTimeSumPlain}{14.293271489\xspace}
\providecommand{\predicateBitpreciseParallelInvariantsResultsAsyncInvariantsPrecAbsPathIncorrectTrueCpuTimeSumPlainHours}{}
  \renewcommand{\predicateBitpreciseParallelInvariantsResultsAsyncInvariantsPrecAbsPathIncorrectTrueCpuTimeSumPlainHours}{0.003970353191388889\xspace}

  % wall-time-sum
\providecommand{\predicateBitpreciseParallelInvariantsResultsAsyncInvariantsPrecAbsPathIncorrectTrueWallTimeSumPlain}{}
  \renewcommand{\predicateBitpreciseParallelInvariantsResultsAsyncInvariantsPrecAbsPathIncorrectTrueWallTimeSumPlain}{4.84083604813\xspace}
\providecommand{\predicateBitpreciseParallelInvariantsResultsAsyncInvariantsPrecAbsPathIncorrectTrueWallTimeSumPlainHours}{}
  \renewcommand{\predicateBitpreciseParallelInvariantsResultsAsyncInvariantsPrecAbsPathIncorrectTrueWallTimeSumPlainHours}{0.0013446766800361111\xspace}

  % cpu-time-avg
\providecommand{\predicateBitpreciseParallelInvariantsResultsAsyncInvariantsPrecAbsPathIncorrectTrueCpuTimeAvgPlain}{}
  \renewcommand{\predicateBitpreciseParallelInvariantsResultsAsyncInvariantsPrecAbsPathIncorrectTrueCpuTimeAvgPlain}{14.293271489\xspace}
\providecommand{\predicateBitpreciseParallelInvariantsResultsAsyncInvariantsPrecAbsPathIncorrectTrueCpuTimeAvgPlainHours}{}
  \renewcommand{\predicateBitpreciseParallelInvariantsResultsAsyncInvariantsPrecAbsPathIncorrectTrueCpuTimeAvgPlainHours}{0.003970353191388889\xspace}

  % wall-time-avg
\providecommand{\predicateBitpreciseParallelInvariantsResultsAsyncInvariantsPrecAbsPathIncorrectTrueWallTimeAvgPlain}{}
  \renewcommand{\predicateBitpreciseParallelInvariantsResultsAsyncInvariantsPrecAbsPathIncorrectTrueWallTimeAvgPlain}{4.84083604813\xspace}
\providecommand{\predicateBitpreciseParallelInvariantsResultsAsyncInvariantsPrecAbsPathIncorrectTrueWallTimeAvgPlainHours}{}
  \renewcommand{\predicateBitpreciseParallelInvariantsResultsAsyncInvariantsPrecAbsPathIncorrectTrueWallTimeAvgPlainHours}{0.0013446766800361111\xspace}

  % inv-succ
\providecommand{\predicateBitpreciseParallelInvariantsResultsAsyncInvariantsPrecAbsPathIncorrectTrueInvSuccPlain}{}
  \renewcommand{\predicateBitpreciseParallelInvariantsResultsAsyncInvariantsPrecAbsPathIncorrectTrueInvSuccPlain}{0\xspace}

  % inv-tries
\providecommand{\predicateBitpreciseParallelInvariantsResultsAsyncInvariantsPrecAbsPathIncorrectTrueInvTriesPlain}{}
  \renewcommand{\predicateBitpreciseParallelInvariantsResultsAsyncInvariantsPrecAbsPathIncorrectTrueInvTriesPlain}{0\xspace}

  % inv-time-sum
\providecommand{\predicateBitpreciseParallelInvariantsResultsAsyncInvariantsPrecAbsPathIncorrectTrueInvTimeSumPlain}{}
  \renewcommand{\predicateBitpreciseParallelInvariantsResultsAsyncInvariantsPrecAbsPathIncorrectTrueInvTimeSumPlain}{0.0\xspace}
\providecommand{\predicateBitpreciseParallelInvariantsResultsAsyncInvariantsPrecAbsPathIncorrectTrueInvTimeSumPlainHours}{}
  \renewcommand{\predicateBitpreciseParallelInvariantsResultsAsyncInvariantsPrecAbsPathIncorrectTrueInvTimeSumPlainHours}{0.0\xspace}

  % finished-main
\providecommand{\predicateBitpreciseParallelInvariantsResultsAsyncInvariantsPrecAbsPathIncorrectTrueFinishedMainPlain}{}
  \renewcommand{\predicateBitpreciseParallelInvariantsResultsAsyncInvariantsPrecAbsPathIncorrectTrueFinishedMainPlain}{1\xspace}

 %% all %%
\providecommand{\predicateBitpreciseParallelInvariantsResultsAsyncInvariantsPrecAbsPathAllPlain}{}
  \renewcommand{\predicateBitpreciseParallelInvariantsResultsAsyncInvariantsPrecAbsPathAllPlain}{3488\xspace}

  % cpu-time-sum
\providecommand{\predicateBitpreciseParallelInvariantsResultsAsyncInvariantsPrecAbsPathAllCpuTimeSumPlain}{}
  \renewcommand{\predicateBitpreciseParallelInvariantsResultsAsyncInvariantsPrecAbsPathAllCpuTimeSumPlain}{1001007.7434740171\xspace}
\providecommand{\predicateBitpreciseParallelInvariantsResultsAsyncInvariantsPrecAbsPathAllCpuTimeSumPlainHours}{}
  \renewcommand{\predicateBitpreciseParallelInvariantsResultsAsyncInvariantsPrecAbsPathAllCpuTimeSumPlainHours}{278.0577065205603\xspace}

  % wall-time-sum
\providecommand{\predicateBitpreciseParallelInvariantsResultsAsyncInvariantsPrecAbsPathAllWallTimeSumPlain}{}
  \renewcommand{\predicateBitpreciseParallelInvariantsResultsAsyncInvariantsPrecAbsPathAllWallTimeSumPlain}{532283.6135136983\xspace}
\providecommand{\predicateBitpreciseParallelInvariantsResultsAsyncInvariantsPrecAbsPathAllWallTimeSumPlainHours}{}
  \renewcommand{\predicateBitpreciseParallelInvariantsResultsAsyncInvariantsPrecAbsPathAllWallTimeSumPlainHours}{147.85655930936065\xspace}

  % cpu-time-avg
\providecommand{\predicateBitpreciseParallelInvariantsResultsAsyncInvariantsPrecAbsPathAllCpuTimeAvgPlain}{}
  \renewcommand{\predicateBitpreciseParallelInvariantsResultsAsyncInvariantsPrecAbsPathAllCpuTimeAvgPlain}{286.98616498681685\xspace}
\providecommand{\predicateBitpreciseParallelInvariantsResultsAsyncInvariantsPrecAbsPathAllCpuTimeAvgPlainHours}{}
  \renewcommand{\predicateBitpreciseParallelInvariantsResultsAsyncInvariantsPrecAbsPathAllCpuTimeAvgPlainHours}{0.07971837916300469\xspace}

  % wall-time-avg
\providecommand{\predicateBitpreciseParallelInvariantsResultsAsyncInvariantsPrecAbsPathAllWallTimeAvgPlain}{}
  \renewcommand{\predicateBitpreciseParallelInvariantsResultsAsyncInvariantsPrecAbsPathAllWallTimeAvgPlain}{152.60424699360618\xspace}
\providecommand{\predicateBitpreciseParallelInvariantsResultsAsyncInvariantsPrecAbsPathAllWallTimeAvgPlainHours}{}
  \renewcommand{\predicateBitpreciseParallelInvariantsResultsAsyncInvariantsPrecAbsPathAllWallTimeAvgPlainHours}{0.04239006860933505\xspace}

  % inv-succ
\providecommand{\predicateBitpreciseParallelInvariantsResultsAsyncInvariantsPrecAbsPathAllInvSuccPlain}{}
  \renewcommand{\predicateBitpreciseParallelInvariantsResultsAsyncInvariantsPrecAbsPathAllInvSuccPlain}{0\xspace}

  % inv-tries
\providecommand{\predicateBitpreciseParallelInvariantsResultsAsyncInvariantsPrecAbsPathAllInvTriesPlain}{}
  \renewcommand{\predicateBitpreciseParallelInvariantsResultsAsyncInvariantsPrecAbsPathAllInvTriesPlain}{0\xspace}

  % inv-time-sum
\providecommand{\predicateBitpreciseParallelInvariantsResultsAsyncInvariantsPrecAbsPathAllInvTimeSumPlain}{}
  \renewcommand{\predicateBitpreciseParallelInvariantsResultsAsyncInvariantsPrecAbsPathAllInvTimeSumPlain}{0.0\xspace}
\providecommand{\predicateBitpreciseParallelInvariantsResultsAsyncInvariantsPrecAbsPathAllInvTimeSumPlainHours}{}
  \renewcommand{\predicateBitpreciseParallelInvariantsResultsAsyncInvariantsPrecAbsPathAllInvTimeSumPlainHours}{0.0\xspace}

  % finished-main
\providecommand{\predicateBitpreciseParallelInvariantsResultsAsyncInvariantsPrecAbsPathAllFinishedMainPlain}{}
  \renewcommand{\predicateBitpreciseParallelInvariantsResultsAsyncInvariantsPrecAbsPathAllFinishedMainPlain}{1126\xspace}

 %% equal-only %%
\providecommand{\predicateBitpreciseParallelInvariantsResultsAsyncInvariantsPrecAbsPathEqualOnlyPlain}{}
  \renewcommand{\predicateBitpreciseParallelInvariantsResultsAsyncInvariantsPrecAbsPathEqualOnlyPlain}{1865\xspace}

  % cpu-time-sum
\providecommand{\predicateBitpreciseParallelInvariantsResultsAsyncInvariantsPrecAbsPathEqualOnlyCpuTimeSumPlain}{}
  \renewcommand{\predicateBitpreciseParallelInvariantsResultsAsyncInvariantsPrecAbsPathEqualOnlyCpuTimeSumPlain}{142105.52873592285\xspace}
\providecommand{\predicateBitpreciseParallelInvariantsResultsAsyncInvariantsPrecAbsPathEqualOnlyCpuTimeSumPlainHours}{}
  \renewcommand{\predicateBitpreciseParallelInvariantsResultsAsyncInvariantsPrecAbsPathEqualOnlyCpuTimeSumPlainHours}{39.473757982200794\xspace}

  % wall-time-sum
\providecommand{\predicateBitpreciseParallelInvariantsResultsAsyncInvariantsPrecAbsPathEqualOnlyWallTimeSumPlain}{}
  \renewcommand{\predicateBitpreciseParallelInvariantsResultsAsyncInvariantsPrecAbsPathEqualOnlyWallTimeSumPlain}{53921.185919035604\xspace}
\providecommand{\predicateBitpreciseParallelInvariantsResultsAsyncInvariantsPrecAbsPathEqualOnlyWallTimeSumPlainHours}{}
  \renewcommand{\predicateBitpreciseParallelInvariantsResultsAsyncInvariantsPrecAbsPathEqualOnlyWallTimeSumPlainHours}{14.978107199732113\xspace}

  % cpu-time-avg
\providecommand{\predicateBitpreciseParallelInvariantsResultsAsyncInvariantsPrecAbsPathEqualOnlyCpuTimeAvgPlain}{}
  \renewcommand{\predicateBitpreciseParallelInvariantsResultsAsyncInvariantsPrecAbsPathEqualOnlyCpuTimeAvgPlain}{76.19599396028035\xspace}
\providecommand{\predicateBitpreciseParallelInvariantsResultsAsyncInvariantsPrecAbsPathEqualOnlyCpuTimeAvgPlainHours}{}
  \renewcommand{\predicateBitpreciseParallelInvariantsResultsAsyncInvariantsPrecAbsPathEqualOnlyCpuTimeAvgPlainHours}{0.021165553877855653\xspace}

  % wall-time-avg
\providecommand{\predicateBitpreciseParallelInvariantsResultsAsyncInvariantsPrecAbsPathEqualOnlyWallTimeAvgPlain}{}
  \renewcommand{\predicateBitpreciseParallelInvariantsResultsAsyncInvariantsPrecAbsPathEqualOnlyWallTimeAvgPlain}{28.91216403165448\xspace}
\providecommand{\predicateBitpreciseParallelInvariantsResultsAsyncInvariantsPrecAbsPathEqualOnlyWallTimeAvgPlainHours}{}
  \renewcommand{\predicateBitpreciseParallelInvariantsResultsAsyncInvariantsPrecAbsPathEqualOnlyWallTimeAvgPlainHours}{0.008031156675459578\xspace}

  % inv-succ
\providecommand{\predicateBitpreciseParallelInvariantsResultsAsyncInvariantsPrecAbsPathEqualOnlyInvSuccPlain}{}
  \renewcommand{\predicateBitpreciseParallelInvariantsResultsAsyncInvariantsPrecAbsPathEqualOnlyInvSuccPlain}{0\xspace}

  % inv-tries
\providecommand{\predicateBitpreciseParallelInvariantsResultsAsyncInvariantsPrecAbsPathEqualOnlyInvTriesPlain}{}
  \renewcommand{\predicateBitpreciseParallelInvariantsResultsAsyncInvariantsPrecAbsPathEqualOnlyInvTriesPlain}{0\xspace}

  % inv-time-sum
\providecommand{\predicateBitpreciseParallelInvariantsResultsAsyncInvariantsPrecAbsPathEqualOnlyInvTimeSumPlain}{}
  \renewcommand{\predicateBitpreciseParallelInvariantsResultsAsyncInvariantsPrecAbsPathEqualOnlyInvTimeSumPlain}{0.0\xspace}
\providecommand{\predicateBitpreciseParallelInvariantsResultsAsyncInvariantsPrecAbsPathEqualOnlyInvTimeSumPlainHours}{}
  \renewcommand{\predicateBitpreciseParallelInvariantsResultsAsyncInvariantsPrecAbsPathEqualOnlyInvTimeSumPlainHours}{0.0\xspace}

  % finished-main
\providecommand{\predicateBitpreciseParallelInvariantsResultsAsyncInvariantsPrecAbsPathEqualOnlyFinishedMainPlain}{}
  \renewcommand{\predicateBitpreciseParallelInvariantsResultsAsyncInvariantsPrecAbsPathEqualOnlyFinishedMainPlain}{1021\xspace}

%%% predicate_bitprecise_parallel_invariants.2016-09-05_0219.results.async-invariants-path %%%
 %% correct %%
\providecommand{\predicateBitpreciseParallelInvariantsResultsAsyncInvariantsPathCorrectPlain}{}
  \renewcommand{\predicateBitpreciseParallelInvariantsResultsAsyncInvariantsPathCorrectPlain}{2097\xspace}

  % cpu-time-sum
\providecommand{\predicateBitpreciseParallelInvariantsResultsAsyncInvariantsPathCorrectCpuTimeSumPlain}{}
  \renewcommand{\predicateBitpreciseParallelInvariantsResultsAsyncInvariantsPathCorrectCpuTimeSumPlain}{191141.67280322302\xspace}
\providecommand{\predicateBitpreciseParallelInvariantsResultsAsyncInvariantsPathCorrectCpuTimeSumPlainHours}{}
  \renewcommand{\predicateBitpreciseParallelInvariantsResultsAsyncInvariantsPathCorrectCpuTimeSumPlainHours}{53.09490911200639\xspace}

  % wall-time-sum
\providecommand{\predicateBitpreciseParallelInvariantsResultsAsyncInvariantsPathCorrectWallTimeSumPlain}{}
  \renewcommand{\predicateBitpreciseParallelInvariantsResultsAsyncInvariantsPathCorrectWallTimeSumPlain}{76317.24601792802\xspace}
\providecommand{\predicateBitpreciseParallelInvariantsResultsAsyncInvariantsPathCorrectWallTimeSumPlainHours}{}
  \renewcommand{\predicateBitpreciseParallelInvariantsResultsAsyncInvariantsPathCorrectWallTimeSumPlainHours}{21.199235004980004\xspace}

  % cpu-time-avg
\providecommand{\predicateBitpreciseParallelInvariantsResultsAsyncInvariantsPathCorrectCpuTimeAvgPlain}{}
  \renewcommand{\predicateBitpreciseParallelInvariantsResultsAsyncInvariantsPathCorrectCpuTimeAvgPlain}{91.15005856138437\xspace}
\providecommand{\predicateBitpreciseParallelInvariantsResultsAsyncInvariantsPathCorrectCpuTimeAvgPlainHours}{}
  \renewcommand{\predicateBitpreciseParallelInvariantsResultsAsyncInvariantsPathCorrectCpuTimeAvgPlainHours}{0.02531946071149566\xspace}

  % wall-time-avg
\providecommand{\predicateBitpreciseParallelInvariantsResultsAsyncInvariantsPathCorrectWallTimeAvgPlain}{}
  \renewcommand{\predicateBitpreciseParallelInvariantsResultsAsyncInvariantsPathCorrectWallTimeAvgPlain}{36.39353648923606\xspace}
\providecommand{\predicateBitpreciseParallelInvariantsResultsAsyncInvariantsPathCorrectWallTimeAvgPlainHours}{}
  \renewcommand{\predicateBitpreciseParallelInvariantsResultsAsyncInvariantsPathCorrectWallTimeAvgPlainHours}{0.010109315691454462\xspace}

  % inv-succ
\providecommand{\predicateBitpreciseParallelInvariantsResultsAsyncInvariantsPathCorrectInvSuccPlain}{}
  \renewcommand{\predicateBitpreciseParallelInvariantsResultsAsyncInvariantsPathCorrectInvSuccPlain}{0\xspace}

  % inv-tries
\providecommand{\predicateBitpreciseParallelInvariantsResultsAsyncInvariantsPathCorrectInvTriesPlain}{}
  \renewcommand{\predicateBitpreciseParallelInvariantsResultsAsyncInvariantsPathCorrectInvTriesPlain}{0\xspace}

  % inv-time-sum
\providecommand{\predicateBitpreciseParallelInvariantsResultsAsyncInvariantsPathCorrectInvTimeSumPlain}{}
  \renewcommand{\predicateBitpreciseParallelInvariantsResultsAsyncInvariantsPathCorrectInvTimeSumPlain}{0.0\xspace}
\providecommand{\predicateBitpreciseParallelInvariantsResultsAsyncInvariantsPathCorrectInvTimeSumPlainHours}{}
  \renewcommand{\predicateBitpreciseParallelInvariantsResultsAsyncInvariantsPathCorrectInvTimeSumPlainHours}{0.0\xspace}

  % finished-main
\providecommand{\predicateBitpreciseParallelInvariantsResultsAsyncInvariantsPathCorrectFinishedMainPlain}{}
  \renewcommand{\predicateBitpreciseParallelInvariantsResultsAsyncInvariantsPathCorrectFinishedMainPlain}{1148\xspace}

 %% incorrect %%
\providecommand{\predicateBitpreciseParallelInvariantsResultsAsyncInvariantsPathIncorrectPlain}{}
  \renewcommand{\predicateBitpreciseParallelInvariantsResultsAsyncInvariantsPathIncorrectPlain}{18\xspace}

  % cpu-time-sum
\providecommand{\predicateBitpreciseParallelInvariantsResultsAsyncInvariantsPathIncorrectCpuTimeSumPlain}{}
  \renewcommand{\predicateBitpreciseParallelInvariantsResultsAsyncInvariantsPathIncorrectCpuTimeSumPlain}{849.5509307929999\xspace}
\providecommand{\predicateBitpreciseParallelInvariantsResultsAsyncInvariantsPathIncorrectCpuTimeSumPlainHours}{}
  \renewcommand{\predicateBitpreciseParallelInvariantsResultsAsyncInvariantsPathIncorrectCpuTimeSumPlainHours}{0.2359863696647222\xspace}

  % wall-time-sum
\providecommand{\predicateBitpreciseParallelInvariantsResultsAsyncInvariantsPathIncorrectWallTimeSumPlain}{}
  \renewcommand{\predicateBitpreciseParallelInvariantsResultsAsyncInvariantsPathIncorrectWallTimeSumPlain}{273.54158592231005\xspace}
\providecommand{\predicateBitpreciseParallelInvariantsResultsAsyncInvariantsPathIncorrectWallTimeSumPlainHours}{}
  \renewcommand{\predicateBitpreciseParallelInvariantsResultsAsyncInvariantsPathIncorrectWallTimeSumPlainHours}{0.07598377386730834\xspace}

  % cpu-time-avg
\providecommand{\predicateBitpreciseParallelInvariantsResultsAsyncInvariantsPathIncorrectCpuTimeAvgPlain}{}
  \renewcommand{\predicateBitpreciseParallelInvariantsResultsAsyncInvariantsPathIncorrectCpuTimeAvgPlain}{47.19727393294444\xspace}
\providecommand{\predicateBitpreciseParallelInvariantsResultsAsyncInvariantsPathIncorrectCpuTimeAvgPlainHours}{}
  \renewcommand{\predicateBitpreciseParallelInvariantsResultsAsyncInvariantsPathIncorrectCpuTimeAvgPlainHours}{0.013110353870262345\xspace}

  % wall-time-avg
\providecommand{\predicateBitpreciseParallelInvariantsResultsAsyncInvariantsPathIncorrectWallTimeAvgPlain}{}
  \renewcommand{\predicateBitpreciseParallelInvariantsResultsAsyncInvariantsPathIncorrectWallTimeAvgPlain}{15.19675477346167\xspace}
\providecommand{\predicateBitpreciseParallelInvariantsResultsAsyncInvariantsPathIncorrectWallTimeAvgPlainHours}{}
  \renewcommand{\predicateBitpreciseParallelInvariantsResultsAsyncInvariantsPathIncorrectWallTimeAvgPlainHours}{0.004221320770406019\xspace}

  % inv-succ
\providecommand{\predicateBitpreciseParallelInvariantsResultsAsyncInvariantsPathIncorrectInvSuccPlain}{}
  \renewcommand{\predicateBitpreciseParallelInvariantsResultsAsyncInvariantsPathIncorrectInvSuccPlain}{0\xspace}

  % inv-tries
\providecommand{\predicateBitpreciseParallelInvariantsResultsAsyncInvariantsPathIncorrectInvTriesPlain}{}
  \renewcommand{\predicateBitpreciseParallelInvariantsResultsAsyncInvariantsPathIncorrectInvTriesPlain}{0\xspace}

  % inv-time-sum
\providecommand{\predicateBitpreciseParallelInvariantsResultsAsyncInvariantsPathIncorrectInvTimeSumPlain}{}
  \renewcommand{\predicateBitpreciseParallelInvariantsResultsAsyncInvariantsPathIncorrectInvTimeSumPlain}{0.0\xspace}
\providecommand{\predicateBitpreciseParallelInvariantsResultsAsyncInvariantsPathIncorrectInvTimeSumPlainHours}{}
  \renewcommand{\predicateBitpreciseParallelInvariantsResultsAsyncInvariantsPathIncorrectInvTimeSumPlainHours}{0.0\xspace}

  % finished-main
\providecommand{\predicateBitpreciseParallelInvariantsResultsAsyncInvariantsPathIncorrectFinishedMainPlain}{}
  \renewcommand{\predicateBitpreciseParallelInvariantsResultsAsyncInvariantsPathIncorrectFinishedMainPlain}{18\xspace}

 %% timeout %%
\providecommand{\predicateBitpreciseParallelInvariantsResultsAsyncInvariantsPathTimeoutPlain}{}
  \renewcommand{\predicateBitpreciseParallelInvariantsResultsAsyncInvariantsPathTimeoutPlain}{1204\xspace}

  % cpu-time-sum
\providecommand{\predicateBitpreciseParallelInvariantsResultsAsyncInvariantsPathTimeoutCpuTimeSumPlain}{}
  \renewcommand{\predicateBitpreciseParallelInvariantsResultsAsyncInvariantsPathTimeoutCpuTimeSumPlain}{740709.5511456358\xspace}
\providecommand{\predicateBitpreciseParallelInvariantsResultsAsyncInvariantsPathTimeoutCpuTimeSumPlainHours}{}
  \renewcommand{\predicateBitpreciseParallelInvariantsResultsAsyncInvariantsPathTimeoutCpuTimeSumPlainHours}{205.75265309600994\xspace}

  % wall-time-sum
\providecommand{\predicateBitpreciseParallelInvariantsResultsAsyncInvariantsPathTimeoutWallTimeSumPlain}{}
  \renewcommand{\predicateBitpreciseParallelInvariantsResultsAsyncInvariantsPathTimeoutWallTimeSumPlain}{415077.7024245233\xspace}
\providecommand{\predicateBitpreciseParallelInvariantsResultsAsyncInvariantsPathTimeoutWallTimeSumPlainHours}{}
  \renewcommand{\predicateBitpreciseParallelInvariantsResultsAsyncInvariantsPathTimeoutWallTimeSumPlainHours}{115.2993617845898\xspace}

  % cpu-time-avg
\providecommand{\predicateBitpreciseParallelInvariantsResultsAsyncInvariantsPathTimeoutCpuTimeAvgPlain}{}
  \renewcommand{\predicateBitpreciseParallelInvariantsResultsAsyncInvariantsPathTimeoutCpuTimeAvgPlain}{615.2072683933852\xspace}
\providecommand{\predicateBitpreciseParallelInvariantsResultsAsyncInvariantsPathTimeoutCpuTimeAvgPlainHours}{}
  \renewcommand{\predicateBitpreciseParallelInvariantsResultsAsyncInvariantsPathTimeoutCpuTimeAvgPlainHours}{0.17089090788705144\xspace}

  % wall-time-avg
\providecommand{\predicateBitpreciseParallelInvariantsResultsAsyncInvariantsPathTimeoutWallTimeAvgPlain}{}
  \renewcommand{\predicateBitpreciseParallelInvariantsResultsAsyncInvariantsPathTimeoutWallTimeAvgPlain}{344.7489222795044\xspace}
\providecommand{\predicateBitpreciseParallelInvariantsResultsAsyncInvariantsPathTimeoutWallTimeAvgPlainHours}{}
  \renewcommand{\predicateBitpreciseParallelInvariantsResultsAsyncInvariantsPathTimeoutWallTimeAvgPlainHours}{0.09576358952208457\xspace}

  % inv-succ
\providecommand{\predicateBitpreciseParallelInvariantsResultsAsyncInvariantsPathTimeoutInvSuccPlain}{}
  \renewcommand{\predicateBitpreciseParallelInvariantsResultsAsyncInvariantsPathTimeoutInvSuccPlain}{0\xspace}

  % inv-tries
\providecommand{\predicateBitpreciseParallelInvariantsResultsAsyncInvariantsPathTimeoutInvTriesPlain}{}
  \renewcommand{\predicateBitpreciseParallelInvariantsResultsAsyncInvariantsPathTimeoutInvTriesPlain}{0\xspace}

  % inv-time-sum
\providecommand{\predicateBitpreciseParallelInvariantsResultsAsyncInvariantsPathTimeoutInvTimeSumPlain}{}
  \renewcommand{\predicateBitpreciseParallelInvariantsResultsAsyncInvariantsPathTimeoutInvTimeSumPlain}{0.0\xspace}
\providecommand{\predicateBitpreciseParallelInvariantsResultsAsyncInvariantsPathTimeoutInvTimeSumPlainHours}{}
  \renewcommand{\predicateBitpreciseParallelInvariantsResultsAsyncInvariantsPathTimeoutInvTimeSumPlainHours}{0.0\xspace}

  % finished-main
\providecommand{\predicateBitpreciseParallelInvariantsResultsAsyncInvariantsPathTimeoutFinishedMainPlain}{}
  \renewcommand{\predicateBitpreciseParallelInvariantsResultsAsyncInvariantsPathTimeoutFinishedMainPlain}{0\xspace}

 %% unknown-or-category-error %%
\providecommand{\predicateBitpreciseParallelInvariantsResultsAsyncInvariantsPathUnknownOrCategoryErrorPlain}{}
  \renewcommand{\predicateBitpreciseParallelInvariantsResultsAsyncInvariantsPathUnknownOrCategoryErrorPlain}{1373\xspace}

  % cpu-time-sum
\providecommand{\predicateBitpreciseParallelInvariantsResultsAsyncInvariantsPathUnknownOrCategoryErrorCpuTimeSumPlain}{}
  \renewcommand{\predicateBitpreciseParallelInvariantsResultsAsyncInvariantsPathUnknownOrCategoryErrorCpuTimeSumPlain}{796197.7155354553\xspace}
\providecommand{\predicateBitpreciseParallelInvariantsResultsAsyncInvariantsPathUnknownOrCategoryErrorCpuTimeSumPlainHours}{}
  \renewcommand{\predicateBitpreciseParallelInvariantsResultsAsyncInvariantsPathUnknownOrCategoryErrorCpuTimeSumPlainHours}{221.16603209318203\xspace}

  % wall-time-sum
\providecommand{\predicateBitpreciseParallelInvariantsResultsAsyncInvariantsPathUnknownOrCategoryErrorWallTimeSumPlain}{}
  \renewcommand{\predicateBitpreciseParallelInvariantsResultsAsyncInvariantsPathUnknownOrCategoryErrorWallTimeSumPlain}{449522.5411646357\xspace}
\providecommand{\predicateBitpreciseParallelInvariantsResultsAsyncInvariantsPathUnknownOrCategoryErrorWallTimeSumPlainHours}{}
  \renewcommand{\predicateBitpreciseParallelInvariantsResultsAsyncInvariantsPathUnknownOrCategoryErrorWallTimeSumPlainHours}{124.86737254573214\xspace}

  % cpu-time-avg
\providecommand{\predicateBitpreciseParallelInvariantsResultsAsyncInvariantsPathUnknownOrCategoryErrorCpuTimeAvgPlain}{}
  \renewcommand{\predicateBitpreciseParallelInvariantsResultsAsyncInvariantsPathUnknownOrCategoryErrorCpuTimeAvgPlain}{579.8963696543739\xspace}
\providecommand{\predicateBitpreciseParallelInvariantsResultsAsyncInvariantsPathUnknownOrCategoryErrorCpuTimeAvgPlainHours}{}
  \renewcommand{\predicateBitpreciseParallelInvariantsResultsAsyncInvariantsPathUnknownOrCategoryErrorCpuTimeAvgPlainHours}{0.16108232490399274\xspace}

  % wall-time-avg
\providecommand{\predicateBitpreciseParallelInvariantsResultsAsyncInvariantsPathUnknownOrCategoryErrorWallTimeAvgPlain}{}
  \renewcommand{\predicateBitpreciseParallelInvariantsResultsAsyncInvariantsPathUnknownOrCategoryErrorWallTimeAvgPlain}{327.4017051454011\xspace}
\providecommand{\predicateBitpreciseParallelInvariantsResultsAsyncInvariantsPathUnknownOrCategoryErrorWallTimeAvgPlainHours}{}
  \renewcommand{\predicateBitpreciseParallelInvariantsResultsAsyncInvariantsPathUnknownOrCategoryErrorWallTimeAvgPlainHours}{0.09094491809594474\xspace}

  % inv-succ
\providecommand{\predicateBitpreciseParallelInvariantsResultsAsyncInvariantsPathUnknownOrCategoryErrorInvSuccPlain}{}
  \renewcommand{\predicateBitpreciseParallelInvariantsResultsAsyncInvariantsPathUnknownOrCategoryErrorInvSuccPlain}{0\xspace}

  % inv-tries
\providecommand{\predicateBitpreciseParallelInvariantsResultsAsyncInvariantsPathUnknownOrCategoryErrorInvTriesPlain}{}
  \renewcommand{\predicateBitpreciseParallelInvariantsResultsAsyncInvariantsPathUnknownOrCategoryErrorInvTriesPlain}{0\xspace}

  % inv-time-sum
\providecommand{\predicateBitpreciseParallelInvariantsResultsAsyncInvariantsPathUnknownOrCategoryErrorInvTimeSumPlain}{}
  \renewcommand{\predicateBitpreciseParallelInvariantsResultsAsyncInvariantsPathUnknownOrCategoryErrorInvTimeSumPlain}{0.0\xspace}
\providecommand{\predicateBitpreciseParallelInvariantsResultsAsyncInvariantsPathUnknownOrCategoryErrorInvTimeSumPlainHours}{}
  \renewcommand{\predicateBitpreciseParallelInvariantsResultsAsyncInvariantsPathUnknownOrCategoryErrorInvTimeSumPlainHours}{0.0\xspace}

  % finished-main
\providecommand{\predicateBitpreciseParallelInvariantsResultsAsyncInvariantsPathUnknownOrCategoryErrorFinishedMainPlain}{}
  \renewcommand{\predicateBitpreciseParallelInvariantsResultsAsyncInvariantsPathUnknownOrCategoryErrorFinishedMainPlain}{1\xspace}

 %% correct-false %%
\providecommand{\predicateBitpreciseParallelInvariantsResultsAsyncInvariantsPathCorrectFalsePlain}{}
  \renewcommand{\predicateBitpreciseParallelInvariantsResultsAsyncInvariantsPathCorrectFalsePlain}{561\xspace}

  % cpu-time-sum
\providecommand{\predicateBitpreciseParallelInvariantsResultsAsyncInvariantsPathCorrectFalseCpuTimeSumPlain}{}
  \renewcommand{\predicateBitpreciseParallelInvariantsResultsAsyncInvariantsPathCorrectFalseCpuTimeSumPlain}{74087.27317134508\xspace}
\providecommand{\predicateBitpreciseParallelInvariantsResultsAsyncInvariantsPathCorrectFalseCpuTimeSumPlainHours}{}
  \renewcommand{\predicateBitpreciseParallelInvariantsResultsAsyncInvariantsPathCorrectFalseCpuTimeSumPlainHours}{20.57979810315141\xspace}

  % wall-time-sum
\providecommand{\predicateBitpreciseParallelInvariantsResultsAsyncInvariantsPathCorrectFalseWallTimeSumPlain}{}
  \renewcommand{\predicateBitpreciseParallelInvariantsResultsAsyncInvariantsPathCorrectFalseWallTimeSumPlain}{31150.352310179383\xspace}
\providecommand{\predicateBitpreciseParallelInvariantsResultsAsyncInvariantsPathCorrectFalseWallTimeSumPlainHours}{}
  \renewcommand{\predicateBitpreciseParallelInvariantsResultsAsyncInvariantsPathCorrectFalseWallTimeSumPlainHours}{8.652875641716495\xspace}

  % cpu-time-avg
\providecommand{\predicateBitpreciseParallelInvariantsResultsAsyncInvariantsPathCorrectFalseCpuTimeAvgPlain}{}
  \renewcommand{\predicateBitpreciseParallelInvariantsResultsAsyncInvariantsPathCorrectFalseCpuTimeAvgPlain}{132.06287552824435\xspace}
\providecommand{\predicateBitpreciseParallelInvariantsResultsAsyncInvariantsPathCorrectFalseCpuTimeAvgPlainHours}{}
  \renewcommand{\predicateBitpreciseParallelInvariantsResultsAsyncInvariantsPathCorrectFalseCpuTimeAvgPlainHours}{0.03668413209117898\xspace}

  % wall-time-avg
\providecommand{\predicateBitpreciseParallelInvariantsResultsAsyncInvariantsPathCorrectFalseWallTimeAvgPlain}{}
  \renewcommand{\predicateBitpreciseParallelInvariantsResultsAsyncInvariantsPathCorrectFalseWallTimeAvgPlain}{55.52647470620211\xspace}
\providecommand{\predicateBitpreciseParallelInvariantsResultsAsyncInvariantsPathCorrectFalseWallTimeAvgPlainHours}{}
  \renewcommand{\predicateBitpreciseParallelInvariantsResultsAsyncInvariantsPathCorrectFalseWallTimeAvgPlainHours}{0.015424020751722807\xspace}

  % inv-succ
\providecommand{\predicateBitpreciseParallelInvariantsResultsAsyncInvariantsPathCorrectFalseInvSuccPlain}{}
  \renewcommand{\predicateBitpreciseParallelInvariantsResultsAsyncInvariantsPathCorrectFalseInvSuccPlain}{0\xspace}

  % inv-tries
\providecommand{\predicateBitpreciseParallelInvariantsResultsAsyncInvariantsPathCorrectFalseInvTriesPlain}{}
  \renewcommand{\predicateBitpreciseParallelInvariantsResultsAsyncInvariantsPathCorrectFalseInvTriesPlain}{0\xspace}

  % inv-time-sum
\providecommand{\predicateBitpreciseParallelInvariantsResultsAsyncInvariantsPathCorrectFalseInvTimeSumPlain}{}
  \renewcommand{\predicateBitpreciseParallelInvariantsResultsAsyncInvariantsPathCorrectFalseInvTimeSumPlain}{0.0\xspace}
\providecommand{\predicateBitpreciseParallelInvariantsResultsAsyncInvariantsPathCorrectFalseInvTimeSumPlainHours}{}
  \renewcommand{\predicateBitpreciseParallelInvariantsResultsAsyncInvariantsPathCorrectFalseInvTimeSumPlainHours}{0.0\xspace}

  % finished-main
\providecommand{\predicateBitpreciseParallelInvariantsResultsAsyncInvariantsPathCorrectFalseFinishedMainPlain}{}
  \renewcommand{\predicateBitpreciseParallelInvariantsResultsAsyncInvariantsPathCorrectFalseFinishedMainPlain}{561\xspace}

 %% correct-true %%
\providecommand{\predicateBitpreciseParallelInvariantsResultsAsyncInvariantsPathCorrectTruePlain}{}
  \renewcommand{\predicateBitpreciseParallelInvariantsResultsAsyncInvariantsPathCorrectTruePlain}{1536\xspace}

  % cpu-time-sum
\providecommand{\predicateBitpreciseParallelInvariantsResultsAsyncInvariantsPathCorrectTrueCpuTimeSumPlain}{}
  \renewcommand{\predicateBitpreciseParallelInvariantsResultsAsyncInvariantsPathCorrectTrueCpuTimeSumPlain}{117054.39963187798\xspace}
\providecommand{\predicateBitpreciseParallelInvariantsResultsAsyncInvariantsPathCorrectTrueCpuTimeSumPlainHours}{}
  \renewcommand{\predicateBitpreciseParallelInvariantsResultsAsyncInvariantsPathCorrectTrueCpuTimeSumPlainHours}{32.515111008854994\xspace}

  % wall-time-sum
\providecommand{\predicateBitpreciseParallelInvariantsResultsAsyncInvariantsPathCorrectTrueWallTimeSumPlain}{}
  \renewcommand{\predicateBitpreciseParallelInvariantsResultsAsyncInvariantsPathCorrectTrueWallTimeSumPlain}{45166.89370774856\xspace}
\providecommand{\predicateBitpreciseParallelInvariantsResultsAsyncInvariantsPathCorrectTrueWallTimeSumPlainHours}{}
  \renewcommand{\predicateBitpreciseParallelInvariantsResultsAsyncInvariantsPathCorrectTrueWallTimeSumPlainHours}{12.54635936326349\xspace}

  % cpu-time-avg
\providecommand{\predicateBitpreciseParallelInvariantsResultsAsyncInvariantsPathCorrectTrueCpuTimeAvgPlain}{}
  \renewcommand{\predicateBitpreciseParallelInvariantsResultsAsyncInvariantsPathCorrectTrueCpuTimeAvgPlain}{76.2072914270039\xspace}
\providecommand{\predicateBitpreciseParallelInvariantsResultsAsyncInvariantsPathCorrectTrueCpuTimeAvgPlainHours}{}
  \renewcommand{\predicateBitpreciseParallelInvariantsResultsAsyncInvariantsPathCorrectTrueCpuTimeAvgPlainHours}{0.02116869206305664\xspace}

  % wall-time-avg
\providecommand{\predicateBitpreciseParallelInvariantsResultsAsyncInvariantsPathCorrectTrueWallTimeAvgPlain}{}
  \renewcommand{\predicateBitpreciseParallelInvariantsResultsAsyncInvariantsPathCorrectTrueWallTimeAvgPlain}{29.405529757648804\xspace}
\providecommand{\predicateBitpreciseParallelInvariantsResultsAsyncInvariantsPathCorrectTrueWallTimeAvgPlainHours}{}
  \renewcommand{\predicateBitpreciseParallelInvariantsResultsAsyncInvariantsPathCorrectTrueWallTimeAvgPlainHours}{0.008168202710458\xspace}

  % inv-succ
\providecommand{\predicateBitpreciseParallelInvariantsResultsAsyncInvariantsPathCorrectTrueInvSuccPlain}{}
  \renewcommand{\predicateBitpreciseParallelInvariantsResultsAsyncInvariantsPathCorrectTrueInvSuccPlain}{0\xspace}

  % inv-tries
\providecommand{\predicateBitpreciseParallelInvariantsResultsAsyncInvariantsPathCorrectTrueInvTriesPlain}{}
  \renewcommand{\predicateBitpreciseParallelInvariantsResultsAsyncInvariantsPathCorrectTrueInvTriesPlain}{0\xspace}

  % inv-time-sum
\providecommand{\predicateBitpreciseParallelInvariantsResultsAsyncInvariantsPathCorrectTrueInvTimeSumPlain}{}
  \renewcommand{\predicateBitpreciseParallelInvariantsResultsAsyncInvariantsPathCorrectTrueInvTimeSumPlain}{0.0\xspace}
\providecommand{\predicateBitpreciseParallelInvariantsResultsAsyncInvariantsPathCorrectTrueInvTimeSumPlainHours}{}
  \renewcommand{\predicateBitpreciseParallelInvariantsResultsAsyncInvariantsPathCorrectTrueInvTimeSumPlainHours}{0.0\xspace}

  % finished-main
\providecommand{\predicateBitpreciseParallelInvariantsResultsAsyncInvariantsPathCorrectTrueFinishedMainPlain}{}
  \renewcommand{\predicateBitpreciseParallelInvariantsResultsAsyncInvariantsPathCorrectTrueFinishedMainPlain}{587\xspace}

 %% incorrect-false %%
\providecommand{\predicateBitpreciseParallelInvariantsResultsAsyncInvariantsPathIncorrectFalsePlain}{}
  \renewcommand{\predicateBitpreciseParallelInvariantsResultsAsyncInvariantsPathIncorrectFalsePlain}{17\xspace}

  % cpu-time-sum
\providecommand{\predicateBitpreciseParallelInvariantsResultsAsyncInvariantsPathIncorrectFalseCpuTimeSumPlain}{}
  \renewcommand{\predicateBitpreciseParallelInvariantsResultsAsyncInvariantsPathIncorrectFalseCpuTimeSumPlain}{834.942784404\xspace}
\providecommand{\predicateBitpreciseParallelInvariantsResultsAsyncInvariantsPathIncorrectFalseCpuTimeSumPlainHours}{}
  \renewcommand{\predicateBitpreciseParallelInvariantsResultsAsyncInvariantsPathIncorrectFalseCpuTimeSumPlainHours}{0.23192855122333333\xspace}

  % wall-time-sum
\providecommand{\predicateBitpreciseParallelInvariantsResultsAsyncInvariantsPathIncorrectFalseWallTimeSumPlain}{}
  \renewcommand{\predicateBitpreciseParallelInvariantsResultsAsyncInvariantsPathIncorrectFalseWallTimeSumPlain}{268.75837206847007\xspace}
\providecommand{\predicateBitpreciseParallelInvariantsResultsAsyncInvariantsPathIncorrectFalseWallTimeSumPlainHours}{}
  \renewcommand{\predicateBitpreciseParallelInvariantsResultsAsyncInvariantsPathIncorrectFalseWallTimeSumPlainHours}{0.0746551033523528\xspace}

  % cpu-time-avg
\providecommand{\predicateBitpreciseParallelInvariantsResultsAsyncInvariantsPathIncorrectFalseCpuTimeAvgPlain}{}
  \renewcommand{\predicateBitpreciseParallelInvariantsResultsAsyncInvariantsPathIncorrectFalseCpuTimeAvgPlain}{49.11428143552941\xspace}
\providecommand{\predicateBitpreciseParallelInvariantsResultsAsyncInvariantsPathIncorrectFalseCpuTimeAvgPlainHours}{}
  \renewcommand{\predicateBitpreciseParallelInvariantsResultsAsyncInvariantsPathIncorrectFalseCpuTimeAvgPlainHours}{0.013642855954313726\xspace}

  % wall-time-avg
\providecommand{\predicateBitpreciseParallelInvariantsResultsAsyncInvariantsPathIncorrectFalseWallTimeAvgPlain}{}
  \renewcommand{\predicateBitpreciseParallelInvariantsResultsAsyncInvariantsPathIncorrectFalseWallTimeAvgPlain}{15.80931600402765\xspace}
\providecommand{\predicateBitpreciseParallelInvariantsResultsAsyncInvariantsPathIncorrectFalseWallTimeAvgPlainHours}{}
  \renewcommand{\predicateBitpreciseParallelInvariantsResultsAsyncInvariantsPathIncorrectFalseWallTimeAvgPlainHours}{0.004391476667785458\xspace}

  % inv-succ
\providecommand{\predicateBitpreciseParallelInvariantsResultsAsyncInvariantsPathIncorrectFalseInvSuccPlain}{}
  \renewcommand{\predicateBitpreciseParallelInvariantsResultsAsyncInvariantsPathIncorrectFalseInvSuccPlain}{0\xspace}

  % inv-tries
\providecommand{\predicateBitpreciseParallelInvariantsResultsAsyncInvariantsPathIncorrectFalseInvTriesPlain}{}
  \renewcommand{\predicateBitpreciseParallelInvariantsResultsAsyncInvariantsPathIncorrectFalseInvTriesPlain}{0\xspace}

  % inv-time-sum
\providecommand{\predicateBitpreciseParallelInvariantsResultsAsyncInvariantsPathIncorrectFalseInvTimeSumPlain}{}
  \renewcommand{\predicateBitpreciseParallelInvariantsResultsAsyncInvariantsPathIncorrectFalseInvTimeSumPlain}{0.0\xspace}
\providecommand{\predicateBitpreciseParallelInvariantsResultsAsyncInvariantsPathIncorrectFalseInvTimeSumPlainHours}{}
  \renewcommand{\predicateBitpreciseParallelInvariantsResultsAsyncInvariantsPathIncorrectFalseInvTimeSumPlainHours}{0.0\xspace}

  % finished-main
\providecommand{\predicateBitpreciseParallelInvariantsResultsAsyncInvariantsPathIncorrectFalseFinishedMainPlain}{}
  \renewcommand{\predicateBitpreciseParallelInvariantsResultsAsyncInvariantsPathIncorrectFalseFinishedMainPlain}{17\xspace}

 %% incorrect-true %%
\providecommand{\predicateBitpreciseParallelInvariantsResultsAsyncInvariantsPathIncorrectTruePlain}{}
  \renewcommand{\predicateBitpreciseParallelInvariantsResultsAsyncInvariantsPathIncorrectTruePlain}{1\xspace}

  % cpu-time-sum
\providecommand{\predicateBitpreciseParallelInvariantsResultsAsyncInvariantsPathIncorrectTrueCpuTimeSumPlain}{}
  \renewcommand{\predicateBitpreciseParallelInvariantsResultsAsyncInvariantsPathIncorrectTrueCpuTimeSumPlain}{14.608146389\xspace}
\providecommand{\predicateBitpreciseParallelInvariantsResultsAsyncInvariantsPathIncorrectTrueCpuTimeSumPlainHours}{}
  \renewcommand{\predicateBitpreciseParallelInvariantsResultsAsyncInvariantsPathIncorrectTrueCpuTimeSumPlainHours}{0.0040578184413888885\xspace}

  % wall-time-sum
\providecommand{\predicateBitpreciseParallelInvariantsResultsAsyncInvariantsPathIncorrectTrueWallTimeSumPlain}{}
  \renewcommand{\predicateBitpreciseParallelInvariantsResultsAsyncInvariantsPathIncorrectTrueWallTimeSumPlain}{4.78321385384\xspace}
\providecommand{\predicateBitpreciseParallelInvariantsResultsAsyncInvariantsPathIncorrectTrueWallTimeSumPlainHours}{}
  \renewcommand{\predicateBitpreciseParallelInvariantsResultsAsyncInvariantsPathIncorrectTrueWallTimeSumPlainHours}{0.0013286705149555557\xspace}

  % cpu-time-avg
\providecommand{\predicateBitpreciseParallelInvariantsResultsAsyncInvariantsPathIncorrectTrueCpuTimeAvgPlain}{}
  \renewcommand{\predicateBitpreciseParallelInvariantsResultsAsyncInvariantsPathIncorrectTrueCpuTimeAvgPlain}{14.608146389\xspace}
\providecommand{\predicateBitpreciseParallelInvariantsResultsAsyncInvariantsPathIncorrectTrueCpuTimeAvgPlainHours}{}
  \renewcommand{\predicateBitpreciseParallelInvariantsResultsAsyncInvariantsPathIncorrectTrueCpuTimeAvgPlainHours}{0.0040578184413888885\xspace}

  % wall-time-avg
\providecommand{\predicateBitpreciseParallelInvariantsResultsAsyncInvariantsPathIncorrectTrueWallTimeAvgPlain}{}
  \renewcommand{\predicateBitpreciseParallelInvariantsResultsAsyncInvariantsPathIncorrectTrueWallTimeAvgPlain}{4.78321385384\xspace}
\providecommand{\predicateBitpreciseParallelInvariantsResultsAsyncInvariantsPathIncorrectTrueWallTimeAvgPlainHours}{}
  \renewcommand{\predicateBitpreciseParallelInvariantsResultsAsyncInvariantsPathIncorrectTrueWallTimeAvgPlainHours}{0.0013286705149555557\xspace}

  % inv-succ
\providecommand{\predicateBitpreciseParallelInvariantsResultsAsyncInvariantsPathIncorrectTrueInvSuccPlain}{}
  \renewcommand{\predicateBitpreciseParallelInvariantsResultsAsyncInvariantsPathIncorrectTrueInvSuccPlain}{0\xspace}

  % inv-tries
\providecommand{\predicateBitpreciseParallelInvariantsResultsAsyncInvariantsPathIncorrectTrueInvTriesPlain}{}
  \renewcommand{\predicateBitpreciseParallelInvariantsResultsAsyncInvariantsPathIncorrectTrueInvTriesPlain}{0\xspace}

  % inv-time-sum
\providecommand{\predicateBitpreciseParallelInvariantsResultsAsyncInvariantsPathIncorrectTrueInvTimeSumPlain}{}
  \renewcommand{\predicateBitpreciseParallelInvariantsResultsAsyncInvariantsPathIncorrectTrueInvTimeSumPlain}{0.0\xspace}
\providecommand{\predicateBitpreciseParallelInvariantsResultsAsyncInvariantsPathIncorrectTrueInvTimeSumPlainHours}{}
  \renewcommand{\predicateBitpreciseParallelInvariantsResultsAsyncInvariantsPathIncorrectTrueInvTimeSumPlainHours}{0.0\xspace}

  % finished-main
\providecommand{\predicateBitpreciseParallelInvariantsResultsAsyncInvariantsPathIncorrectTrueFinishedMainPlain}{}
  \renewcommand{\predicateBitpreciseParallelInvariantsResultsAsyncInvariantsPathIncorrectTrueFinishedMainPlain}{1\xspace}

 %% all %%
\providecommand{\predicateBitpreciseParallelInvariantsResultsAsyncInvariantsPathAllPlain}{}
  \renewcommand{\predicateBitpreciseParallelInvariantsResultsAsyncInvariantsPathAllPlain}{3488\xspace}

  % cpu-time-sum
\providecommand{\predicateBitpreciseParallelInvariantsResultsAsyncInvariantsPathAllCpuTimeSumPlain}{}
  \renewcommand{\predicateBitpreciseParallelInvariantsResultsAsyncInvariantsPathAllCpuTimeSumPlain}{988188.9392694727\xspace}
\providecommand{\predicateBitpreciseParallelInvariantsResultsAsyncInvariantsPathAllCpuTimeSumPlainHours}{}
  \renewcommand{\predicateBitpreciseParallelInvariantsResultsAsyncInvariantsPathAllCpuTimeSumPlainHours}{274.49692757485354\xspace}

  % wall-time-sum
\providecommand{\predicateBitpreciseParallelInvariantsResultsAsyncInvariantsPathAllWallTimeSumPlain}{}
  \renewcommand{\predicateBitpreciseParallelInvariantsResultsAsyncInvariantsPathAllWallTimeSumPlain}{526113.3287684853\xspace}
\providecommand{\predicateBitpreciseParallelInvariantsResultsAsyncInvariantsPathAllWallTimeSumPlainHours}{}
  \renewcommand{\predicateBitpreciseParallelInvariantsResultsAsyncInvariantsPathAllWallTimeSumPlainHours}{146.14259132457926\xspace}

  % cpu-time-avg
\providecommand{\predicateBitpreciseParallelInvariantsResultsAsyncInvariantsPathAllCpuTimeAvgPlain}{}
  \renewcommand{\predicateBitpreciseParallelInvariantsResultsAsyncInvariantsPathAllCpuTimeAvgPlain}{283.31104910248644\xspace}
\providecommand{\predicateBitpreciseParallelInvariantsResultsAsyncInvariantsPathAllCpuTimeAvgPlainHours}{}
  \renewcommand{\predicateBitpreciseParallelInvariantsResultsAsyncInvariantsPathAllCpuTimeAvgPlainHours}{0.07869751363957957\xspace}

  % wall-time-avg
\providecommand{\predicateBitpreciseParallelInvariantsResultsAsyncInvariantsPathAllWallTimeAvgPlain}{}
  \renewcommand{\predicateBitpreciseParallelInvariantsResultsAsyncInvariantsPathAllWallTimeAvgPlain}{150.8352433395887\xspace}
\providecommand{\predicateBitpreciseParallelInvariantsResultsAsyncInvariantsPathAllWallTimeAvgPlainHours}{}
  \renewcommand{\predicateBitpreciseParallelInvariantsResultsAsyncInvariantsPathAllWallTimeAvgPlainHours}{0.041898678705441304\xspace}

  % inv-succ
\providecommand{\predicateBitpreciseParallelInvariantsResultsAsyncInvariantsPathAllInvSuccPlain}{}
  \renewcommand{\predicateBitpreciseParallelInvariantsResultsAsyncInvariantsPathAllInvSuccPlain}{0\xspace}

  % inv-tries
\providecommand{\predicateBitpreciseParallelInvariantsResultsAsyncInvariantsPathAllInvTriesPlain}{}
  \renewcommand{\predicateBitpreciseParallelInvariantsResultsAsyncInvariantsPathAllInvTriesPlain}{0\xspace}

  % inv-time-sum
\providecommand{\predicateBitpreciseParallelInvariantsResultsAsyncInvariantsPathAllInvTimeSumPlain}{}
  \renewcommand{\predicateBitpreciseParallelInvariantsResultsAsyncInvariantsPathAllInvTimeSumPlain}{0.0\xspace}
\providecommand{\predicateBitpreciseParallelInvariantsResultsAsyncInvariantsPathAllInvTimeSumPlainHours}{}
  \renewcommand{\predicateBitpreciseParallelInvariantsResultsAsyncInvariantsPathAllInvTimeSumPlainHours}{0.0\xspace}

  % finished-main
\providecommand{\predicateBitpreciseParallelInvariantsResultsAsyncInvariantsPathAllFinishedMainPlain}{}
  \renewcommand{\predicateBitpreciseParallelInvariantsResultsAsyncInvariantsPathAllFinishedMainPlain}{1167\xspace}

 %% equal-only %%
\providecommand{\predicateBitpreciseParallelInvariantsResultsAsyncInvariantsPathEqualOnlyPlain}{}
  \renewcommand{\predicateBitpreciseParallelInvariantsResultsAsyncInvariantsPathEqualOnlyPlain}{1865\xspace}

  % cpu-time-sum
\providecommand{\predicateBitpreciseParallelInvariantsResultsAsyncInvariantsPathEqualOnlyCpuTimeSumPlain}{}
  \renewcommand{\predicateBitpreciseParallelInvariantsResultsAsyncInvariantsPathEqualOnlyCpuTimeSumPlain}{136393.1281000529\xspace}
\providecommand{\predicateBitpreciseParallelInvariantsResultsAsyncInvariantsPathEqualOnlyCpuTimeSumPlainHours}{}
  \renewcommand{\predicateBitpreciseParallelInvariantsResultsAsyncInvariantsPathEqualOnlyCpuTimeSumPlainHours}{37.886980027792475\xspace}

  % wall-time-sum
\providecommand{\predicateBitpreciseParallelInvariantsResultsAsyncInvariantsPathEqualOnlyWallTimeSumPlain}{}
  \renewcommand{\predicateBitpreciseParallelInvariantsResultsAsyncInvariantsPathEqualOnlyWallTimeSumPlain}{51105.93448376244\xspace}
\providecommand{\predicateBitpreciseParallelInvariantsResultsAsyncInvariantsPathEqualOnlyWallTimeSumPlainHours}{}
  \renewcommand{\predicateBitpreciseParallelInvariantsResultsAsyncInvariantsPathEqualOnlyWallTimeSumPlainHours}{14.196092912156233\xspace}

  % cpu-time-avg
\providecommand{\predicateBitpreciseParallelInvariantsResultsAsyncInvariantsPathEqualOnlyCpuTimeAvgPlain}{}
  \renewcommand{\predicateBitpreciseParallelInvariantsResultsAsyncInvariantsPathEqualOnlyCpuTimeAvgPlain}{73.13304455766912\xspace}
\providecommand{\predicateBitpreciseParallelInvariantsResultsAsyncInvariantsPathEqualOnlyCpuTimeAvgPlainHours}{}
  \renewcommand{\predicateBitpreciseParallelInvariantsResultsAsyncInvariantsPathEqualOnlyCpuTimeAvgPlainHours}{0.020314734599352534\xspace}

  % wall-time-avg
\providecommand{\predicateBitpreciseParallelInvariantsResultsAsyncInvariantsPathEqualOnlyWallTimeAvgPlain}{}
  \renewcommand{\predicateBitpreciseParallelInvariantsResultsAsyncInvariantsPathEqualOnlyWallTimeAvgPlain}{27.402645835797554\xspace}
\providecommand{\predicateBitpreciseParallelInvariantsResultsAsyncInvariantsPathEqualOnlyWallTimeAvgPlainHours}{}
  \renewcommand{\predicateBitpreciseParallelInvariantsResultsAsyncInvariantsPathEqualOnlyWallTimeAvgPlainHours}{0.007611846065499321\xspace}

  % inv-succ
\providecommand{\predicateBitpreciseParallelInvariantsResultsAsyncInvariantsPathEqualOnlyInvSuccPlain}{}
  \renewcommand{\predicateBitpreciseParallelInvariantsResultsAsyncInvariantsPathEqualOnlyInvSuccPlain}{0\xspace}

  % inv-tries
\providecommand{\predicateBitpreciseParallelInvariantsResultsAsyncInvariantsPathEqualOnlyInvTriesPlain}{}
  \renewcommand{\predicateBitpreciseParallelInvariantsResultsAsyncInvariantsPathEqualOnlyInvTriesPlain}{0\xspace}

  % inv-time-sum
\providecommand{\predicateBitpreciseParallelInvariantsResultsAsyncInvariantsPathEqualOnlyInvTimeSumPlain}{}
  \renewcommand{\predicateBitpreciseParallelInvariantsResultsAsyncInvariantsPathEqualOnlyInvTimeSumPlain}{0.0\xspace}
\providecommand{\predicateBitpreciseParallelInvariantsResultsAsyncInvariantsPathEqualOnlyInvTimeSumPlainHours}{}
  \renewcommand{\predicateBitpreciseParallelInvariantsResultsAsyncInvariantsPathEqualOnlyInvTimeSumPlainHours}{0.0\xspace}

  % finished-main
\providecommand{\predicateBitpreciseParallelInvariantsResultsAsyncInvariantsPathEqualOnlyFinishedMainPlain}{}
  \renewcommand{\predicateBitpreciseParallelInvariantsResultsAsyncInvariantsPathEqualOnlyFinishedMainPlain}{1045\xspace}

%%% predicate_bitprecise_parallel_invariants.2016-09-05_0219.results.async-invariants-abs %%%
 %% correct %%
\providecommand{\predicateBitpreciseParallelInvariantsResultsAsyncInvariantsAbsCorrectPlain}{}
  \renewcommand{\predicateBitpreciseParallelInvariantsResultsAsyncInvariantsAbsCorrectPlain}{2104\xspace}

  % cpu-time-sum
\providecommand{\predicateBitpreciseParallelInvariantsResultsAsyncInvariantsAbsCorrectCpuTimeSumPlain}{}
  \renewcommand{\predicateBitpreciseParallelInvariantsResultsAsyncInvariantsAbsCorrectCpuTimeSumPlain}{199877.50180124154\xspace}
\providecommand{\predicateBitpreciseParallelInvariantsResultsAsyncInvariantsAbsCorrectCpuTimeSumPlainHours}{}
  \renewcommand{\predicateBitpreciseParallelInvariantsResultsAsyncInvariantsAbsCorrectCpuTimeSumPlainHours}{55.52152827812265\xspace}

  % wall-time-sum
\providecommand{\predicateBitpreciseParallelInvariantsResultsAsyncInvariantsAbsCorrectWallTimeSumPlain}{}
  \renewcommand{\predicateBitpreciseParallelInvariantsResultsAsyncInvariantsAbsCorrectWallTimeSumPlain}{81094.2989759324\xspace}
\providecommand{\predicateBitpreciseParallelInvariantsResultsAsyncInvariantsAbsCorrectWallTimeSumPlainHours}{}
  \renewcommand{\predicateBitpreciseParallelInvariantsResultsAsyncInvariantsAbsCorrectWallTimeSumPlainHours}{22.526194159981223\xspace}

  % cpu-time-avg
\providecommand{\predicateBitpreciseParallelInvariantsResultsAsyncInvariantsAbsCorrectCpuTimeAvgPlain}{}
  \renewcommand{\predicateBitpreciseParallelInvariantsResultsAsyncInvariantsAbsCorrectCpuTimeAvgPlain}{94.99881264317564\xspace}
\providecommand{\predicateBitpreciseParallelInvariantsResultsAsyncInvariantsAbsCorrectCpuTimeAvgPlainHours}{}
  \renewcommand{\predicateBitpreciseParallelInvariantsResultsAsyncInvariantsAbsCorrectCpuTimeAvgPlainHours}{0.02638855906754879\xspace}

  % wall-time-avg
\providecommand{\predicateBitpreciseParallelInvariantsResultsAsyncInvariantsAbsCorrectWallTimeAvgPlain}{}
  \renewcommand{\predicateBitpreciseParallelInvariantsResultsAsyncInvariantsAbsCorrectWallTimeAvgPlain}{38.542917764226424\xspace}
\providecommand{\predicateBitpreciseParallelInvariantsResultsAsyncInvariantsAbsCorrectWallTimeAvgPlainHours}{}
  \renewcommand{\predicateBitpreciseParallelInvariantsResultsAsyncInvariantsAbsCorrectWallTimeAvgPlainHours}{0.01070636604561845\xspace}

  % inv-succ
\providecommand{\predicateBitpreciseParallelInvariantsResultsAsyncInvariantsAbsCorrectInvSuccPlain}{}
  \renewcommand{\predicateBitpreciseParallelInvariantsResultsAsyncInvariantsAbsCorrectInvSuccPlain}{0\xspace}

  % inv-tries
\providecommand{\predicateBitpreciseParallelInvariantsResultsAsyncInvariantsAbsCorrectInvTriesPlain}{}
  \renewcommand{\predicateBitpreciseParallelInvariantsResultsAsyncInvariantsAbsCorrectInvTriesPlain}{0\xspace}

  % inv-time-sum
\providecommand{\predicateBitpreciseParallelInvariantsResultsAsyncInvariantsAbsCorrectInvTimeSumPlain}{}
  \renewcommand{\predicateBitpreciseParallelInvariantsResultsAsyncInvariantsAbsCorrectInvTimeSumPlain}{0.0\xspace}
\providecommand{\predicateBitpreciseParallelInvariantsResultsAsyncInvariantsAbsCorrectInvTimeSumPlainHours}{}
  \renewcommand{\predicateBitpreciseParallelInvariantsResultsAsyncInvariantsAbsCorrectInvTimeSumPlainHours}{0.0\xspace}

  % finished-main
\providecommand{\predicateBitpreciseParallelInvariantsResultsAsyncInvariantsAbsCorrectFinishedMainPlain}{}
  \renewcommand{\predicateBitpreciseParallelInvariantsResultsAsyncInvariantsAbsCorrectFinishedMainPlain}{1154\xspace}

 %% incorrect %%
\providecommand{\predicateBitpreciseParallelInvariantsResultsAsyncInvariantsAbsIncorrectPlain}{}
  \renewcommand{\predicateBitpreciseParallelInvariantsResultsAsyncInvariantsAbsIncorrectPlain}{18\xspace}

  % cpu-time-sum
\providecommand{\predicateBitpreciseParallelInvariantsResultsAsyncInvariantsAbsIncorrectCpuTimeSumPlain}{}
  \renewcommand{\predicateBitpreciseParallelInvariantsResultsAsyncInvariantsAbsIncorrectCpuTimeSumPlain}{902.6310161410001\xspace}
\providecommand{\predicateBitpreciseParallelInvariantsResultsAsyncInvariantsAbsIncorrectCpuTimeSumPlainHours}{}
  \renewcommand{\predicateBitpreciseParallelInvariantsResultsAsyncInvariantsAbsIncorrectCpuTimeSumPlainHours}{0.2507308378169445\xspace}

  % wall-time-sum
\providecommand{\predicateBitpreciseParallelInvariantsResultsAsyncInvariantsAbsIncorrectWallTimeSumPlain}{}
  \renewcommand{\predicateBitpreciseParallelInvariantsResultsAsyncInvariantsAbsIncorrectWallTimeSumPlain}{288.06467247008004\xspace}
\providecommand{\predicateBitpreciseParallelInvariantsResultsAsyncInvariantsAbsIncorrectWallTimeSumPlainHours}{}
  \renewcommand{\predicateBitpreciseParallelInvariantsResultsAsyncInvariantsAbsIncorrectWallTimeSumPlainHours}{0.08001796457502224\xspace}

  % cpu-time-avg
\providecommand{\predicateBitpreciseParallelInvariantsResultsAsyncInvariantsAbsIncorrectCpuTimeAvgPlain}{}
  \renewcommand{\predicateBitpreciseParallelInvariantsResultsAsyncInvariantsAbsIncorrectCpuTimeAvgPlain}{50.1461675633889\xspace}
\providecommand{\predicateBitpreciseParallelInvariantsResultsAsyncInvariantsAbsIncorrectCpuTimeAvgPlainHours}{}
  \renewcommand{\predicateBitpreciseParallelInvariantsResultsAsyncInvariantsAbsIncorrectCpuTimeAvgPlainHours}{0.01392949098983025\xspace}

  % wall-time-avg
\providecommand{\predicateBitpreciseParallelInvariantsResultsAsyncInvariantsAbsIncorrectWallTimeAvgPlain}{}
  \renewcommand{\predicateBitpreciseParallelInvariantsResultsAsyncInvariantsAbsIncorrectWallTimeAvgPlain}{16.003592915004447\xspace}
\providecommand{\predicateBitpreciseParallelInvariantsResultsAsyncInvariantsAbsIncorrectWallTimeAvgPlainHours}{}
  \renewcommand{\predicateBitpreciseParallelInvariantsResultsAsyncInvariantsAbsIncorrectWallTimeAvgPlainHours}{0.004445442476390124\xspace}

  % inv-succ
\providecommand{\predicateBitpreciseParallelInvariantsResultsAsyncInvariantsAbsIncorrectInvSuccPlain}{}
  \renewcommand{\predicateBitpreciseParallelInvariantsResultsAsyncInvariantsAbsIncorrectInvSuccPlain}{0\xspace}

  % inv-tries
\providecommand{\predicateBitpreciseParallelInvariantsResultsAsyncInvariantsAbsIncorrectInvTriesPlain}{}
  \renewcommand{\predicateBitpreciseParallelInvariantsResultsAsyncInvariantsAbsIncorrectInvTriesPlain}{0\xspace}

  % inv-time-sum
\providecommand{\predicateBitpreciseParallelInvariantsResultsAsyncInvariantsAbsIncorrectInvTimeSumPlain}{}
  \renewcommand{\predicateBitpreciseParallelInvariantsResultsAsyncInvariantsAbsIncorrectInvTimeSumPlain}{0.0\xspace}
\providecommand{\predicateBitpreciseParallelInvariantsResultsAsyncInvariantsAbsIncorrectInvTimeSumPlainHours}{}
  \renewcommand{\predicateBitpreciseParallelInvariantsResultsAsyncInvariantsAbsIncorrectInvTimeSumPlainHours}{0.0\xspace}

  % finished-main
\providecommand{\predicateBitpreciseParallelInvariantsResultsAsyncInvariantsAbsIncorrectFinishedMainPlain}{}
  \renewcommand{\predicateBitpreciseParallelInvariantsResultsAsyncInvariantsAbsIncorrectFinishedMainPlain}{18\xspace}

 %% timeout %%
\providecommand{\predicateBitpreciseParallelInvariantsResultsAsyncInvariantsAbsTimeoutPlain}{}
  \renewcommand{\predicateBitpreciseParallelInvariantsResultsAsyncInvariantsAbsTimeoutPlain}{1198\xspace}

  % cpu-time-sum
\providecommand{\predicateBitpreciseParallelInvariantsResultsAsyncInvariantsAbsTimeoutCpuTimeSumPlain}{}
  \renewcommand{\predicateBitpreciseParallelInvariantsResultsAsyncInvariantsAbsTimeoutCpuTimeSumPlain}{736867.8009033024\xspace}
\providecommand{\predicateBitpreciseParallelInvariantsResultsAsyncInvariantsAbsTimeoutCpuTimeSumPlainHours}{}
  \renewcommand{\predicateBitpreciseParallelInvariantsResultsAsyncInvariantsAbsTimeoutCpuTimeSumPlainHours}{204.68550025091733\xspace}

  % wall-time-sum
\providecommand{\predicateBitpreciseParallelInvariantsResultsAsyncInvariantsAbsTimeoutWallTimeSumPlain}{}
  \renewcommand{\predicateBitpreciseParallelInvariantsResultsAsyncInvariantsAbsTimeoutWallTimeSumPlain}{414103.85633969516\xspace}
\providecommand{\predicateBitpreciseParallelInvariantsResultsAsyncInvariantsAbsTimeoutWallTimeSumPlainHours}{}
  \renewcommand{\predicateBitpreciseParallelInvariantsResultsAsyncInvariantsAbsTimeoutWallTimeSumPlainHours}{115.02884898324865\xspace}

  % cpu-time-avg
\providecommand{\predicateBitpreciseParallelInvariantsResultsAsyncInvariantsAbsTimeoutCpuTimeAvgPlain}{}
  \renewcommand{\predicateBitpreciseParallelInvariantsResultsAsyncInvariantsAbsTimeoutCpuTimeAvgPlain}{615.0816368141088\xspace}
\providecommand{\predicateBitpreciseParallelInvariantsResultsAsyncInvariantsAbsTimeoutCpuTimeAvgPlainHours}{}
  \renewcommand{\predicateBitpreciseParallelInvariantsResultsAsyncInvariantsAbsTimeoutCpuTimeAvgPlainHours}{0.17085601022614133\xspace}

  % wall-time-avg
\providecommand{\predicateBitpreciseParallelInvariantsResultsAsyncInvariantsAbsTimeoutWallTimeAvgPlain}{}
  \renewcommand{\predicateBitpreciseParallelInvariantsResultsAsyncInvariantsAbsTimeoutWallTimeAvgPlain}{345.6626513686938\xspace}
\providecommand{\predicateBitpreciseParallelInvariantsResultsAsyncInvariantsAbsTimeoutWallTimeAvgPlainHours}{}
  \renewcommand{\predicateBitpreciseParallelInvariantsResultsAsyncInvariantsAbsTimeoutWallTimeAvgPlainHours}{0.0960174031579705\xspace}

  % inv-succ
\providecommand{\predicateBitpreciseParallelInvariantsResultsAsyncInvariantsAbsTimeoutInvSuccPlain}{}
  \renewcommand{\predicateBitpreciseParallelInvariantsResultsAsyncInvariantsAbsTimeoutInvSuccPlain}{0\xspace}

  % inv-tries
\providecommand{\predicateBitpreciseParallelInvariantsResultsAsyncInvariantsAbsTimeoutInvTriesPlain}{}
  \renewcommand{\predicateBitpreciseParallelInvariantsResultsAsyncInvariantsAbsTimeoutInvTriesPlain}{0\xspace}

  % inv-time-sum
\providecommand{\predicateBitpreciseParallelInvariantsResultsAsyncInvariantsAbsTimeoutInvTimeSumPlain}{}
  \renewcommand{\predicateBitpreciseParallelInvariantsResultsAsyncInvariantsAbsTimeoutInvTimeSumPlain}{0.0\xspace}
\providecommand{\predicateBitpreciseParallelInvariantsResultsAsyncInvariantsAbsTimeoutInvTimeSumPlainHours}{}
  \renewcommand{\predicateBitpreciseParallelInvariantsResultsAsyncInvariantsAbsTimeoutInvTimeSumPlainHours}{0.0\xspace}

  % finished-main
\providecommand{\predicateBitpreciseParallelInvariantsResultsAsyncInvariantsAbsTimeoutFinishedMainPlain}{}
  \renewcommand{\predicateBitpreciseParallelInvariantsResultsAsyncInvariantsAbsTimeoutFinishedMainPlain}{0\xspace}

 %% unknown-or-category-error %%
\providecommand{\predicateBitpreciseParallelInvariantsResultsAsyncInvariantsAbsUnknownOrCategoryErrorPlain}{}
  \renewcommand{\predicateBitpreciseParallelInvariantsResultsAsyncInvariantsAbsUnknownOrCategoryErrorPlain}{1366\xspace}

  % cpu-time-sum
\providecommand{\predicateBitpreciseParallelInvariantsResultsAsyncInvariantsAbsUnknownOrCategoryErrorCpuTimeSumPlain}{}
  \renewcommand{\predicateBitpreciseParallelInvariantsResultsAsyncInvariantsAbsUnknownOrCategoryErrorCpuTimeSumPlain}{792841.3160317022\xspace}
\providecommand{\predicateBitpreciseParallelInvariantsResultsAsyncInvariantsAbsUnknownOrCategoryErrorCpuTimeSumPlainHours}{}
  \renewcommand{\predicateBitpreciseParallelInvariantsResultsAsyncInvariantsAbsUnknownOrCategoryErrorCpuTimeSumPlainHours}{220.23369889769506\xspace}

  % wall-time-sum
\providecommand{\predicateBitpreciseParallelInvariantsResultsAsyncInvariantsAbsUnknownOrCategoryErrorWallTimeSumPlain}{}
  \renewcommand{\predicateBitpreciseParallelInvariantsResultsAsyncInvariantsAbsUnknownOrCategoryErrorWallTimeSumPlain}{448715.1259589222\xspace}
\providecommand{\predicateBitpreciseParallelInvariantsResultsAsyncInvariantsAbsUnknownOrCategoryErrorWallTimeSumPlainHours}{}
  \renewcommand{\predicateBitpreciseParallelInvariantsResultsAsyncInvariantsAbsUnknownOrCategoryErrorWallTimeSumPlainHours}{124.64309054414505\xspace}

  % cpu-time-avg
\providecommand{\predicateBitpreciseParallelInvariantsResultsAsyncInvariantsAbsUnknownOrCategoryErrorCpuTimeAvgPlain}{}
  \renewcommand{\predicateBitpreciseParallelInvariantsResultsAsyncInvariantsAbsUnknownOrCategoryErrorCpuTimeAvgPlain}{580.4109194961217\xspace}
\providecommand{\predicateBitpreciseParallelInvariantsResultsAsyncInvariantsAbsUnknownOrCategoryErrorCpuTimeAvgPlainHours}{}
  \renewcommand{\predicateBitpreciseParallelInvariantsResultsAsyncInvariantsAbsUnknownOrCategoryErrorCpuTimeAvgPlainHours}{0.16122525541558935\xspace}

  % wall-time-avg
\providecommand{\predicateBitpreciseParallelInvariantsResultsAsyncInvariantsAbsUnknownOrCategoryErrorWallTimeAvgPlain}{}
  \renewcommand{\predicateBitpreciseParallelInvariantsResultsAsyncInvariantsAbsUnknownOrCategoryErrorWallTimeAvgPlain}{328.4883791792988\xspace}
\providecommand{\predicateBitpreciseParallelInvariantsResultsAsyncInvariantsAbsUnknownOrCategoryErrorWallTimeAvgPlainHours}{}
  \renewcommand{\predicateBitpreciseParallelInvariantsResultsAsyncInvariantsAbsUnknownOrCategoryErrorWallTimeAvgPlainHours}{0.09124677199424967\xspace}

  % inv-succ
\providecommand{\predicateBitpreciseParallelInvariantsResultsAsyncInvariantsAbsUnknownOrCategoryErrorInvSuccPlain}{}
  \renewcommand{\predicateBitpreciseParallelInvariantsResultsAsyncInvariantsAbsUnknownOrCategoryErrorInvSuccPlain}{0\xspace}

  % inv-tries
\providecommand{\predicateBitpreciseParallelInvariantsResultsAsyncInvariantsAbsUnknownOrCategoryErrorInvTriesPlain}{}
  \renewcommand{\predicateBitpreciseParallelInvariantsResultsAsyncInvariantsAbsUnknownOrCategoryErrorInvTriesPlain}{0\xspace}

  % inv-time-sum
\providecommand{\predicateBitpreciseParallelInvariantsResultsAsyncInvariantsAbsUnknownOrCategoryErrorInvTimeSumPlain}{}
  \renewcommand{\predicateBitpreciseParallelInvariantsResultsAsyncInvariantsAbsUnknownOrCategoryErrorInvTimeSumPlain}{0.0\xspace}
\providecommand{\predicateBitpreciseParallelInvariantsResultsAsyncInvariantsAbsUnknownOrCategoryErrorInvTimeSumPlainHours}{}
  \renewcommand{\predicateBitpreciseParallelInvariantsResultsAsyncInvariantsAbsUnknownOrCategoryErrorInvTimeSumPlainHours}{0.0\xspace}

  % finished-main
\providecommand{\predicateBitpreciseParallelInvariantsResultsAsyncInvariantsAbsUnknownOrCategoryErrorFinishedMainPlain}{}
  \renewcommand{\predicateBitpreciseParallelInvariantsResultsAsyncInvariantsAbsUnknownOrCategoryErrorFinishedMainPlain}{1\xspace}

 %% correct-false %%
\providecommand{\predicateBitpreciseParallelInvariantsResultsAsyncInvariantsAbsCorrectFalsePlain}{}
  \renewcommand{\predicateBitpreciseParallelInvariantsResultsAsyncInvariantsAbsCorrectFalsePlain}{572\xspace}

  % cpu-time-sum
\providecommand{\predicateBitpreciseParallelInvariantsResultsAsyncInvariantsAbsCorrectFalseCpuTimeSumPlain}{}
  \renewcommand{\predicateBitpreciseParallelInvariantsResultsAsyncInvariantsAbsCorrectFalseCpuTimeSumPlain}{82501.03526299485\xspace}
\providecommand{\predicateBitpreciseParallelInvariantsResultsAsyncInvariantsAbsCorrectFalseCpuTimeSumPlainHours}{}
  \renewcommand{\predicateBitpreciseParallelInvariantsResultsAsyncInvariantsAbsCorrectFalseCpuTimeSumPlainHours}{22.91695423972079\xspace}

  % wall-time-sum
\providecommand{\predicateBitpreciseParallelInvariantsResultsAsyncInvariantsAbsCorrectFalseWallTimeSumPlain}{}
  \renewcommand{\predicateBitpreciseParallelInvariantsResultsAsyncInvariantsAbsCorrectFalseWallTimeSumPlain}{35416.835423938166\xspace}
\providecommand{\predicateBitpreciseParallelInvariantsResultsAsyncInvariantsAbsCorrectFalseWallTimeSumPlainHours}{}
  \renewcommand{\predicateBitpreciseParallelInvariantsResultsAsyncInvariantsAbsCorrectFalseWallTimeSumPlainHours}{9.838009839982824\xspace}

  % cpu-time-avg
\providecommand{\predicateBitpreciseParallelInvariantsResultsAsyncInvariantsAbsCorrectFalseCpuTimeAvgPlain}{}
  \renewcommand{\predicateBitpreciseParallelInvariantsResultsAsyncInvariantsAbsCorrectFalseCpuTimeAvgPlain}{144.23257913110987\xspace}
\providecommand{\predicateBitpreciseParallelInvariantsResultsAsyncInvariantsAbsCorrectFalseCpuTimeAvgPlainHours}{}
  \renewcommand{\predicateBitpreciseParallelInvariantsResultsAsyncInvariantsAbsCorrectFalseCpuTimeAvgPlainHours}{0.040064605314197185\xspace}

  % wall-time-avg
\providecommand{\predicateBitpreciseParallelInvariantsResultsAsyncInvariantsAbsCorrectFalseWallTimeAvgPlain}{}
  \renewcommand{\predicateBitpreciseParallelInvariantsResultsAsyncInvariantsAbsCorrectFalseWallTimeAvgPlain}{61.91754444744435\xspace}
\providecommand{\predicateBitpreciseParallelInvariantsResultsAsyncInvariantsAbsCorrectFalseWallTimeAvgPlainHours}{}
  \renewcommand{\predicateBitpreciseParallelInvariantsResultsAsyncInvariantsAbsCorrectFalseWallTimeAvgPlainHours}{0.017199317902067874\xspace}

  % inv-succ
\providecommand{\predicateBitpreciseParallelInvariantsResultsAsyncInvariantsAbsCorrectFalseInvSuccPlain}{}
  \renewcommand{\predicateBitpreciseParallelInvariantsResultsAsyncInvariantsAbsCorrectFalseInvSuccPlain}{0\xspace}

  % inv-tries
\providecommand{\predicateBitpreciseParallelInvariantsResultsAsyncInvariantsAbsCorrectFalseInvTriesPlain}{}
  \renewcommand{\predicateBitpreciseParallelInvariantsResultsAsyncInvariantsAbsCorrectFalseInvTriesPlain}{0\xspace}

  % inv-time-sum
\providecommand{\predicateBitpreciseParallelInvariantsResultsAsyncInvariantsAbsCorrectFalseInvTimeSumPlain}{}
  \renewcommand{\predicateBitpreciseParallelInvariantsResultsAsyncInvariantsAbsCorrectFalseInvTimeSumPlain}{0.0\xspace}
\providecommand{\predicateBitpreciseParallelInvariantsResultsAsyncInvariantsAbsCorrectFalseInvTimeSumPlainHours}{}
  \renewcommand{\predicateBitpreciseParallelInvariantsResultsAsyncInvariantsAbsCorrectFalseInvTimeSumPlainHours}{0.0\xspace}

  % finished-main
\providecommand{\predicateBitpreciseParallelInvariantsResultsAsyncInvariantsAbsCorrectFalseFinishedMainPlain}{}
  \renewcommand{\predicateBitpreciseParallelInvariantsResultsAsyncInvariantsAbsCorrectFalseFinishedMainPlain}{572\xspace}

 %% correct-true %%
\providecommand{\predicateBitpreciseParallelInvariantsResultsAsyncInvariantsAbsCorrectTruePlain}{}
  \renewcommand{\predicateBitpreciseParallelInvariantsResultsAsyncInvariantsAbsCorrectTruePlain}{1532\xspace}

  % cpu-time-sum
\providecommand{\predicateBitpreciseParallelInvariantsResultsAsyncInvariantsAbsCorrectTrueCpuTimeSumPlain}{}
  \renewcommand{\predicateBitpreciseParallelInvariantsResultsAsyncInvariantsAbsCorrectTrueCpuTimeSumPlain}{117376.4665382471\xspace}
\providecommand{\predicateBitpreciseParallelInvariantsResultsAsyncInvariantsAbsCorrectTrueCpuTimeSumPlainHours}{}
  \renewcommand{\predicateBitpreciseParallelInvariantsResultsAsyncInvariantsAbsCorrectTrueCpuTimeSumPlainHours}{32.60457403840197\xspace}

  % wall-time-sum
\providecommand{\predicateBitpreciseParallelInvariantsResultsAsyncInvariantsAbsCorrectTrueWallTimeSumPlain}{}
  \renewcommand{\predicateBitpreciseParallelInvariantsResultsAsyncInvariantsAbsCorrectTrueWallTimeSumPlain}{45677.46355199429\xspace}
\providecommand{\predicateBitpreciseParallelInvariantsResultsAsyncInvariantsAbsCorrectTrueWallTimeSumPlainHours}{}
  \renewcommand{\predicateBitpreciseParallelInvariantsResultsAsyncInvariantsAbsCorrectTrueWallTimeSumPlainHours}{12.688184319998413\xspace}

  % cpu-time-avg
\providecommand{\predicateBitpreciseParallelInvariantsResultsAsyncInvariantsAbsCorrectTrueCpuTimeAvgPlain}{}
  \renewcommand{\predicateBitpreciseParallelInvariantsResultsAsyncInvariantsAbsCorrectTrueCpuTimeAvgPlain}{76.61649251843805\xspace}
\providecommand{\predicateBitpreciseParallelInvariantsResultsAsyncInvariantsAbsCorrectTrueCpuTimeAvgPlainHours}{}
  \renewcommand{\predicateBitpreciseParallelInvariantsResultsAsyncInvariantsAbsCorrectTrueCpuTimeAvgPlainHours}{0.021282359032899458\xspace}

  % wall-time-avg
\providecommand{\predicateBitpreciseParallelInvariantsResultsAsyncInvariantsAbsCorrectTrueWallTimeAvgPlain}{}
  \renewcommand{\predicateBitpreciseParallelInvariantsResultsAsyncInvariantsAbsCorrectTrueWallTimeAvgPlain}{29.81557673106677\xspace}
\providecommand{\predicateBitpreciseParallelInvariantsResultsAsyncInvariantsAbsCorrectTrueWallTimeAvgPlainHours}{}
  \renewcommand{\predicateBitpreciseParallelInvariantsResultsAsyncInvariantsAbsCorrectTrueWallTimeAvgPlainHours}{0.008282104647518548\xspace}

  % inv-succ
\providecommand{\predicateBitpreciseParallelInvariantsResultsAsyncInvariantsAbsCorrectTrueInvSuccPlain}{}
  \renewcommand{\predicateBitpreciseParallelInvariantsResultsAsyncInvariantsAbsCorrectTrueInvSuccPlain}{0\xspace}

  % inv-tries
\providecommand{\predicateBitpreciseParallelInvariantsResultsAsyncInvariantsAbsCorrectTrueInvTriesPlain}{}
  \renewcommand{\predicateBitpreciseParallelInvariantsResultsAsyncInvariantsAbsCorrectTrueInvTriesPlain}{0\xspace}

  % inv-time-sum
\providecommand{\predicateBitpreciseParallelInvariantsResultsAsyncInvariantsAbsCorrectTrueInvTimeSumPlain}{}
  \renewcommand{\predicateBitpreciseParallelInvariantsResultsAsyncInvariantsAbsCorrectTrueInvTimeSumPlain}{0.0\xspace}
\providecommand{\predicateBitpreciseParallelInvariantsResultsAsyncInvariantsAbsCorrectTrueInvTimeSumPlainHours}{}
  \renewcommand{\predicateBitpreciseParallelInvariantsResultsAsyncInvariantsAbsCorrectTrueInvTimeSumPlainHours}{0.0\xspace}

  % finished-main
\providecommand{\predicateBitpreciseParallelInvariantsResultsAsyncInvariantsAbsCorrectTrueFinishedMainPlain}{}
  \renewcommand{\predicateBitpreciseParallelInvariantsResultsAsyncInvariantsAbsCorrectTrueFinishedMainPlain}{582\xspace}

 %% incorrect-false %%
\providecommand{\predicateBitpreciseParallelInvariantsResultsAsyncInvariantsAbsIncorrectFalsePlain}{}
  \renewcommand{\predicateBitpreciseParallelInvariantsResultsAsyncInvariantsAbsIncorrectFalsePlain}{18\xspace}

  % cpu-time-sum
\providecommand{\predicateBitpreciseParallelInvariantsResultsAsyncInvariantsAbsIncorrectFalseCpuTimeSumPlain}{}
  \renewcommand{\predicateBitpreciseParallelInvariantsResultsAsyncInvariantsAbsIncorrectFalseCpuTimeSumPlain}{902.6310161410001\xspace}
\providecommand{\predicateBitpreciseParallelInvariantsResultsAsyncInvariantsAbsIncorrectFalseCpuTimeSumPlainHours}{}
  \renewcommand{\predicateBitpreciseParallelInvariantsResultsAsyncInvariantsAbsIncorrectFalseCpuTimeSumPlainHours}{0.2507308378169445\xspace}

  % wall-time-sum
\providecommand{\predicateBitpreciseParallelInvariantsResultsAsyncInvariantsAbsIncorrectFalseWallTimeSumPlain}{}
  \renewcommand{\predicateBitpreciseParallelInvariantsResultsAsyncInvariantsAbsIncorrectFalseWallTimeSumPlain}{288.06467247008004\xspace}
\providecommand{\predicateBitpreciseParallelInvariantsResultsAsyncInvariantsAbsIncorrectFalseWallTimeSumPlainHours}{}
  \renewcommand{\predicateBitpreciseParallelInvariantsResultsAsyncInvariantsAbsIncorrectFalseWallTimeSumPlainHours}{0.08001796457502224\xspace}

  % cpu-time-avg
\providecommand{\predicateBitpreciseParallelInvariantsResultsAsyncInvariantsAbsIncorrectFalseCpuTimeAvgPlain}{}
  \renewcommand{\predicateBitpreciseParallelInvariantsResultsAsyncInvariantsAbsIncorrectFalseCpuTimeAvgPlain}{50.1461675633889\xspace}
\providecommand{\predicateBitpreciseParallelInvariantsResultsAsyncInvariantsAbsIncorrectFalseCpuTimeAvgPlainHours}{}
  \renewcommand{\predicateBitpreciseParallelInvariantsResultsAsyncInvariantsAbsIncorrectFalseCpuTimeAvgPlainHours}{0.01392949098983025\xspace}

  % wall-time-avg
\providecommand{\predicateBitpreciseParallelInvariantsResultsAsyncInvariantsAbsIncorrectFalseWallTimeAvgPlain}{}
  \renewcommand{\predicateBitpreciseParallelInvariantsResultsAsyncInvariantsAbsIncorrectFalseWallTimeAvgPlain}{16.003592915004447\xspace}
\providecommand{\predicateBitpreciseParallelInvariantsResultsAsyncInvariantsAbsIncorrectFalseWallTimeAvgPlainHours}{}
  \renewcommand{\predicateBitpreciseParallelInvariantsResultsAsyncInvariantsAbsIncorrectFalseWallTimeAvgPlainHours}{0.004445442476390124\xspace}

  % inv-succ
\providecommand{\predicateBitpreciseParallelInvariantsResultsAsyncInvariantsAbsIncorrectFalseInvSuccPlain}{}
  \renewcommand{\predicateBitpreciseParallelInvariantsResultsAsyncInvariantsAbsIncorrectFalseInvSuccPlain}{0\xspace}

  % inv-tries
\providecommand{\predicateBitpreciseParallelInvariantsResultsAsyncInvariantsAbsIncorrectFalseInvTriesPlain}{}
  \renewcommand{\predicateBitpreciseParallelInvariantsResultsAsyncInvariantsAbsIncorrectFalseInvTriesPlain}{0\xspace}

  % inv-time-sum
\providecommand{\predicateBitpreciseParallelInvariantsResultsAsyncInvariantsAbsIncorrectFalseInvTimeSumPlain}{}
  \renewcommand{\predicateBitpreciseParallelInvariantsResultsAsyncInvariantsAbsIncorrectFalseInvTimeSumPlain}{0.0\xspace}
\providecommand{\predicateBitpreciseParallelInvariantsResultsAsyncInvariantsAbsIncorrectFalseInvTimeSumPlainHours}{}
  \renewcommand{\predicateBitpreciseParallelInvariantsResultsAsyncInvariantsAbsIncorrectFalseInvTimeSumPlainHours}{0.0\xspace}

  % finished-main
\providecommand{\predicateBitpreciseParallelInvariantsResultsAsyncInvariantsAbsIncorrectFalseFinishedMainPlain}{}
  \renewcommand{\predicateBitpreciseParallelInvariantsResultsAsyncInvariantsAbsIncorrectFalseFinishedMainPlain}{18\xspace}

 %% incorrect-true %%
\providecommand{\predicateBitpreciseParallelInvariantsResultsAsyncInvariantsAbsIncorrectTruePlain}{}
  \renewcommand{\predicateBitpreciseParallelInvariantsResultsAsyncInvariantsAbsIncorrectTruePlain}{0\xspace}

  % cpu-time-sum
\providecommand{\predicateBitpreciseParallelInvariantsResultsAsyncInvariantsAbsIncorrectTrueCpuTimeSumPlain}{}
  \renewcommand{\predicateBitpreciseParallelInvariantsResultsAsyncInvariantsAbsIncorrectTrueCpuTimeSumPlain}{0.0\xspace}
\providecommand{\predicateBitpreciseParallelInvariantsResultsAsyncInvariantsAbsIncorrectTrueCpuTimeSumPlainHours}{}
  \renewcommand{\predicateBitpreciseParallelInvariantsResultsAsyncInvariantsAbsIncorrectTrueCpuTimeSumPlainHours}{0.0\xspace}

  % wall-time-sum
\providecommand{\predicateBitpreciseParallelInvariantsResultsAsyncInvariantsAbsIncorrectTrueWallTimeSumPlain}{}
  \renewcommand{\predicateBitpreciseParallelInvariantsResultsAsyncInvariantsAbsIncorrectTrueWallTimeSumPlain}{0.0\xspace}
\providecommand{\predicateBitpreciseParallelInvariantsResultsAsyncInvariantsAbsIncorrectTrueWallTimeSumPlainHours}{}
  \renewcommand{\predicateBitpreciseParallelInvariantsResultsAsyncInvariantsAbsIncorrectTrueWallTimeSumPlainHours}{0.0\xspace}

  % cpu-time-avg
\providecommand{\predicateBitpreciseParallelInvariantsResultsAsyncInvariantsAbsIncorrectTrueCpuTimeAvgPlain}{}
  \renewcommand{\predicateBitpreciseParallelInvariantsResultsAsyncInvariantsAbsIncorrectTrueCpuTimeAvgPlain}{NaN\xspace}
\providecommand{\predicateBitpreciseParallelInvariantsResultsAsyncInvariantsAbsIncorrectTrueCpuTimeAvgPlainHours}{}
  \renewcommand{\predicateBitpreciseParallelInvariantsResultsAsyncInvariantsAbsIncorrectTrueCpuTimeAvgPlainHours}{NaN\xspace}

  % wall-time-avg
\providecommand{\predicateBitpreciseParallelInvariantsResultsAsyncInvariantsAbsIncorrectTrueWallTimeAvgPlain}{}
  \renewcommand{\predicateBitpreciseParallelInvariantsResultsAsyncInvariantsAbsIncorrectTrueWallTimeAvgPlain}{NaN\xspace}
\providecommand{\predicateBitpreciseParallelInvariantsResultsAsyncInvariantsAbsIncorrectTrueWallTimeAvgPlainHours}{}
  \renewcommand{\predicateBitpreciseParallelInvariantsResultsAsyncInvariantsAbsIncorrectTrueWallTimeAvgPlainHours}{NaN\xspace}

  % inv-succ
\providecommand{\predicateBitpreciseParallelInvariantsResultsAsyncInvariantsAbsIncorrectTrueInvSuccPlain}{}
  \renewcommand{\predicateBitpreciseParallelInvariantsResultsAsyncInvariantsAbsIncorrectTrueInvSuccPlain}{0\xspace}

  % inv-tries
\providecommand{\predicateBitpreciseParallelInvariantsResultsAsyncInvariantsAbsIncorrectTrueInvTriesPlain}{}
  \renewcommand{\predicateBitpreciseParallelInvariantsResultsAsyncInvariantsAbsIncorrectTrueInvTriesPlain}{0\xspace}

  % inv-time-sum
\providecommand{\predicateBitpreciseParallelInvariantsResultsAsyncInvariantsAbsIncorrectTrueInvTimeSumPlain}{}
  \renewcommand{\predicateBitpreciseParallelInvariantsResultsAsyncInvariantsAbsIncorrectTrueInvTimeSumPlain}{0.0\xspace}
\providecommand{\predicateBitpreciseParallelInvariantsResultsAsyncInvariantsAbsIncorrectTrueInvTimeSumPlainHours}{}
  \renewcommand{\predicateBitpreciseParallelInvariantsResultsAsyncInvariantsAbsIncorrectTrueInvTimeSumPlainHours}{0.0\xspace}

  % finished-main
\providecommand{\predicateBitpreciseParallelInvariantsResultsAsyncInvariantsAbsIncorrectTrueFinishedMainPlain}{}
  \renewcommand{\predicateBitpreciseParallelInvariantsResultsAsyncInvariantsAbsIncorrectTrueFinishedMainPlain}{0\xspace}

 %% all %%
\providecommand{\predicateBitpreciseParallelInvariantsResultsAsyncInvariantsAbsAllPlain}{}
  \renewcommand{\predicateBitpreciseParallelInvariantsResultsAsyncInvariantsAbsAllPlain}{3488\xspace}

  % cpu-time-sum
\providecommand{\predicateBitpreciseParallelInvariantsResultsAsyncInvariantsAbsAllCpuTimeSumPlain}{}
  \renewcommand{\predicateBitpreciseParallelInvariantsResultsAsyncInvariantsAbsAllCpuTimeSumPlain}{993621.4488490853\xspace}
\providecommand{\predicateBitpreciseParallelInvariantsResultsAsyncInvariantsAbsAllCpuTimeSumPlainHours}{}
  \renewcommand{\predicateBitpreciseParallelInvariantsResultsAsyncInvariantsAbsAllCpuTimeSumPlainHours}{276.00595801363477\xspace}

  % wall-time-sum
\providecommand{\predicateBitpreciseParallelInvariantsResultsAsyncInvariantsAbsAllWallTimeSumPlain}{}
  \renewcommand{\predicateBitpreciseParallelInvariantsResultsAsyncInvariantsAbsAllWallTimeSumPlain}{530097.4896073246\xspace}
\providecommand{\predicateBitpreciseParallelInvariantsResultsAsyncInvariantsAbsAllWallTimeSumPlainHours}{}
  \renewcommand{\predicateBitpreciseParallelInvariantsResultsAsyncInvariantsAbsAllWallTimeSumPlainHours}{147.24930266870126\xspace}

  % cpu-time-avg
\providecommand{\predicateBitpreciseParallelInvariantsResultsAsyncInvariantsAbsAllCpuTimeAvgPlain}{}
  \renewcommand{\predicateBitpreciseParallelInvariantsResultsAsyncInvariantsAbsAllCpuTimeAvgPlain}{284.86853464710015\xspace}
\providecommand{\predicateBitpreciseParallelInvariantsResultsAsyncInvariantsAbsAllCpuTimeAvgPlainHours}{}
  \renewcommand{\predicateBitpreciseParallelInvariantsResultsAsyncInvariantsAbsAllCpuTimeAvgPlainHours}{0.07913014851308338\xspace}

  % wall-time-avg
\providecommand{\predicateBitpreciseParallelInvariantsResultsAsyncInvariantsAbsAllWallTimeAvgPlain}{}
  \renewcommand{\predicateBitpreciseParallelInvariantsResultsAsyncInvariantsAbsAllWallTimeAvgPlain}{151.9774912865036\xspace}
\providecommand{\predicateBitpreciseParallelInvariantsResultsAsyncInvariantsAbsAllWallTimeAvgPlainHours}{}
  \renewcommand{\predicateBitpreciseParallelInvariantsResultsAsyncInvariantsAbsAllWallTimeAvgPlainHours}{0.04221596980180656\xspace}

  % inv-succ
\providecommand{\predicateBitpreciseParallelInvariantsResultsAsyncInvariantsAbsAllInvSuccPlain}{}
  \renewcommand{\predicateBitpreciseParallelInvariantsResultsAsyncInvariantsAbsAllInvSuccPlain}{0\xspace}

  % inv-tries
\providecommand{\predicateBitpreciseParallelInvariantsResultsAsyncInvariantsAbsAllInvTriesPlain}{}
  \renewcommand{\predicateBitpreciseParallelInvariantsResultsAsyncInvariantsAbsAllInvTriesPlain}{0\xspace}

  % inv-time-sum
\providecommand{\predicateBitpreciseParallelInvariantsResultsAsyncInvariantsAbsAllInvTimeSumPlain}{}
  \renewcommand{\predicateBitpreciseParallelInvariantsResultsAsyncInvariantsAbsAllInvTimeSumPlain}{0.0\xspace}
\providecommand{\predicateBitpreciseParallelInvariantsResultsAsyncInvariantsAbsAllInvTimeSumPlainHours}{}
  \renewcommand{\predicateBitpreciseParallelInvariantsResultsAsyncInvariantsAbsAllInvTimeSumPlainHours}{0.0\xspace}

  % finished-main
\providecommand{\predicateBitpreciseParallelInvariantsResultsAsyncInvariantsAbsAllFinishedMainPlain}{}
  \renewcommand{\predicateBitpreciseParallelInvariantsResultsAsyncInvariantsAbsAllFinishedMainPlain}{1173\xspace}

 %% equal-only %%
\providecommand{\predicateBitpreciseParallelInvariantsResultsAsyncInvariantsAbsEqualOnlyPlain}{}
  \renewcommand{\predicateBitpreciseParallelInvariantsResultsAsyncInvariantsAbsEqualOnlyPlain}{1865\xspace}

  % cpu-time-sum
\providecommand{\predicateBitpreciseParallelInvariantsResultsAsyncInvariantsAbsEqualOnlyCpuTimeSumPlain}{}
  \renewcommand{\predicateBitpreciseParallelInvariantsResultsAsyncInvariantsAbsEqualOnlyCpuTimeSumPlain}{137644.2518049231\xspace}
\providecommand{\predicateBitpreciseParallelInvariantsResultsAsyncInvariantsAbsEqualOnlyCpuTimeSumPlainHours}{}
  \renewcommand{\predicateBitpreciseParallelInvariantsResultsAsyncInvariantsAbsEqualOnlyCpuTimeSumPlainHours}{38.234514390256415\xspace}

  % wall-time-sum
\providecommand{\predicateBitpreciseParallelInvariantsResultsAsyncInvariantsAbsEqualOnlyWallTimeSumPlain}{}
  \renewcommand{\predicateBitpreciseParallelInvariantsResultsAsyncInvariantsAbsEqualOnlyWallTimeSumPlain}{51860.656197540775\xspace}
\providecommand{\predicateBitpreciseParallelInvariantsResultsAsyncInvariantsAbsEqualOnlyWallTimeSumPlainHours}{}
  \renewcommand{\predicateBitpreciseParallelInvariantsResultsAsyncInvariantsAbsEqualOnlyWallTimeSumPlainHours}{14.405737832650216\xspace}

  % cpu-time-avg
\providecommand{\predicateBitpreciseParallelInvariantsResultsAsyncInvariantsAbsEqualOnlyCpuTimeAvgPlain}{}
  \renewcommand{\predicateBitpreciseParallelInvariantsResultsAsyncInvariantsAbsEqualOnlyCpuTimeAvgPlain}{73.80388836725099\xspace}
\providecommand{\predicateBitpreciseParallelInvariantsResultsAsyncInvariantsAbsEqualOnlyCpuTimeAvgPlainHours}{}
  \renewcommand{\predicateBitpreciseParallelInvariantsResultsAsyncInvariantsAbsEqualOnlyCpuTimeAvgPlainHours}{0.020501080102014166\xspace}

  % wall-time-avg
\providecommand{\predicateBitpreciseParallelInvariantsResultsAsyncInvariantsAbsEqualOnlyWallTimeAvgPlain}{}
  \renewcommand{\predicateBitpreciseParallelInvariantsResultsAsyncInvariantsAbsEqualOnlyWallTimeAvgPlain}{27.807322357930712\xspace}
\providecommand{\predicateBitpreciseParallelInvariantsResultsAsyncInvariantsAbsEqualOnlyWallTimeAvgPlainHours}{}
  \renewcommand{\predicateBitpreciseParallelInvariantsResultsAsyncInvariantsAbsEqualOnlyWallTimeAvgPlainHours}{0.0077242562105363085\xspace}

  % inv-succ
\providecommand{\predicateBitpreciseParallelInvariantsResultsAsyncInvariantsAbsEqualOnlyInvSuccPlain}{}
  \renewcommand{\predicateBitpreciseParallelInvariantsResultsAsyncInvariantsAbsEqualOnlyInvSuccPlain}{0\xspace}

  % inv-tries
\providecommand{\predicateBitpreciseParallelInvariantsResultsAsyncInvariantsAbsEqualOnlyInvTriesPlain}{}
  \renewcommand{\predicateBitpreciseParallelInvariantsResultsAsyncInvariantsAbsEqualOnlyInvTriesPlain}{0\xspace}

  % inv-time-sum
\providecommand{\predicateBitpreciseParallelInvariantsResultsAsyncInvariantsAbsEqualOnlyInvTimeSumPlain}{}
  \renewcommand{\predicateBitpreciseParallelInvariantsResultsAsyncInvariantsAbsEqualOnlyInvTimeSumPlain}{0.0\xspace}
\providecommand{\predicateBitpreciseParallelInvariantsResultsAsyncInvariantsAbsEqualOnlyInvTimeSumPlainHours}{}
  \renewcommand{\predicateBitpreciseParallelInvariantsResultsAsyncInvariantsAbsEqualOnlyInvTimeSumPlainHours}{0.0\xspace}

  % finished-main
\providecommand{\predicateBitpreciseParallelInvariantsResultsAsyncInvariantsAbsEqualOnlyFinishedMainPlain}{}
  \renewcommand{\predicateBitpreciseParallelInvariantsResultsAsyncInvariantsAbsEqualOnlyFinishedMainPlain}{1042\xspace}

%%% predicate_bitprecise_parallel_invariants.2016-09-05_0219.results.async-invariants-abs-path %%%
 %% correct %%
\providecommand{\predicateBitpreciseParallelInvariantsResultsAsyncInvariantsAbsPathCorrectPlain}{}
  \renewcommand{\predicateBitpreciseParallelInvariantsResultsAsyncInvariantsAbsPathCorrectPlain}{2096\xspace}

  % cpu-time-sum
\providecommand{\predicateBitpreciseParallelInvariantsResultsAsyncInvariantsAbsPathCorrectCpuTimeSumPlain}{}
  \renewcommand{\predicateBitpreciseParallelInvariantsResultsAsyncInvariantsAbsPathCorrectCpuTimeSumPlain}{193002.19719524757\xspace}
\providecommand{\predicateBitpreciseParallelInvariantsResultsAsyncInvariantsAbsPathCorrectCpuTimeSumPlainHours}{}
  \renewcommand{\predicateBitpreciseParallelInvariantsResultsAsyncInvariantsAbsPathCorrectCpuTimeSumPlainHours}{53.611721443124324\xspace}

  % wall-time-sum
\providecommand{\predicateBitpreciseParallelInvariantsResultsAsyncInvariantsAbsPathCorrectWallTimeSumPlain}{}
  \renewcommand{\predicateBitpreciseParallelInvariantsResultsAsyncInvariantsAbsPathCorrectWallTimeSumPlain}{77753.14130711908\xspace}
\providecommand{\predicateBitpreciseParallelInvariantsResultsAsyncInvariantsAbsPathCorrectWallTimeSumPlainHours}{}
  \renewcommand{\predicateBitpreciseParallelInvariantsResultsAsyncInvariantsAbsPathCorrectWallTimeSumPlainHours}{21.598094807533077\xspace}

  % cpu-time-avg
\providecommand{\predicateBitpreciseParallelInvariantsResultsAsyncInvariantsAbsPathCorrectCpuTimeAvgPlain}{}
  \renewcommand{\predicateBitpreciseParallelInvariantsResultsAsyncInvariantsAbsPathCorrectCpuTimeAvgPlain}{92.0812009519311\xspace}
\providecommand{\predicateBitpreciseParallelInvariantsResultsAsyncInvariantsAbsPathCorrectCpuTimeAvgPlainHours}{}
  \renewcommand{\predicateBitpreciseParallelInvariantsResultsAsyncInvariantsAbsPathCorrectCpuTimeAvgPlainHours}{0.025578111375536415\xspace}

  % wall-time-avg
\providecommand{\predicateBitpreciseParallelInvariantsResultsAsyncInvariantsAbsPathCorrectWallTimeAvgPlain}{}
  \renewcommand{\predicateBitpreciseParallelInvariantsResultsAsyncInvariantsAbsPathCorrectWallTimeAvgPlain}{37.09596436408353\xspace}
\providecommand{\predicateBitpreciseParallelInvariantsResultsAsyncInvariantsAbsPathCorrectWallTimeAvgPlainHours}{}
  \renewcommand{\predicateBitpreciseParallelInvariantsResultsAsyncInvariantsAbsPathCorrectWallTimeAvgPlainHours}{0.010304434545578757\xspace}

  % inv-succ
\providecommand{\predicateBitpreciseParallelInvariantsResultsAsyncInvariantsAbsPathCorrectInvSuccPlain}{}
  \renewcommand{\predicateBitpreciseParallelInvariantsResultsAsyncInvariantsAbsPathCorrectInvSuccPlain}{0\xspace}

  % inv-tries
\providecommand{\predicateBitpreciseParallelInvariantsResultsAsyncInvariantsAbsPathCorrectInvTriesPlain}{}
  \renewcommand{\predicateBitpreciseParallelInvariantsResultsAsyncInvariantsAbsPathCorrectInvTriesPlain}{0\xspace}

  % inv-time-sum
\providecommand{\predicateBitpreciseParallelInvariantsResultsAsyncInvariantsAbsPathCorrectInvTimeSumPlain}{}
  \renewcommand{\predicateBitpreciseParallelInvariantsResultsAsyncInvariantsAbsPathCorrectInvTimeSumPlain}{0.0\xspace}
\providecommand{\predicateBitpreciseParallelInvariantsResultsAsyncInvariantsAbsPathCorrectInvTimeSumPlainHours}{}
  \renewcommand{\predicateBitpreciseParallelInvariantsResultsAsyncInvariantsAbsPathCorrectInvTimeSumPlainHours}{0.0\xspace}

  % finished-main
\providecommand{\predicateBitpreciseParallelInvariantsResultsAsyncInvariantsAbsPathCorrectFinishedMainPlain}{}
  \renewcommand{\predicateBitpreciseParallelInvariantsResultsAsyncInvariantsAbsPathCorrectFinishedMainPlain}{1148\xspace}

 %% incorrect %%
\providecommand{\predicateBitpreciseParallelInvariantsResultsAsyncInvariantsAbsPathIncorrectPlain}{}
  \renewcommand{\predicateBitpreciseParallelInvariantsResultsAsyncInvariantsAbsPathIncorrectPlain}{19\xspace}

  % cpu-time-sum
\providecommand{\predicateBitpreciseParallelInvariantsResultsAsyncInvariantsAbsPathIncorrectCpuTimeSumPlain}{}
  \renewcommand{\predicateBitpreciseParallelInvariantsResultsAsyncInvariantsAbsPathIncorrectCpuTimeSumPlain}{969.275954974\xspace}
\providecommand{\predicateBitpreciseParallelInvariantsResultsAsyncInvariantsAbsPathIncorrectCpuTimeSumPlainHours}{}
  \renewcommand{\predicateBitpreciseParallelInvariantsResultsAsyncInvariantsAbsPathIncorrectCpuTimeSumPlainHours}{0.26924332082611113\xspace}

  % wall-time-sum
\providecommand{\predicateBitpreciseParallelInvariantsResultsAsyncInvariantsAbsPathIncorrectWallTimeSumPlain}{}
  \renewcommand{\predicateBitpreciseParallelInvariantsResultsAsyncInvariantsAbsPathIncorrectWallTimeSumPlain}{313.06105780597005\xspace}
\providecommand{\predicateBitpreciseParallelInvariantsResultsAsyncInvariantsAbsPathIncorrectWallTimeSumPlainHours}{}
  \renewcommand{\predicateBitpreciseParallelInvariantsResultsAsyncInvariantsAbsPathIncorrectWallTimeSumPlainHours}{0.08696140494610279\xspace}

  % cpu-time-avg
\providecommand{\predicateBitpreciseParallelInvariantsResultsAsyncInvariantsAbsPathIncorrectCpuTimeAvgPlain}{}
  \renewcommand{\predicateBitpreciseParallelInvariantsResultsAsyncInvariantsAbsPathIncorrectCpuTimeAvgPlain}{51.014523946\xspace}
\providecommand{\predicateBitpreciseParallelInvariantsResultsAsyncInvariantsAbsPathIncorrectCpuTimeAvgPlainHours}{}
  \renewcommand{\predicateBitpreciseParallelInvariantsResultsAsyncInvariantsAbsPathIncorrectCpuTimeAvgPlainHours}{0.01417070109611111\xspace}

  % wall-time-avg
\providecommand{\predicateBitpreciseParallelInvariantsResultsAsyncInvariantsAbsPathIncorrectWallTimeAvgPlain}{}
  \renewcommand{\predicateBitpreciseParallelInvariantsResultsAsyncInvariantsAbsPathIncorrectWallTimeAvgPlain}{16.47689777926158\xspace}
\providecommand{\predicateBitpreciseParallelInvariantsResultsAsyncInvariantsAbsPathIncorrectWallTimeAvgPlainHours}{}
  \renewcommand{\predicateBitpreciseParallelInvariantsResultsAsyncInvariantsAbsPathIncorrectWallTimeAvgPlainHours}{0.004576916049794884\xspace}

  % inv-succ
\providecommand{\predicateBitpreciseParallelInvariantsResultsAsyncInvariantsAbsPathIncorrectInvSuccPlain}{}
  \renewcommand{\predicateBitpreciseParallelInvariantsResultsAsyncInvariantsAbsPathIncorrectInvSuccPlain}{0\xspace}

  % inv-tries
\providecommand{\predicateBitpreciseParallelInvariantsResultsAsyncInvariantsAbsPathIncorrectInvTriesPlain}{}
  \renewcommand{\predicateBitpreciseParallelInvariantsResultsAsyncInvariantsAbsPathIncorrectInvTriesPlain}{0\xspace}

  % inv-time-sum
\providecommand{\predicateBitpreciseParallelInvariantsResultsAsyncInvariantsAbsPathIncorrectInvTimeSumPlain}{}
  \renewcommand{\predicateBitpreciseParallelInvariantsResultsAsyncInvariantsAbsPathIncorrectInvTimeSumPlain}{0.0\xspace}
\providecommand{\predicateBitpreciseParallelInvariantsResultsAsyncInvariantsAbsPathIncorrectInvTimeSumPlainHours}{}
  \renewcommand{\predicateBitpreciseParallelInvariantsResultsAsyncInvariantsAbsPathIncorrectInvTimeSumPlainHours}{0.0\xspace}

  % finished-main
\providecommand{\predicateBitpreciseParallelInvariantsResultsAsyncInvariantsAbsPathIncorrectFinishedMainPlain}{}
  \renewcommand{\predicateBitpreciseParallelInvariantsResultsAsyncInvariantsAbsPathIncorrectFinishedMainPlain}{19\xspace}

 %% timeout %%
\providecommand{\predicateBitpreciseParallelInvariantsResultsAsyncInvariantsAbsPathTimeoutPlain}{}
  \renewcommand{\predicateBitpreciseParallelInvariantsResultsAsyncInvariantsAbsPathTimeoutPlain}{1196\xspace}

  % cpu-time-sum
\providecommand{\predicateBitpreciseParallelInvariantsResultsAsyncInvariantsAbsPathTimeoutCpuTimeSumPlain}{}
  \renewcommand{\predicateBitpreciseParallelInvariantsResultsAsyncInvariantsAbsPathTimeoutCpuTimeSumPlain}{735648.5586207315\xspace}
\providecommand{\predicateBitpreciseParallelInvariantsResultsAsyncInvariantsAbsPathTimeoutCpuTimeSumPlainHours}{}
  \renewcommand{\predicateBitpreciseParallelInvariantsResultsAsyncInvariantsAbsPathTimeoutCpuTimeSumPlainHours}{204.3468218390921\xspace}

  % wall-time-sum
\providecommand{\predicateBitpreciseParallelInvariantsResultsAsyncInvariantsAbsPathTimeoutWallTimeSumPlain}{}
  \renewcommand{\predicateBitpreciseParallelInvariantsResultsAsyncInvariantsAbsPathTimeoutWallTimeSumPlain}{411627.32938433904\xspace}
\providecommand{\predicateBitpreciseParallelInvariantsResultsAsyncInvariantsAbsPathTimeoutWallTimeSumPlainHours}{}
  \renewcommand{\predicateBitpreciseParallelInvariantsResultsAsyncInvariantsAbsPathTimeoutWallTimeSumPlainHours}{114.34092482898306\xspace}

  % cpu-time-avg
\providecommand{\predicateBitpreciseParallelInvariantsResultsAsyncInvariantsAbsPathTimeoutCpuTimeAvgPlain}{}
  \renewcommand{\predicateBitpreciseParallelInvariantsResultsAsyncInvariantsAbsPathTimeoutCpuTimeAvgPlain}{615.0907680775347\xspace}
\providecommand{\predicateBitpreciseParallelInvariantsResultsAsyncInvariantsAbsPathTimeoutCpuTimeAvgPlainHours}{}
  \renewcommand{\predicateBitpreciseParallelInvariantsResultsAsyncInvariantsAbsPathTimeoutCpuTimeAvgPlainHours}{0.17085854668820408\xspace}

  % wall-time-avg
\providecommand{\predicateBitpreciseParallelInvariantsResultsAsyncInvariantsAbsPathTimeoutWallTimeAvgPlain}{}
  \renewcommand{\predicateBitpreciseParallelInvariantsResultsAsyncInvariantsAbsPathTimeoutWallTimeAvgPlain}{344.17000784643733\xspace}
\providecommand{\predicateBitpreciseParallelInvariantsResultsAsyncInvariantsAbsPathTimeoutWallTimeAvgPlainHours}{}
  \renewcommand{\predicateBitpreciseParallelInvariantsResultsAsyncInvariantsAbsPathTimeoutWallTimeAvgPlainHours}{0.0956027799573437\xspace}

  % inv-succ
\providecommand{\predicateBitpreciseParallelInvariantsResultsAsyncInvariantsAbsPathTimeoutInvSuccPlain}{}
  \renewcommand{\predicateBitpreciseParallelInvariantsResultsAsyncInvariantsAbsPathTimeoutInvSuccPlain}{0\xspace}

  % inv-tries
\providecommand{\predicateBitpreciseParallelInvariantsResultsAsyncInvariantsAbsPathTimeoutInvTriesPlain}{}
  \renewcommand{\predicateBitpreciseParallelInvariantsResultsAsyncInvariantsAbsPathTimeoutInvTriesPlain}{0\xspace}

  % inv-time-sum
\providecommand{\predicateBitpreciseParallelInvariantsResultsAsyncInvariantsAbsPathTimeoutInvTimeSumPlain}{}
  \renewcommand{\predicateBitpreciseParallelInvariantsResultsAsyncInvariantsAbsPathTimeoutInvTimeSumPlain}{0.0\xspace}
\providecommand{\predicateBitpreciseParallelInvariantsResultsAsyncInvariantsAbsPathTimeoutInvTimeSumPlainHours}{}
  \renewcommand{\predicateBitpreciseParallelInvariantsResultsAsyncInvariantsAbsPathTimeoutInvTimeSumPlainHours}{0.0\xspace}

  % finished-main
\providecommand{\predicateBitpreciseParallelInvariantsResultsAsyncInvariantsAbsPathTimeoutFinishedMainPlain}{}
  \renewcommand{\predicateBitpreciseParallelInvariantsResultsAsyncInvariantsAbsPathTimeoutFinishedMainPlain}{0\xspace}

 %% unknown-or-category-error %%
\providecommand{\predicateBitpreciseParallelInvariantsResultsAsyncInvariantsAbsPathUnknownOrCategoryErrorPlain}{}
  \renewcommand{\predicateBitpreciseParallelInvariantsResultsAsyncInvariantsAbsPathUnknownOrCategoryErrorPlain}{1373\xspace}

  % cpu-time-sum
\providecommand{\predicateBitpreciseParallelInvariantsResultsAsyncInvariantsAbsPathUnknownOrCategoryErrorCpuTimeSumPlain}{}
  \renewcommand{\predicateBitpreciseParallelInvariantsResultsAsyncInvariantsAbsPathUnknownOrCategoryErrorCpuTimeSumPlain}{795961.3278808318\xspace}
\providecommand{\predicateBitpreciseParallelInvariantsResultsAsyncInvariantsAbsPathUnknownOrCategoryErrorCpuTimeSumPlainHours}{}
  \renewcommand{\predicateBitpreciseParallelInvariantsResultsAsyncInvariantsAbsPathUnknownOrCategoryErrorCpuTimeSumPlainHours}{221.1003688557866\xspace}

  % wall-time-sum
\providecommand{\predicateBitpreciseParallelInvariantsResultsAsyncInvariantsAbsPathUnknownOrCategoryErrorWallTimeSumPlain}{}
  \renewcommand{\predicateBitpreciseParallelInvariantsResultsAsyncInvariantsAbsPathUnknownOrCategoryErrorWallTimeSumPlain}{448743.69779111416\xspace}
\providecommand{\predicateBitpreciseParallelInvariantsResultsAsyncInvariantsAbsPathUnknownOrCategoryErrorWallTimeSumPlainHours}{}
  \renewcommand{\predicateBitpreciseParallelInvariantsResultsAsyncInvariantsAbsPathUnknownOrCategoryErrorWallTimeSumPlainHours}{124.65102716419838\xspace}

  % cpu-time-avg
\providecommand{\predicateBitpreciseParallelInvariantsResultsAsyncInvariantsAbsPathUnknownOrCategoryErrorCpuTimeAvgPlain}{}
  \renewcommand{\predicateBitpreciseParallelInvariantsResultsAsyncInvariantsAbsPathUnknownOrCategoryErrorCpuTimeAvgPlain}{579.7242009328709\xspace}
\providecommand{\predicateBitpreciseParallelInvariantsResultsAsyncInvariantsAbsPathUnknownOrCategoryErrorCpuTimeAvgPlainHours}{}
  \renewcommand{\predicateBitpreciseParallelInvariantsResultsAsyncInvariantsAbsPathUnknownOrCategoryErrorCpuTimeAvgPlainHours}{0.1610345002591308\xspace}

  % wall-time-avg
\providecommand{\predicateBitpreciseParallelInvariantsResultsAsyncInvariantsAbsPathUnknownOrCategoryErrorWallTimeAvgPlain}{}
  \renewcommand{\predicateBitpreciseParallelInvariantsResultsAsyncInvariantsAbsPathUnknownOrCategoryErrorWallTimeAvgPlain}{326.8344485004473\xspace}
\providecommand{\predicateBitpreciseParallelInvariantsResultsAsyncInvariantsAbsPathUnknownOrCategoryErrorWallTimeAvgPlainHours}{}
  \renewcommand{\predicateBitpreciseParallelInvariantsResultsAsyncInvariantsAbsPathUnknownOrCategoryErrorWallTimeAvgPlainHours}{0.0907873468056798\xspace}

  % inv-succ
\providecommand{\predicateBitpreciseParallelInvariantsResultsAsyncInvariantsAbsPathUnknownOrCategoryErrorInvSuccPlain}{}
  \renewcommand{\predicateBitpreciseParallelInvariantsResultsAsyncInvariantsAbsPathUnknownOrCategoryErrorInvSuccPlain}{0\xspace}

  % inv-tries
\providecommand{\predicateBitpreciseParallelInvariantsResultsAsyncInvariantsAbsPathUnknownOrCategoryErrorInvTriesPlain}{}
  \renewcommand{\predicateBitpreciseParallelInvariantsResultsAsyncInvariantsAbsPathUnknownOrCategoryErrorInvTriesPlain}{0\xspace}

  % inv-time-sum
\providecommand{\predicateBitpreciseParallelInvariantsResultsAsyncInvariantsAbsPathUnknownOrCategoryErrorInvTimeSumPlain}{}
  \renewcommand{\predicateBitpreciseParallelInvariantsResultsAsyncInvariantsAbsPathUnknownOrCategoryErrorInvTimeSumPlain}{0.0\xspace}
\providecommand{\predicateBitpreciseParallelInvariantsResultsAsyncInvariantsAbsPathUnknownOrCategoryErrorInvTimeSumPlainHours}{}
  \renewcommand{\predicateBitpreciseParallelInvariantsResultsAsyncInvariantsAbsPathUnknownOrCategoryErrorInvTimeSumPlainHours}{0.0\xspace}

  % finished-main
\providecommand{\predicateBitpreciseParallelInvariantsResultsAsyncInvariantsAbsPathUnknownOrCategoryErrorFinishedMainPlain}{}
  \renewcommand{\predicateBitpreciseParallelInvariantsResultsAsyncInvariantsAbsPathUnknownOrCategoryErrorFinishedMainPlain}{2\xspace}

 %% correct-false %%
\providecommand{\predicateBitpreciseParallelInvariantsResultsAsyncInvariantsAbsPathCorrectFalsePlain}{}
  \renewcommand{\predicateBitpreciseParallelInvariantsResultsAsyncInvariantsAbsPathCorrectFalsePlain}{568\xspace}

  % cpu-time-sum
\providecommand{\predicateBitpreciseParallelInvariantsResultsAsyncInvariantsAbsPathCorrectFalseCpuTimeSumPlain}{}
  \renewcommand{\predicateBitpreciseParallelInvariantsResultsAsyncInvariantsAbsPathCorrectFalseCpuTimeSumPlain}{77219.60555845998\xspace}
\providecommand{\predicateBitpreciseParallelInvariantsResultsAsyncInvariantsAbsPathCorrectFalseCpuTimeSumPlainHours}{}
  \renewcommand{\predicateBitpreciseParallelInvariantsResultsAsyncInvariantsAbsPathCorrectFalseCpuTimeSumPlainHours}{21.44989043290555\xspace}

  % wall-time-sum
\providecommand{\predicateBitpreciseParallelInvariantsResultsAsyncInvariantsAbsPathCorrectFalseWallTimeSumPlain}{}
  \renewcommand{\predicateBitpreciseParallelInvariantsResultsAsyncInvariantsAbsPathCorrectFalseWallTimeSumPlain}{32850.11300874305\xspace}
\providecommand{\predicateBitpreciseParallelInvariantsResultsAsyncInvariantsAbsPathCorrectFalseWallTimeSumPlainHours}{}
  \renewcommand{\predicateBitpreciseParallelInvariantsResultsAsyncInvariantsAbsPathCorrectFalseWallTimeSumPlainHours}{9.125031391317513\xspace}

  % cpu-time-avg
\providecommand{\predicateBitpreciseParallelInvariantsResultsAsyncInvariantsAbsPathCorrectFalseCpuTimeAvgPlain}{}
  \renewcommand{\predicateBitpreciseParallelInvariantsResultsAsyncInvariantsAbsPathCorrectFalseCpuTimeAvgPlain}{135.9500097860211\xspace}
\providecommand{\predicateBitpreciseParallelInvariantsResultsAsyncInvariantsAbsPathCorrectFalseCpuTimeAvgPlainHours}{}
  \renewcommand{\predicateBitpreciseParallelInvariantsResultsAsyncInvariantsAbsPathCorrectFalseCpuTimeAvgPlainHours}{0.03776389160722808\xspace}

  % wall-time-avg
\providecommand{\predicateBitpreciseParallelInvariantsResultsAsyncInvariantsAbsPathCorrectFalseWallTimeAvgPlain}{}
  \renewcommand{\predicateBitpreciseParallelInvariantsResultsAsyncInvariantsAbsPathCorrectFalseWallTimeAvgPlain}{57.834706001308184\xspace}
\providecommand{\predicateBitpreciseParallelInvariantsResultsAsyncInvariantsAbsPathCorrectFalseWallTimeAvgPlainHours}{}
  \renewcommand{\predicateBitpreciseParallelInvariantsResultsAsyncInvariantsAbsPathCorrectFalseWallTimeAvgPlainHours}{0.016065196111474497\xspace}

  % inv-succ
\providecommand{\predicateBitpreciseParallelInvariantsResultsAsyncInvariantsAbsPathCorrectFalseInvSuccPlain}{}
  \renewcommand{\predicateBitpreciseParallelInvariantsResultsAsyncInvariantsAbsPathCorrectFalseInvSuccPlain}{0\xspace}

  % inv-tries
\providecommand{\predicateBitpreciseParallelInvariantsResultsAsyncInvariantsAbsPathCorrectFalseInvTriesPlain}{}
  \renewcommand{\predicateBitpreciseParallelInvariantsResultsAsyncInvariantsAbsPathCorrectFalseInvTriesPlain}{0\xspace}

  % inv-time-sum
\providecommand{\predicateBitpreciseParallelInvariantsResultsAsyncInvariantsAbsPathCorrectFalseInvTimeSumPlain}{}
  \renewcommand{\predicateBitpreciseParallelInvariantsResultsAsyncInvariantsAbsPathCorrectFalseInvTimeSumPlain}{0.0\xspace}
\providecommand{\predicateBitpreciseParallelInvariantsResultsAsyncInvariantsAbsPathCorrectFalseInvTimeSumPlainHours}{}
  \renewcommand{\predicateBitpreciseParallelInvariantsResultsAsyncInvariantsAbsPathCorrectFalseInvTimeSumPlainHours}{0.0\xspace}

  % finished-main
\providecommand{\predicateBitpreciseParallelInvariantsResultsAsyncInvariantsAbsPathCorrectFalseFinishedMainPlain}{}
  \renewcommand{\predicateBitpreciseParallelInvariantsResultsAsyncInvariantsAbsPathCorrectFalseFinishedMainPlain}{568\xspace}

 %% correct-true %%
\providecommand{\predicateBitpreciseParallelInvariantsResultsAsyncInvariantsAbsPathCorrectTruePlain}{}
  \renewcommand{\predicateBitpreciseParallelInvariantsResultsAsyncInvariantsAbsPathCorrectTruePlain}{1528\xspace}

  % cpu-time-sum
\providecommand{\predicateBitpreciseParallelInvariantsResultsAsyncInvariantsAbsPathCorrectTrueCpuTimeSumPlain}{}
  \renewcommand{\predicateBitpreciseParallelInvariantsResultsAsyncInvariantsAbsPathCorrectTrueCpuTimeSumPlain}{115782.591636788\xspace}
\providecommand{\predicateBitpreciseParallelInvariantsResultsAsyncInvariantsAbsPathCorrectTrueCpuTimeSumPlainHours}{}
  \renewcommand{\predicateBitpreciseParallelInvariantsResultsAsyncInvariantsAbsPathCorrectTrueCpuTimeSumPlainHours}{32.16183101021889\xspace}

  % wall-time-sum
\providecommand{\predicateBitpreciseParallelInvariantsResultsAsyncInvariantsAbsPathCorrectTrueWallTimeSumPlain}{}
  \renewcommand{\predicateBitpreciseParallelInvariantsResultsAsyncInvariantsAbsPathCorrectTrueWallTimeSumPlain}{44903.02829837621\xspace}
\providecommand{\predicateBitpreciseParallelInvariantsResultsAsyncInvariantsAbsPathCorrectTrueWallTimeSumPlainHours}{}
  \renewcommand{\predicateBitpreciseParallelInvariantsResultsAsyncInvariantsAbsPathCorrectTrueWallTimeSumPlainHours}{12.473063416215613\xspace}

  % cpu-time-avg
\providecommand{\predicateBitpreciseParallelInvariantsResultsAsyncInvariantsAbsPathCorrectTrueCpuTimeAvgPlain}{}
  \renewcommand{\predicateBitpreciseParallelInvariantsResultsAsyncInvariantsAbsPathCorrectTrueCpuTimeAvgPlain}{75.77394740627487\xspace}
\providecommand{\predicateBitpreciseParallelInvariantsResultsAsyncInvariantsAbsPathCorrectTrueCpuTimeAvgPlainHours}{}
  \renewcommand{\predicateBitpreciseParallelInvariantsResultsAsyncInvariantsAbsPathCorrectTrueCpuTimeAvgPlainHours}{0.021048318723965242\xspace}

  % wall-time-avg
\providecommand{\predicateBitpreciseParallelInvariantsResultsAsyncInvariantsAbsPathCorrectTrueWallTimeAvgPlain}{}
  \renewcommand{\predicateBitpreciseParallelInvariantsResultsAsyncInvariantsAbsPathCorrectTrueWallTimeAvgPlain}{29.38679862459176\xspace}
\providecommand{\predicateBitpreciseParallelInvariantsResultsAsyncInvariantsAbsPathCorrectTrueWallTimeAvgPlainHours}{}
  \renewcommand{\predicateBitpreciseParallelInvariantsResultsAsyncInvariantsAbsPathCorrectTrueWallTimeAvgPlainHours}{0.008162999617942155\xspace}

  % inv-succ
\providecommand{\predicateBitpreciseParallelInvariantsResultsAsyncInvariantsAbsPathCorrectTrueInvSuccPlain}{}
  \renewcommand{\predicateBitpreciseParallelInvariantsResultsAsyncInvariantsAbsPathCorrectTrueInvSuccPlain}{0\xspace}

  % inv-tries
\providecommand{\predicateBitpreciseParallelInvariantsResultsAsyncInvariantsAbsPathCorrectTrueInvTriesPlain}{}
  \renewcommand{\predicateBitpreciseParallelInvariantsResultsAsyncInvariantsAbsPathCorrectTrueInvTriesPlain}{0\xspace}

  % inv-time-sum
\providecommand{\predicateBitpreciseParallelInvariantsResultsAsyncInvariantsAbsPathCorrectTrueInvTimeSumPlain}{}
  \renewcommand{\predicateBitpreciseParallelInvariantsResultsAsyncInvariantsAbsPathCorrectTrueInvTimeSumPlain}{0.0\xspace}
\providecommand{\predicateBitpreciseParallelInvariantsResultsAsyncInvariantsAbsPathCorrectTrueInvTimeSumPlainHours}{}
  \renewcommand{\predicateBitpreciseParallelInvariantsResultsAsyncInvariantsAbsPathCorrectTrueInvTimeSumPlainHours}{0.0\xspace}

  % finished-main
\providecommand{\predicateBitpreciseParallelInvariantsResultsAsyncInvariantsAbsPathCorrectTrueFinishedMainPlain}{}
  \renewcommand{\predicateBitpreciseParallelInvariantsResultsAsyncInvariantsAbsPathCorrectTrueFinishedMainPlain}{580\xspace}

 %% incorrect-false %%
\providecommand{\predicateBitpreciseParallelInvariantsResultsAsyncInvariantsAbsPathIncorrectFalsePlain}{}
  \renewcommand{\predicateBitpreciseParallelInvariantsResultsAsyncInvariantsAbsPathIncorrectFalsePlain}{18\xspace}

  % cpu-time-sum
\providecommand{\predicateBitpreciseParallelInvariantsResultsAsyncInvariantsAbsPathIncorrectFalseCpuTimeSumPlain}{}
  \renewcommand{\predicateBitpreciseParallelInvariantsResultsAsyncInvariantsAbsPathIncorrectFalseCpuTimeSumPlain}{952.794109115\xspace}
\providecommand{\predicateBitpreciseParallelInvariantsResultsAsyncInvariantsAbsPathIncorrectFalseCpuTimeSumPlainHours}{}
  \renewcommand{\predicateBitpreciseParallelInvariantsResultsAsyncInvariantsAbsPathIncorrectFalseCpuTimeSumPlainHours}{0.2646650303097222\xspace}

  % wall-time-sum
\providecommand{\predicateBitpreciseParallelInvariantsResultsAsyncInvariantsAbsPathIncorrectFalseWallTimeSumPlain}{}
  \renewcommand{\predicateBitpreciseParallelInvariantsResultsAsyncInvariantsAbsPathIncorrectFalseWallTimeSumPlain}{307.70907187457004\xspace}
\providecommand{\predicateBitpreciseParallelInvariantsResultsAsyncInvariantsAbsPathIncorrectFalseWallTimeSumPlainHours}{}
  \renewcommand{\predicateBitpreciseParallelInvariantsResultsAsyncInvariantsAbsPathIncorrectFalseWallTimeSumPlainHours}{0.08547474218738056\xspace}

  % cpu-time-avg
\providecommand{\predicateBitpreciseParallelInvariantsResultsAsyncInvariantsAbsPathIncorrectFalseCpuTimeAvgPlain}{}
  \renewcommand{\predicateBitpreciseParallelInvariantsResultsAsyncInvariantsAbsPathIncorrectFalseCpuTimeAvgPlain}{52.93300606194444\xspace}
\providecommand{\predicateBitpreciseParallelInvariantsResultsAsyncInvariantsAbsPathIncorrectFalseCpuTimeAvgPlainHours}{}
  \renewcommand{\predicateBitpreciseParallelInvariantsResultsAsyncInvariantsAbsPathIncorrectFalseCpuTimeAvgPlainHours}{0.014703612794984567\xspace}

  % wall-time-avg
\providecommand{\predicateBitpreciseParallelInvariantsResultsAsyncInvariantsAbsPathIncorrectFalseWallTimeAvgPlain}{}
  \renewcommand{\predicateBitpreciseParallelInvariantsResultsAsyncInvariantsAbsPathIncorrectFalseWallTimeAvgPlain}{17.094948437476113\xspace}
\providecommand{\predicateBitpreciseParallelInvariantsResultsAsyncInvariantsAbsPathIncorrectFalseWallTimeAvgPlainHours}{}
  \renewcommand{\predicateBitpreciseParallelInvariantsResultsAsyncInvariantsAbsPathIncorrectFalseWallTimeAvgPlainHours}{0.004748596788187809\xspace}

  % inv-succ
\providecommand{\predicateBitpreciseParallelInvariantsResultsAsyncInvariantsAbsPathIncorrectFalseInvSuccPlain}{}
  \renewcommand{\predicateBitpreciseParallelInvariantsResultsAsyncInvariantsAbsPathIncorrectFalseInvSuccPlain}{0\xspace}

  % inv-tries
\providecommand{\predicateBitpreciseParallelInvariantsResultsAsyncInvariantsAbsPathIncorrectFalseInvTriesPlain}{}
  \renewcommand{\predicateBitpreciseParallelInvariantsResultsAsyncInvariantsAbsPathIncorrectFalseInvTriesPlain}{0\xspace}

  % inv-time-sum
\providecommand{\predicateBitpreciseParallelInvariantsResultsAsyncInvariantsAbsPathIncorrectFalseInvTimeSumPlain}{}
  \renewcommand{\predicateBitpreciseParallelInvariantsResultsAsyncInvariantsAbsPathIncorrectFalseInvTimeSumPlain}{0.0\xspace}
\providecommand{\predicateBitpreciseParallelInvariantsResultsAsyncInvariantsAbsPathIncorrectFalseInvTimeSumPlainHours}{}
  \renewcommand{\predicateBitpreciseParallelInvariantsResultsAsyncInvariantsAbsPathIncorrectFalseInvTimeSumPlainHours}{0.0\xspace}

  % finished-main
\providecommand{\predicateBitpreciseParallelInvariantsResultsAsyncInvariantsAbsPathIncorrectFalseFinishedMainPlain}{}
  \renewcommand{\predicateBitpreciseParallelInvariantsResultsAsyncInvariantsAbsPathIncorrectFalseFinishedMainPlain}{18\xspace}

 %% incorrect-true %%
\providecommand{\predicateBitpreciseParallelInvariantsResultsAsyncInvariantsAbsPathIncorrectTruePlain}{}
  \renewcommand{\predicateBitpreciseParallelInvariantsResultsAsyncInvariantsAbsPathIncorrectTruePlain}{1\xspace}

  % cpu-time-sum
\providecommand{\predicateBitpreciseParallelInvariantsResultsAsyncInvariantsAbsPathIncorrectTrueCpuTimeSumPlain}{}
  \renewcommand{\predicateBitpreciseParallelInvariantsResultsAsyncInvariantsAbsPathIncorrectTrueCpuTimeSumPlain}{16.481845859\xspace}
\providecommand{\predicateBitpreciseParallelInvariantsResultsAsyncInvariantsAbsPathIncorrectTrueCpuTimeSumPlainHours}{}
  \renewcommand{\predicateBitpreciseParallelInvariantsResultsAsyncInvariantsAbsPathIncorrectTrueCpuTimeSumPlainHours}{0.004578290516388889\xspace}

  % wall-time-sum
\providecommand{\predicateBitpreciseParallelInvariantsResultsAsyncInvariantsAbsPathIncorrectTrueWallTimeSumPlain}{}
  \renewcommand{\predicateBitpreciseParallelInvariantsResultsAsyncInvariantsAbsPathIncorrectTrueWallTimeSumPlain}{5.3519859314\xspace}
\providecommand{\predicateBitpreciseParallelInvariantsResultsAsyncInvariantsAbsPathIncorrectTrueWallTimeSumPlainHours}{}
  \renewcommand{\predicateBitpreciseParallelInvariantsResultsAsyncInvariantsAbsPathIncorrectTrueWallTimeSumPlainHours}{0.0014866627587222221\xspace}

  % cpu-time-avg
\providecommand{\predicateBitpreciseParallelInvariantsResultsAsyncInvariantsAbsPathIncorrectTrueCpuTimeAvgPlain}{}
  \renewcommand{\predicateBitpreciseParallelInvariantsResultsAsyncInvariantsAbsPathIncorrectTrueCpuTimeAvgPlain}{16.481845859\xspace}
\providecommand{\predicateBitpreciseParallelInvariantsResultsAsyncInvariantsAbsPathIncorrectTrueCpuTimeAvgPlainHours}{}
  \renewcommand{\predicateBitpreciseParallelInvariantsResultsAsyncInvariantsAbsPathIncorrectTrueCpuTimeAvgPlainHours}{0.004578290516388889\xspace}

  % wall-time-avg
\providecommand{\predicateBitpreciseParallelInvariantsResultsAsyncInvariantsAbsPathIncorrectTrueWallTimeAvgPlain}{}
  \renewcommand{\predicateBitpreciseParallelInvariantsResultsAsyncInvariantsAbsPathIncorrectTrueWallTimeAvgPlain}{5.3519859314\xspace}
\providecommand{\predicateBitpreciseParallelInvariantsResultsAsyncInvariantsAbsPathIncorrectTrueWallTimeAvgPlainHours}{}
  \renewcommand{\predicateBitpreciseParallelInvariantsResultsAsyncInvariantsAbsPathIncorrectTrueWallTimeAvgPlainHours}{0.0014866627587222221\xspace}

  % inv-succ
\providecommand{\predicateBitpreciseParallelInvariantsResultsAsyncInvariantsAbsPathIncorrectTrueInvSuccPlain}{}
  \renewcommand{\predicateBitpreciseParallelInvariantsResultsAsyncInvariantsAbsPathIncorrectTrueInvSuccPlain}{0\xspace}

  % inv-tries
\providecommand{\predicateBitpreciseParallelInvariantsResultsAsyncInvariantsAbsPathIncorrectTrueInvTriesPlain}{}
  \renewcommand{\predicateBitpreciseParallelInvariantsResultsAsyncInvariantsAbsPathIncorrectTrueInvTriesPlain}{0\xspace}

  % inv-time-sum
\providecommand{\predicateBitpreciseParallelInvariantsResultsAsyncInvariantsAbsPathIncorrectTrueInvTimeSumPlain}{}
  \renewcommand{\predicateBitpreciseParallelInvariantsResultsAsyncInvariantsAbsPathIncorrectTrueInvTimeSumPlain}{0.0\xspace}
\providecommand{\predicateBitpreciseParallelInvariantsResultsAsyncInvariantsAbsPathIncorrectTrueInvTimeSumPlainHours}{}
  \renewcommand{\predicateBitpreciseParallelInvariantsResultsAsyncInvariantsAbsPathIncorrectTrueInvTimeSumPlainHours}{0.0\xspace}

  % finished-main
\providecommand{\predicateBitpreciseParallelInvariantsResultsAsyncInvariantsAbsPathIncorrectTrueFinishedMainPlain}{}
  \renewcommand{\predicateBitpreciseParallelInvariantsResultsAsyncInvariantsAbsPathIncorrectTrueFinishedMainPlain}{1\xspace}

 %% all %%
\providecommand{\predicateBitpreciseParallelInvariantsResultsAsyncInvariantsAbsPathAllPlain}{}
  \renewcommand{\predicateBitpreciseParallelInvariantsResultsAsyncInvariantsAbsPathAllPlain}{3488\xspace}

  % cpu-time-sum
\providecommand{\predicateBitpreciseParallelInvariantsResultsAsyncInvariantsAbsPathAllCpuTimeSumPlain}{}
  \renewcommand{\predicateBitpreciseParallelInvariantsResultsAsyncInvariantsAbsPathAllCpuTimeSumPlain}{989932.8010310552\xspace}
\providecommand{\predicateBitpreciseParallelInvariantsResultsAsyncInvariantsAbsPathAllCpuTimeSumPlainHours}{}
  \renewcommand{\predicateBitpreciseParallelInvariantsResultsAsyncInvariantsAbsPathAllCpuTimeSumPlainHours}{274.98133361973754\xspace}

  % wall-time-sum
\providecommand{\predicateBitpreciseParallelInvariantsResultsAsyncInvariantsAbsPathAllWallTimeSumPlain}{}
  \renewcommand{\predicateBitpreciseParallelInvariantsResultsAsyncInvariantsAbsPathAllWallTimeSumPlain}{526809.9001560397\xspace}
\providecommand{\predicateBitpreciseParallelInvariantsResultsAsyncInvariantsAbsPathAllWallTimeSumPlainHours}{}
  \renewcommand{\predicateBitpreciseParallelInvariantsResultsAsyncInvariantsAbsPathAllWallTimeSumPlainHours}{146.33608337667772\xspace}

  % cpu-time-avg
\providecommand{\predicateBitpreciseParallelInvariantsResultsAsyncInvariantsAbsPathAllCpuTimeAvgPlain}{}
  \renewcommand{\predicateBitpreciseParallelInvariantsResultsAsyncInvariantsAbsPathAllCpuTimeAvgPlain}{283.8110094699126\xspace}
\providecommand{\predicateBitpreciseParallelInvariantsResultsAsyncInvariantsAbsPathAllCpuTimeAvgPlainHours}{}
  \renewcommand{\predicateBitpreciseParallelInvariantsResultsAsyncInvariantsAbsPathAllCpuTimeAvgPlainHours}{0.07883639151942017\xspace}

  % wall-time-avg
\providecommand{\predicateBitpreciseParallelInvariantsResultsAsyncInvariantsAbsPathAllWallTimeAvgPlain}{}
  \renewcommand{\predicateBitpreciseParallelInvariantsResultsAsyncInvariantsAbsPathAllWallTimeAvgPlain}{151.03494843923158\xspace}
\providecommand{\predicateBitpreciseParallelInvariantsResultsAsyncInvariantsAbsPathAllWallTimeAvgPlainHours}{}
  \renewcommand{\predicateBitpreciseParallelInvariantsResultsAsyncInvariantsAbsPathAllWallTimeAvgPlainHours}{0.04195415234423099\xspace}

  % inv-succ
\providecommand{\predicateBitpreciseParallelInvariantsResultsAsyncInvariantsAbsPathAllInvSuccPlain}{}
  \renewcommand{\predicateBitpreciseParallelInvariantsResultsAsyncInvariantsAbsPathAllInvSuccPlain}{0\xspace}

  % inv-tries
\providecommand{\predicateBitpreciseParallelInvariantsResultsAsyncInvariantsAbsPathAllInvTriesPlain}{}
  \renewcommand{\predicateBitpreciseParallelInvariantsResultsAsyncInvariantsAbsPathAllInvTriesPlain}{0\xspace}

  % inv-time-sum
\providecommand{\predicateBitpreciseParallelInvariantsResultsAsyncInvariantsAbsPathAllInvTimeSumPlain}{}
  \renewcommand{\predicateBitpreciseParallelInvariantsResultsAsyncInvariantsAbsPathAllInvTimeSumPlain}{0.0\xspace}
\providecommand{\predicateBitpreciseParallelInvariantsResultsAsyncInvariantsAbsPathAllInvTimeSumPlainHours}{}
  \renewcommand{\predicateBitpreciseParallelInvariantsResultsAsyncInvariantsAbsPathAllInvTimeSumPlainHours}{0.0\xspace}

  % finished-main
\providecommand{\predicateBitpreciseParallelInvariantsResultsAsyncInvariantsAbsPathAllFinishedMainPlain}{}
  \renewcommand{\predicateBitpreciseParallelInvariantsResultsAsyncInvariantsAbsPathAllFinishedMainPlain}{1169\xspace}

 %% equal-only %%
\providecommand{\predicateBitpreciseParallelInvariantsResultsAsyncInvariantsAbsPathEqualOnlyPlain}{}
  \renewcommand{\predicateBitpreciseParallelInvariantsResultsAsyncInvariantsAbsPathEqualOnlyPlain}{1865\xspace}

  % cpu-time-sum
\providecommand{\predicateBitpreciseParallelInvariantsResultsAsyncInvariantsAbsPathEqualOnlyCpuTimeSumPlain}{}
  \renewcommand{\predicateBitpreciseParallelInvariantsResultsAsyncInvariantsAbsPathEqualOnlyCpuTimeSumPlain}{138145.95442634786\xspace}
\providecommand{\predicateBitpreciseParallelInvariantsResultsAsyncInvariantsAbsPathEqualOnlyCpuTimeSumPlainHours}{}
  \renewcommand{\predicateBitpreciseParallelInvariantsResultsAsyncInvariantsAbsPathEqualOnlyCpuTimeSumPlainHours}{38.373876229541075\xspace}

  % wall-time-sum
\providecommand{\predicateBitpreciseParallelInvariantsResultsAsyncInvariantsAbsPathEqualOnlyWallTimeSumPlain}{}
  \renewcommand{\predicateBitpreciseParallelInvariantsResultsAsyncInvariantsAbsPathEqualOnlyWallTimeSumPlain}{51897.45970511367\xspace}
\providecommand{\predicateBitpreciseParallelInvariantsResultsAsyncInvariantsAbsPathEqualOnlyWallTimeSumPlainHours}{}
  \renewcommand{\predicateBitpreciseParallelInvariantsResultsAsyncInvariantsAbsPathEqualOnlyWallTimeSumPlainHours}{14.415961029198243\xspace}

  % cpu-time-avg
\providecommand{\predicateBitpreciseParallelInvariantsResultsAsyncInvariantsAbsPathEqualOnlyCpuTimeAvgPlain}{}
  \renewcommand{\predicateBitpreciseParallelInvariantsResultsAsyncInvariantsAbsPathEqualOnlyCpuTimeAvgPlain}{74.07289781573611\xspace}
\providecommand{\predicateBitpreciseParallelInvariantsResultsAsyncInvariantsAbsPathEqualOnlyCpuTimeAvgPlainHours}{}
  \renewcommand{\predicateBitpreciseParallelInvariantsResultsAsyncInvariantsAbsPathEqualOnlyCpuTimeAvgPlainHours}{0.020575804948815588\xspace}

  % wall-time-avg
\providecommand{\predicateBitpreciseParallelInvariantsResultsAsyncInvariantsAbsPathEqualOnlyWallTimeAvgPlain}{}
  \renewcommand{\predicateBitpreciseParallelInvariantsResultsAsyncInvariantsAbsPathEqualOnlyWallTimeAvgPlain}{27.827056142152102\xspace}
\providecommand{\predicateBitpreciseParallelInvariantsResultsAsyncInvariantsAbsPathEqualOnlyWallTimeAvgPlainHours}{}
  \renewcommand{\predicateBitpreciseParallelInvariantsResultsAsyncInvariantsAbsPathEqualOnlyWallTimeAvgPlainHours}{0.007729737817264473\xspace}

  % inv-succ
\providecommand{\predicateBitpreciseParallelInvariantsResultsAsyncInvariantsAbsPathEqualOnlyInvSuccPlain}{}
  \renewcommand{\predicateBitpreciseParallelInvariantsResultsAsyncInvariantsAbsPathEqualOnlyInvSuccPlain}{0\xspace}

  % inv-tries
\providecommand{\predicateBitpreciseParallelInvariantsResultsAsyncInvariantsAbsPathEqualOnlyInvTriesPlain}{}
  \renewcommand{\predicateBitpreciseParallelInvariantsResultsAsyncInvariantsAbsPathEqualOnlyInvTriesPlain}{0\xspace}

  % inv-time-sum
\providecommand{\predicateBitpreciseParallelInvariantsResultsAsyncInvariantsAbsPathEqualOnlyInvTimeSumPlain}{}
  \renewcommand{\predicateBitpreciseParallelInvariantsResultsAsyncInvariantsAbsPathEqualOnlyInvTimeSumPlain}{0.0\xspace}
\providecommand{\predicateBitpreciseParallelInvariantsResultsAsyncInvariantsAbsPathEqualOnlyInvTimeSumPlainHours}{}
  \renewcommand{\predicateBitpreciseParallelInvariantsResultsAsyncInvariantsAbsPathEqualOnlyInvTimeSumPlainHours}{0.0\xspace}

  % finished-main
\providecommand{\predicateBitpreciseParallelInvariantsResultsAsyncInvariantsAbsPathEqualOnlyFinishedMainPlain}{}
  \renewcommand{\predicateBitpreciseParallelInvariantsResultsAsyncInvariantsAbsPathEqualOnlyFinishedMainPlain}{1044\xspace}

%%% predicate_base_parallel.2016-09-03_1701.results.pred-bitvectors %%%
 %% correct %%
\providecommand{\predicateBaseParallelResultsPredBitvectorsCorrectPlain}{}
  \renewcommand{\predicateBaseParallelResultsPredBitvectorsCorrectPlain}{2050\xspace}

  % cpu-time-sum
\providecommand{\predicateBaseParallelResultsPredBitvectorsCorrectCpuTimeSumPlain}{}
  \renewcommand{\predicateBaseParallelResultsPredBitvectorsCorrectCpuTimeSumPlain}{186174.12572727102\xspace}
\providecommand{\predicateBaseParallelResultsPredBitvectorsCorrectCpuTimeSumPlainHours}{}
  \renewcommand{\predicateBaseParallelResultsPredBitvectorsCorrectCpuTimeSumPlainHours}{51.71503492424195\xspace}

  % wall-time-sum
\providecommand{\predicateBaseParallelResultsPredBitvectorsCorrectWallTimeSumPlain}{}
  \renewcommand{\predicateBaseParallelResultsPredBitvectorsCorrectWallTimeSumPlain}{77716.21838975082\xspace}
\providecommand{\predicateBaseParallelResultsPredBitvectorsCorrectWallTimeSumPlainHours}{}
  \renewcommand{\predicateBaseParallelResultsPredBitvectorsCorrectWallTimeSumPlainHours}{21.58783844159745\xspace}

  % cpu-time-avg
\providecommand{\predicateBaseParallelResultsPredBitvectorsCorrectCpuTimeAvgPlain}{}
  \renewcommand{\predicateBaseParallelResultsPredBitvectorsCorrectCpuTimeAvgPlain}{90.81664669622977\xspace}
\providecommand{\predicateBaseParallelResultsPredBitvectorsCorrectCpuTimeAvgPlainHours}{}
  \renewcommand{\predicateBaseParallelResultsPredBitvectorsCorrectCpuTimeAvgPlainHours}{0.02522684630450827\xspace}

  % wall-time-avg
\providecommand{\predicateBaseParallelResultsPredBitvectorsCorrectWallTimeAvgPlain}{}
  \renewcommand{\predicateBaseParallelResultsPredBitvectorsCorrectWallTimeAvgPlain}{37.91035043402479\xspace}
\providecommand{\predicateBaseParallelResultsPredBitvectorsCorrectWallTimeAvgPlainHours}{}
  \renewcommand{\predicateBaseParallelResultsPredBitvectorsCorrectWallTimeAvgPlainHours}{0.01053065289834022\xspace}

  % inv-succ
\providecommand{\predicateBaseParallelResultsPredBitvectorsCorrectInvSuccPlain}{}
  \renewcommand{\predicateBaseParallelResultsPredBitvectorsCorrectInvSuccPlain}{0\xspace}

  % inv-tries
\providecommand{\predicateBaseParallelResultsPredBitvectorsCorrectInvTriesPlain}{}
  \renewcommand{\predicateBaseParallelResultsPredBitvectorsCorrectInvTriesPlain}{0\xspace}

  % inv-time-sum
\providecommand{\predicateBaseParallelResultsPredBitvectorsCorrectInvTimeSumPlain}{}
  \renewcommand{\predicateBaseParallelResultsPredBitvectorsCorrectInvTimeSumPlain}{0.0\xspace}
\providecommand{\predicateBaseParallelResultsPredBitvectorsCorrectInvTimeSumPlainHours}{}
  \renewcommand{\predicateBaseParallelResultsPredBitvectorsCorrectInvTimeSumPlainHours}{0.0\xspace}

  % finished-main
\providecommand{\predicateBaseParallelResultsPredBitvectorsCorrectFinishedMainPlain}{}
  \renewcommand{\predicateBaseParallelResultsPredBitvectorsCorrectFinishedMainPlain}{1109\xspace}

 %% incorrect %%
\providecommand{\predicateBaseParallelResultsPredBitvectorsIncorrectPlain}{}
  \renewcommand{\predicateBaseParallelResultsPredBitvectorsIncorrectPlain}{18\xspace}

  % cpu-time-sum
\providecommand{\predicateBaseParallelResultsPredBitvectorsIncorrectCpuTimeSumPlain}{}
  \renewcommand{\predicateBaseParallelResultsPredBitvectorsIncorrectCpuTimeSumPlain}{905.4540134499999\xspace}
\providecommand{\predicateBaseParallelResultsPredBitvectorsIncorrectCpuTimeSumPlainHours}{}
  \renewcommand{\predicateBaseParallelResultsPredBitvectorsIncorrectCpuTimeSumPlainHours}{0.2515150037361111\xspace}

  % wall-time-sum
\providecommand{\predicateBaseParallelResultsPredBitvectorsIncorrectWallTimeSumPlain}{}
  \renewcommand{\predicateBaseParallelResultsPredBitvectorsIncorrectWallTimeSumPlain}{296.6754686832\xspace}
\providecommand{\predicateBaseParallelResultsPredBitvectorsIncorrectWallTimeSumPlainHours}{}
  \renewcommand{\predicateBaseParallelResultsPredBitvectorsIncorrectWallTimeSumPlainHours}{0.082409852412\xspace}

  % cpu-time-avg
\providecommand{\predicateBaseParallelResultsPredBitvectorsIncorrectCpuTimeAvgPlain}{}
  \renewcommand{\predicateBaseParallelResultsPredBitvectorsIncorrectCpuTimeAvgPlain}{50.30300074722222\xspace}
\providecommand{\predicateBaseParallelResultsPredBitvectorsIncorrectCpuTimeAvgPlainHours}{}
  \renewcommand{\predicateBaseParallelResultsPredBitvectorsIncorrectCpuTimeAvgPlainHours}{0.013973055763117283\xspace}

  % wall-time-avg
\providecommand{\predicateBaseParallelResultsPredBitvectorsIncorrectWallTimeAvgPlain}{}
  \renewcommand{\predicateBaseParallelResultsPredBitvectorsIncorrectWallTimeAvgPlain}{16.4819704824\xspace}
\providecommand{\predicateBaseParallelResultsPredBitvectorsIncorrectWallTimeAvgPlainHours}{}
  \renewcommand{\predicateBaseParallelResultsPredBitvectorsIncorrectWallTimeAvgPlainHours}{0.004578325134000001\xspace}

  % inv-succ
\providecommand{\predicateBaseParallelResultsPredBitvectorsIncorrectInvSuccPlain}{}
  \renewcommand{\predicateBaseParallelResultsPredBitvectorsIncorrectInvSuccPlain}{0\xspace}

  % inv-tries
\providecommand{\predicateBaseParallelResultsPredBitvectorsIncorrectInvTriesPlain}{}
  \renewcommand{\predicateBaseParallelResultsPredBitvectorsIncorrectInvTriesPlain}{0\xspace}

  % inv-time-sum
\providecommand{\predicateBaseParallelResultsPredBitvectorsIncorrectInvTimeSumPlain}{}
  \renewcommand{\predicateBaseParallelResultsPredBitvectorsIncorrectInvTimeSumPlain}{0.0\xspace}
\providecommand{\predicateBaseParallelResultsPredBitvectorsIncorrectInvTimeSumPlainHours}{}
  \renewcommand{\predicateBaseParallelResultsPredBitvectorsIncorrectInvTimeSumPlainHours}{0.0\xspace}

  % finished-main
\providecommand{\predicateBaseParallelResultsPredBitvectorsIncorrectFinishedMainPlain}{}
  \renewcommand{\predicateBaseParallelResultsPredBitvectorsIncorrectFinishedMainPlain}{18\xspace}

 %% timeout %%
\providecommand{\predicateBaseParallelResultsPredBitvectorsTimeoutPlain}{}
  \renewcommand{\predicateBaseParallelResultsPredBitvectorsTimeoutPlain}{1246\xspace}

  % cpu-time-sum
\providecommand{\predicateBaseParallelResultsPredBitvectorsTimeoutCpuTimeSumPlain}{}
  \renewcommand{\predicateBaseParallelResultsPredBitvectorsTimeoutCpuTimeSumPlain}{766027.2173757756\xspace}
\providecommand{\predicateBaseParallelResultsPredBitvectorsTimeoutCpuTimeSumPlainHours}{}
  \renewcommand{\predicateBaseParallelResultsPredBitvectorsTimeoutCpuTimeSumPlainHours}{212.78533815993765\xspace}

  % wall-time-sum
\providecommand{\predicateBaseParallelResultsPredBitvectorsTimeoutWallTimeSumPlain}{}
  \renewcommand{\predicateBaseParallelResultsPredBitvectorsTimeoutWallTimeSumPlain}{431836.77718257194\xspace}
\providecommand{\predicateBaseParallelResultsPredBitvectorsTimeoutWallTimeSumPlainHours}{}
  \renewcommand{\predicateBaseParallelResultsPredBitvectorsTimeoutWallTimeSumPlainHours}{119.95466032849221\xspace}

  % cpu-time-avg
\providecommand{\predicateBaseParallelResultsPredBitvectorsTimeoutCpuTimeAvgPlain}{}
  \renewcommand{\predicateBaseParallelResultsPredBitvectorsTimeoutCpuTimeAvgPlain}{614.7890990174764\xspace}
\providecommand{\predicateBaseParallelResultsPredBitvectorsTimeoutCpuTimeAvgPlainHours}{}
  \renewcommand{\predicateBaseParallelResultsPredBitvectorsTimeoutCpuTimeAvgPlainHours}{0.17077474972707676\xspace}

  % wall-time-avg
\providecommand{\predicateBaseParallelResultsPredBitvectorsTimeoutWallTimeAvgPlain}{}
  \renewcommand{\predicateBaseParallelResultsPredBitvectorsTimeoutWallTimeAvgPlain}{346.578472859207\xspace}
\providecommand{\predicateBaseParallelResultsPredBitvectorsTimeoutWallTimeAvgPlainHours}{}
  \renewcommand{\predicateBaseParallelResultsPredBitvectorsTimeoutWallTimeAvgPlainHours}{0.09627179801644639\xspace}

  % inv-succ
\providecommand{\predicateBaseParallelResultsPredBitvectorsTimeoutInvSuccPlain}{}
  \renewcommand{\predicateBaseParallelResultsPredBitvectorsTimeoutInvSuccPlain}{0\xspace}

  % inv-tries
\providecommand{\predicateBaseParallelResultsPredBitvectorsTimeoutInvTriesPlain}{}
  \renewcommand{\predicateBaseParallelResultsPredBitvectorsTimeoutInvTriesPlain}{0\xspace}

  % inv-time-sum
\providecommand{\predicateBaseParallelResultsPredBitvectorsTimeoutInvTimeSumPlain}{}
  \renewcommand{\predicateBaseParallelResultsPredBitvectorsTimeoutInvTimeSumPlain}{0.0\xspace}
\providecommand{\predicateBaseParallelResultsPredBitvectorsTimeoutInvTimeSumPlainHours}{}
  \renewcommand{\predicateBaseParallelResultsPredBitvectorsTimeoutInvTimeSumPlainHours}{0.0\xspace}

  % finished-main
\providecommand{\predicateBaseParallelResultsPredBitvectorsTimeoutFinishedMainPlain}{}
  \renewcommand{\predicateBaseParallelResultsPredBitvectorsTimeoutFinishedMainPlain}{0\xspace}

 %% unknown-or-category-error %%
\providecommand{\predicateBaseParallelResultsPredBitvectorsUnknownOrCategoryErrorPlain}{}
  \renewcommand{\predicateBaseParallelResultsPredBitvectorsUnknownOrCategoryErrorPlain}{1420\xspace}

  % cpu-time-sum
\providecommand{\predicateBaseParallelResultsPredBitvectorsUnknownOrCategoryErrorCpuTimeSumPlain}{}
  \renewcommand{\predicateBaseParallelResultsPredBitvectorsUnknownOrCategoryErrorCpuTimeSumPlain}{826071.5568969803\xspace}
\providecommand{\predicateBaseParallelResultsPredBitvectorsUnknownOrCategoryErrorCpuTimeSumPlainHours}{}
  \renewcommand{\predicateBaseParallelResultsPredBitvectorsUnknownOrCategoryErrorCpuTimeSumPlainHours}{229.4643213602723\xspace}

  % wall-time-sum
\providecommand{\predicateBaseParallelResultsPredBitvectorsUnknownOrCategoryErrorWallTimeSumPlain}{}
  \renewcommand{\predicateBaseParallelResultsPredBitvectorsUnknownOrCategoryErrorWallTimeSumPlain}{469011.1509220535\xspace}
\providecommand{\predicateBaseParallelResultsPredBitvectorsUnknownOrCategoryErrorWallTimeSumPlainHours}{}
  \renewcommand{\predicateBaseParallelResultsPredBitvectorsUnknownOrCategoryErrorWallTimeSumPlainHours}{130.28087525612597\xspace}

  % cpu-time-avg
\providecommand{\predicateBaseParallelResultsPredBitvectorsUnknownOrCategoryErrorCpuTimeAvgPlain}{}
  \renewcommand{\predicateBaseParallelResultsPredBitvectorsUnknownOrCategoryErrorCpuTimeAvgPlain}{581.7405330260425\xspace}
\providecommand{\predicateBaseParallelResultsPredBitvectorsUnknownOrCategoryErrorCpuTimeAvgPlainHours}{}
  \renewcommand{\predicateBaseParallelResultsPredBitvectorsUnknownOrCategoryErrorCpuTimeAvgPlainHours}{0.16159459250723401\xspace}

  % wall-time-avg
\providecommand{\predicateBaseParallelResultsPredBitvectorsUnknownOrCategoryErrorWallTimeAvgPlain}{}
  \renewcommand{\predicateBaseParallelResultsPredBitvectorsUnknownOrCategoryErrorWallTimeAvgPlain}{330.2895429028546\xspace}
\providecommand{\predicateBaseParallelResultsPredBitvectorsUnknownOrCategoryErrorWallTimeAvgPlainHours}{}
  \renewcommand{\predicateBaseParallelResultsPredBitvectorsUnknownOrCategoryErrorWallTimeAvgPlainHours}{0.09174709525079294\xspace}

  % inv-succ
\providecommand{\predicateBaseParallelResultsPredBitvectorsUnknownOrCategoryErrorInvSuccPlain}{}
  \renewcommand{\predicateBaseParallelResultsPredBitvectorsUnknownOrCategoryErrorInvSuccPlain}{0\xspace}

  % inv-tries
\providecommand{\predicateBaseParallelResultsPredBitvectorsUnknownOrCategoryErrorInvTriesPlain}{}
  \renewcommand{\predicateBaseParallelResultsPredBitvectorsUnknownOrCategoryErrorInvTriesPlain}{0\xspace}

  % inv-time-sum
\providecommand{\predicateBaseParallelResultsPredBitvectorsUnknownOrCategoryErrorInvTimeSumPlain}{}
  \renewcommand{\predicateBaseParallelResultsPredBitvectorsUnknownOrCategoryErrorInvTimeSumPlain}{0.0\xspace}
\providecommand{\predicateBaseParallelResultsPredBitvectorsUnknownOrCategoryErrorInvTimeSumPlainHours}{}
  \renewcommand{\predicateBaseParallelResultsPredBitvectorsUnknownOrCategoryErrorInvTimeSumPlainHours}{0.0\xspace}

  % finished-main
\providecommand{\predicateBaseParallelResultsPredBitvectorsUnknownOrCategoryErrorFinishedMainPlain}{}
  \renewcommand{\predicateBaseParallelResultsPredBitvectorsUnknownOrCategoryErrorFinishedMainPlain}{3\xspace}

 %% correct-false %%
\providecommand{\predicateBaseParallelResultsPredBitvectorsCorrectFalsePlain}{}
  \renewcommand{\predicateBaseParallelResultsPredBitvectorsCorrectFalsePlain}{541\xspace}

  % cpu-time-sum
\providecommand{\predicateBaseParallelResultsPredBitvectorsCorrectFalseCpuTimeSumPlain}{}
  \renewcommand{\predicateBaseParallelResultsPredBitvectorsCorrectFalseCpuTimeSumPlain}{73533.12947292808\xspace}
\providecommand{\predicateBaseParallelResultsPredBitvectorsCorrectFalseCpuTimeSumPlainHours}{}
  \renewcommand{\predicateBaseParallelResultsPredBitvectorsCorrectFalseCpuTimeSumPlainHours}{20.42586929803558\xspace}

  % wall-time-sum
\providecommand{\predicateBaseParallelResultsPredBitvectorsCorrectFalseWallTimeSumPlain}{}
  \renewcommand{\predicateBaseParallelResultsPredBitvectorsCorrectFalseWallTimeSumPlain}{33148.401274206684\xspace}
\providecommand{\predicateBaseParallelResultsPredBitvectorsCorrectFalseWallTimeSumPlainHours}{}
  \renewcommand{\predicateBaseParallelResultsPredBitvectorsCorrectFalseWallTimeSumPlainHours}{9.20788924283519\xspace}

  % cpu-time-avg
\providecommand{\predicateBaseParallelResultsPredBitvectorsCorrectFalseCpuTimeAvgPlain}{}
  \renewcommand{\predicateBaseParallelResultsPredBitvectorsCorrectFalseCpuTimeAvgPlain}{135.9207568815676\xspace}
\providecommand{\predicateBaseParallelResultsPredBitvectorsCorrectFalseCpuTimeAvgPlainHours}{}
  \renewcommand{\predicateBaseParallelResultsPredBitvectorsCorrectFalseCpuTimeAvgPlainHours}{0.037755765800435444\xspace}

  % wall-time-avg
\providecommand{\predicateBaseParallelResultsPredBitvectorsCorrectFalseWallTimeAvgPlain}{}
  \renewcommand{\predicateBaseParallelResultsPredBitvectorsCorrectFalseWallTimeAvgPlain}{61.27246076563158\xspace}
\providecommand{\predicateBaseParallelResultsPredBitvectorsCorrectFalseWallTimeAvgPlainHours}{}
  \renewcommand{\predicateBaseParallelResultsPredBitvectorsCorrectFalseWallTimeAvgPlainHours}{0.017020127990453215\xspace}

  % inv-succ
\providecommand{\predicateBaseParallelResultsPredBitvectorsCorrectFalseInvSuccPlain}{}
  \renewcommand{\predicateBaseParallelResultsPredBitvectorsCorrectFalseInvSuccPlain}{0\xspace}

  % inv-tries
\providecommand{\predicateBaseParallelResultsPredBitvectorsCorrectFalseInvTriesPlain}{}
  \renewcommand{\predicateBaseParallelResultsPredBitvectorsCorrectFalseInvTriesPlain}{0\xspace}

  % inv-time-sum
\providecommand{\predicateBaseParallelResultsPredBitvectorsCorrectFalseInvTimeSumPlain}{}
  \renewcommand{\predicateBaseParallelResultsPredBitvectorsCorrectFalseInvTimeSumPlain}{0.0\xspace}
\providecommand{\predicateBaseParallelResultsPredBitvectorsCorrectFalseInvTimeSumPlainHours}{}
  \renewcommand{\predicateBaseParallelResultsPredBitvectorsCorrectFalseInvTimeSumPlainHours}{0.0\xspace}

  % finished-main
\providecommand{\predicateBaseParallelResultsPredBitvectorsCorrectFalseFinishedMainPlain}{}
  \renewcommand{\predicateBaseParallelResultsPredBitvectorsCorrectFalseFinishedMainPlain}{541\xspace}

 %% correct-true %%
\providecommand{\predicateBaseParallelResultsPredBitvectorsCorrectTruePlain}{}
  \renewcommand{\predicateBaseParallelResultsPredBitvectorsCorrectTruePlain}{1509\xspace}

  % cpu-time-sum
\providecommand{\predicateBaseParallelResultsPredBitvectorsCorrectTrueCpuTimeSumPlain}{}
  \renewcommand{\predicateBaseParallelResultsPredBitvectorsCorrectTrueCpuTimeSumPlain}{112640.99625434304\xspace}
\providecommand{\predicateBaseParallelResultsPredBitvectorsCorrectTrueCpuTimeSumPlainHours}{}
  \renewcommand{\predicateBaseParallelResultsPredBitvectorsCorrectTrueCpuTimeSumPlainHours}{31.2891656262064\xspace}

  % wall-time-sum
\providecommand{\predicateBaseParallelResultsPredBitvectorsCorrectTrueWallTimeSumPlain}{}
  \renewcommand{\predicateBaseParallelResultsPredBitvectorsCorrectTrueWallTimeSumPlain}{44567.817115544116\xspace}
\providecommand{\predicateBaseParallelResultsPredBitvectorsCorrectTrueWallTimeSumPlainHours}{}
  \renewcommand{\predicateBaseParallelResultsPredBitvectorsCorrectTrueWallTimeSumPlainHours}{12.379949198762255\xspace}

  % cpu-time-avg
\providecommand{\predicateBaseParallelResultsPredBitvectorsCorrectTrueCpuTimeAvgPlain}{}
  \renewcommand{\predicateBaseParallelResultsPredBitvectorsCorrectTrueCpuTimeAvgPlain}{74.646120778226\xspace}
\providecommand{\predicateBaseParallelResultsPredBitvectorsCorrectTrueCpuTimeAvgPlainHours}{}
  \renewcommand{\predicateBaseParallelResultsPredBitvectorsCorrectTrueCpuTimeAvgPlainHours}{0.020735033549507224\xspace}

  % wall-time-avg
\providecommand{\predicateBaseParallelResultsPredBitvectorsCorrectTrueWallTimeAvgPlain}{}
  \renewcommand{\predicateBaseParallelResultsPredBitvectorsCorrectTrueWallTimeAvgPlain}{29.534670056689276\xspace}
\providecommand{\predicateBaseParallelResultsPredBitvectorsCorrectTrueWallTimeAvgPlainHours}{}
  \renewcommand{\predicateBaseParallelResultsPredBitvectorsCorrectTrueWallTimeAvgPlainHours}{0.008204075015747022\xspace}

  % inv-succ
\providecommand{\predicateBaseParallelResultsPredBitvectorsCorrectTrueInvSuccPlain}{}
  \renewcommand{\predicateBaseParallelResultsPredBitvectorsCorrectTrueInvSuccPlain}{0\xspace}

  % inv-tries
\providecommand{\predicateBaseParallelResultsPredBitvectorsCorrectTrueInvTriesPlain}{}
  \renewcommand{\predicateBaseParallelResultsPredBitvectorsCorrectTrueInvTriesPlain}{0\xspace}

  % inv-time-sum
\providecommand{\predicateBaseParallelResultsPredBitvectorsCorrectTrueInvTimeSumPlain}{}
  \renewcommand{\predicateBaseParallelResultsPredBitvectorsCorrectTrueInvTimeSumPlain}{0.0\xspace}
\providecommand{\predicateBaseParallelResultsPredBitvectorsCorrectTrueInvTimeSumPlainHours}{}
  \renewcommand{\predicateBaseParallelResultsPredBitvectorsCorrectTrueInvTimeSumPlainHours}{0.0\xspace}

  % finished-main
\providecommand{\predicateBaseParallelResultsPredBitvectorsCorrectTrueFinishedMainPlain}{}
  \renewcommand{\predicateBaseParallelResultsPredBitvectorsCorrectTrueFinishedMainPlain}{568\xspace}

 %% incorrect-false %%
\providecommand{\predicateBaseParallelResultsPredBitvectorsIncorrectFalsePlain}{}
  \renewcommand{\predicateBaseParallelResultsPredBitvectorsIncorrectFalsePlain}{18\xspace}

  % cpu-time-sum
\providecommand{\predicateBaseParallelResultsPredBitvectorsIncorrectFalseCpuTimeSumPlain}{}
  \renewcommand{\predicateBaseParallelResultsPredBitvectorsIncorrectFalseCpuTimeSumPlain}{905.4540134499999\xspace}
\providecommand{\predicateBaseParallelResultsPredBitvectorsIncorrectFalseCpuTimeSumPlainHours}{}
  \renewcommand{\predicateBaseParallelResultsPredBitvectorsIncorrectFalseCpuTimeSumPlainHours}{0.2515150037361111\xspace}

  % wall-time-sum
\providecommand{\predicateBaseParallelResultsPredBitvectorsIncorrectFalseWallTimeSumPlain}{}
  \renewcommand{\predicateBaseParallelResultsPredBitvectorsIncorrectFalseWallTimeSumPlain}{296.6754686832\xspace}
\providecommand{\predicateBaseParallelResultsPredBitvectorsIncorrectFalseWallTimeSumPlainHours}{}
  \renewcommand{\predicateBaseParallelResultsPredBitvectorsIncorrectFalseWallTimeSumPlainHours}{0.082409852412\xspace}

  % cpu-time-avg
\providecommand{\predicateBaseParallelResultsPredBitvectorsIncorrectFalseCpuTimeAvgPlain}{}
  \renewcommand{\predicateBaseParallelResultsPredBitvectorsIncorrectFalseCpuTimeAvgPlain}{50.30300074722222\xspace}
\providecommand{\predicateBaseParallelResultsPredBitvectorsIncorrectFalseCpuTimeAvgPlainHours}{}
  \renewcommand{\predicateBaseParallelResultsPredBitvectorsIncorrectFalseCpuTimeAvgPlainHours}{0.013973055763117283\xspace}

  % wall-time-avg
\providecommand{\predicateBaseParallelResultsPredBitvectorsIncorrectFalseWallTimeAvgPlain}{}
  \renewcommand{\predicateBaseParallelResultsPredBitvectorsIncorrectFalseWallTimeAvgPlain}{16.4819704824\xspace}
\providecommand{\predicateBaseParallelResultsPredBitvectorsIncorrectFalseWallTimeAvgPlainHours}{}
  \renewcommand{\predicateBaseParallelResultsPredBitvectorsIncorrectFalseWallTimeAvgPlainHours}{0.004578325134000001\xspace}

  % inv-succ
\providecommand{\predicateBaseParallelResultsPredBitvectorsIncorrectFalseInvSuccPlain}{}
  \renewcommand{\predicateBaseParallelResultsPredBitvectorsIncorrectFalseInvSuccPlain}{0\xspace}

  % inv-tries
\providecommand{\predicateBaseParallelResultsPredBitvectorsIncorrectFalseInvTriesPlain}{}
  \renewcommand{\predicateBaseParallelResultsPredBitvectorsIncorrectFalseInvTriesPlain}{0\xspace}

  % inv-time-sum
\providecommand{\predicateBaseParallelResultsPredBitvectorsIncorrectFalseInvTimeSumPlain}{}
  \renewcommand{\predicateBaseParallelResultsPredBitvectorsIncorrectFalseInvTimeSumPlain}{0.0\xspace}
\providecommand{\predicateBaseParallelResultsPredBitvectorsIncorrectFalseInvTimeSumPlainHours}{}
  \renewcommand{\predicateBaseParallelResultsPredBitvectorsIncorrectFalseInvTimeSumPlainHours}{0.0\xspace}

  % finished-main
\providecommand{\predicateBaseParallelResultsPredBitvectorsIncorrectFalseFinishedMainPlain}{}
  \renewcommand{\predicateBaseParallelResultsPredBitvectorsIncorrectFalseFinishedMainPlain}{18\xspace}

 %% incorrect-true %%
\providecommand{\predicateBaseParallelResultsPredBitvectorsIncorrectTruePlain}{}
  \renewcommand{\predicateBaseParallelResultsPredBitvectorsIncorrectTruePlain}{0\xspace}

  % cpu-time-sum
\providecommand{\predicateBaseParallelResultsPredBitvectorsIncorrectTrueCpuTimeSumPlain}{}
  \renewcommand{\predicateBaseParallelResultsPredBitvectorsIncorrectTrueCpuTimeSumPlain}{0.0\xspace}
\providecommand{\predicateBaseParallelResultsPredBitvectorsIncorrectTrueCpuTimeSumPlainHours}{}
  \renewcommand{\predicateBaseParallelResultsPredBitvectorsIncorrectTrueCpuTimeSumPlainHours}{0.0\xspace}

  % wall-time-sum
\providecommand{\predicateBaseParallelResultsPredBitvectorsIncorrectTrueWallTimeSumPlain}{}
  \renewcommand{\predicateBaseParallelResultsPredBitvectorsIncorrectTrueWallTimeSumPlain}{0.0\xspace}
\providecommand{\predicateBaseParallelResultsPredBitvectorsIncorrectTrueWallTimeSumPlainHours}{}
  \renewcommand{\predicateBaseParallelResultsPredBitvectorsIncorrectTrueWallTimeSumPlainHours}{0.0\xspace}

  % cpu-time-avg
\providecommand{\predicateBaseParallelResultsPredBitvectorsIncorrectTrueCpuTimeAvgPlain}{}
  \renewcommand{\predicateBaseParallelResultsPredBitvectorsIncorrectTrueCpuTimeAvgPlain}{NaN\xspace}
\providecommand{\predicateBaseParallelResultsPredBitvectorsIncorrectTrueCpuTimeAvgPlainHours}{}
  \renewcommand{\predicateBaseParallelResultsPredBitvectorsIncorrectTrueCpuTimeAvgPlainHours}{NaN\xspace}

  % wall-time-avg
\providecommand{\predicateBaseParallelResultsPredBitvectorsIncorrectTrueWallTimeAvgPlain}{}
  \renewcommand{\predicateBaseParallelResultsPredBitvectorsIncorrectTrueWallTimeAvgPlain}{NaN\xspace}
\providecommand{\predicateBaseParallelResultsPredBitvectorsIncorrectTrueWallTimeAvgPlainHours}{}
  \renewcommand{\predicateBaseParallelResultsPredBitvectorsIncorrectTrueWallTimeAvgPlainHours}{NaN\xspace}

  % inv-succ
\providecommand{\predicateBaseParallelResultsPredBitvectorsIncorrectTrueInvSuccPlain}{}
  \renewcommand{\predicateBaseParallelResultsPredBitvectorsIncorrectTrueInvSuccPlain}{0\xspace}

  % inv-tries
\providecommand{\predicateBaseParallelResultsPredBitvectorsIncorrectTrueInvTriesPlain}{}
  \renewcommand{\predicateBaseParallelResultsPredBitvectorsIncorrectTrueInvTriesPlain}{0\xspace}

  % inv-time-sum
\providecommand{\predicateBaseParallelResultsPredBitvectorsIncorrectTrueInvTimeSumPlain}{}
  \renewcommand{\predicateBaseParallelResultsPredBitvectorsIncorrectTrueInvTimeSumPlain}{0.0\xspace}
\providecommand{\predicateBaseParallelResultsPredBitvectorsIncorrectTrueInvTimeSumPlainHours}{}
  \renewcommand{\predicateBaseParallelResultsPredBitvectorsIncorrectTrueInvTimeSumPlainHours}{0.0\xspace}

  % finished-main
\providecommand{\predicateBaseParallelResultsPredBitvectorsIncorrectTrueFinishedMainPlain}{}
  \renewcommand{\predicateBaseParallelResultsPredBitvectorsIncorrectTrueFinishedMainPlain}{0\xspace}

 %% all %%
\providecommand{\predicateBaseParallelResultsPredBitvectorsAllPlain}{}
  \renewcommand{\predicateBaseParallelResultsPredBitvectorsAllPlain}{3488\xspace}

  % cpu-time-sum
\providecommand{\predicateBaseParallelResultsPredBitvectorsAllCpuTimeSumPlain}{}
  \renewcommand{\predicateBaseParallelResultsPredBitvectorsAllCpuTimeSumPlain}{1013151.1366376984\xspace}
\providecommand{\predicateBaseParallelResultsPredBitvectorsAllCpuTimeSumPlainHours}{}
  \renewcommand{\predicateBaseParallelResultsPredBitvectorsAllCpuTimeSumPlainHours}{281.43087128824953\xspace}

  % wall-time-sum
\providecommand{\predicateBaseParallelResultsPredBitvectorsAllWallTimeSumPlain}{}
  \renewcommand{\predicateBaseParallelResultsPredBitvectorsAllWallTimeSumPlain}{547024.0447804875\xspace}
\providecommand{\predicateBaseParallelResultsPredBitvectorsAllWallTimeSumPlainHours}{}
  \renewcommand{\predicateBaseParallelResultsPredBitvectorsAllWallTimeSumPlainHours}{151.95112355013543\xspace}

  % cpu-time-avg
\providecommand{\predicateBaseParallelResultsPredBitvectorsAllCpuTimeAvgPlain}{}
  \renewcommand{\predicateBaseParallelResultsPredBitvectorsAllCpuTimeAvgPlain}{290.46764238466125\xspace}
\providecommand{\predicateBaseParallelResultsPredBitvectorsAllCpuTimeAvgPlainHours}{}
  \renewcommand{\predicateBaseParallelResultsPredBitvectorsAllCpuTimeAvgPlainHours}{0.08068545621796146\xspace}

  % wall-time-avg
\providecommand{\predicateBaseParallelResultsPredBitvectorsAllWallTimeAvgPlain}{}
  \renewcommand{\predicateBaseParallelResultsPredBitvectorsAllWallTimeAvgPlain}{156.83028806780032\xspace}
\providecommand{\predicateBaseParallelResultsPredBitvectorsAllWallTimeAvgPlainHours}{}
  \renewcommand{\predicateBaseParallelResultsPredBitvectorsAllWallTimeAvgPlainHours}{0.043563968907722316\xspace}

  % inv-succ
\providecommand{\predicateBaseParallelResultsPredBitvectorsAllInvSuccPlain}{}
  \renewcommand{\predicateBaseParallelResultsPredBitvectorsAllInvSuccPlain}{0\xspace}

  % inv-tries
\providecommand{\predicateBaseParallelResultsPredBitvectorsAllInvTriesPlain}{}
  \renewcommand{\predicateBaseParallelResultsPredBitvectorsAllInvTriesPlain}{0\xspace}

  % inv-time-sum
\providecommand{\predicateBaseParallelResultsPredBitvectorsAllInvTimeSumPlain}{}
  \renewcommand{\predicateBaseParallelResultsPredBitvectorsAllInvTimeSumPlain}{0.0\xspace}
\providecommand{\predicateBaseParallelResultsPredBitvectorsAllInvTimeSumPlainHours}{}
  \renewcommand{\predicateBaseParallelResultsPredBitvectorsAllInvTimeSumPlainHours}{0.0\xspace}

  % finished-main
\providecommand{\predicateBaseParallelResultsPredBitvectorsAllFinishedMainPlain}{}
  \renewcommand{\predicateBaseParallelResultsPredBitvectorsAllFinishedMainPlain}{1130\xspace}

 %% equal-only %%
\providecommand{\predicateBaseParallelResultsPredBitvectorsEqualOnlyPlain}{}
  \renewcommand{\predicateBaseParallelResultsPredBitvectorsEqualOnlyPlain}{1865\xspace}

  % cpu-time-sum
\providecommand{\predicateBaseParallelResultsPredBitvectorsEqualOnlyCpuTimeSumPlain}{}
  \renewcommand{\predicateBaseParallelResultsPredBitvectorsEqualOnlyCpuTimeSumPlain}{143762.873377509\xspace}
\providecommand{\predicateBaseParallelResultsPredBitvectorsEqualOnlyCpuTimeSumPlainHours}{}
  \renewcommand{\predicateBaseParallelResultsPredBitvectorsEqualOnlyCpuTimeSumPlainHours}{39.9341314937525\xspace}

  % wall-time-sum
\providecommand{\predicateBaseParallelResultsPredBitvectorsEqualOnlyWallTimeSumPlain}{}
  \renewcommand{\predicateBaseParallelResultsPredBitvectorsEqualOnlyWallTimeSumPlain}{56238.797879459176\xspace}
\providecommand{\predicateBaseParallelResultsPredBitvectorsEqualOnlyWallTimeSumPlainHours}{}
  \renewcommand{\predicateBaseParallelResultsPredBitvectorsEqualOnlyWallTimeSumPlainHours}{15.621888299849772\xspace}

  % cpu-time-avg
\providecommand{\predicateBaseParallelResultsPredBitvectorsEqualOnlyCpuTimeAvgPlain}{}
  \renewcommand{\predicateBaseParallelResultsPredBitvectorsEqualOnlyCpuTimeAvgPlain}{77.08465060456247\xspace}
\providecommand{\predicateBaseParallelResultsPredBitvectorsEqualOnlyCpuTimeAvgPlainHours}{}
  \renewcommand{\predicateBaseParallelResultsPredBitvectorsEqualOnlyCpuTimeAvgPlainHours}{0.021412402945711798\xspace}

  % wall-time-avg
\providecommand{\predicateBaseParallelResultsPredBitvectorsEqualOnlyWallTimeAvgPlain}{}
  \renewcommand{\predicateBaseParallelResultsPredBitvectorsEqualOnlyWallTimeAvgPlain}{30.15485140989768\xspace}
\providecommand{\predicateBaseParallelResultsPredBitvectorsEqualOnlyWallTimeAvgPlainHours}{}
  \renewcommand{\predicateBaseParallelResultsPredBitvectorsEqualOnlyWallTimeAvgPlainHours}{0.008376347613860467\xspace}

  % inv-succ
\providecommand{\predicateBaseParallelResultsPredBitvectorsEqualOnlyInvSuccPlain}{}
  \renewcommand{\predicateBaseParallelResultsPredBitvectorsEqualOnlyInvSuccPlain}{0\xspace}

  % inv-tries
\providecommand{\predicateBaseParallelResultsPredBitvectorsEqualOnlyInvTriesPlain}{}
  \renewcommand{\predicateBaseParallelResultsPredBitvectorsEqualOnlyInvTriesPlain}{0\xspace}

  % inv-time-sum
\providecommand{\predicateBaseParallelResultsPredBitvectorsEqualOnlyInvTimeSumPlain}{}
  \renewcommand{\predicateBaseParallelResultsPredBitvectorsEqualOnlyInvTimeSumPlain}{0.0\xspace}
\providecommand{\predicateBaseParallelResultsPredBitvectorsEqualOnlyInvTimeSumPlainHours}{}
  \renewcommand{\predicateBaseParallelResultsPredBitvectorsEqualOnlyInvTimeSumPlainHours}{0.0\xspace}

  % finished-main
\providecommand{\predicateBaseParallelResultsPredBitvectorsEqualOnlyFinishedMainPlain}{}
  \renewcommand{\predicateBaseParallelResultsPredBitvectorsEqualOnlyFinishedMainPlain}{1049\xspace}


\begin{table}
 \caption{Details on all parallel analyses using invariants and their baselines}
 \label{table:parallel}
\begin{adjustbox}{max width=\textwidth}
  \begin{tabular}{l
                  S[table-format=4.0, round-mode=off, round-precision=3]
                  S[table-format=3.0, round-mode=off, round-precision=3]
                  S[table-format=1.0, round-mode=off, round-precision=3]
                  S[table-format=2.0, round-mode=off, round-precision=3]
                  S[table-format=4.0, round-mode=off, round-precision=3]
                  S[table-format=3.0, round-mode=figures, round-precision=3]
                  S[table-format=2.1, round-mode=figures, round-precision=3]
                  S[table-format=3.0, round-mode=figures, round-precision=3]
                  S[table-format=2.1, round-mode=figures, round-precision=3]}
\toprule
 & \multicolumn{2}{c}{\textbf{correct}} & \multicolumn{2}{c}{\textbf{wrong}} & \multicolumn{1}{c}{\textbf{Main Succ}} & \multicolumn{2}{c}{\textbf{Wall time (h)}} & \multicolumn{2}{c}{\textbf{CPU time (h)}}\\              
 \textbf{async-}& \multicolumn{1}{c}{proof} & \multicolumn{1}{c}{alarm} & \multicolumn{1}{c}{proof} & \multicolumn{1}{c}{alarm} & \multicolumn{1}{c}{correct} & \multicolumn{1}{c}{all} & \multicolumn{1}{c}{equal} & \multicolumn{1}{c}{all} & \multicolumn{1}{c}{equal} \\
\cmidrule(lr){1-1}\cmidrule(lr){2-3}\cmidrule(lr){4-5}\cmidrule(lr){6-6}\cmidrule(lr){7-8}\cmidrule(lr){9-10}

\textbf{base300} & \predicateBaseResultsPredBitvectorsCorrectTruePlain & \predicateBaseResultsPredBitvectorsCorrectFalsePlain & \predicateBaseResultsPredBitvectorsIncorrectTruePlain & \predicateBaseResultsPredBitvectorsIncorrectFalsePlain & \predicateBaseResultsPredBitvectorsCorrectFinishedMainPlain & \predicateBaseResultsPredBitvectorsAllWallTimeSumPlainHours & \predicateBaseResultsPredBitvectorsEqualOnlyWallTimeSumPlainHours & \predicateBaseResultsPredBitvectorsAllCpuTimeSumPlainHours & \predicateBaseResultsPredBitvectorsEqualOnlyCpuTimeSumPlainHours \\
\textbf{base600} & \predicateBaseLongtimeoutResultsPredBitvectorsCorrectTruePlain & \predicateBaseLongtimeoutResultsPredBitvectorsCorrectFalsePlain & \predicateBaseLongtimeoutResultsPredBitvectorsIncorrectTruePlain & \predicateBaseLongtimeoutResultsPredBitvectorsIncorrectFalsePlain & \predicateBaseLongtimeoutResultsPredBitvectorsCorrectFinishedMainPlain & \predicateBaseLongtimeoutResultsPredBitvectorsAllWallTimeSumPlainHours & \predicateBaseLongtimeoutResultsPredBitvectorsEqualOnlyWallTimeSumPlainHours & \predicateBaseLongtimeoutResultsPredBitvectorsAllCpuTimeSumPlainHours & \predicateBaseLongtimeoutResultsPredBitvectorsEqualOnlyCpuTimeSumPlainHours \\
\textbf{basePar} & \predicateBaseParallelResultsPredBitvectorsCorrectTruePlain & \predicateBaseParallelResultsPredBitvectorsCorrectFalsePlain & \predicateBaseParallelResultsPredBitvectorsIncorrectTruePlain & \predicateBaseParallelResultsPredBitvectorsIncorrectFalsePlain & \predicateBaseParallelResultsPredBitvectorsCorrectFinishedMainPlain & \predicateBaseParallelResultsPredBitvectorsAllWallTimeSumPlainHours & \predicateBaseParallelResultsPredBitvectorsEqualOnlyWallTimeSumPlainHours & \predicateBaseParallelResultsPredBitvectorsAllCpuTimeSumPlainHours & \predicateBaseParallelResultsPredBitvectorsEqualOnlyCpuTimeSumPlainHours \\
\textbf{abs} & \predicateBitpreciseParallelInvariantsResultsAsyncInvariantsAbsCorrectTruePlain & \predicateBitpreciseParallelInvariantsResultsAsyncInvariantsAbsCorrectFalsePlain & \predicateBitpreciseParallelInvariantsResultsAsyncInvariantsAbsIncorrectTruePlain & \predicateBitpreciseParallelInvariantsResultsAsyncInvariantsAbsIncorrectFalsePlain & \predicateBitpreciseParallelInvariantsResultsAsyncInvariantsAbsCorrectFinishedMainPlain & \predicateBitpreciseParallelInvariantsResultsAsyncInvariantsAbsAllWallTimeSumPlainHours & \predicateBitpreciseParallelInvariantsResultsAsyncInvariantsAbsEqualOnlyWallTimeSumPlainHours & \predicateBitpreciseParallelInvariantsResultsAsyncInvariantsAbsAllCpuTimeSumPlainHours & \predicateBitpreciseParallelInvariantsResultsAsyncInvariantsAbsEqualOnlyCpuTimeSumPlainHours \\
\textbf{path} & \predicateBitpreciseParallelInvariantsResultsAsyncInvariantsPathCorrectTruePlain & \predicateBitpreciseParallelInvariantsResultsAsyncInvariantsPathCorrectFalsePlain & \predicateBitpreciseParallelInvariantsResultsAsyncInvariantsPathIncorrectTruePlain & \predicateBitpreciseParallelInvariantsResultsAsyncInvariantsPathIncorrectFalsePlain & \predicateBitpreciseParallelInvariantsResultsAsyncInvariantsPathCorrectFinishedMainPlain & \predicateBitpreciseParallelInvariantsResultsAsyncInvariantsPathAllWallTimeSumPlainHours & \predicateBitpreciseParallelInvariantsResultsAsyncInvariantsPathEqualOnlyWallTimeSumPlainHours & \predicateBitpreciseParallelInvariantsResultsAsyncInvariantsPathAllCpuTimeSumPlainHours & \predicateBitpreciseParallelInvariantsResultsAsyncInvariantsPathEqualOnlyCpuTimeSumPlainHours \\
\textbf{prec} & \predicateBitpreciseParallelInvariantsResultsAsyncInvariantsPrecCorrectTruePlain & \predicateBitpreciseParallelInvariantsResultsAsyncInvariantsPrecCorrectFalsePlain & \predicateBitpreciseParallelInvariantsResultsAsyncInvariantsPrecIncorrectTruePlain & \predicateBitpreciseParallelInvariantsResultsAsyncInvariantsPrecIncorrectFalsePlain & \predicateBitpreciseParallelInvariantsResultsAsyncInvariantsPrecCorrectFinishedMainPlain & \predicateBitpreciseParallelInvariantsResultsAsyncInvariantsPrecAllWallTimeSumPlainHours & \predicateBitpreciseParallelInvariantsResultsAsyncInvariantsPrecEqualOnlyWallTimeSumPlainHours & \predicateBitpreciseParallelInvariantsResultsAsyncInvariantsPrecAllCpuTimeSumPlainHours & \predicateBitpreciseParallelInvariantsResultsAsyncInvariantsPrecEqualOnlyCpuTimeSumPlainHours \\
\textbf{prec-path} & \predicateBitpreciseParallelInvariantsResultsAsyncInvariantsPrecPathCorrectTruePlain & \predicateBitpreciseParallelInvariantsResultsAsyncInvariantsPrecPathCorrectFalsePlain & \predicateBitpreciseParallelInvariantsResultsAsyncInvariantsPrecPathIncorrectTruePlain & \predicateBitpreciseParallelInvariantsResultsAsyncInvariantsPrecPathIncorrectFalsePlain & \predicateBitpreciseParallelInvariantsResultsAsyncInvariantsPrecPathCorrectFinishedMainPlain & \predicateBitpreciseParallelInvariantsResultsAsyncInvariantsPrecPathAllWallTimeSumPlainHours & \predicateBitpreciseParallelInvariantsResultsAsyncInvariantsPrecPathEqualOnlyWallTimeSumPlainHours & \predicateBitpreciseParallelInvariantsResultsAsyncInvariantsPrecPathAllCpuTimeSumPlainHours & \predicateBitpreciseParallelInvariantsResultsAsyncInvariantsPrecPathEqualOnlyCpuTimeSumPlainHours \\
\textbf{abs-path} & \predicateBitpreciseParallelInvariantsResultsAsyncInvariantsAbsPathCorrectTruePlain & \predicateBitpreciseParallelInvariantsResultsAsyncInvariantsAbsPathCorrectFalsePlain & \predicateBitpreciseParallelInvariantsResultsAsyncInvariantsAbsPathIncorrectTruePlain & \predicateBitpreciseParallelInvariantsResultsAsyncInvariantsAbsPathIncorrectFalsePlain & \predicateBitpreciseParallelInvariantsResultsAsyncInvariantsAbsPathCorrectFinishedMainPlain & \predicateBitpreciseParallelInvariantsResultsAsyncInvariantsAbsPathAllWallTimeSumPlainHours & \predicateBitpreciseParallelInvariantsResultsAsyncInvariantsAbsPathEqualOnlyWallTimeSumPlainHours & \predicateBitpreciseParallelInvariantsResultsAsyncInvariantsAbsPathAllCpuTimeSumPlainHours & \predicateBitpreciseParallelInvariantsResultsAsyncInvariantsAbsPathEqualOnlyCpuTimeSumPlainHours \\
\textbf{prec-abs} & \predicateBitpreciseParallelInvariantsResultsAsyncInvariantsPrecAbsCorrectTruePlain & \predicateBitpreciseParallelInvariantsResultsAsyncInvariantsPrecAbsCorrectFalsePlain & \predicateBitpreciseParallelInvariantsResultsAsyncInvariantsPrecAbsIncorrectTruePlain & \predicateBitpreciseParallelInvariantsResultsAsyncInvariantsPrecAbsIncorrectFalsePlain & \predicateBitpreciseParallelInvariantsResultsAsyncInvariantsPrecAbsCorrectFinishedMainPlain & \predicateBitpreciseParallelInvariantsResultsAsyncInvariantsPrecAbsAllWallTimeSumPlainHours & \predicateBitpreciseParallelInvariantsResultsAsyncInvariantsPrecAbsEqualOnlyWallTimeSumPlainHours & \predicateBitpreciseParallelInvariantsResultsAsyncInvariantsPrecAbsAllCpuTimeSumPlainHours & \predicateBitpreciseParallelInvariantsResultsAsyncInvariantsPrecAbsEqualOnlyCpuTimeSumPlainHours \\
\textbf{prec-abs-path} & \predicateBitpreciseParallelInvariantsResultsAsyncInvariantsPrecAbsPathCorrectTruePlain & \predicateBitpreciseParallelInvariantsResultsAsyncInvariantsPrecAbsPathCorrectFalsePlain & \predicateBitpreciseParallelInvariantsResultsAsyncInvariantsPrecAbsPathIncorrectTruePlain & \predicateBitpreciseParallelInvariantsResultsAsyncInvariantsPrecAbsPathIncorrectFalsePlain & \predicateBitpreciseParallelInvariantsResultsAsyncInvariantsPrecAbsPathCorrectFinishedMainPlain & \predicateBitpreciseParallelInvariantsResultsAsyncInvariantsPrecAbsPathAllWallTimeSumPlainHours & \predicateBitpreciseParallelInvariantsResultsAsyncInvariantsPrecAbsPathEqualOnlyWallTimeSumPlainHours & \predicateBitpreciseParallelInvariantsResultsAsyncInvariantsPrecAbsPathAllCpuTimeSumPlainHours & \predicateBitpreciseParallelInvariantsResultsAsyncInvariantsPrecAbsPathEqualOnlyCpuTimeSumPlainHours \\



\bottomrule
 \end{tabular}
 \end{adjustbox}

\end{table}

\autoref{table:parallel} shows many interesting facts about this way of generating and using invariants. At first the overall CPU time of all analyses using parallel analyses is about \SI{87}{\percent} higher
than \textbf{base300}. This is not surprising as we configured the parallel analyses to be able to use at most \SI{600}{\second}, and there are two analyses running concurrently. Compared to \textbf{base600} the
overall CPU time is only increasing by approximately \SI{6}{\percent}. When only looking at the correctly and equally analyzed tasks, the parallel analyses consequently take about \SI{90}{\percent} more CPU time than
\textbf{base300} and \textbf{base600}.
By looking at the wall time the picture changes. Over all tasks the wall time of \textbf{base600} is \SI{240}{\hour} and the wall times of the parallel analyses are around \SI{150}{\hour}, about \SI{37}{\percent} lower. For the 
correctly and equally analyzed tasks the wall time for \textbf{base300} and \textbf{base600} is shorter than any of the parallel analyses. It is noticeable that while the number of correctly 
analyzed tasks is increasing for all parallel configurations, the number of wrongly analyzed tasks decreases by 9, about \SI{33}{\percent}. In the next paragraphs we will only use \textbf{basePar} for 
comparisons, because this baseline is closest to the configurations using invariants.

\textbf{basePar} is the slowest of all parallel configurations, meaning that the usage of invariants boosts the CPU and wall time of the analyses. The wall time is overall about \SI{3}{\percent} higher and for the 
equal tasks about \SI{8}{\percent} higher. For the CPU time it is overall about \SI{2}{\percent} higher, and \SI{4}{\percent} higher for the equally and correctly analyzed tasks.
While the consumed time decreases when using invariants, the number of 
correctly analyzed tasks rises between \num{25} and \num{54}, making the performance of the analyses strictly better than \textbf{basePar}.

%file:///localhome/stieglma/workspace/CPAcheckerTrunk/test/programs/benchmarks/heap-manipulation/sll_to_dll_rev_false-unreach-call.i
By taking a closer look at the configurations using invariants we can see that the configurations where the invariants are conjoined to the path formula are analyzing one unsafe task wrongly and report 
that there is no specification violation. By digging deeper we found out that at some point, the conjunction of path formula and invariant became unsatisfiable, immediately leading to the wrong result.
When comparing the path formula and the invariant at this point, one can see that the invariant assumes that a variable (a pointer) has the address zero, where it has another address in the path formula.
This is no encoding issue in the way we thought about it, instead it has to do with how (aliased) pointers are handled in different \acp{CPA}. While the \PredicateCPA{} handles such cases with
uninterpreted functions and does not expect value assignments to such variables, there is no special handling in the \InvariantsCPA{} at all. The \InvariantsCPA{} uses a separate \ac{CPA} for that, and the formulas generated by the \InvariantsCPA{} do 
unfortunately contain assumptions about pointers being 0.\,\sidenote[57][-1cm]{The full path formula, the invariant, and the interpolant for the program
\texttt{heap-manipulation/sll\\\_to\_dll\_rev\_false-unr\\each-call.i} can be found on our supplementary web page.}
This leads to the unwanted behavior we observed here. That this problem leads to a different results only one time in over \num{3400} verification tasks makes the problem 
even harder to find. A reason for this issue not having bad effects on the \textbf{-abs} configurations is that the interfering parts of the path formula are removed during abstraction due to the 
precision.
\setcounter{sidenote}{58}

%As for all invariant usage approaches besides conjoining them to the path formula the verification task is analyzed correctly we think that the problem lies in the usage strategy.

Apart from the wrongly analyzed task there are more differences, for example regarding the number of tasks where the result comes from the main analysis using the \PredicateCPA{} and not from the additional 
analysis using the \InvariantsCPA{}. From \num{2104} correctly analyzed tasks with \textbf{async-abs}, \num{1154} results are reported by the main analysis, about \SI{55}{\percent}. For all configurations where the invariants 
are appended to the precision, this ratio is about \SI{53}{\percent}, meaning that the main analysis became slower such that more results are given by the analysis with the \InvariantsCPA{}. This can also be seen 
by the wall and CPU times, which are higher for all \textbf{-prec} configurations.

In \autoref{title:combiningInv} we made some assumptions about the performance of the different approaches, stating that \textbf{-abs}, \textbf{-path}, \textbf{-prec} and \mbox{\textbf{-prec-path}} look most 
promising. While \textbf{-prec} suffers from bad performance due to more necessary computations during the abstraction, we need to be sure that we have invariants (with correct encoding) to be able
to safely use them with \textbf{-abs}. From \autoref{table:parallel} we can see that \textbf{async-abs} is the best configuration we have used, which is an indicator that our formula conversion works 
very well. Additionally we found out that the performance drawback of appending invariants to the precision is considerable. The configurations conjoining the invariants to the path formula are almost as 
good as \textbf{async-abs}, however, they have one wrongly analyzed unsafe result, making the conclusion about their validity impossible. 








