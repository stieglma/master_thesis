\chapter{Motivation}
Predicate analysis is one of the main approaches in software verification. Its success is based on the recent improvements made to \acs{SMT} solvers which are mainly used as back-end for solving
the formulas created during the analysis.

Another huge enhancement, \acl{ABE}~\cite{Beyer:PredicateAbstraction}, was invented in 2010 and is implemented in the \CPAchecker{} framework as a part of the 
\PredicateCPA{}. With this work we try to further enhance the \PredicateCPA{} by generating and using auxiliary invariants. The invariants should make the analysis faster and less dependent on the
interpolation abilities of the \acs{SMT} solvers. Interpolation is nice to have and easy to use on the one hand, but inappropriate interpolants may lead to loops being unnecessarily unrolled and therefore
to longer run times. We try to circumvent this issue by heuristics and separate analyses which generate invariants that can be used as a replacement for interpolation with \acs{SMT} solvers.

One of the main contributions of this thesis is the development of an algorithm to concurrently run several analyses on the same verification task. Besides that,
we formalize different invariant generation and usage strategies,evaluate their impact on a predicate analysis, and show their individual strengths and weaknesses.
The concurrent execution of analyses comes hand in hand with the necessity of 
communication between these analyses. While there was some work on passing information from an earlier running analysis to a later one~\cite{Beyer:PrecisionReuse}
no one has extended this to passing results from one concurrently running analysis to another one. This is, however, necessary for computing invariants with one analysis,
which should then be used in another analysis.

\clearpage
\section*{Structure of This Work}
This work is structured into three main parts, the first part contains all background information and related work, including in-depth information about the \CPAchecker{} framework and all kinds of 
projects using auxiliary invariants to enhance their analyses. The second part is about our additions; we explain different invariant generation heuristics, as well as how they can be used in the 
\PredicateCPA{}. For more flexibility, we do also create a new algorithm, which allows the concurrent execution of several analyses in one verification run.
The third and last part consists of the evaluation, giving detailed insights into our experiments and the advantages and drawbacks we found. We do also mention the restrictions
and challenges we had, and finally conclude this thesis with a summary and an outlook on how the process of using invariants can be further extended and improved.